\documentclass[deligne.tex]{subfiles}

\begin{document}
	\subsection*{1.1.11} To verify the anti-commutativity of the 9th square, as
	the morphism of triangles $(X'',Y'',Z'')\ra(X'[1],Y'[1],Z'[1])$ factors
	as the composition of two morphisms of triangles
	$(X'',Y'',Z'')\ra(A,Y'',Z'[1])\ra(X'[1],Y'[1],Z'[1])$, where the second
	arrow is the rotation of $(Z',A,Y'')\ra(Z',X'[1],Y'[1])$, it suffices
	to verify that the triangle $(Z',X'[1],Y'[1])$ which appears in this last
	morphism of triangles has all arrows induced by the arrows of
	$(X',Y',Z')$ or translates of them (with the same parity).
	This is not hard to check from the diagram (1).
	(The stated explanation appears to be an un-explanation.)
	
	\subsection*{1.3.3} Though it is not stated explicitly, it is immediate from
	the definition $\mathcal D^{\leq n}:=\mathcal D^{\leq 0}[-n]$,
	$\mathcal D^{\geq n}:=\mathcal D^{\geq 0}[-n]$ that
	$(\tau_{\leq n}X)[m]=\tau_{\leq n-m}(X[m])$ and
	$(\tau_{\geq n}X)[m]=\tau_{\geq n-m}(X[m])$.
	Namely, for $X$ in $\mathcal D$ and $T$ in $\mathcal D^{\leq n}$,
	$T=T'[-m]$ for some $T'$ in $\mathcal D^{\leq n-m}$, so
	\begin{align*}
		&\Hom(T,(\tau_{\leq n-m}(X[m]))[-m])
		=\Hom(T'[-m], (\tau_{\leq n-m}(X[m]))[-m]) \\
		=&\Hom(T'[-m],\tau_{\leq n-m}(X[m]))[-m]
		=\Hom(T',X[m])[-m]
		=\Hom(T'[-m],X) \\
		=&\Hom(T,X).
	\end{align*}
	
	\subsection*{1.4.2.1} The argument for why the derived functors continue to
	satisfy the stated adjunctions is as follows (this argument is also found
	in SGA 4 Exp. XVIII 3.1.4.11). Given $F,G$ an adjoint pair
	of functors on abelian categories
	\begin{equation*}\begin{tikzcd}\mathcal A\arrow[r,shift left=.5ex,"{F}"]&\mathcal B \arrow[l,shift left=.5ex,"{G}"]\end{tikzcd}\end{equation*}
	where both categories have enough injectives and $L$ is exact.
	The functors $F$ and $G$ extend to
	functors $D^+(\mathcal A)\rightleftarrows D^+(\mathcal B)$.
	Given $K^\dotp\in D^+(\mathcal A),L^\dotp\in D^+(\mathcal B)$,
	we may assume $L^\dotp$ is a complex of injective objects;
	we have an isomorphism of triple complexes
	\begin{equation*}
		\begin{tikzcd}
		\Hom^\dotp(F(K^\dotp),L)\arrow[r,leftrightarrow,"{\sim}"]&\Hom^\dotp(K^\dotp,G(L^\dotp))\arrow[l].
		\end{tikzcd}
	\end{equation*}
	As $G$ preserves injectives, taking $H^0$ of the associated simple complex 
	(calculated with products) finds the desired
	\begin{equation*}
	\begin{tikzcd}
		\Hom_{K(\mathcal B)}(F(K^\dotp),L^\dotp)\arrow[r,leftrightarrow,"{\sim}"]&
		\Hom_{K(\mathcal A)}(K^\dotp,G(L^\dotp)),\arrow[l]
	\end{tikzcd}
	\end{equation*}
	where both sides are also $\Hom$ in the respective derived categories,
	since $L^\dotp$ and $G(L^\dotp)$ are complexes of injectives.
	
	\subsection*{1.4.4} The question is, why are the adjoints to the Verdier 
	quotients fully faithful? Let's consider the quotient
	$Q:\mathcal T\ra\mathcal T/\mathcal U$, where $\mathcal U$ is the strictly
	full coreflective triangulated subcategory of $\mathcal T$;
	$(\mathcal U,\mathcal V)$ form a t-structure on $\mathcal T$;
	$\mathcal U={}^\perp\mathcal V$, and $\mathcal V=\mathcal U^\perp$.
	Since the embedding $u:\mathcal U\ra\mathcal T$ admits a right adjoint
	$u_\dotp$ it follows that $Q$ admits a right adjoint $Q_\dotp$~\cite[I 6-5]{CD}.
	There is a natural isomorphism of functors
	$Q_\dotp\circ Q\xrightarrow{\sim}v\circ v^\dotp$,
	where $v^\dotp$ is the left adjoint to the inclusion
	$v:\mathcal V\ra\mathcal T$~\cite[I 6-6]{CD}.
	The functor $v^\dotp$ is nothing other
	than $\tau_{\geq0}$ for the t-structure $(\mathcal U,\mathcal V)$,
	and therefore, restricted to $\mathcal V$,
	$v\circ v^\dotp=\id|_{\mathcal V}$.
	On the other hand, the functor $Q$ when restricted to $\mathcal V$
	is fully faithful~\cite[I 5-3]{CD}. Therefore $Q_\dotp$ restricted to the
	essential image of $\mathcal V$ under $Q$ is fully faithful.
	This essential image is all of $\mathcal T/\mathcal U$, since every object
	$X$ in $\mathcal T$ belongs to an exact triangle $(U,X,V)$ with $U,V$
	objects in $\mathcal U,\mathcal V$, respectively.
	The assertion that $v^\dotp$ yields an equivalence
	$\underline v^\dotp:\mathcal T/\mathcal U\xrightarrow{\sim}\mathcal V$
	(functor obtained by
	applying the universal property of $\mathcal T/\mathcal U$ to $v^\dotp$) 
	is easy since $Q\circ v:\mathcal V\ra\mathcal T/\mathcal U$ is an 
	equivalence, and $\underline v^\dotp\circ Q\circ v=v^\dotp\circ v=\id$.
	The corresponding
	statement for $\mathcal T\ra\mathcal T/\mathcal V$ follows identically.
	
	\subsection*{1.4.6} In part b), there is the following consideration.
	Given triangulated categories $\mathcal T,\mathcal T'$, a thick 
	subcategory $\mathcal U\subset\mathcal T$ with Verdier quotient	$Q:\mathcal T\ra\mathcal T/\mathcal U$, exact
	functors $F,G:\mathcal T/\mathcal U\ra\mathcal T'$ and a natural 
	transformation $\overline\varphi:F\circ Q\ra G\circ Q$, there is an 
	obvious candidate for a natural transformation $\varphi:F\ra G$, since
	$\operatorname{Ob}(\mathcal T/\mathcal U)=\operatorname{Ob}(\mathcal T)$.
	But is it still a natural transformation? Let $f:X\ra Y$ in
	$\mathcal T/\mathcal U$ be represented by $X\xleftarrow{s}Z\xrightarrow{a}Y$,
	where $s$ is in the saturated multiplicative system of morphisms 
	corresponding to $\mathcal U$. The commutative diamond
	\begin{equation*}
	\begin{tikzcd}
		&&F(Z)\arrow[dl,swap,"F(s)"]\arrow[dr,"\varphi(Z)"] \\
		&F(X)\arrow[dl,swap,"\id"]\arrow[dr,"\varphi(X)"]&&
		G(Z)\arrow[dl,swap,"G(s)"]\arrow[dr,"G(a)"]\\
		F(X)&&G(X)&&G(Y)
	\end{tikzcd}
	\end{equation*}
	shows that $\varphi(X)\circ G(f)$ coincides with
	$F(X)\xleftarrow{F(s)}F(Z)\xrightarrow{G(a)\circ\varphi(Z)}G(Y)$, but as
	\begin{equation*}
	\begin{tikzcd}
		%F(X)\arrow[d,"\varphi(X)"]& \arrow[l,swap,"F(s)"]
		F(Z)\arrow[r,"F(a)"]\arrow[d,"\varphi(Z)"]
		&F(Y)\arrow[d,"\varphi(Y)"] \\
		%G(X)& \arrow[l,swap,"G(s)"]
		G(Z)\arrow[r,"G(a)"]
		&G(Y)
	\end{tikzcd}
	\end{equation*}
	commutes, this morphism is just
	$F(X)\xleftarrow{F(s)}F(Z)\xrightarrow{\varphi(Y)\circ F(a)}G(Y)=F(f)\circ\varphi(Y)$, which shows that $\varphi$ is indeed a natural 
	transformation.
	
	\subsection*{1.4.7} In c), by 1.1.9, the morphism $B\ra C$ is the unique such
	that completes the morphism of triangles
	$(B,j_!j^*X[1],j_*j^*X[1])\ra(C,X[1],j_*j^*X[1])$.
	
	\subsection*{1.4.13} The distinguished triangle
	$(\tau_{\leq p}^FX,X,i_*\tau_{>p}i^*X)$ is the distinguished triangle
	$(A,Y,i_*\tau_{>p}i^*Y)$ of 1.4.10, since as remarked, $X=Y$ since
	$\tau_{>0}j^\ast X=0$. To check that $A=\tau_{\leq p}^FX$, note
	that $A$ belongs to $\mathcal D^{\leq p}$, as
	$i^\ast A\simeq\tau_{\leq p}i^\ast X$, and that if $T$ belongs to
	$\mathcal D^{\leq p}$, then by applying $\Hom(T,-)$ to the above
	distinguished triangle and observing that as $i_\ast$ is t-exact, it
	commutes with truncation, so
	$i_\ast\tau_{>p}i^\ast X[-1]$ lies in $\mathcal D^{>p+1}$,
	$\Hom(T,i_\ast\tau_{>p}i^*X)=0=\Hom^{-1}(T,i_\ast\tau_{>p}i^\ast X)$,
	and $\Hom(T,X)\simeq\Hom(T,A)$.
	
	The distinguished triangle $(\tau_{\leq p-1}^FX,X,i_*\tau_{>p-1}i^*X)$
	and the fact that $i_*$ commutes with truncation establishes
	$i_*\tau_{>p-1}i^*X$ as $\tau_{>p-1}X$ for the t-structure on 
	$\mathcal D$; applying $\tau_{\leq p}$ and passing it through the $i_*$
	gives the statement about cohomology.
	
	\subsection*{1.4.14} To find the dual statement at the end of the proof,
	reverse arrows and exchange $j_!\leftrightarrow j_*$,
	$i^*\leftrightarrow i^!$ to obtain the distinguished triangle
	$(i_*i^*X[-1],j_!Y,X)$, then use (b'), the isomorphism
	$j_*/j_!\simeq i^!j_![1]$ of 1.4.6.4, (and the note to 1.3.3) to write
	\begin{equation*}
		i_*i^*X[-1]=i_*(\tau_{\leq p-1}(j_*/j_!)Y)[-1]
		=i_*\tau_{\leq p}((j_*/j_!)Y[-1])
		=i_*\tau_{\leq p}i^!j_!Y,
	\end{equation*}
	establishing $X$ as $\tau^F_{\geq p+1}j_!Y$.
	
	\subsection*{1.4.17.1} A little note: $^p i^*X$ is the largest quotient of $X$
	belonging to $\mathcal C_F$. First we check that it is a quotient from
	1.4.17 (ii). Then, suppose $A$ belongs to $\mathcal C_F$ and
	$X\twoheadrightarrow A$; then $^pi^*X\twoheadrightarrow{}^pi^*A
	\xrightarrow{\sim}A$, as ${}^pi_*$ is fully faithful, so the adjunction
	morphism ${}^pi^*{}^pi_*\ra\id$ is an isomorphism, and ${}^pi^*X$
	is indeed the largest quotient of $X$ in $\mathcal C_F$.
	Dually for ${}^pi^!$.
	
	\subsection*{1.4.18} A little note about $T$ faithful: as $^pj^*$ is an exact
	functor, if $^pj^*f_1=0$, this means that 
	${}^pj^*\im(f_1)=\im(^pj^*f_1)=0$, which is to say that $\im f_1$
	belongs to $\overline{\mathcal C}_F$.
		
	\subsection*{1.4.23} In the distinguished triangle
	$(i_*H^0i^!j_!B,\tau_{\geq0}^Fj_!B,\tau_{\geq1}^Fj_!B)$,
	as $j_!B=\tau_{\geq0}^Uj_!B$,
	$\tau_{\geq p}=\tau_{\geq p}^F\tau_{\geq p}^U$,
	and $j_!$ is right t-exact, $\tau_{\geq0}^Fj_!B$ sits in $\mathcal C$.
	Likewise, $i_*$ is t-exact, so $i_*H^0i^!j_!B$ also sits in
	$\mathcal C$, and from the long exact sequence of $H^i$ one finds
	that $\tau_{\geq1}^Fj_!B$ is in $\mathcal D^{[-1,0]}$.
	
	\subsection*{2.1.2} In the discussion `\emph{Si les foncteurs $^\circ i^!_S$
	sont de dimension cohomologique finie…}' it is claimed that there is a
	neighborhood of $S$ in which $H^j\tau_{<a}K$ is supported on $S$.
	To find such a neighborhood, simply discard $\overline S-S$ and the
	closure of any stratum which doesn't meet $S$. The assumption that the
	closure of each stratum is a union of strata implies that the induced
	stratification of the resulting neighborhood of $S$ has the property
	that every stratum contains $S$ in its closure, and therefore
	$H^j\tau_{<a}K$ vanishes on every stratum distinct from $S$.
	By construction, $S$ is a closed set in this neighborhood.
	
	As for the isomorphism
	$H^i(i_S^*\tau_{<a}K)\xleftarrow{\sim}H^i(i_S^!\tau_{<a}K)$ for $i<a$,
	let us replace $X$ by the neighborhood above, in which case
	the adjunction morphism $\tau_{>a}K\ra{i_S}_*i_S^*\tau_{<a}K$ is an
	isomorphism as $^\circ{i_S}_*$ and $^\circ i_S^*$ are exact and the 
	induced morphism on cohomology
	$H^j(\tau_{>a}K)\ra{i_S}_*i_S^*H^j(\tau_{<a}K)$ is an
	isomorphism for all $j$. Therefore by 1.4.1.2,
	$i^!_S\tau_{<a}K\xrightarrow{\sim}i_S^!{i_S}_*i_S^*\tau_{<a}K\xrightarrow{\sim}i_S^*\tau_{<a}K$.
	
	\subsection*{2.1.13} To get the desired conclusion from the spectral sequence
	$R^pj_*H^qK\Rightarrow H^{p+q}Rj_*K$, recall that the locally constant
	constructible sheaves form a weak Serre subcategory of the category of 
	constructible sheaves of $\mathcal O$-modules. For $j^!$ when $j$ is a
	closed immersion, just use that $j_!=j_*$ is exact in the long exact
	sequence of cohomology for the distinguished triangle
	$(j_!j^!K,K,k_*k^*K)$.
	
	One can deduce that the truncation operators $\tau_{\leq p}$ and
	$\tau_{\geq p}$ respect $D_c(X,R)$ from the proof of 
	1.4.10 by induction on the number of strata à la 2.1.3.
	
	\subsection*{2.1.14} For the isomorphism $i^!L\simeq i^*L\otimes_{\ZZ}\Or[-d]$,
	combine Proposition 4.3.6 from Dimca, \emph{Sheaves in Topology} with
	the description of the relative dualizing complex in, e.g.,
	Remark 3.3.5 in Kashiwara-Schapira, \emph{Sheaves on Manifolds}, taking
	$S=\{\ast\}$ there and noting
	that $f^!A_X$ is the relative dualizing complex $\omega_{Y/X}$
	(Definition 3.1.16), and the orientation sheaves are self-dual
	(so that Deligne's $\Or$ coincides with $\Or_T\otimes i^*\Or_S$).
	The `non-characteristic' hypothesis in Dimca's Proposition is trivial
	in light of the result that precedes it and the fact that the
	$H^mj^!K$ are locally constant.
	
	Note that by the description of $i^!$ in terms of local cohomology,
	this gives a statement like, $H^n_T(S,\mathcal F)$ vanishes in degrees
	less than the codimension of $T$, for $\mathcal F$ a locally free sheaf
	(with obvious extension to $\mathcal F$ a bounded below
	complex with locally free $H^n$).
	
	\subsection*{2.1.16}\label{BBD:2.1.16}
	In the remarks about Verdier duality, the fact that the functor $D$
	exchanges $j_!\leftrightarrow j_*$, $j^!\leftrightarrow j^*$ follows by 
	taking $L$ to be the dualizing complex in the local formulas of adjunction
	\begin{align*}
		j_*DK&=j_*R\underline\Hom(K,j^!L)\xrightarrow{\sim}
		R\underline\Hom(j_!K,L)=Dj_!K \\
		Dj^*K&=R\underline\Hom(j^*K,j^!L)\xrightarrow{\sim}
		j^!R\underline\Hom(K,L)=j^!DK
	\end{align*}
	(the latter may be found, e.g. in SGA 4 Exp. XVIII 3.1.12.2,
	or Dimca, 3.3.7, but see especially SGA 5 Exp. I 1.12).
	These formulæ hold for more general morphisms than the inclusion of a locally closed subscheme; the condition is compactifiability.

	In \emph{Th. finitude} \S4 `Bidualité locale,' Deligne
	puts Verdier duality for étale cohomology on firm footing in the case of
	$a:X\ra S$ a scheme of finite type over $S$ a regular scheme of 
	dimension 0 or 1.
	If $A=\ZZ/n$, $K_S$ constant sheaf on $S$ with value $A$, and $K$ in 
	$D_{ctf}^b(X,A)$, put $K_X:=Ra^!K_S$ and $DK:=R\underline\Hom(K,K_X)$.
	Then $K_X$ is dualizing; i.e.
	\begin{theorem*}[Deligne]
		$K\xra\sim DDK$.
	\end{theorem*}
	This involutivity establishes the stated duality in the formalism, since
	we may write
	\begin{equation*}
		Dj_*K=Dj_*DDK=DDj_!DK=j_!DK.
	\end{equation*}
	It is essential that the cohomology sheaves be locally
	constant of finite rank when restricted to each stratum and that $R$ 
	have the proscribed properties so that Poincaré duality holds on each 
	stratum. In fact, the definition of perverse sheaf is engineered 
	expressly so that on each stratum we have Poincaré duality, and this
	data determines the Verdier dual of the sheaf on the stratified space.
	
	About the formula $H^iDK=(H^{-d-i}K)^\vee\otimes\Or$, let's do it
	instead in the case $R=\ZZ/\ell$, in which case the dualizing
	complex is $\ZZ/\ell(d)[2d]$ and the formula is
	\begin{equation*}
		H^iDK=(H^{-2d-i}K)^\vee(d).\tag{$\dagger$}
	\end{equation*}
	This is simply Poincaré duality (SGAA XVIII 3.2.5).
	(In light of
	\hyperref[thfin:2.1]{the note to 2.1 in \emph{Th. finitude}}, this is
	the immediate consequence of the weakly convergent spectral
	sequence
	\begin{equation*}
		E_2^{pq}=\operatorname{Ext}^p(H^{-q}(K),\ZZ/\ell(d))\Rightarrow
		\operatorname{Ext}^{p+q}(K,\ZZ/\ell(d))
	\end{equation*}	
	which collapses at the $E_2$ page.)
	
	The business about exchanging $^{p^*}H^i$ and $^pH^{-i}$ is seen to be
	true for $H^0$, then use $H^iK=H^0(K[i])$ and
	$D(K[n])=R\underline\Hom(K[n],-)=R\underline\Hom(K,-)[-n]=(DK)[-n]$
	
	\subsection*{2.2.2}
	A word about the `trivial' implication (ii)$\Rightarrow$(iii).
	As each $S$ in $\mathcal T$ is lisse equidimensional, applying (ii) to
	each irreducible component we find there exists a Zariski dense open 
	$i:U\hookrightarrow S$ such that $i^*H^ii_S^*K=H^ii^*i_S^*K$
	(resp. $i^*H^ii_S^!K=H^ii^!i_S^!K$) vanish in degrees $i>p(S)$
	(resp. $i<p(S)$); as $H^ii_S^*K$ and $H^ii_S^!K$ are locally constant,
	this implies they also vanish when their restrictions to $U$ do.
	
	In the second paragraph, `\emph{il reste à montrer que chacune implique 
	que $H^iK=0$ pour $i>b$ (resp. $i<a$).}' Without securing this, we would 
	not have that $\tau_{\geq a}K$ (resp. $\tau_{\leq b}K$) belongs to 
	$D^b_c$ and the proof in the first paragraph wouldn't apply.
	In both cases, the verification for $i>b$ is easy as $^\circ i^*$ is
	exact.
	The verification for (iii) $i<a$ follows the proof of (2.1.2.1) exactly, 
	after $\mathcal T$ has been replaced by a finer stratification.
	The verification for (ii) follows immediately by the noetherian property
	from the stated claim that for each irreducible subvariety $S'$ and each
	$i<a$, there exists a dense open $S$ of $S'$ such that $H^iK$ vanishes on 
	$S$. To prove this claim we proceed by descending induction on $\dim S'$,
	the maximal nonvacuous case being easy since such an $S'$ is an 
	irreducible component of $X$, and $H^iK$ vanishes on the dense open $S$ 
	obtained from (ii) since $S$ is open in $X$ and $i_S^!=i_S^*$ is exact.
	The case of general $S'$ will follow from the argument of the proof of 
	(2.1.2.1) (see note for 2.1.2) if we can find a neighborhood $U$ of the 
	$S$ obtained from (ii) so that $\supp((H^j\tau_{<a}K)|_U)\subset S$.
	Begin with the irreducible component containing $S'$; the 
	inductive assumption gives an open set which has either empty or nonempty 
	intersection with $S'$. If nonempty, then this intersection is the 
	desired open of $S'$ of the claim.
	If empty, $S'$ belongs to the complement. Throw away all irreducible
	components of the complement that do not contain $S'$ and apply the same
	process to the irreducible component of the complement that contains 
	$S'$. After finitely many steps we are left only with $S'$, and we throw
	away $S'-S$. The open neighborhood $U$ is the set we are left with after 
	throwing away finitely many closed sets of $X$, and $S$ is closed in $U$.
	
	\subsection*{2.2.3} In the discussion of the intermediate extension,
	the triangle $(\tau_{<t}i^*j_*A,i^*j_*A,\tau_{\geq t}i^*j_*A)$ is 
	distinguished, not the one written, and if $\tau_{\geq t}i^*j_*A$ is in
	$^pD_c^{\geq0}$, then indeed
	$^p\tau_{<0}i^*j_*A\xrightarrow{\sim}\tau_{<t}i^*j_*A$
	and $\tau_{\geq t}i^*j_*A\simeq{}^p\tau_{\geq 0}i^*j_*A$; the latter
	isomorphism establishes an isomorphism
	$^p\tau_{<0}^Fj_*A\xrightarrow{\sim}\tau_{<t}^Fj_*A$, which differs
	by one character from what is written.
	
	\subsection*{2.2.8} The discussion of $R\underline\Hom$ differs from what
	is proved in 2.1.20 in that there is no longer a fixed stratification.
	Fortunately, if $p\leq a,b\geq q$, there are only finitely many solutions
	to $i=m-n$ for a fixed $i$ and for $n\leq a$, $m\geq b$. Therefore for
	each $i$, we can apply the reasoning of 2.1.20 to a common refinement
	of only finitely many stratifications.
	
\subsection*{2.2.10}\label{BBD:2.2.10}
As remarked in \hyperref[BBD:2.1.16]{the note to 2.1.16}, we will an
involutive Verdier duality regardless of whether each stratum is smooth
or the sheaf on it is lisse (or has lisse cohomology).
The reason we restrict to smooth strata and complexes of sheaves $K$ with
lisse cohomology on each stratum is so that we can determine precisely the
degrees in which the Verdier dual $DK$ is concentrated due to the simple
description of the dualizing complex and the simple computation of
$R\underline\Hom$
(c.f. \hyperref[thfin:1.6]{the note to \emph{Th. finitude} 1.6}).
This is necessary to make Verdier duality a compatible operation vis à vis
the t-structures attached to a given perversity function and its dual.

The program outlined in this paragraph achieves both the smoothness of the
strata and the lisseté of the cohomology sheaves on each stratum so that 
the latter property is moreover respected by the six functors (and hence
also by Verdier duality).
Here is an explication of the smoothness condition (a).
	\begin{lemma*}[EGA $0_{IV}$ 22.5.8 \& IV 6.7.6, 6.7.8, Stacks tags \href{https://stacks.math.columbia.edu/tag/07EL}{\texttt{07EL}} \& \href{https://stacks.math.columbia.edu/tag/038X}{\texttt{038X}}]
		Let $X$ be a scheme locally of finite type over a field $k$ and
		$x\in X$. Then the following are equivalent:
		\begin{enumerate}[label=(\roman*)]
			\item $X\ra\Spec k$ is smooth at $x$.
			\item $X$ is geometrically regular at $x$, i.e. for every finite extension $k'$ of $k$, the semi-local ring $(\mathcal O_X)_x\otimes_k k'$ is regular.
			\item $X\times_k \overline k$ is regular at every point lying over $x$.
		\end{enumerate}
	\end{lemma*}
	The smoothness condition on strata is that over $\overline k$, each
	stratum $S$, with the reduced subscheme structure, is smooth.
	The claim is then that on $S$ equidimensional of dimension $d$, the
	dualizing complex is given by $\ZZ/\ell(d)[2d]$.
	After replacing $k$ by its perfect closure, anodyne operation with
	respect to the étale topology, we may assume $S$ is of finite type over a
	perfect field $k$. The fact that $S\times_k\overline k$ is smooth implies
	(in light of the lemma and Stacks tag 
	\href{https://stacks.math.columbia.edu/tag/030U}{\texttt{030U}})
	that $S$, with its reduced scheme structure, is smooth.
	In this case, the fact about the dualizing complex is standard
	(SGAA Exp. XVIII 3.2.5).
	
	\subsection*{2.2.14} A brief review of Galois cohomology of a finite
	field $k$, to recall why the groups $H^i(\Gal(\overline k/k'),\ZZ/\ell)$
	are finite for every finite extension $k'$ of $k$. We have
	\begin{align*}
		H^0&=H^1=\ZZ/\ell \\
		H^i&=0\quad i>1.
	\end{align*}
	The case of $H^0$ is obvious as it corresponds to taking
	$G=\Gal(\overline k/k')$-invariants of a trivial
	$G$-module. The $H^1$ is the corollary of a formula given in Serre,
	\emph{Corps Locaux}, Ch.\,XIII Prop.\,1 (p.\,197 in the 1968 édition Hermann).
	The vanishing in
	degrees $>1$ is because a finite field is $C_1$ and hence has finite
	cohomological dimension; now see (1.6) in
	\emph{Arcata}, SGA 4$\frac12$. For more details see Serre,
	\emph{Cohomologie Galoisienne}, Ch.\,II, \S3.
	
	Let $\mathcal D_n= D_{ctf}^b(X,\ZZ/\ell^n)$,
	$K,L$ objects of $\mathcal D_n$, $G=\Gal(\overline k,k)$,
	and $f:X\ra\Spec k$ the structure morphism.
	Why does the above imply that $\Hom_{\mathcal D_n}(K,L)$ is finite?
	Deligne's finiteness theorems show that the sheaves
	$R\underline\Hom(K,L)$ belong to $\mathcal D_n$.
	The Hochschild-Serre spectral sequence gives
	\begin{equation*}
		E^2_{ij}=H^i(G,R^j\Gamma(X\times_k\overline k,R\underline\Hom(K,L)))\Rightarrow R^{i+j}\Gamma(X,R\underline\Hom(K,L)).\tag{$\dagger$}
	\end{equation*}
	As $R\Gamma(X\times_k\overline k,R\underline\Hom(K,L))$ coincides with
	the stalk of the constructible sheaf $Rf_*R\underline\Hom(K,L)$
	at any geometric point of $\Spec k$, it belongs to $D^+_{c}(\ZZ/\ell^n)$
	(even $D^+_{ctf}$; c.f. Th.\,finitude~1.7).
	Fix $j$ and let
	$A:=R^j\Gamma(X\times_k\overline k,R\underline\Hom(K,L))$; it is a finite
	$G$-module. Let $U$ denote the kernel of $G\ra\Aut A$; it is an open
	normal subgroup of finite index corresponding to a finite extension
	$k'$ of $k$. The Galois group $U=\Gal(\overline k,k')$ acts trivially
	on $A$ and a simple dévissage reducing to the case $\ZZ/\ell$ shows that
	the $H^i(U,A)$ are finite. As $G/U$ is a finite group, the spectral 
	sequence (c.f. \emph{Cohomologie Galoisienne} \S2.6b)
	\begin{equation*}
		H^p(G/U,H^q(U,A))\Rightarrow H^{p+q}(G,A)
	\end{equation*}
	shows that the groups $H^i(G,A)$ are also finite, and therefore that the
	objects on the $E^2$ page of ($\dagger$) are finite $\ZZ/\ell^n$-modules
	so that $R^0\Gamma(X,R\underline\Hom(K,L))=\Hom_{\mathcal D_n}(K,L)$ is finite.

	\subsection*{2.2.16} See
	\hyperref[WeilII_1.1.2]{note to Weil II (1.1.2)}.
	In that paper, the functor $H^i(K)$ is defined as the pro-sheaf which is
	the projective system defined by the $H^i(K_n)$; the corresponding
	projective system is AR-isomorphic to an $(\ell\ZZ)$-adic sheaf in the 
	naïve sense. This allows us to upgrade the pointwise exact sequence (*)
	of those notes to the corresponding sequence of sheaves (2.2.16.1).
	There, he uses the notation $K\Lotimes\ZZ/\ell/n$ for $K_n$, whereas
	here $\otimes$ is used instead of $\Lotimes$. In the interest of
	consistency, I will continue with the notation
	$K\Lotimes\ZZ/\ell^n$ for $K_n$.
	So in this paragraph, we are implicitly in the AR category or in the
	category of pro-sheaves.
	(In Weil II, Deligne uses the definition
	of $\ZZ_\ell$-sheaf as any pro-sheaf in the essential image of the
	$\ZZ_\ell$-sheaves; i.e. the $(\ell\ZZ)$-adic objects of the category
	of abelian constructible sheaves.)
	For the business about $H^0$ inducing an equivalence
	between $D_c^b(X,\ZZ_\ell)^{\leq0}\cap D_c^b(X,\ZZ_\ell)^{\geq0}$ and
	$\ZZ_\ell$-constructible sheaves, this is simple, since for such a $K$
	we can represent
	$K\Lotimes\ZZ/\ell$ by a sheaf concentrated in degree 0.
	As every complex of flat sheaves representing the $K_n$ admits an
	$\ell$-adic filtration with successive quotients quasi-isomorphic to
	$K\Lotimes\ZZ/\ell$, we can represent $K$ by a projective system of
	flat sheaves concentrated in degree 0. In this case, $K$ is a bona fide
	$\ell$-adic sheaf.
	
	As for checking whether $K$ belongs to $D_c^b(X,\ZZ_\ell)^{\leq0}$,
	the statement is punctual and we may consider the problem in
	$D_{\mathrm{parf}}$. If $H^i(K\Lotimes\ZZ/\ell^n)$ is null for one $n$,
	then $H^i(K)\otimes\ZZ/\ell^n$ is null by (2.2.16.1), so
	$H^i(K)\otimes\ZZ/\ell$ is null. This imples by the exact sequence
	\begin{equation*}
		0\ra\ell H^i(K)/\ell^n H^i(K)\ra H^i(K)/\ell^n H^i(K)\ra H^i(K)/\ell H^i(K)\ra0
	\end{equation*}
	and the fact that
	$H^i(K)/\ell^{n-1}\twoheadrightarrow\ell H^i(K)/\ell^n$ surjects
	that $H^i(K)\otimes\ZZ/\ell^n$ is null for all $n$ and so $K$ is in
	$D_c^b(X,\ZZ_\ell)^{\leq0}$. On the other hand, if $K$ is in
	$D_c^b(X,\ZZ_\ell)^{\leq0}$, the exact sequence (2.2.16.1) for $n=1$
	tells us that $H^i(K\Lotimes\ZZ/\ell)=0$ for $i>0$.
	The $\ell$-adic filtration on flat complexes 
	representing $K\Lotimes\ZZ/\ell^n$ has successive quotients
	quasi-isomorphic to $K\Lotimes\ZZ/\ell$ and the sequence of cohomology
	then establishes that $H^i(K\Lotimes\ZZ/\ell^n)$ is null.
	
	\subsection*{2.2.17} On the equivalent conditions: 
	suppose $K\Lotimes_{\ZZ_\ell}\ZZ/\ell^n$ is in
	$D^b_{\mathcal S,\mathcal L}(X,\ZZ/\ell^n)$ for one $n$.
	In the spirit of \hyperref[WeilII_1.1.2]{Weil II (1.1.2) claim a)},
	observe that as the projective system $H^i(K\Lotimes\ZZ/\ell^k)$
	is noetherian AR-$(\ell\ZZ)$-adic, the projective system
	$H^i(K\Lotimes\ZZ/\ell^{kn})$ is noetherian AR-($\ell^n\ZZ$)-adic
	and AR-isomorphic by (3.2.3) of that section (i.e. \cite[V, 3.2.3]{SGA5})
	to the $(\ell^n\ZZ)$-adic system
	$(\mathscr K^i_{k+r}/\ell^{kn}\mathscr K^i_{k+r})_{k\in\NN}$ for 
	some integer $r\geq 0$, where $\mathscr K^i_k$ denotes the projective 
	system of universal images of the system $H^i(K\Lotimes\ZZ/\ell^{kn})$.
	As $K\Lotimes\ZZ/\ell^n$ is in $D^b_{\mathcal S,\mathcal L}(X,\ZZ/\ell^n)$,
	the $H^i(K\Lotimes\ZZ/\ell^n)$ are locally constant on the strata in
	$\mathcal S$. Taking successive quotients on the $\ell^n$-adic
	filtration on bounded complexes of flat sheaves representing the
	$K\Lotimes\ZZ/\ell^{kn}$, we find that the $H^i(K\Lotimes\ZZ/\ell^{kn})$
	are also locally constant on the strata. Therefore the sheaves
	$\mathscr K_k^i$ are, as well as the sheaves
	$\mathscr K^i_{k+r}/\ell^{(k+1)n}\mathscr K^i_{k+r}$.
	For $k=0$, the latter sheaf is isomorphic to $H^i(K)\otimes\ZZ/\ell^n$,
	so we have shown that $H^i(K)\otimes\ZZ/\ell^n$ is locally constant on
	the strata $\mathcal S$.
	
	(An equivalent way to argue is again to use the description of
	\cite[V, 3.2.3]{SGA5} and just note that when computing the universal
	image subsheaves of $H^i(K\Lotimes\ZZ/\ell^{kn})$, one can restrict to
	looking at the images of the sheaves $H^i(K\Lotimes\ZZ/\ell^{kn}))$, and 
	when finding an $r$, if $r$ works, then $s$ works for any $s\geq r$, so
	$r$ can be taken to be a multiple of $kn$.)
	
	As $H^i(K)\otimes\ZZ/\ell^n$ is locally constant and includes into
	$H^i(K\Lotimes\ZZ/\ell^n)$, on each stratum $S\in\mathcal S$ consider a
	Jordan-Hölder series for both. The constituents of the former are a 
	subset of the constituents of the latter and therefore also belong to
	$\mathcal L(S)$. We have shown that $H^i(K)\otimes\ZZ/\ell^n$, and
	therefore $H^i(K)\otimes\ZZ/\ell$, is
	$(\mathcal S,\mathcal L)$-constructible.
	
	On the other hand, suppose the $H^i(K)\otimes\ZZ/\ell$ are
	$(\mathcal S,\mathcal L)$-constructible, and without loss of generality 
	let $K$ be in $D_c^b(X,\ZZ_\ell)^{\leq 0}$. Proceed by recurrence on
	$-j$; in the case $j=0$, $H^1(K)=0$ and
	$H^0(K)\otimes\ZZ/\ell\xra\sim H^0(K\Lotimes\ZZ/\ell)$. The $\ell$-adic
	filtration on any bounded complex of flat sheaves representing
	$K\Lotimes\ZZ/\ell^n$ has successive quotients quasi-isomorphic to
	$K\Lotimes\ZZ/\ell$; taking the long exact sequence of cohomology finds
	that $H^0(K\Lotimes\ZZ/\ell^n)\xra\sim H^0(K)\otimes\ZZ/\ell^n$ are
	$(\mathcal S,\mathcal L)$-constructible for all $n$.
	In particular, this implies that each
	$\Tor_1^{\ZZ_\ell}(H^0(K),\ZZ/\ell^n)$ is
	$(\mathcal S,\mathcal L)$-constructible.
	To see why, note that the increasing sequence of $\ell$-adic subsheaves
	$\ker\ell^a\subset H^0(K)$ must stabilize, say at $a=N$ as $H^0(K)$ is
	noetherian. Let $\mathscr K$ denote $\ker \ell^N$; it is an $\ell$-adic 
	subsheaf of $H^0(K)$ and is $(\mathcal S,\mathcal L)$-constructible
	since $H^0(K)\otimes\ZZ/\ell^N$ is. Then
	\begin{equation*}
		\Tor^{\ZZ_\ell}_1(H^0(K),\ZZ/\ell^n)
		\simeq\Tor^{\ZZ_\ell}_1(\mathscr K,\ZZ/\ell^n)
		\simeq\mathscr K[\ell^n]
	\end{equation*}
	This last sheaf is easily seen to be
	$(\mathcal S,\mathcal L)$-constructible, as it is locally constant
	wherever $\mathscr K$ is, and as a subsheaf, its constituents on a
	stratum are a subset of the constituents of $\mathscr K$ on that stratum.
	Since $\Tor^{\ZZ_\ell}_1(H^0(K),\ZZ/\ell^n)$ and
	$H^{-1}(K)\otimes\ZZ/\ell$ are
	$(\mathcal S,\mathcal L)$-constructible, (2.2.16.1)
	shows that $H^{-1}(K\Lotimes\ZZ/\ell)$ is
	$(\mathcal S,\mathcal L)$-constructible, hence that
	$H^{-1}(K\Lotimes\ZZ/\ell^n)$ is for all $n$. 
	The argument above then shows that $H^{-1}(K)\otimes\ZZ/\ell^n$ are
	$(\mathcal S,\mathcal L)$-constructible for all $n$, and hence
	that $\Tor_1^{\ZZ_\ell}(H^{-1}(K),\ZZ/\ell^n)$ are for all $n$, etc.
	
	In order to proceed to define the t-structure in imitation of 2.2.10,
	one needs to extend the six functors to $D_c^b(X,\ZZ_\ell)$. This is
	trivial because they commute with reduction modulo $\ell^n$; see
	\hyperref[weilII:1.1.2c]{note to Weil II 1.1.2c}.
	Then the claim about $K$ belonging to $^pD_c^{\leq0}(X,\ZZ_\ell)$ iff
	its reduction modulo $\ell$ belongs to $^pD_c^{\leq0}(X,\ZZ/\ell)$ is
	also trivial.
	
\subsection*{2.2.18}\label{BBD:2.2.18}
See \hyperref[weilII:1.1.3]{note to Weil II 1.1.3}.
Multiplication by $\ell$ on complexes of flat sheaves representing $K$
in $D_c^b(X,\ZZ_\ell)$ induces a multiplication by $\ell$ on their 
cohomology. To see that the image 
$D^b_{\mathcal S,\mathcal L}(X,\ZZ_\ell)\otimes\QQ_\ell$
consists of those $K$ such that each $H^iK$ is the $\QQ_\ell\otimes$
of an $(\mathcal S,\mathcal L)$-constructible $\ZZ_\ell$-sheaf, well,
certainly it is contained in it. On the other hand, given $K$ in
$D_c^b(X,\ZZ_\ell)$
with each $H^iK$ the $\QQ_\ell\otimes$ of an
$(\mathcal S,\mathcal L)$-constructible $\ZZ_\ell$-sheaf $\mathscr F_i$,
then for each of finitely many nonzero $i$, there exist nonzero 
$a_i,b_i\in\ZZ_\ell$ such that $a_iH^iK=b\mathscr F_i$; $b_i\mathscr F_i$ is
a subsheaf of $\mathscr F_i$ and is also
$(\mathcal S,\mathcal L)$-constructible.
Then $(\prod_i a_i)K$ has cohomology sheaves which are
$(\mathcal S,\mathcal L)$-constructible, showing the reverse containment.

Turning now to the claim that the forgetful functor $\omega$ induces an
equivalence
\begin{align*}
	D_c^b(X,E_\lambda)\ra
	\{&\text{category of objects $K$ of $D_c^b(X,\QQ_\ell)$ equipped with
	a morphism} \\
	&\text{of $\QQ_\ell$-algebras $E_\lambda\ra\End(K)$}\},
\end{align*}
the essential surjectivity of $\omega$ follows from the fact that if $K$ is 
in $D_c^b(X,\QQ_\ell)$ and equipped with an action $\phi:E_\lambda\ra\End(K)$, 
and $K\otimes_{\QQ_\ell}E_\lambda$ has $E_\lambda$ acting on itself, then
there is an $E_\lambda$-equivariant imbedding $K\ra K\otimes E_\lambda$ with 
retraction $r$. Let $\alpha$ be a primitive element for the extension
$E_\lambda/\QQ_\ell$ so that $E_\lambda\simeq\QQ(\alpha)$, and let
$d$ denote the degree $[E_\lambda:\QQ_\ell]$. The maps are given by
\begin{align*}
	i:K&\ra K\otimes E_\lambda &
	r:K\otimes E_\lambda &\ra K \\
	K&\mapsto \frac1d\sum_{i=0}^{d-1}\phi(\alpha)^iK\otimes\alpha^{-i} &
	K\otimes a&\mapsto\phi(a)K.
\end{align*}
This implies that $K$ is indeed a direct factor of $K\otimes E_\lambda$;
see Neeman, \emph{Triangulated Categories} 1.2.10. This in turn gives an
idempotent in $\End(K\otimes E_\lambda)$, and if the image of this 
idempotent is represented in $D_c^b(X,E_\lambda)$, then this implies a
splitting of $K\otimes E_\lambda$ in $D_c^b(X,E_\lambda)$ which is sent by
$\omega$ to the direct factor $K$.
A category in which every idempotent splits is called alternatively
Cauchy complete, idempotent complete, or Karoubi complete
(see SGA 4, I 8.7.8), so we are done if
we show that $D_c^b(X,E_\lambda)$ is Karoubi complete.
The splitting of an idempotent $e$ in the endomorphism ring of an object
in some category is equivalent to the existence of the equalizer
$i=\ker(e,\id)$ or the coequalizer $r=\coker(e,\id)$, and this 
(co)equalizer, if it exists, is an absolute (co)limit; i.e. it is preserved
by every functor (see Proposition 1 of Borceaux and Dejean,
\emph{Cauchy Completion in Category Theory}). As $D_c^b(X,E_\lambda)$
is a projective limit of categories $D_{ctf}^b(X,R/m^n)$ for $R$ the ring
of integers in $E_\lambda$, it is easily seen that if the categories
$\mathcal D_n:=D_{ctf}^b(X,R/m^n)$ are Karoubi complete, if $e$ is an 
idempotent in $\End(K\otimes E)$, its reductions in $\mathcal D_n$ are
idempotents which split, and as these splittings are absolute (co)limits,
they automatically give an object in $D_c^b(X,E_\lambda)$ splitting $e$.
It will suffice to show that $\mathcal D_n$ is Karoubi complete.
Let's say that a triangulated category has direct sums if it has
(arbitrary) categorical direct sums and if the (arbitrary) direct sum of
distinguished triangles is distinguished.
B\"okstedt and Neeman show in \emph{Homotopy limits in triangulated categories} 3.2
that if a triangulated category has direct sums, it is Karoubi complete.
(Actually, all that is needed is that countable coproducts of objects in
the triangulated category exist; a simple exposition is Neeman,
\emph{Triangulated Categories} \S1.6, in particular (1.6.8).)
The category $D(X,R/m^n)$ is therefore Karoubi complete.
An object $C$ of an additive category with arbitrary direct sums is said to
be compact if $\Hom(C,-)$ commutes with arbitrary direct sums.
Any direct summand of a compact object $C$ is compact, since a finite
colimit of compact objects is compact*.
Therefore as $D(X,R/m^n)$ is Karoubi complete, so
is the full subcategory generated by compact objects.
That $D_{ctf}^b(X,R/m^n)$ coincides with the subcategory generated by
compact objects of $D(X,R/m^n)$ is 6.4.8 of Bhatt-Scholze,
\emph{The pro-étale topology for schemes}, after you recall that the
objects of $D^b_{ctf}(X,R/m^n)$ can be represented by bounded complexes
of $R/m^n$-flat constructible sheaves (\emph{Rapport} 4.6).

(*) To recognize $A$, a direct summand of $C$, as a finite colimit, note
that for the retraction $A\xra i C\xra p A$, $A$ is both the 
equalizer and coequalizer of $\id_C$ and $ip$ (by the maps $i$ and $p$,
respectively). As coequalizer, $A$ is a finite colimit; we can 
therefore write
\begin{equation*}
	\Hom\Big(\underset{\bullet\rightrightarrows\bullet}{\colim}\ C,\bigoplus_{i\in I}U_i\Big)
	=\lim_{\bullet\rightrightarrows\bullet}\Hom\Big(C,\bigoplus_{i\in I}U_i\Big)
	=\lim_{\bullet\rightrightarrows\bullet}\bigoplus_{i\in I}\Hom(C,U_i),
\end{equation*}
where the last equality is because $C$ is compact; now use that finite 
limits commute with filtered colimits in Set to write
\begin{equation*}
	\lim_{\bullet\rightrightarrows\bullet}\bigoplus_{i\in I}\Hom(C,U_i)
	=\bigoplus_{i\in I}\lim_{\bullet\rightrightarrows\bullet}\Hom(C,U_i)
	=\bigoplus_{i\in I}\Hom(\underset{\bullet\rightrightarrows\bullet}{\colim}\ C,U_i).
\end{equation*}
As $A=\underset{\bullet\rightrightarrows\bullet}{\colim}\ C$, this proves
$A$ compact.

\subsection*{2.2.19} The fact that $i^!f_*=f_*i^!$ is SGAA XVIII 3.1.12.3.
The `argument habituel d'homotopie' referenced in the last sentence of the
proof is referring to, e.g. Th. 5.7.1 in Godement,
\emph{Théorie des faisceaux}, but a more accessible reference is
\href{https://stacks.math.columbia.edu/tag/09UY}{Stacks \texttt{09UY}}.
If you have two open coverings $\mathcal U=(U_i)_{i\in I}$ and
$\mathcal V=(V_j)_{j\in J}$ of a space $X$, and $\mathcal U$ is a refinement
of $\mathcal V$, so that we can choose a map $\phi:I\ra J$ such that
$U_i\subset V_{\phi(i)}$ for all $i\in I$. This induces a map of \v Cech
complexes
$\phi^*:\check C^\dotp(\mathcal V,\mathscr F)\ra\check C^\dotp(\mathcal U,\mathscr F)$
for any sheaf $\mathscr F$ on $X$.
The result then says that if you have a second $\phi':I\ra J$ such that
$U_i\subset V_{\phi'(i)}$, the maps $\phi^*$, $\phi'^*$ are homotopic.
This instantly implies that if $\mathcal U,\mathcal V$ are mutual 
refinements, they have the same \v Cech cohomology, since in this case we
can choose $\phi:I\ra J$ and $\psi:J\ra I$ satisfying\
$U_i\subset V_{\phi(i)}$ and $V_j\subset U_{\psi(j)}$; the maps
$\phi\circ\psi$ and $\psi\circ\phi$ must then induce the identity on
\v Cech cohomology as they are homotopic to the identity on the chain level.
Our case is formally equivalent to this statement for $\check H^0$.

In more words: $A$ is defined as an equalizer of two maps
\begin{equation*}
	\prod_I\ ^pj_{i*}A_{U_i}\rightrightarrows\prod_{I\times I}\ ^pj_{ij*}A_{U_{ij}}
\end{equation*}
where one map $d^0_0$ is the product over $(i,j)\in I\times I$ of the maps
\begin{equation*}
	\prod_I\ ^pj_{i*}A_{U_i}\xra{\pr_i}\ ^pj_{i*}A_{U_i}\ra\ ^pj_{\dotp j*}A_{U_{\dotp j}},
\end{equation*}
(here the second map is $j_{i*}$ of the unit of the adjunction on $U_i$ if
$\dotp=i$ and 0 if $\dotp\ne i$),
and the other map $d^0_1$ is the product over $(i,j)\in I\times I$ of the maps
\begin{equation*}
	\prod_I\ ^pj_{i*}A_{U_i}\xra{\pr_j}\ ^pj_{j*}A_{U_j}\ra\ ^pj_{i\dotp*}A_{U_{i\dotp}}
\end{equation*}
(the second map is $j_{j*}$ of the unit of the adjunction on $U_j$ if
$\dotp=j$ and 0 if $\dotp\ne j$).
This recognizes $A$ as $\ker(d^0_0-d^0_1)$.
Let $f$ denote maps of the type
\begin{equation*}
	U_{i_0}\times\cdots\times U_{i_a}\times\cdots U_{i_p}
	\ra U_{i_0}\times\cdots\times\hat U_{i_a}\times\cdots U_{i_p}.
\end{equation*}
In higher degrees, the differential is determined by the formula
\begin{equation*}
	\pr_{i_0\ldots i_{p+1}}\circ d
	=\sum_{a=0}^{p+1} (-1)^a\
	^pj_{i_0\ldots\hat i_a\ldots i_p*}(\eta(A_{i_0\ldots i_p}))
	\circ\pr_{i_0\ldots \hat i_a\ldots i_{p+1}},
\end{equation*}
where $\eta$ is the unit of the adjunction $\id\ra\ ^pf_*\ ^pf^*$.
Let $\mathcal U$ be the cover of $X$ as above and $s:I\ra J$ satisfying
$s_i:U_i\ra V_{s(i)}$ as above.
The chain map induced by $s$ has the explicit description
\begin{equation*}
	\pr_{i_A}\circ s^*=\ ^pj_{s(i_A)*}(\eta(A_{i_A}))\circ\pr_{s(i_A)},
\end{equation*}
where $i_A$ is a multi-index $i_0i_1\ldots i_p$, 
$s(i_A)=s(i_0)\ldots s(i_p)$, and $\eta$ is the unit of
the adjunction $\id\ra\ ^ps_{i_A*}\ ^ps_{i_A}^*$ on $V_{s(i_A)}$,
where
$s_{i_A}:U_{i_0}\times\cdots\times U_{i_p}\ra V_{s(i_0)}\times\cdots\times
V_{s(i_p)}$ is deduced by taking the product of the maps $s_i.$

Now given another map $t:I\ra J$ satisfying $t_i:U_i\ra V_{t(i)}$,
let $0\leq a\leq p$ and let $f$ now denote maps of the sort
\begin{equation*}
	U_{i_0}\times\cdots\times U_{i_p}
	\ra V_{s(i_0)}\times\cdots\times V_{s(i_a)}\times
	V_{t(i_a)}\times\cdots\times V_{t(i_p)},
\end{equation*}
where here we use the map $U_{i_a}\ra V_{s(i_a)}\times V_{t(i_a)}$.
We set up a homotopy $h$ by the formula
\begin{ceqn}\begin{equation*}
	\pr_{i_0\ldots i_p}\circ h
	=\sum_{a=0}^p (-1)^a\
	^pj_{s(i_0)\ldots s(i_a)t(i_a)\ldots t(i_p)*}(\eta(A_{s(i_0)\ldots s(i_a)t(i_a)\ldots t(i_p)})
	\circ\pr_{s(i_0)\ldots s(i_a)t(i_a)\ldots t(i_p)}
\end{equation*}\end{ceqn}
where here $\eta$ is the unit of the adjunction
$\id\ra\ ^pf_*\ ^pf^*$.

Now we can follow the argument of
\href{https://stacks.math.columbia.edu/tag/01FP}{Stacks \texttt{01FP}}.

\subsection*{3.1.2}
(3.1.2.7) The morphism $K_{n+1}\ra K_n$ in $D\mathcal A$ can be represented
by $K_{n+1}\xleftarrow{\alpha} K'_{n+1}\xra\beta K_n$ with $\alpha$ a
quasi-isomorphism and $\beta$ a homotopy class of morphisms.
Replacing $K_n$ by $K_n\oplus\operatorname{cone}(K_{n+1}')$ allows us to
write a commutative square
\begin{equation*}\begin{tikzcd}
	K_{n+1}'\arrow[d,"\alpha"]\arrow[r,"{(\beta,\iota)}"]
	&K_n\oplus\operatorname{cone}(K_{n+1}') \\
	K_{n+1}\arrow[r]&K_n\arrow[u,"\gamma"']
\end{tikzcd}\end{equation*}
where $\iota$ denotes the canonical map
$K_{n+1}'\ra\operatorname{cone}K'_{n+1}$.
Now $(\beta,\iota)$ is injective and $\gamma$ is a homotopy equivalence
so that it can be inverted. Doing this finitely many times allows us to
produce a filtered complex of objects $(K,F)$ of $\mathcal A$ with the
sequence of $F^iK$ isomorphic in $D\mathcal A$ to the given sequence.

(3.1.2.8) The intersection of two subobjects $A,B$ of an object $C$ in an
abelian category $\mathcal A$ is defined as the pullback
\begin{equation*}\begin{tikzcd}
	A\cap B\arrow[d]\arrow[r]&A\arrow[d] \\
	B\arrow[r]&C
\end{tikzcd}\end{equation*}
or equivalently by the exact sequence
\begin{equation*}
	0\ra A\cap B\ra A\oplus B\ra C.
\end{equation*}
The sum $A+B$ is defined as the image of $A\oplus B$ under the same map
$A\oplus B\ra C$ defined by the monomorphisms $A\ra C$ and $B\ra C$.
The inclusion
\begin{equation*}
	\sum_{i+j=p}F^i\cap G^j\subset\bigcap_{i+j=p+1}F^i+G^j
\end{equation*}
is evident from the fact that if $a+b=p$, every pair $(i,j)$ satisfying
$i+j=p+1$ also satisfies $i\leq a$ or $j\leq b$.
For the reverse inclusion, find some $a$ in the intersection; then we
have expressions $a=f_i+g_j$ for $f_i\in F^i, g_j\in G^j,i+j=p+1$; as the 
filtration is finite \& decreasing there exists an $n$ such that
$a=f_n$ and $g_n=0$.
Then $a-f_{n+1}=g_{p+1-(n+1)}$ is in $F^n\cap G^{p-n}$.
Likewise $f_{n+i}-f_{n+i+1}=g_{p+1-(n+i+1)}-g_{p+1-(n+i)}$ is in
$F^{n+i}\cap G^{p-(n+i)}$ for $i\geq0$.
By finiteness, $f_{n+i}=0$ for $i\gg0$, so that
\begin{equation*}
	a=\sum_{i=0}^{\infty} f_{n+i}-f_{n+i+1}\qquad
	(f_{n+i}-f_{n+i+1}\in F^{n+i}\cap G^{p-(n+i)})
\end{equation*}
has only finitely many nonzero terms.

\subsection*{3.1.3} This paragraph should be read along with
Illusie \emph{Complexe cotangent et déformations I} Ch.\,V \S1, cited
\cite{Cotangent}. Equations (3.1.3.4) and (3.1.3.5) are written with the
notation
$\Hom^{p+q}$ and $\Hom_{D\mathcal A}^{p+q}$ and $\Hom_{DF\mathcal A}^{p+q}$ 
instead of
$\Ext^{p+q}$ and $\Ext_{D\mathcal A}^{p+q}$ and $\Ext_{DF\mathcal A}^{p+q}$.
This notation, which continues into (3.1.4) and beyond, is standard but
creates something of a conflict, as it should not be conflated with the
$\Hom^n$ defined in the first part of (3.1.3), as the $\Hom^n$ there
specifies a component of a familiar chain complex. For objects $K,L$ of
$D\mathcal A$, $\Ext^n(K,L)=\Hom_{D\mathcal A}(K,L[n])$; in these notes we
will tend to write $\Ext^n$ instead of $\Hom^n$ even when Deligne writes
$\Hom^n$.

\subsection*{3.1.4} The statement of the proposition of course has grammatical grouping
`lorsque \{$n=0$ ou $-1$\} et $i>j$' and `lorsque \{$n=0$ ou $-1$\} et 
$n+i-j<0$'. The proof of (i) is obtained from the distinguished triangle
given by taking the sequence of cohomology and using the vanishing of $H^0$
and $H^{-1}$ of $F^{-\infty}/F^0$. In the proof of (ii), the differential
at $(0,0)$ on the $E_r$ page for goes $(0,0)\xra{d}(r,1-r)$;
as $E_1^{pq}=0$ for $p<0$, $E_\infty^{00}=\cap_r\ker d_r$.
The assumptions give that $E_r^{pq}=0$ whenever $q<0$ and either $p+q=0$ or
$p+q=1$, so that $\ker d_r=E_2^{00}$ for $r\geq2$ and therefore
$E_2^{00}=E_\infty^{00}$. The assumption that $E_1^{pq}=0$ if $n=0$ and
$q<0$ implies that the only nonzero graded piece in the filtration on
$\Hom_{DF\mathcal A}(K,L)=\Ext^0_{DF\mathcal A}(K,L)$ is the 
0\textsuperscript{th} and it coincides with $E_2^{00}$.

\emph{Note:}
$\Ext^n(\Gr_F^iK,\Gr_F^jL)=\Ext^{n+i-j}(\Gr_F^iK[i],\Gr_F^jL[j])$ so
for example the condition $\Ext^{n}(\Gr_F^iK[-i],\Gr_F^jL[-j])=0$ for
$n<0$ equates to $\Ext^{n-j+i}(\Gr_F^iK,\Gr_F^jL)=0$ for $n<0$, which
implies $\Ext^n(\Gr_F^iK,\Gr_F^jL)=0$ whenever $i>j$ and $n<1$.

\begin{remark}[vague]
	As discussed in \hyperref[BBD:3.1.7]{the note to (3.1.7)} below,
	considering the $\Gr_F^i$ as filtered objects in $\mathcal DF$,
	$\Hom^n_{\mathcal DF}(\Gr_F^iK,\Gr_F^jK)=0$ when $i>j$ (any $n$).
	Of course, forgetting the filtration,
	$\Hom^n_{\mathcal D}(\omega\Gr_F^iK,\omega\Gr_F^jK)$ need not vanish.
	The result of (3.1.4) (i) can be seen as saying that this is precisely
	the difference between computing $\Hom$ in $\mathcal DF$ versus in 
	$\mathcal D$, and that moreover it only need be checked for $n=0$ and 
	$-1$.
\end{remark}


\subsection*{3.1.6} The filtration $T_i$ is now taken to be an \emph{increasing} 
filtration, corresponding to the \emph{decreasing} filtration $T^{-i}$.
Under this correspondence $(\Gr_i^TK)[i]$ becomes $(\Gr_T^{-i}K)[i]$.
$\Hom^n(A,B)$ should be written $\Ext^n(A,B)=\Hom(A,B[n])$, null for $n<0$
when $A,B$ are in $\mathcal C$ by (1.3.1) (i).

\subsection*{3.1.7}\label{BBD:3.1.7}
\emph{A priori} there is the matter of the uniqueness of the arrow of
degree 1 $\Gr_F^iK\ra\Gr_F^{i+1}K$ `defined' by the distinguished triangle
$(\Gr_F^{i+1},F^i/F^{i+2},\Gr_F^i)$.
Recall \cite[1.1.10]{BBD} which says that this arrow is unique
(and therefore actually defined by the property that
$(\Gr_F^{i+1},F^i/F^{i+2},\Gr_F^i)$ is distinguished) if
$\Hom^{-1}_{\mathcal DF}(\Gr_F^{i+1}K,\Gr^i_FK)=0$.
\begin{multline*}
	\Hom^{-1}(\Gr_F^{i+1}K,\Gr^i_FK)=\Hom(\Gr_F^{i+1}K,(\Gr^i_FK)[-1]) \\
	=\Hom(\Gr_F^{i+1}K,\Gr^i_F(K[-1]))=0
\end{multline*}
(all $\Hom$ are in $\mathcal DF$) as of course $F^{i+1}\Gr^i_F=0$.

For $d^{i+1}\circ d^i=0$ the point is that in the commutative 
diagram of distinguished triangles below, the two morphisms
$F^{i+1}/F^{i+3}\ra\Gr_F^{i+1}$ coincide.
\begin{equation*}\begin{tikzcd}
	\Gr_F^i[-1]\arrow[r]&F^{i+1}/F^{i+3}\arrow[r]\arrow[d]
	&F^i/F^{i+3}\arrow[d] \\
	&\Gr_F^{i+1}\arrow[r]\arrow[d,equals]&F^i/F^{i+2}\arrow[r]\arrow[d]
	&\Gr_F^i \\
	F^{i+1}/F^{i+3}\arrow[r]&\Gr_F^{i+1}\arrow[r]&\Gr_F^{i+2}[1]
\end{tikzcd}\end{equation*}

\subsection*{3.1.8}\label{BBD:3.1.8}
On the subject of the differential $d_1(f)=df-fd$,
we have to compute it. Illusie discusses the differential in
\cite[V 1.4.10]{Cotangent}.
The spectral sequence (3.1.3.4) is isomorphic to the spectral sequence
\begin{equation*}
	E_1^{pq}=H^{p+q}\Gr_F^pR\Hom(L,M)\Rightarrow\Ext^{p+q}(L,M).
\end{equation*}
We replace $M$ by a filtered injective resolution of $M$, which is 
therefore (non-canonically) termwise-split as a direct sum of its graded
pieces, which are also injective
(\href{https://stacks.math.columbia.edu/tag/05TP}{\texttt{05TP}}), and
we compute in the abelian category.
General considerations
(\href{https://stacks.math.columbia.edu/tag/012N}{\texttt{012N}}) dictate
that the differential $d_1$ is equal to the coboundary map
associated to the short exact sequence of complexes
\begin{ceqn}\begin{equation*}
	0\ra\Gr^{p+1}(\Hom(L,M))\ra F^p/F^{p+2} \Hom(L,M)
	\ra\Gr^p(\Hom(L,M)))\ra0,
\end{equation*}\end{ceqn}
where the differential on $\Hom(L,M)$ is defined by
\begin{equation*}
	\Hom^n(L,M)=\prod_i\Hom(L^i,M^{i+n}),\quad d^p=d_L+(-1)^{n+1}d_M.
\end{equation*}
We need only consider $n=0$ and for simplicity we look at a single $i$ at a
time. Basically we need to translate this differential via the isomorphism
\begin{align*}
	&E_1^{pq}\simeq\oplus_k\Ext^{p+q}(\Gr^kL,\Gr^{k+p}M),\quad\text{or, more properly, through} \\
	&\Gr^p(\Hom(L^i,M^i))\simeq\oplus_k\Hom(\Gr^kL^i,\Gr^{k+p}M^i).
\end{align*}
The diagrams on the next page, combined with the fact that $d^0=d_L-d_M$,
show that $d_1$, the coboundary map induced 
by $d^0$ and the relevant exact sequence, coincides with
\begin{equation*}
	\prod_i\Hom(\Gr^kL^i,\Gr^{k+p}M^i)\ni (f_i)\mapsto (f_id''_L-d'_Mf_{i-1}),
\end{equation*}
for each $k$, where $d'_M$ and $d''_M$ are induced on $\Hom$ by
coboundaries associated to the exact sequences
\begin{align*}
	&0\ra\Gr^{k+p+1}M\ra F^{k+p}M/F^{k+p+2}M\ra\Gr^{k+p}M\ra0\quad\text{and} \\
	&0\ra\Gr^kL\ra F^{k-1}/F^{k+1}L\ra\Gr^{k-1}L\ra0,\quad\text{respectively.}
\end{align*}
So $(f_i)$ belongs to $\ker d_1$ if $f_id''_L-d'_Mf_{i-1}=0$ for all $i$,
and all that is left to observe is that $d_M'$ and $d_L''$ correspond with
Deligne's differential $d$.

\newpage
\newgeometry{bottom=2cm,left=2cm,right=2cm}
\begin{landscape}\begin{small}
\begin{ceqn}\begin{equation*}\begin{tikzcd}[column sep=small]
	&\;\arrow[d]&\;\arrow[d]&\;\arrow[d] \\
	0\arrow[r]
	&\Gr^{p+1}(\Hom(L^i,M^i))\arrow[r]\arrow[d]
	&F^p/F^{p+2}(\Hom(L^i,M^i))\arrow[r]\arrow[d,"d^0=d_L-d_M"]
	&\Gr^p(\Hom(L^i,M^i))\arrow[r]\arrow[d]
	&0 \\
	0\arrow[r]
	&\Gr^{p+1}(\Hom(L^{i-1},M^i)\oplus\Hom(L^i,M^{i+1}))\arrow[r,"\alpha"]\arrow[d]
	&F^p/F^{p+2}(\Hom(L^{i-1},M^i)\oplus\Hom(L^i,M^{i+1}))\arrow[r]\arrow[d]
	&\Gr^p(\Hom(L^{i-1},M^i)\oplus\Hom(L^i,M^{i+1}))\arrow[r]\arrow[d]
	&0\\
	&\;&\;&\;
\end{tikzcd}\end{equation*}
\end{ceqn}\end{small}
Let $a,b\in\ZZ$ (we care about $(a,b)=(i-1,i),(i,i)$, or $(i,i+1)$)
and fix retractions to the monomorphisms
$F^p/F^{p+k}M^b:=F^pM^b/F^{p+k}M^b\hookrightarrow F^{p+1}/F^{p+k}M^b$
so that we can identify $M^b$ with the direct sum of its graded pieces,
which are almost all zero.
\begin{equation*}
	F^p/F^{p+2}\Hom(L^a,M^b)
	\simeq\oplus_kF^p/F^{p+2}\Hom(L^a,\Gr^kM^b)
	\simeq\oplus_k\Hom(F^{k-p-1}/F^{k-p+1}L^a,\Gr^kM^b)
\end{equation*}
\begin{ceqn}\begin{equation*}\begin{tikzcd}[column sep=small]
	0\arrow[r]&\Gr^{p+1}\Hom(L^a,M^b)\arrow[r]\arrow[dd,"\scalebox{1.5}{\rotatebox{-90}{\text{\Huge$\sim$}}}"]
	&F^p/F^{p+2}\Hom(L^a,M^b)\arrow[r]\arrow[d,phantom,"{\rotatebox{-90}{\text{\Large$\simeq$}}}"]
	&\Gr^p\Hom(L^a,M^b)\arrow[r]\arrow[ddd,"\scalebox{1.5}{\rotatebox{-90}{\text{\Huge$\sim$}}}"]
	&0 \\
	&&\oplus_k\Hom(F^{k-p-1}/F^{k-p+1}L^a,\Gr^kM^b)\arrow[ur]\arrow[d,"\Gr"] \\
	&\oplus_k\Hom(\Gr^{k-p-1}L^a,\Gr^kM^b)\arrow[r]\arrow[ur] \arrow[d,phantom,"{\rotatebox{-90}{\text{\Large$\simeq$}}}"]
	&\oplus_k\Hom(\Gr^{k-p-1}L^a\oplus\Gr^{k-p}L^a,\Gr^kM^b) \arrow[d,phantom,"{\rotatebox{-90}{\text{\Large$\simeq$}}}"] \\
	0\arrow[r]&\oplus_k\Hom(\Gr^{k-p}L^a,\Gr^{k+1}M^b)\arrow[r]
	&\oplus_k\Hom(\Gr^{k-p}L^a,F^k/F^{k+2}M^b)\arrow[r]
	&\oplus_k\Hom(\Gr^{k-p}L^a,\Gr^kM^b)\arrow[r]&0
\end{tikzcd}\end{equation*}\end{ceqn}
The above diagram commutes.
The bottom row and diagonal row are induced by the exact sequences
\begin{align*}
	&0\ra\Gr^{k+p+1}M^b\ra F^{k+p}/F^{k+p+2}M^b\ra\Gr^{k+p}M^b\ra0\quad\text{and} \\
	&0\ra\Gr^kL^a\ra F^{k-1}/F^{k+1}L^a\ra\Gr^{k-1}L^a\ra0,\quad\text{respectively.}
\end{align*}
\end{landscape}
\restoregeometry
Returning to the proof of (3.1.8), let's examine why the filtered complex
$X$ in the first solution is taken by $G$ to a bounded complex of objects 
of $\mathcal C$ isomorphic to the given one.
In the (unconventionally indexed) double complex
\begin{equation*}\begin{tikzcd}
	&\;&\;&\; \\
	\;\arrow[r]&K^{i+1,j-1}\arrow[u]\arrow[r]
	&K^{i+1,j}\arrow[u]\arrow[r]
	&K^{i+1,j+1}\arrow[u]\arrow[r]&\; \\
	\;\arrow[r]&K^{i,j-1}\arrow[u]\arrow[r]
	&K^{i,j}\arrow[u,"d'"]\arrow[r,"d''"]
	&K^{i,j+1}\arrow[u]\arrow[r]&\; \\
	\;\arrow[r]&K^{i-1,j-1}\arrow[u]\arrow[r]
	&K^{i-1,j}\arrow[u]\arrow[r]
	&K^{i-1,j+1}\arrow[u]\arrow[r]&\; \\
	&\;\arrow[u]&\;\arrow[u]&\;\arrow[u]
\end{tikzcd}\end{equation*}
the terms $X^n$ correspond to the diagonals with
slope $-1$ capturing all terms with biindex summing to $n$.
\begin{equation*}
	\Gr^iX\simeq(K^i[-i],(-1)^id'')=K^i[-i].
\end{equation*}
It remains only to show that the morphism of degree one in the 
distinguished triangle associated to the exact sequence
\begin{equation*}
	0\ra\Gr^{i+1}X\xra f F^i/F^{i+2}X\xra g\Gr^iX\ra0
\end{equation*}
coincides with $d'^i$. Actually we will show that it coincides with
$-d'^i$. This is OK, as any complex $L=(L,d)$ is isomorphic to the
complex $L'=(L,-d)$ by the morphisms $(-1)^i\id:L^i\ra L^i$ which induce an
isomorphism of chain complexes $L\xra\sim L'$.
Therefore we will still have shown that $0\ra K^a\ra\cdots\ra K^b\ra0$ is
in the essential image of $G$.
We have an isomorphism of distinguished triangles
\begin{equation*}\begin{tikzcd}
	\Gr^{i+1}X\arrow[r,"f"]
	&F^i/F^{i+2}X\arrow[r,"g"]
	&\Gr^iX\arrow[r]
	&\Gr^{i+1}X[1] \\
	\Gr^{i+1}X\arrow[r]\arrow[u,equals]
	&\operatorname{Cyl}(f)\arrow[u]\arrow[r]
	&C(f)\arrow[u,"\gamma"]\arrow[r,"\delta"]
	&\Gr^{i+1}X[1]\arrow[u,equals]
\end{tikzcd}\end{equation*}
where $\delta:C(f)=\Gr^{i+1}X[1]\oplus F^i/F^{i+2}X\ra \Gr^{i+1}X[1]$
corresponds componentwise to projection to the first factor and
$\gamma:C(f)\ra\Gr^iX$ is given by the composition
$C(f)\xra{\pr_2}F^i/F^{i+2}X\xra g\Gr^iX$, where here $\pr_2$ denotes
abusively the morphism of complexes induced by componentwise projection 
to the second factor.
\begin{align*}
	&C(f)=K^{i+1}[i]\oplus K^i[-i]\oplus K^{i+1}[-i-1]
	\quad\text{with differential} \\
	&\begin{multlined}d^n(k^{i+1,n-i},k^{i,n-i},k^{i+1,n-i-1}) \\
	=((-1)^id'',(-1)^id'',f(k^{i+1,n-i})+d'k^{i,n-i}+(-1)^{i+1}d''k^{i+1,n-i-1}).\end{multlined}
\end{align*}
It suffices to show that $-d'\circ\gamma$ is homotopic to $\delta$.
Consider the homotopy given by
\begin{align*}
	&h^n:C(f)^n=K^{i+1,n-i}\oplus K^{i,n-i}\oplus K^{i+1,n-i-1}
	\xra{\pr_3} K^{i+1,n-i-1}=(K^{i+1}[-i])^{n-1} \\
	&dh(k^{i+1,n-i},k^{i,n-i},k^{i+1,n-i-1})
	=(-1)^id''k^{i+1,n-i-1} \\
	&hd(k^{i+1,n-i},k^{i,n-i},k^{i+1,n-i-1})
	=d'(k^{i,n-i})+(-1)^{i+1}d''k^{i+1,n-i-1}+k^{i+1,n-i} \\
	&\text{so that}\quad
	dh+hd=k^{i+1,n-i}+d'(k^{i,n-i}).
\end{align*}
Therefore $h$ defines a homotopy between the two morphisms
\begin{equation*}\begin{tikzcd}
	C(f)\arrow[r,shift left,"\delta"]\arrow[r,shift right,"-d'\circ\gamma"']
	&K^{i+1}[-i].
\end{tikzcd}\end{equation*}
As for the construction of the differential of the complex $X$, let's
check that
\begin{equation*}
	d_2=H_0+H_1+H_2=(-1)^id''+d'+\sum_i(-1)^{i+1}H_2^i
\end{equation*}
satisfies (*); i.e. that $d_2\circ d_2$ is of filtration $\geq3$.
To do this, it suffices to look at the terms of $d_2\circ d_2$ which are
of filtration $<3$. We denote by $(\geq n)$ terms of filtration $\geq n$.
It suffices to compute what $d_2\circ d_2$ does to $K^{i,j}$.
\begin{align*}
	d_2\circ d_2
	&=d_1\circ d_1+d_1\circ H_2+H_2\circ D_1+H_2\circ H_2 \\
	&=d'^{i+1}\circ d'^i+(-1)^{i+2}d''\circ H_2+H_2\circ(-1)^id'' \\
	&=d''\circ H_2^i+H_2^i\circ d''+(-1)^{2i+3}d''\circ H_2^i+(-1)^{2i+1}H_2^i\circ d''+(\geq3) \\
	&=0+(\geq3)
\end{align*}
as $H_2\circ H_2$ is of filtration 4 and $d'\circ H_2$ and
$H_2\circ d'$ are of filtration 3.

In the case $p\geq2$, $\varphi^i:K^i\ra K^{i+p+1}[1-p]$ is a morphism of
complexes. This is because the component of degree $p+1$ of $d_p\circ d_p$
is bihomogenous of degree $(p+1,1-p)$: indeed, it is a sum of morphisms
obtained by composing two morphisms bihomogenous of degree $(a,-a+1)$ and
$(b,-b+1)$, respectively, so that $a+b=p+1$. Let's verify that with
\begin{equation*}
	H_{p+1}=\sum_i(-1)^{i+p}H^i_{p+1},\quad
	d_{p+1}\circ d_{p+1}\,\,\text{is of filtration $\geq p+2$.}
\end{equation*}
We check what $d_{p+1}\circ d_{p+1}$ does to $K^{i,j}$.
\begin{ceqn}\begin{align*}
	d_{p+1}\circ d_{p+1}
	&=\Big(d_p+\sum_i(-1)^{i+p}H_{p+1}^i\Big)\circ\Big(d_p+\sum_i(-1)^{i+p}H_{p+1}^i\Big) \\
	&=\varphi+H_0\circ H_{p+1}+H_{p+1}\circ H_0+(\geq p+2) \\
	&=\varphi^i+(-1)^{i+p+1}d''\circ (-1)^{i+p}H_{p+1}^i+(-1)^{i+p}H^i_{p+1}\circ (-1)^id''+(\geq p+2) \\
	&=\varphi^i-d''\circ H^i_{p+1}+(-1)^pH_{p+1}^i\circ d''+(\geq p+2) \\
	&=0+(\geq p+2),
\end{align*}\end{ceqn}
in light of the identity that shows $\varphi^i$ homotopic to 0 (as a 
morphism of complexes $K^i\ra K^{i+p+1}[1-p]$) with homotopy $H^i_{p+1}$.

\subsubsection*{2ème solution}
Let's check that the object $(C,F)$ of $\mathcal D^bF_{\text{b\^ete}}$ 
defined up to isomorphism by the distinguished triangle
\begin{equation*}\begin{tikzcd}
	&(A,F)\arrow[dr,"\tilde f"] \\
	(A,F[1])\arrow[ur,"\id_A"]\arrow[rr]&&(A,F)\arrow[r]&(C,F)\arrow[r]&\;
\end{tikzcd}\end{equation*}
is sent by $G$ to a complex in $C^b(\mathcal C)$ isomorphic to
$0\ra K^a\ra\cdots\ra K^b\ra0$.
There is some ambiguity in the definition of $A,B$ and $f$. As we will see,
we want $f^{p+1}:K^p\ra K^{p+1}$ to coincide with $-d^p$, $G(B)$ to 
coincide with the bête truncation
$\sigma_{\geq p+1}(\cdots\ra K^i\xra{d^i}\cdots)$ of the desired
complex, and $G(A)$ to coincide with the translation of the bête truncation 
$\sigma_{\leq p}(\cdots\ra K^i\xra{d^i}\cdots)[-1]$.
We have a commutative diagram with distinguished rows and columns.
\begin{equation*}\begin{tikzcd}
	\Gr_F^{i+2}A\arrow[r]\arrow[d]
	&\Gr_F^{i+1}B\arrow[r]\arrow[d]
	&\Gr_F^{i+1}C\arrow[r]\arrow[d]&\; \\
	F^{i+1}/F^{i+3}A\arrow[r]\arrow[d]
	&F^i/F^{i+2}B\arrow[r]\arrow[d]
	&F^i/F^{i+2}C\arrow[r]\arrow[d]&\; \\
	\Gr_F^{i+1}A\arrow[r]\arrow[d]
	&\Gr^i_FB\arrow[r]\arrow[d]
	&\Gr_F^iC\arrow[r]\arrow[d]&\; \\
	\;&\;&\;
\end{tikzcd}\end{equation*}
When $i>p$, $\Gr_{F[1]}^iA=\Gr_F^{i+1}A=0$, the left column vanishes, and
we find an isomorphism of bête-truncated complexes
$\sigma_{>p}G(B)\xra\sim\sigma_{>p}G(C)$:
\begin{ceqn}\begin{equation*}\begin{tikzcd}
	(\Gr_F^{p+1}C)[p+1]\arrow[r,"d_C^{p+1}"]
	&\cdots\arrow[r]&(\Gr_F^{b}C)[b]\arrow[r] 
	&(\Gr_F^{b+1}C)[b+1]\arrow[r]
	&\cdots \\
	(\Gr_F^{p+1}B)[p+1]\arrow[r]\arrow[u,"\sim"]\arrow[d,"\sim"]
	&\cdots\arrow[r]
	&(\Gr_F^{b}B)[b]\arrow[r]\arrow[u,"\sim"]\arrow[d,"\sim"]
	&(\Gr_F^{b+1}B)[b+1]\arrow[r]\arrow[u,"\sim"]\arrow[d,"\sim"]
	&\cdots \\\
	K^{p+1}\arrow[r,"d^{p+1}"]&\cdots\arrow[r]
	&K^b\arrow[r]&0\arrow[r]&\cdots
\end{tikzcd}\end{equation*}\end{ceqn}
When $i\leq p$, $\Gr_F^iB=0$ and $\Gr^i_FC\xra\sim(\Gr_F^{i+1}A)[1]$.
When $i<p$, moreover the entire middle column vanishes, finding an
isomorphism of bête truncated complexes
$\sigma_{<p}G(C)\xra\sim\sigma_{\leq p}(G(A)[1])$: 
\begin{ceqn}\begin{equation*}\begin{tikzcd}
	\cdots\arrow[r]&(\Gr_F^{a-1}C)[a-1]\arrow[r]\arrow[d,"\sim"]
	&(\Gr_F^{a}C)[a]\arrow[r,"d_C^a"]\arrow[d,"\sim"]
	&\cdots\arrow[r]&(\Gr_F^{p}C)[p]\arrow[d,"\sim"] \\
	\cdots\arrow[r]&(\Gr_F^{a-1}A)[a]\arrow[r]\arrow[d,"\sim"]
	&(\Gr_F^{a}A)[a+1]\arrow[r]\arrow[d,"\sim"]
	&\cdots\arrow[r]&(\Gr_F^{p}A)[p+1]\arrow[d,"\sim"] \\
	\cdots\arrow[r]&0\arrow[r]
	&K^a\arrow[r,"{d^a}"]&\cdots\arrow[r]&K^p
\end{tikzcd}\end{equation*}\end{ceqn}
We will be done if we can join these diagrams with a commutative diagram
\begin{equation*}\begin{tikzcd}
	(\Gr_F^{p}C)[p]\arrow[d,"\sim"]\arrow[r,"d_C^p"]\arrow[d,"\sim"]
	&(\Gr_F^{p+1}C)[p+1] \\
	(\Gr_F^{p}A)[p+1]\arrow[d,"\sim"]\arrow[r,dashed]
	&(\Gr_F^{p+1}B)[p+1]\arrow[d,"\sim"]\arrow[u,"\sim"] \\
	K^p\arrow[r,"d^p"]&K^{p+1}
\end{tikzcd}\end{equation*}
When $i=p$, part of our earlier commutative diagram looks like
\begin{equation*}\begin{tikzcd}
	&\Gr_F^{p+1}B\arrow[r,"\sim"]\arrow[d,"\sim","\beta"']
	&\Gr_F^{p+1}C\arrow[d] \\
	F^{p+1}/F^{p+3}A\arrow[r,"\alpha"]\arrow[d,"\sim", "\gamma"']
	&F^p/F^{p+2}B\arrow[r]
	&F^p/F^{p+2}C\arrow[r]\arrow[d]&\; \\
	\Gr_F^{p+1}A&&\Gr_F^pC\arrow[d,"{d_C^p[-p]}"] \\ &&\;
\end{tikzcd}\end{equation*}
with distinguished row and column, inducing an isomorphism of 
distinguished triangles
\begin{equation*}\begin{tikzcd}
	\Gr^{p+1}_FB\arrow[r]\arrow[d,"\sim"]
	&F^p/F^{p+2}C\arrow[d,equals]\arrow[r]
	&(\Gr_F^{p+1}A)[1]\arrow[r,"{-(\beta^{-1}\circ\alpha\circ\gamma^{-1})[1]}"]&[40pt]\;&\; \\
	\Gr_F^{p+1}C\arrow[r]&F^p/F^{p+2}C\arrow[r]
	&\Gr_F^pC\arrow[u,"\sim"]\arrow[r,"{d_C^p[-p]}"]&\;&.
\end{tikzcd}\end{equation*}
The commutative diagram
\begin{ceqn}\begin{equation*}\begin{tikzcd}
	&K^p[-p-1]\arrow[r,"{f^{p+1}[-p-1]}"]
	&[10pt]K^{p+1}[-p-1] \\
	\Gr^p_{F[1]}A\arrow[r,equals]
	&\Gr_F^{p+1}A\arrow[r,"\delta"]\arrow[d]\arrow[u,"{\rotatebox{-90}{\text{\Huge$\sim$}}}"]
	&\Gr_F^{p+1}B\arrow[d,"\sim","\beta"']\arrow[u,"{\rotatebox{-90}{\text{\Huge$\sim$}}}"] \\
	F^{p+1}/F^{p+3}A=(F[1])^p/(F[1])^{p+2}A\arrow[u,"\sim","\gamma"']\arrow[r,"\id_A"]\arrow[rr,bend right=15,"\alpha"]
	&F^p/F^{p+2}A\arrow[r,"\tilde f"]&F^p/F^{p+2}B
\end{tikzcd}\end{equation*}\end{ceqn}
shows that
$\delta=\beta^{-1}\circ\alpha\circ\gamma^{-1}:\Gr_F^{p+1}A\ra\Gr_F^{p+1}B$
has the property that $-\delta[p+1]$ makes a commutative diagram
\begin{equation*}\begin{tikzcd}
	(\Gr_F^{p}C)[p]\arrow[d,"\sim"]\arrow[r,"d_C^p"]\arrow[d,"\sim"]
	&(\Gr_F^{p+1}C)[p+1] \\
	(\Gr_F^{p}A)[p+1]\arrow[d,"\sim"]\arrow[r,"{-\delta[p+1]}"]
	&(\Gr_F^{p+1}B)[p+1]\arrow[d,"\sim"]\arrow[u,"\sim"] \\
	K^p\arrow[r,"-f^{p+1}"]&K^{p+1}
\end{tikzcd}\end{equation*}
This is why we should take $f^{p+1}=-d^p$.
\begin{remark}
	There is a related construction involving
	décalage as defined by Deligne in \emph{Théorie de Hodge II} (1.3.3).
	(The choice of name, décalage, refers not to a shift of the filtration,
	but rather to a shift in the page numbering of the spectral sequences
	associated to the filtration.) The décalage functor
	$(X,F)\mapsto(X,\Dec(F))$ is defined componentwise by
	\begin{equation*}
		(\Dec F)^a(X^i)
		:=\ker(d:F^{i+a}(X^i)\ra F^{i+a}(X^{i+1})/F^{i+a+1}(X^{i+1})).
	\end{equation*}
	By construction, the differential on $(\Dec F)^a$ coincides with the
	coboundary map on cohomology so that if we define (for each $a$) the
	complex $H^a(X,F)^\dotp$ with $H^a(X,F)^i:=H^i\Gr_F^{i+a}X$ with
	differential coming from the coboundary, the obvious map
	\begin{equation*}
		\Gr_{\Dec F}^aX\longrightarrow H^a(X,F)
	\end{equation*}
	is a quasi-isomorphism.
	There is a t-structure on the filtered derived category $DF$ with
	\begin{align*}
		&(DF)^{\leq0}=\{(X,F)\text{ s.t. $\Gr^iX$ is acyclic in degrees $>i$}\} \\
		&(DF)^{\geq0}=\{(X,F)\text{ s.t. $\Gr^iX$ is acyclic in degrees $<i$}\}
	\end{align*}
	whose heart coincides with the category of bête-filtered complexes.
	The truncation $\tau_{\leq0}$ for this t-structure is the functor
	\begin{equation*}
		(X,F)\mapsto((\Dec F)^0X,F\cap(\Dec F)^0)
	\end{equation*}
	which lands in $(DF)^{\leq0}$ since $(\Dec F)^0$ lies in $F^{i+1}$
	in degrees $>i$.
	Under the equivalence of categories $G$, the corresponding cohomological
	functor is the functor $H^\dotp$.
	This is (3.11) in Beilinson's \emph{Notes on Absolute Hodge Cohomology}.
\end{remark}


\subsection*{3.1.9}
The isomorphisms (3.1.9.2) render anticommutative the given square because
the isomorphisms $\Gr^i_F(K[1])=(\Gr^i_FK)[1]$ together with their
translations render anticommutative the square
\begin{equation*}\begin{tikzcd}
	\Gr^i_F(K[1])\arrow[d,equals]\arrow[r,"{d_1[-i]}"]
	&\Gr^{i+1}_F(K[1])[1]\arrow[d,equals] \\
	(\Gr^i_FK)[1]\arrow[r,"{d_0[1-i]}"]& (\Gr^{i+1}_FK)[2]
\end{tikzcd}\end{equation*}
where we have added subscripts for clarity.
As discussed in \hyperref[BBD:3.1.7]{the note to (3.1.7)} above,
$d_0[1-i]:(\Gr^i_FK)[1]\ra(\Gr_F^{i+1}K)[2]$ is the translation of the
morphism $d_0[-i]$ defined by the distinguished triangle
\begin{equation*}\begin{tikzcd}
	\Gr^{i+1}_FK\arrow[r,"u"]&F^i/F^{i+2}K\arrow[r,"v"]
	&\Gr^i_FK\arrow[r,"{d_0[-i]}"]&\;
\end{tikzcd}\end{equation*}
On the other hand, $d_1[-i]:\Gr_F^i(K[1])\ra\Gr_F^{i+1}(K[1])$ is the 
morphism defined by the triangle in the second row
\begin{equation*}\begin{tikzcd}
	(\Gr^{i+1}_FK)[1]\arrow[r,"{u[1]}"]\arrow[d,equals]
	&(F^i/F^{i+2}K)[1]\arrow[r,"{v[1]}"]\arrow[d,equals]
	&(\Gr_F^iK)[1]\arrow[r]\arrow[d,equals]&\; \\
	\Gr^{i+1}_F(K[1])\arrow[r]&F^i/F^{i+2}(K[1])\arrow[r]
	&\Gr_F^i(K[1])\arrow[r,"{d_1[-i]}"]&\;
\end{tikzcd}\end{equation*}
once we show that this triangle is distinguished.
If we can show there is a morphism of degree 1 $d_1[-i]$ making this
triangle distinguished, this morphism of degree 1 will be unique by the 
considerations of \hyperref[BBD:3.1.7]{the note to (3.1.7)}.
The diagram
\begin{equation*}\begin{tikzcd}
	(\Gr^{i+1}_FK)[1]\arrow[r,"{-u[1]}"]\arrow[d,"\id"]
	&(F^i/F^{i+2}K)[1]\arrow[r,"{-v[1]}"]\arrow[d,"-\id"]
	&(\Gr_F^iK)[1]\arrow[r,"{-d_0[1-i]}"]\arrow[d,"\id"]&\; \\
	(\Gr^{i+1}_FK)[1]\arrow[r,"{u[1]}"]\arrow[d,equals]
	&(F^i/F^{i+2}K)[1]\arrow[r,"{v[1]}"]\arrow[d,equals]
	&(\Gr_F^iK)[1]\arrow[r,"{-d_0[1-i]}"]\arrow[d,equals]&\; \\
	\Gr^{i+1}_F(K[1])\arrow[r]&F^i/F^{i+2}(K[1])\arrow[r]
	&\Gr_F^i(K[1])\arrow[r,"{d_1[-i]}"]&\;
\end{tikzcd}\end{equation*}
is an isomorphism of triangles with the top row distinguished.
We conclude $d_1[-i]=-d_0[1-i]$ after the canonical isomorphisms
$\Gr^i_F(K[1])[j]=(\Gr_F^iK)[1+j]$.

\subsection*{3.1.11} The point is that we already computed the differential
on the $E_1$ page to be $d_1(f)=df-fd$ in
\hyperref[BBD:3.1.8]{the note to 3.1.8}, and the image of
$d_1:E_1^{-1,0}\ra E_1^{0,0}$ consists of precisely the homotopies.
As computed in 3.1.8, the kernel of $d_1:E_1^{0,0}\ra E_1^{1,0}$ consists
of precisely the morphisms of complexes in $C^b(\mathcal C)$.

\subsection*{3.1.12}
We can choose a representative $f$ for a homotopy class of morphisms of
complexes in $K^b(\mathcal C)$.
Typos \& clarifications:
`$G(C(f),\tilde F)=C(G(f))$. En particulier,
si $(K,F)$ et $(L,F)$ sont bêtes, $(C(f),\tilde F)$ l'est aussi,
et $C(\omega f)=\omega (C(f),\tilde F)$ s'identifie à
$\real(C(f))$.'
The isomorphism $G(C(f),\tilde F)=C(G(f))$ relies on the earlier 
calculation $G(K[1],F[1])=(GK)[1]$.
Interestingly,
\begin{equation*}
	G(K,L,C(f))=\omega((K,F),(L,F),(C(f),\tilde F))=(K,L,C(f))
\end{equation*}
where the middle triangle is not distinguished, but after forgetting the
filtration, it is.

\subsection*{3.1.13} We will compute the differentials of $G(\Sigma)$ to
verify that $\mathbf s\Sigma^*=G(\Sigma)$, but before we do, a few words
on the matter of sign that was also an issue in the discussion of both
solutions to (3.1.8) in \hyperref[BBD:3.1.8]{the note to 3.1.8}.
Given a map $f:A^*\ra B^*$ of complexes (considered as a double complex
with first degree concentrated in 0 and 1) and the filtration by first
degree on the associated simple complex $K$, it is easy to compute that the
coboundary $\partial$ on cohomology associated to the exact sequence
\begin{equation*}\begin{tikzcd}
	0\arrow[r]&\Gr^{1}K\arrow[d,equals]\arrow[r,"v"]
	&F^0/F^{2}K\arrow[d,equals]\arrow[r]&\Gr^0K\arrow[d,equals]\arrow[r]&0\\
	0\arrow[r]&B[-1]\arrow[r]&B[-1]\oplus A\arrow[r]&A\arrow[r]&0
\end{tikzcd}\end{equation*}
coincides with $f$, as the differential on $F^0/F^2K\simeq B[-1]\oplus A$
is given by
\begin{equation*}
	\begin{pmatrix}d_{B[-1]}&f\\0&d_A\end{pmatrix}.
\end{equation*}
The issue appears to be that when one defines the differential on the cone 
of a morphism of complexes $u:L\ra M$ in the usual way as
\begin{equation*}
	\begin{pmatrix}d_{L[1]}&0\\u&d_M\end{pmatrix},
\end{equation*}
then given the exact sequence of complexes with coboundary $\partial$
\begin{equation*}
	0\ra L\xra u M\ra\coker u\ra 0,
\end{equation*}
the morphism of degree one $\delta$ in the distinguished triangle
$(L,\operatorname{Cyl}(u),C(u))$ coincides on cohomology with $-\partial$
after the isomorphism with $(L,M,\coker u)$.
(This is consistent with Weibel's book but Gelfand-Manin say, I believe
incorrectly, $\partial$.)

Applying this in the case of $u:\Gr^1K\ra F^0/F^2K$, we expect that the map
of degree one $\Gr^0K\ra(\Gr^1K)[1]$ will coincide with $-f$.
As before, this isn't really a problem in view of the isomorphism
\begin{equation*}\begin{tikzcd}
	0\arrow[r]&A\arrow[r,"f"]\arrow[d,"\id"]
	&B\arrow[r,"g"]\arrow[d,"-\id"]
	&C\arrow[r]\arrow[d,"\id"]&0 \\
	0\arrow[r]&A\arrow[r,"-f"]&B\arrow[r,"-g"]&C\arrow[r]&0.
\end{tikzcd}\end{equation*}
It is also possible to define the differential of the
cone of the morphism $u$ to be
\begin{equation*}
	\begin{pmatrix}d_{L[1]}&0\\-u&d_M\end{pmatrix},
\end{equation*}
and although less standard, this would correct our sign, but we don't
proceed this way.

Returning to the notation in the beginning of this paragraph, we will show
that the differentials on $G(\mathbf s\Sigma^*)$ coincide with the
differentials of $\Sigma$ with the sign reversed. In fact, it suffices to
do so for the single morphism of complexes $f$. In the double complex
\begin{equation*}\begin{tikzcd}
	\;&\; \\
	A^p\arrow[r,"f^p"]\arrow[u]&B^p\arrow[u] \\
	A^{p-1}\arrow[u]\arrow[r,"f^{p-1}"]&B^{p-1}\arrow[u] \\
	\;\arrow[u]&\;\arrow[u]
\end{tikzcd}\end{equation*}
we consider first degree concentrated in $0,1$, and we let $K$ denote the
associated simple complex. With $v$ as above,
\begin{equation*}\begin{tikzcd}
	C(v)\arrow[r,"\delta"]\arrow[d]&(\Gr^1K)[1]\arrow[d,equals] \\
	\Gr^0K\arrow[r,"f"]&(\Gr^1K)[1]
\end{tikzcd}\quad\simeq\quad
\begin{tikzcd}
	B\oplus B[-1]\oplus A
	\arrow[r,"\delta"]\arrow[d,"\gamma"]
	&B\arrow[d,equals] \\
	A\arrow[r,"f"]&B
\end{tikzcd}
\end{equation*}
where $\delta$ is projection to the first factor, $\gamma$ to the last,
and we wish to show these squares commute up to homotopy.
The differential on $C(v)\simeq B\oplus B[-1]\oplus A$ is given by
\begin{equation*}\begin{pmatrix}
	d_B \\
	\id&d_{B[-1]}&f\\
	&&d_A
\end{pmatrix}\end{equation*}
and we compute $dh-hd$ for the homotopy
$h^p:B^p\oplus B^{p-1}\oplus A^p\ra B^{p-1}$ which is projection to the 
middle factor:
\begin{align*}
	&dh(b^p,b^{p-1},a^p)=db^{p-1} \\
	&hd(b^p,b^{p-1},a^p)=b^p+f(a^p)-db^{p-1}.
\end{align*}
This shows that $\delta\sim-f$ so that
if $\Sigma^*=A^*\xra f B^*\xra g C^*$, the differentials of
$G(\mathbf s\Sigma^*)$ coincide with $-f$ and $-g$ which, as discussed above, is ok.

\subsection*{3.1.14} We check that the cone $C$ of the inclusion
$\iota:\tau_{\leq i}K^*\ra\tau_{\leq i}'K^*$ is homotopic to the complex
$L$ given by $\ker d\ra K^i\ra\im d$ in degrees $i-1,i,i+1$.
The morphisms of complexes $f$ and $g$ are given componentwise by
\begin{ceqn}\begin{equation*}\begin{tikzcd}
	\cdots\arrow[r]
	&0\arrow[r]\arrow[d,"f^{i-2}"]
	&\ker d\arrow[r]\arrow[d,"f^{i-1}"]
	&K^i\arrow[r]\arrow[d,"f^i"]
	&\im d\arrow[r]\arrow[d,"f^{i+1}"]
	&0\arrow[r]\arrow[d,"f^{i+2}"]
	&\cdots \\
	\cdots\arrow[r]
	&K^{i-1}\arrow[d,phantom,"\oplus"]\arrow[r,"{d[1]}"]\arrow[dr,"\iota^{i-1}"]
	&\ker d\arrow[r]\arrow[d,phantom,"\oplus"]\arrow[dr,"\iota^{i}"]
	&0\arrow[r]\arrow[d,phantom,"\oplus"]
	&0\arrow[r]\arrow[d,phantom,"\oplus"]
	&0\arrow[r]\arrow[d,phantom,"\oplus"]
	&\cdots \\
	\cdots\arrow[r]
	&K^{i-2}\arrow[r,"d"]\arrow[d,"g^{i-2}"]
	&K^{i-1}\arrow[r,"d"]\arrow[d,"g^{i-1}"]
	&K^{i}\arrow[r,"d"]\arrow[d,"g^i"]
	&\im d\arrow[r]\arrow[d,"g^{i+1}"]
	&0\arrow[r]\arrow[d,"g^{i+2}"]
	&\cdots \\
	\cdots\arrow[r]
	&0\arrow[r]
	&\ker d\arrow[r]
	&K^i\arrow[r]
	&\im d\arrow[r]
	&0\arrow[r]
	&\cdots
\end{tikzcd}\end{equation*}\end{ceqn}
where $f^{i-1}$ is inclusion into the first factor and $f^i,f^{i+1}$
are inclusion into the second factor, while $g^i,g^{i+1}$ are projection to
the second factor, $g^j=0$ for $j<i-1$, and $g^{i-1}=\pr_1-d\circ\pr_2$.
Evidently $g\circ f=\id_L$ and we need to construct a homotopy $h$ so that
$f\circ g=\id_C+dh+hd$. Set each $h^p:C^p\ra C^{p-1}$ for $p>i-1$ to zero,
while for $p\leq i-1$, let $h^p$ be the projection to the
second factor of $C^p$ followed by inclusion into the first factor of
$C^{p-1}$ so that if we only draw arrows for the components of $h$ which
are nonzero, $h$ is given by
\begin{ceqn}\begin{equation*}\begin{tikzcd}[row sep=tiny]
	\;\arrow[d,phantom,"\cdots"]
	&K^{i-2}\arrow[d,phantom,"\oplus"]
	&K^{i-1}\arrow[d,phantom,"\oplus"]
	&\ker d\arrow[d,phantom,"\oplus"]
	&0\arrow[d,phantom,"\oplus"]
	&0\arrow[d,phantom,"\oplus"]
	&0\arrow[d,phantom,"\oplus"]
	&\;\arrow[d,phantom,"\cdots"] \\
	\;
	&K^{i-3}
	&K^{i-2}\arrow[ul,"\id"']
	&K^{i-1} \arrow[ul,"\id"']
	&K^{i}
	&\im d
	&0
	&\;
\end{tikzcd}\end{equation*}\end{ceqn}
Now let's examine why an isomorphism between $H^iK^*[-i]$ and a cone on
$\real(\tau_{\leq{i-1}}K^*)\ra\real(\tau_{\leq i}K^*)$ is enough to 
conclude that $\real(\tau_{\leq i}K^*)\simeq\tau_{\leq i}(\real K^*)$
and $^pH^i\real K^*\simeq H^iK^*$.
As $K^*$ is an object of $C^b(\mathcal C)$, there are integers $a,b$
such that $K^i=0$ for $i<a$ and $i>b$.
Then $\tau_{\leq a-1}K^*=0$ and we find
$\real(\tau_{\leq a}K^*)\simeq(H^aK^*)[-a]$.
As $H^iK^*$ is in $\mathcal C$, proceeding inductively we find that
$^pH^j(\real(\tau_{\leq i}K^*))$ equals $H^jK^*$ for $j\leq i$ and is null
for $j>i$. As $H^{b+1}K^*=0$,
$\real(\tau_{\leq b} K^*)\simeq\real(\tau_{\leq b+1}K^*)=\real K^*$ so that
$\real(\tau_{\leq b}K*)\simeq\real K^*\simeq\tau_{\leq b}\real K^*$ since
as we have seen, $^pH^j(\real K^*)=0$ for $j>b$.
Now supposing $\real(\tau_{\leq i}K^*)\simeq\tau_{\leq i}(\real K^*)$ has
been established, the isomorphism of distinguished triangles
\begin{equation*}\begin{tikzcd}
	\real(\tau_{\leq i-1}K^*)\arrow[r]\arrow[d,dashed,"\sim"]
	&\real(\tau_{\leq i} K^*)\arrow[r]\arrow[d,"\sim"]
	&(H^iK^*)[-i]\arrow[r]\arrow[d,"\sim"]&\; \\
	\tau_{\leq i-1}(\real K^*)\arrow[r]
	&\tau_{\leq i}(\real K^*)\arrow[r]
	&^pH^i(\real K^*)[-i]\arrow[r]&\;
\end{tikzcd}\end{equation*}
achieves the same for $i-1$.

Now we show that the cone $C^*$ on the morphism
$\iota:\tau'_{\leq i-1}K^*\ra\tau_{\leq i}K^*$ is homotopic to the complex
$L^*$ reduced to $\im d^{i-1}\ra\ker d^i$ in degrees $i-1$ and $i$. 
The story is the same as what we just did. We define morphisms
$f:L^*\ra C^*$ and $g:C^*\ra L^*$ which compose to morphisms homotopic to
the identity.
\begin{ceqn}\begin{equation*}\begin{tikzcd}
	\cdots\arrow[r]
	&0\arrow[r]\arrow[d]
	&0\arrow[r]\arrow[d]
	&\im d^{i-1}\arrow[r]\arrow[d,"f^{i-1}"]
	&\im d\arrow[r]\arrow[d,"f^i"]
	&0\arrow[r]\arrow[d]
	&\cdots \\
	\cdots\arrow[r]
	&K^{i-2}\arrow[d,phantom,"\oplus"]\arrow[r,"{d[1]}"]\arrow[dr,"\iota^{i-2}"]
	&K^{i-1}\arrow[r]\arrow[d,phantom,"\oplus"]\arrow[dr,"\iota^{i-1}"]
	&\im d^{i-1}\arrow[r]\arrow[d,phantom,"\oplus"]\arrow[dr,"\iota^i"]
	&0\arrow[r]\arrow[d,phantom,"\oplus"]
	&0\arrow[r]\arrow[d,phantom,"\oplus"]
	&\cdots \\
	\cdots\arrow[r]
	&K^{i-3}\arrow[r,"d"]\arrow[d]
	&K^{i-2}\arrow[r]\arrow[d]
	&K^{i-1}\arrow[r]\arrow[d,"g^{i-1}"]
	&\ker d^i\arrow[r]\arrow[d,"g^i"]
	&0\arrow[r]\arrow[d]
	&\cdots \\
	\cdots\arrow[r]
 	&0\arrow[r]
	&0\arrow[r]
	&\im d^{i-1}\arrow[r]
	&\im d\arrow[r]
	&0\arrow[r]
	&\cdots
\end{tikzcd}\end{equation*}\end{ceqn}
Here the morphisms $f^i,g^i$ is the obvious inclusion and projection, while
$f^{i-1}$ is the inclusion to the first factor and
$g^{i-1}=\pr_1-d^{i-1}\circ\pr_2$. Then $g\circ f=\id_{L^*}$ and
$f\circ g=\id_{C^*}+dh+hd$ where $h$ is the homotopy with all components
nonzero except for the arrows in the below diagram.
\begin{ceqn}\begin{equation*}\begin{tikzcd}[row sep=tiny]
	\;\arrow[d,phantom,"\cdots"]
	&K^{i-2}\arrow[d,phantom,"\oplus"]
	&K^{i-1}\arrow[d,phantom,"\oplus"]
	&\im d^{i-1}\arrow[d,phantom,"\oplus"]
	&0\arrow[d,phantom,"\oplus"]
	&0\arrow[d,phantom,"\oplus"]
	&0\arrow[d,phantom,"\oplus"]
	&\;\arrow[d,phantom,"\cdots"] \\
	\;
	&K^{i-3}
	&K^{i-2}\arrow[ul,"\id"']
	&K^{i-1} \arrow[ul,"\id"']
	&\ker d^i
	&0
	&0
	&\;
\end{tikzcd}\end{equation*}\end{ceqn}
The complex $L^*$ is the cone of the morphism
\begin{equation*}
	\im(d^{i-1})[-i]\ra\ker(d^i)[-i]
\end{equation*}
(shifted by $[-i]$ not $[i]$) and the point is that the realization functor
sends an object $D$ of $\mathcal C$ shifted by $i$ to the same object 
shifted by $i$: the complex $D[i]$ with one-step filtration
$F^j(D[i])=D[i]$ for $j\leq-i$, $F^j(D[i])=0$ for $j>-i$ is
bête-filtered with $\Gr^{-i}(D[i])[-i]=D$ and all other $\Gr^j=0$, so that
$\real(D[i])=D[i]$. Of course, the exact sequence
\begin{equation*}
	0\ra\im d^{i-1}\ra\ker d^i\ra H^iK^*\ra 0
\end{equation*}
gives rise to a distinguished triangle (where the cone is unique up to
unique isomorphism)
\begin{equation*}
	(\im d^{i-1})[-i]\ra(\ker d^i)[-i]\ra (H^iK^*)[-i]\ra.
\end{equation*}
\begin{remark}
	The functor $\real$ is exact and t-exact.
\end{remark}

\subsection*{3.1.16}
The claim about the characterization of the $\Hom_{D\mathcal C}(A,B[n])$
is effectively the classical Tôhoku statement, with the twist that it
suffices to efface morphisms one at a time, rather than the whole $\Hom$
group at once. Swan calls this weaker effaceability property described in
3.1.16 `weak effaceability.'
Grothendieck's statement about a $\delta$-functor $F^n$ being universal if
$F^n$ is effaceable for $n>0$ holds with `effaceable' replaced by `weakly
effaceable' if $F$ takes values in a module category. The $\delta$-functor
$\Hom_{D\mathcal C}(A,-[n])$ takes values in abelian groups and is weakly
effaceable.
See Buchsbaum, \emph{Satellites and Universal Functors} (4.2) \& (4.3) and
Swan, \emph{Cup products in sheaf cohomology, pure injectives, and a substitute for projective resolutions} (1.1) \& (6.1) for the full story.

The bit about every object of $D^b\mathcal C$ se dévissant en objets de
$\mathcal C$ is combined with the exact sequence of $\Hom^i$ of
proposition 1-2 of `état 0' of Verdier's thesis in SGA $4\frac12$.

\subsection*{3.1.17} As remark (ii) is a corollary of remark (i), let it
suffice to prove (i). There is some trivial ambiguity: where remark (i)
writes `pour $A,B$ in $\mathcal C$' it means `for any $A,B$ in
$\mathcal C$.' On the one hand, if $f\in\Hom_{\mathcal D}(A,B[N])$
is in the image of $\real$, say $\real(\tilde f)=f$, then $\tilde f$ can
be represented in $\Hom_{D\mathcal C}(A,B[N])$ by a Yoneda Ext
and either the monomorphism at the left end or the epimorphism at the right
will efface $\tilde f$, and, after realization, $f$ (see below).
On the other hand,
let's suppose $f$ is effaceable in the sense of (3.1.16) and we have an
epimorphism $u:A'\twoheadrightarrow A$ and monomorphism
$v'[-N]:B\hookrightarrow B'$ such that $v'fu=0$.
In light of (1.2.3), these morphisms give rise to distinguished triangles
$(A'',A',A)$ and $(B,B',B'')$ in $\mathcal D$ which are unique up to unique
isomorphism, and, in view of (1.1.9), a morphism of triangles
\begin{equation*}\begin{tikzcd}
	A'\arrow[r,"u"]\arrow[d,dashed,"g"]
	&A\arrow[r]\arrow[d,"f"]
	&A''[1]\arrow[r]\arrow[d,dashed,"h"]&\; \\
	B''[N-1]\arrow[r]&B[N]\arrow[r,"v'"]&B'[N]\arrow[r]&\;
\end{tikzcd}\end{equation*}
where now the two dashed arrows lie in a $\Hom^{N-1}_{\mathcal C}$.
All the data of this morphism of triangles (except $f$) is lifted uniquely
to $D\mathcal C$, where we can complete it to a morphism of triangles
(in possibly several different ways) and then apply $\real$. We would like
to show that $\real$ establishes a bijection between the morphisms
$\tilde f$ which complete the lift of the above to a morphism of triangles
in $D\mathcal C$ and those $f$ which complete the morphism of triangles in
$\mathcal D$. This will simultaneously show that our original $f$ is in the
image of $\real$ and prove that
$\real:\Hom^N_{D\mathcal C}(A,B[N])\ra\Hom^N_{\mathcal D}(A,B[N])$ is
injective.

We have a commutative diagram with exact rows and columns on the next page.
From this diagram we see that if $g\in\Hom^{-1}(A',B''[N])$ and
$h\in\Hom(A''[1],B'[N])$ such that there is at least one $f\in\Hom(A,B[N])$
such that $fu$ is in the image of $g$ and $v'f$ is in the image of $h$, the
set of all $f\in\Hom(A,B[N])$ satisfying this same property is in bijection
with
\begin{align*}
	&X\cap Y,\quad\text{where} \\
	&X:=\coker(\Hom^{-1}(A',B[N])\ra\Hom(A''[1],B[N]))\quad\text{and} \\
	&Y:=\coker(\Hom^{-1}(A,B'[N])\ra\Hom^{-1}(A,B''[N])).
\end{align*}
Both $X$ and $Y$ are considered as subgroups of $\Hom(A,B[N])$.
The point is that $\real$ sends $X$ and $Y$ ($\Hom$ taken in $D\mathcal C$)
isomorphically to $X$ and $Y$, respectively ($\Hom$ taken in $\mathcal D$)
by the hypothesis that $\real$ is an isomorphism on $\Hom^n$ for $n<N$.
Therefore $\real$ does the same for $X\cap Y$. This proves our claim.

\newpage
\newgeometry{bottom=2cm,left=6cm,right=1.375in} % 2.75 in / 2 = 1.375
\begin{landscape}\begin{small}\begin{ceqn}\begin{equation*}\begin{tikzcd}[row sep=large,column sep=small]
&\vdots&\vdots&\vdots&\vdots&\vdots \\
\cdots\arrow[r]
&\Hom^{-1}(A',B'[N])\arrow[r]\arrow[u]
&\Hom^{-1}(A',B''[N])\arrow[r]\arrow[u]
&\Hom(A',B[N])\arrow[r]\arrow[u]
&\Hom(A',B'[N])\arrow[r]\arrow[u]
&\Hom(A',B''[N])\arrow[r]\arrow[u]
&\cdots
\\
\cdots\arrow[r]
&\Hom^{-1}(A,B'[N])\arrow[r]\arrow[u]
&\Hom^{-1}(A,B''[N])\arrow[r]\arrow[u]
&\Hom(A,B[N])\arrow[r,"{v'}"]\arrow[u,"u"]
&\Hom(A,B'[N])\arrow[r]\arrow[u]
&\Hom(A,B''[N])\arrow[r]\arrow[u]
&\cdots
\\
\cdots\arrow[r]
&\Hom^{-1}(A''[1],B'[N])\arrow[r]\arrow[u]
&\Hom^{-1}(A''[1],B''[N])\arrow[r]\arrow[u]
&\Hom(A''[1],B[N])\arrow[r]\arrow[u]
&\Hom(A''[1],B'[N])\arrow[r]\arrow[u]
&\Hom(A''[1],B''[N])\arrow[r]\arrow[u]
&\cdots
\\
\cdots\arrow[r]
&\Hom^{-2}(A',B'[N])\arrow[r]\arrow[u]
&\Hom^{-2}(A',B''[N])\arrow[r]\arrow[u]
&\Hom^{-1}(A',B[N])\arrow[r]\arrow[u]
&\Hom^{-1}(A',B'[N])\arrow[r]\arrow[u]
&\Hom^{-1}(A',B''[N])\arrow[r]\arrow[u]
&\cdots \\
&\vdots\arrow[u]&\vdots\arrow[u]&\vdots\arrow[u]&\vdots\arrow[u]&\vdots\arrow[u] \\
\end{tikzcd}\end{equation*}\end{ceqn}\end{small}\end{landscape}
\restoregeometry

\subsubsection*{Annihilating Yoneda $\Ext$s}\label{sec:killExts}
Now we verify that given $f\in\Hom_{D\mathcal C}(A,B[N])$ represented by 
the Yoneda Ext
\begin{equation*}
	Q:=\cdots\ra0\ra B=K^{-N}\xra\alpha K^{-N+1}\ra\cdots\ra K^0\xra\beta A\ra0\ra\cdots
\end{equation*}
either the monomorphism $\alpha$ or the epimorphism $\beta$ will efface 
$f$. The postcomposition with $\alpha[N]$ is represented by the diagram
\begin{ceqn}\begin{equation*}\begin{tikzcd}
	\cdots\arrow[r]&0\arrow[r]&0\arrow[r]
	&0\arrow[r]&\cdots\arrow[r]&A\arrow[r]
	&0\arrow[r]&\cdots \\
	\cdots\arrow[r]&0\arrow[r]\arrow[u]\arrow[d]
	&B=K^{-N}\arrow[r,"\alpha"]\arrow[d,"\id_B"]\arrow[u]
	&K^{-N+1}\arrow[r]\arrow[d]\arrow[u]
	&\cdots\arrow[r]&K^0\arrow[r]\arrow[u,"\beta"]\arrow[d]
	&0\arrow[r]\arrow[d]\arrow[u]&\cdots \\
	\cdots\arrow[r]&0\arrow[r]\arrow[d]
	&B\arrow[r]\arrow[d,"\alpha"]
	&0\arrow[r]\arrow[d]&\cdots\arrow[r]
	&0\arrow[r]\arrow[d]
	&0\arrow[r]\arrow[d]&\cdots \\
	\cdots\arrow[r]&0\arrow[r]&K^{-N+1}\arrow[r]
	&0\arrow[r]&\cdots\arrow[r]&0\arrow[r]
	&0\arrow[r]&\cdots
\end{tikzcd}\end{equation*}\end{ceqn}
with $\beta$ inducing a quasi-isomorphism, and the point is that the
morphism of complexes
\begin{ceqn}\begin{equation*}\begin{tikzcd}
	\cdots\arrow[r]&0\arrow[r]\arrow[d]
	&B=K^{-N}\arrow[r,"\alpha"]\arrow[d,"\alpha"]
	&K^{-N+1}\arrow[r]\arrow[d]
	&\cdots\arrow[r]&K^0\arrow[r]\arrow[d]
	&0\arrow[r]\arrow[d]&\cdots \\
	\cdots\arrow[r]&0\arrow[r]&K^{-N+1}\arrow[r]
	&0\arrow[r]&\cdots\arrow[r]&0\arrow[r]
	&0\arrow[r]&\cdots
\end{tikzcd}\end{equation*}\end{ceqn}
is homotopic to 0 via the obvious homotopy.
On the other hand, the precomposition with $\beta$ is represented by the
diagram
\begin{ceqn}\begin{equation*}\begin{tikzcd}
	\cdots\arrow[r]&0\arrow[r]\arrow[d]&0\arrow[r]\arrow[d]
	&0\arrow[r]\arrow[d]&\cdots\arrow[r]&K_0\arrow[r]\arrow[d,"\beta"]
	&0\arrow[r]\arrow[d]&\cdots \\
	\cdots\arrow[r]&0\arrow[r]&0\arrow[r]
	&0\arrow[r]&\cdots\arrow[r]&A\arrow[r]
	&0\arrow[r]&\cdots \\
	\cdots\arrow[r]&0\arrow[r]\arrow[u]\arrow[d]
	&B=K^{-N}\arrow[r,"\alpha"]\arrow[d,"\id_B"]\arrow[u]
	&K^{-N+1}\arrow[r]\arrow[d]\arrow[u]
	&\cdots\arrow[r]&K^0\arrow[r]\arrow[u,"\beta"]\arrow[d]
	&0\arrow[r]\arrow[d]\arrow[u]&\cdots \\
	\cdots\arrow[r]&0\arrow[r]
	&B\arrow[r]
	&0\arrow[r]&\cdots\arrow[r]
	&0\arrow[r]
	&0\arrow[r]&\cdots
\end{tikzcd}\end{equation*}\end{ceqn}
and here the point is that this morphism is in the same equivalence class
as the morphism
\begin{ceqn}\begin{equation*}\begin{tikzcd}
	\cdots\arrow[r]&0\arrow[r]\arrow[d]&0\arrow[r]\arrow[d]
	&0\arrow[r]\arrow[d]&\cdots\arrow[r]&K_0\arrow[r]\arrow[d,"\id"]
	&0\arrow[r]\arrow[d]&\cdots \\
	\cdots\arrow[r]&0\arrow[r]\arrow[d]
	&B=K^{-N}\arrow[r,"\alpha"]\arrow[d,"\id"]
	&K^{-N+1}\arrow[r]\arrow[d]
	&\cdots\arrow[r]&K^0\arrow[r]\arrow[d]
	&0\arrow[r]\arrow[d]&\cdots \\
	\cdots\arrow[r]&0\arrow[r]
	&B\arrow[r]
	&0\arrow[r]&\cdots\arrow[r]
	&0\arrow[r]
	&0\arrow[r]&\cdots
\end{tikzcd}\end{equation*}\end{ceqn}
as the diagram
\begin{equation*}\begin{tikzcd}
	&K^0\arrow[dl,"\id"]\arrow[dr,"\id"] \\
	K^0\arrow[dr,"\beta"]&&Q\arrow[dl,"\beta"] \\
	&A
\end{tikzcd}\end{equation*}
commutes, where here the arrow decorations refer only to the degree 0
component of the morphism, the other components being obvious.
Now it is obvious this composition is null.

\subsection*{4.1.1} The following proof of Artin's theorem was given to me
by Sasha Beilinson. The $\eta$ notation, along with the
construction of a retraction, is discussed after the proof.

Below $D(X):=D^b_c(X,\ZZ/\ell^n),D(X)^{\leq0}:=\{\mathcal F\in D(X):\dim\supp H^i\mathcal F\leq-i\}$, i.e., this is $^pD(X)^{\leq0}$ where
$p$ is the middle perverse $t$-structure.
\begin{theorem*}[Artin]
For an affine map $f:X\ra Y$ of schemes of finite type over a field $k$
(with $\operatorname{char} k$ prime to $\ell$) the functor
$f_*:D(X)\ra D(Y)$ is right $t$-exact.
\end{theorem*}
\begin{proof}
Pick $\mathcal F\in D(X)^{\leq0}$; we want to show that
$f_*\mathcal F\in D(Y)^{\leq0}$. Let $d(\mathcal F)$ be the dimension of
support of $\mathcal F$. We use induction by $d(\mathcal F)$, so we assume
that for every $f,k$ as in the theorem and $\mathcal G\in D(X)^{\leq0}$
with $d(\mathcal G)<d(\mathcal F)$ one has $f_*\mathcal G\in D(Y)^{\leq0}$.

(o) \emph{We can assume that $k$ is algebraically closed} (since $f_*$
commutes with the base change to an algebraic closure of $k$).

(i) \emph{It is enough to show that for every closed point $y\in Y$ the
complex $(f_*\mathcal F)_y$ is connective} (i.e., acyclic in degrees $>0$):
We need to check that for a point $\eta$ of $Y$ of dimension $\delta>0$
the complex $(f_*\mathcal F)_\eta[-\delta]$ is connective. Let $Z\subset Y$
be the closure of $\eta$. Replacing $Y$ by an étale neighborhood of $\eta$,
choose a retraction $Y\ra Z$ (it exists since $k$ is perfect). Consider
$Y$ as a $Z$-scheme and $f$ as a map of $Z$-schemes. Let $f^o:X^o\ra Y^o$
be the map of the generic fibers (over $\eta=\Spec k_\eta\in Z$); this is
an affine morphism of $k_\eta$-schemes.
Since $\mathcal F|_{X^o}\in D(X^o)^{\leq-\delta}$ and
$d(\mathcal F|_{X^o})\leq d(\mathcal F)-\delta$, one has
$f^o_*(\mathcal F|_{X^o})[-\delta]\in D(Y^o)^{\leq0}$ by the induction
assumption applied to $f^o$, $k_\eta$, and $\mathcal F|_{X^o}[-\delta]$.
Now $(f_*\mathcal F)|_{Y^o}=f_*^o(\mathcal F|_{X^o})$, hence
$(f_*\mathcal F)_\eta[-\delta]=f_*^o(\mathcal F|_{X^o})_\eta[-\delta]$
is connective, q.e.d.

(ii) \emph{The case when $f$ is an open embedding $j:X\hookrightarrow\overline X$ with $Q:=\overline X-X$ a Cartier divisor:}
We can assume that $y$ as in (i) lies in $Q$ which is a principal divisor 
$h=0$. Let $K$ be the field of fractions of the henselian local ring at
$0\in\A^1_k$, $\tilde K$ its separable closure, $G:=\Gal(\tilde K/K)$, and
$\Psi=\Psi_h:D(X)\ra D(Q)$ be the nearby cycles functor. One has 
$(j_*\mathcal F)_y=R\Gamma(G,\Psi(\mathcal F)_y)$, so, since $G$ has
cohomological dimension 1, it is enough to check that
$\Psi(\mathcal F)_y[-1]$ is connective. By definition, $\Psi(\mathcal F)_y$
is inductive limit of complexes
$R\Gamma(U_{\tilde K},\mathcal F_{U_{\tilde K}})$ where $U/\overline X$
runs the category of affine étale neighborhoods of $y$,
$U_{\tilde K}:=U\times_{\A^1}\Spec\tilde K$, $\mathcal F_{U_{\tilde K}}$ is
the pullback of $\mathcal F$ by the map $U_{\tilde K}\ra X$.
Since $\mathcal F_{U_{\tilde K}}\in D(U_{\tilde K})^{\leq-1}$ and
$D(\mathcal F_{U_{\tilde K}})<d(\mathcal F)$, each complex
$R\Gamma(U_{\tilde K},\mathcal F_{U_{\tilde K}})[-1]$ is connective by the
induction assumption applied to the affine map
$U_{\tilde K}\ra\Spec\tilde K$, and so $\Psi(\mathcal F)_y[-1]$ is
connective.

(iii) \emph{The case when $f$ is the projection} $p:X=\A^1\times Y\ra Y$:
For $y$ as in (i) consider the complementary embeddings 
$i_y:X_y=\A^1_y\hookrightarrow X$, $j_y:X-X_y\hookrightarrow X$.
Applying $p_*(-)_y$ to the exact triangle
$j_{y!}j_y^*\mathcal F\ra\mathcal F\ra i_{y*}i_y^*\mathcal F$ we see that 
it is enough to show that $(p_*i_{y*}i_y^*\mathcal F)_y$ and
$(p_*j_{y!}j_y^*\mathcal F)_y$ are connective.

$(p_*i_{y*}i_y^*\mathcal F)_y$ \emph{is connective:}
One has $(p_*i_{y*}i_y^*\mathcal F)_y=R\Gamma(\A^1_k,i_y^*\mathcal F)$, so 
it is enough to check that for every successive quotient of the (usual)
canonical filtration on $i_y^*\mathcal F$ the complex
$R\Gamma(\A^1_k,\Gr_ni_y^*\mathcal F)$ is connective.
Since $\Gr_{>0}\mathcal F=0$ and $\Gr_0\mathcal F$ is supported at finitely 
many points, we are reduced to the claim that for a (usual) sheaf
$\mathcal G$ on $\A^1_k$ one has $H^{>1}(\A^1_k,\mathcal G)=0$ which is
SGA 4 IX 5.7.

$(p_*j_{y!}j_y^*\mathcal F)_y$ \emph{is connective:}
One has $\mathcal G:=j_{y!}j_y^*\mathcal F\in D(X)^{\geq0}$.
Consider the open embedding
$j:X=(\PP^1-\{\infty\})\times Y\hookrightarrow\overline X:=\PP^1\times Y$.
Let $\overline p:\overline X\ra Y$ be the projection, and
$\overline{i}_y:\PP^1_k\ra\overline X$ be the embedding
$\overline{i}_y(a)=(a,y)$. Then $p=\overline pj$,
$(p_*\mathcal G)_y=(\overline p_*j_*\mathcal G)_y=R\Gamma(\PP^1_k,\overline{i}^*_yj_*\mathcal G)$ be proper base change, and so
$(p_*\mathcal G)_y=(j_*\mathcal G)_{\overline{i}_y(\infty)}$
since $i_y^*\mathcal G=0$. We are done by (ii) applied to $j$ and
$\mathcal G$.

(iv) \emph{The general case:} It is enough to write $f$ as a composition
$f=f_nf_{n-1}\ldots f_0$ of affine maps $f_i$ such that our claim is true
for each $f_i$ (indeed, the sheaves
$\mathcal F_i:=(f_if_{i-1}\ldots f_0)_*\mathcal F$
satisfy $d(\mathcal F_i)\leq d(\mathcal F)$, and so
$\mathcal F_i=f_{i*}\mathcal F_{i-1}\in D^{\leq0}$ by induction by $i$).
Now locally on $Y$ we can factor $f$ as composition
$X\hookrightarrow\A^n\times Y\ra Y$ where $\hookrightarrow$ is a closed
embedding and $\ra$ is the projection. Thus
$f=f_nf_{n-1}\ldots f_0$ where $f_0$ is $\hookrightarrow$ and $f_i$ is the 
projection $\A^{n-i+1}\times Y\ra\A^{n-i}\times Y$ for $i>0$. Our claim is
true for $f_0$ since $f_{0*}$ is t-exact and for $f_i$, $i>0$, by (iii),
and we are done.
\end{proof}

\subsubsection*{Constructing retractions}
In the above, $\eta$ is used simultaneously for a point of the topological
space of the scheme $Y$ and for a geometric point centered on this
(scheme-theoretic) point.
Let $\overline\eta$ be a geometric point of $Y$ centered on a point
$\eta$ of $Y$ and $Z=\overline{\{\eta\}}$.
As $k$ is perfect, the smooth locus of $Z$ is nonempty, so we may
assume that $Z$ factors as $Z\xra h\A^\delta\ra\Spec k$ for $h$ étale.
Suppose $Y=\Spec A$, $Z=\Spec B$ and form the pullback
\begin{equation*}\begin{tikzcd}
	A'\arrow[d]\arrow[r]\arrow[dr,"\ulcorner",phantom,near start]&k[x_1,\ldots,x_\delta]\arrow[d] \\
	A\arrow[r]&B
\end{tikzcd}\end{equation*}
of rings; it is easy to find a retract $k[x_1,\ldots,x_\delta]\ra A'$.
Let $Y'=\Spec A'$.
\begin{equation*}\begin{tikzcd}
	Z\arrow[r,hook]\arrow[d,"h"]&Y\arrow[d] \\
	\A^\delta\arrow[r,hook]&Y'\arrow[l,bend right,dashed]
\end{tikzcd}\end{equation*}
The map $Z\times_{\A^\delta}Y\ra Y$ is étale and the base change by
$Z\ra Y$ is given by $Z\times_{\A^\delta}Z\ra Z$, which admits the diagonal
as section. As $h$ is étale, this diagonal is is an isomorphism onto a
connected component of $Z\times_{\A^\delta}Z$ (SGA 1 I 9.3) and we identify
$Z$ with this component. Let $U$ denote
$Z\times_{\A^\delta}Y$ minus the closed subscheme $Z\times_{\A^\delta}Z-Z$;
$U$ is an étale neighborhood of $\overline\eta$ and is equipped with a 
retract $U\ra Z$, namely the one
\begin{equation*}
	Z\hookrightarrow U=Z\times_{\A^\delta}Y-(Z\times_{\A_\delta}Z-Z)
	\ra Z
\end{equation*}
induced by the first projection $Z\times_{\A^\delta}Y\ra Z$.


\subsection*{4.1.7}
The \v Cech spectral sequence is also called the Cartan-Leray spectral
sequence and its existence in an arbitrary topos is established in
SGAA, Exp. V 3.3.

\subsection*{4.1.8} The only thing worth mentioning is that the entire
second paragraph is implicitly local to $U_i$. After all, on $U_i$ we have 
that $\tau_{\leq-i}K$ is in $^pD^{\leq0}$, and to show that $K|{U_i}$ is in
$^pD^{\leq0}$, it suffices to show that $H^q(\tau_{>-i}K)|{U_i}$ is 0 for 
all $q$ (i.e. for all $q>-i$). Proceeding by descending induction on $q>-i$,
the induction step consists of showing that $H^i(V\cap W_i,\mathcal L)=0$
for all affine open $V$ implies that $\mathcal L=0$.

\subsection*{4.2.5}
As $f^*$ and $R\underline\Hom$ commute with reduction modulo $\ell^n$, it is
enough to prove the statement in the category $D^b_{ctf}(X,\ZZ/\ell^n)$,
and the equality $\Gamma H^0R\underline\Hom=H^0R\Gamma R\underline\Hom$
holds in $D(X,\ZZ/\ell^n)$ because, as $R\underline\Hom$ is in
$D_c^{\geq0}$, $H^0=\tau_{\leq0}\ker d_0$, and $\Gamma$ commutes with the
formation of kernels. For (4.2.5.3), the retraction
$\mathcal H\ra\ ^\circ f_*f^*\mathcal H$ is $^\circ f_*(\eta(f^*\mathcal H)$
where $\eta$ is the unit $\id\ra\ ^\circ e_*e^*$:
\begin{equation*}
	\mathcal H\ra\ ^\circ f_*f^*\mathcal H
	\ra\ ^\circ f_*\,^\circ e_* e^*f^*\mathcal H=\mathcal H,
\end{equation*}
and it remains to check that the first arrow is surjective.
The references in the rest of this paragraph are to SGAA Exposé XV.
By, (1.1), it suffices to show that
$H^0(Y',\mathcal H)\ra H^0(X',f'^*\mathcal H)$ is bijective for each
$Y'\ra Y$ étale. Replacing $Y$ by $Y'$, we know by (1.5) that $f$ is
$(-1)$-acyclic; i.e. that
$\alpha:H^0(Y,\mathcal H)\ra H^0(X,f^*\mathcal H)$ is injective, as $f$ is
surjective. Moreover, $f$ is locally acyclic as it is smooth.
Then, (1.16) shows that $\alpha$ is surjective iff for every geometric
point $\overline y$ of $Y$ algebraic over a point $y$ of $Y$,
$\overline\alpha:H^0(\overline y,\mathcal H_{\overline y})\ra H^0(X_{\overline y},f^*\mathcal H|X_{\overline y})$ is, so we may assume
$Y$ is the spectrum of an algebraically closed field, in which case $X$ is
connected, as the fibers of $f$ were assumed geometrically connected.
Then $\alpha$ is seen to be bijective by another application of (1.1),
which reduces the matter to the corresponding question for a constant sheaf.
Note that the existence of a retraction discussed above is irrelevant to
this argument.

\subsection*{4.2.6}
This paragraph, as written, is nonsense.
The correct (equivalent) statements are

(a)\quad $u^*$ identifies $\mathcal A$ with a subcategory of $\mathcal B$ closed under subquotients. \\
\indent(b)\quad the unit of adjunction $\eta_!:\id_{\mathcal B}\ra u^*u_!$ is a natural epimorphism;\\
\indent(b$'$)\quad the counit of adjunction $\vep_*:u^*u_*\ra\id_{\mathcal B}$ is a natural monomorphism.

Note that statement (a) is different from $\mathcal A$ being épaisse, as an
épaisse subcategory is also closed under extensions.
A general remark: an adjunction $F:\mathcal C\leftrightarrows\mathcal D:G$ 
with unit $\eta$ and counit $\vep$ is called idempotent if it can be
factored as a reflection and a coreflection, in which case many things are
true; see the nLab page `idempotent adjunction.' In particular, $\vep F$ and
$\eta G$ are natural isomorphisms. In our situation, both $u_!\dashv u^*$
and $u^*\dashv u_*$ are idempotent adjunctions as they are reflective and
coreflective, respectively, so we get that $\vep_*u^*$ and $\eta_!u^*$ are
natural isomorphisms. Of course, $u_!$ is right exact and $u_*$ is left exact. We prove (a)$\Leftrightarrow$(b$'$); dual arguments give
(a)$\Leftrightarrow$(b).

To prove (a)$\Rightarrow$(b$'$), let $B$ be an object of $\mathcal B$.
In the commutative diagram with exact rows,
\begin{equation*}\begin{tikzcd}
	0\arrow[r]&\ker\vep_*(B)\arrow[r]&u^*u_*B\arrow[r]&B \\
	0\arrow[r]&u^*u_*\ker\vep_*(B)\arrow[u,"\alpha"]\arrow[r]&u^*u_*u^*u_*B\arrow[u,"\beta"]\arrow[r,"\gamma"]
	&u^*u_*B\arrow[u]
\end{tikzcd}\end{equation*}
$\ker\vep_*(B)$ is in the essential image of $\mathcal A$ as $\mathcal A$ is
closed under subobjects. Therefore $\alpha$, $\beta$, and $\gamma$ are
isomorphisms, which shows $u^*u_*\ker\vep_*(B)=0$ and therefore
$\ker\vep_*(B)=0$.

Now let's prove (b$'$)$\Rightarrow$(a). In the commutative diagram
\begin{equation*}\begin{tikzcd}
	0\arrow[r]&A\arrow[r]&u^*B\arrow[r]&C\arrow[r]&0 \\
	0\arrow[r]&u^*u_*A\arrow[r]\arrow[u]
	&u^*u_*u^*B\arrow[u]\arrow[r]&u^*u_*C\arrow[u]
\end{tikzcd}\end{equation*}
with exact rows, the middle arrow is an isomorphism and the outer arrows
are monomorphisms. By the four-lemma, the first arrow is also an
epimorphism, hence an isomorphism, identifying $A$ in the essential image of
$\mathcal A$. Therefore this essential image is closed under subobjects and
hence also under subquotients, as $u^*$ is exact.

Identifying $\mathcal A$ with its essential image, a full subcategory of
$\mathcal B$, every object $B$ of $\mathcal B$ has a largest subobject in
$\mathcal A$, viz. $u^*u_*B$, and a largest quotient in $\mathcal A$, viz.
$u^*u_!B$. To see this, simply observe that both candidates are indeed in
$\mathcal A$, and if $A$ is in $\mathcal A$ and a subobject of $\mathcal B$,
then $A\simeq u^*u_*A\hookrightarrow u^*u_*B\hookrightarrow B$, and dually.

The example adjunction is backwards:
$u_!(X\ra Y)=(Y\xra{\sim}Y)$ is the left adjoint,
$u_*(X\ra Y)=(X\xra\sim X)$ is the right adjoint, since the diagrams
\begin{equation*}\begin{tikzcd}
	A\arrow[r]&B\arrow[d] && A\arrow[r,"\sim"]\arrow[d]&B \\
	C\arrow[r,"\sim"]&D && C\arrow[r]&D
\end{tikzcd}\end{equation*}
have unique completions to commutative squares.
Then $\vep_*:u^*u_*\ra\id_{\mathcal B}$ needn't be a natural monomorphism, 
as a commutative diagram of the sort
\begin{equation*}\begin{tikzcd}
	A\arrow[d]\arrow[r]&B\arrow[d] \\
	C\arrow[r,"\sim"]\arrow[d,"\id"]&C\arrow[d] \\
	C\arrow[r]&D
\end{tikzcd}\end{equation*}
determines the morphism $A\ra C$ but isn't enough to determine the morphism
$B\ra D$.

\subsection*{4.2.6.1}
The Jordan-H\"older theorem holds in any abelian category, and since $u^*$
is exact and preserves simple objects, identifying $\mathcal A$ with its
essential image, it follows that the components of any $A$ in $\mathcal A$
also belong to $\mathcal A$. To show that $\mathcal A$ is closed under
subobjects, and therefore subquotients, it will suffice to show that if
$0\ra B\ra A\ra S\ra0$ with $A$ in $\mathcal A$, $B$ in $\mathcal B$, and
$S$ simple, then $B$ is in $\mathcal A$. Since $S$ is simple, it is a
component of $A$ and therefore in $\mathcal A$. The five-lemma, applied to
the diagram with exact rows obtained by applying $\vep_*$
\begin{equation*}\begin{tikzcd}
	0\arrow[r]&B\arrow[r]&A\arrow[r]&S \\
	0\arrow[r]&u^*u_*B\arrow[r]\arrow[u]
	&u^*u_*A\arrow[u,"\sim"]\arrow[r]&u^*u_*S\arrow[u,"\sim"],
\end{tikzcd}\end{equation*}
shows that indeed $u^*u_*B\xra\sim B$, and $B$ is in $\mathcal A$.

\subsection*{4.2.6.2}
Notes on the proof are below; the proof depends on the middle perversity
insofar as the commutativity of intermediate extension with $f^*[d]$ relies
on the relative dimension coinciding with the change in perversity
between an irreducible component and its inverse image by $f$.
The proof doesn't work for $\ZZ_\ell$-sheaves as it relies on 4.3.1 which
fails for $\ZZ_\ell$-cohomology.

\subsubsection*{Commutation of $f^*[d]$ with intermediate extension}
We wish to show that $f^*[d]j_{*!}=j_{*!}f^*[d]$.
The transitivity of $j_{*!}$ for $j$ the inclusion of a locally closed
subset allows us to factorize the inclusion of $V$ as the open immersion
$V\hookrightarrow\overline V$ followed by the closed immersion
$\overline V\hookrightarrow Y$; the latter posing no problem as in this case
lower shriek, lower star, and intermediate extension all coincide and 
commute with $f^*[d]$ by smooth base change. Reduced to the case
of open immersion $Y=\overline V$, let $m$ be minimal such that $F=Y-V$ is
of dimension $\leq m$, and put $t:=p(2m)$. We can find a stratification of
$F$ into strata satisfying 2.2.10 (a) so that for
$A$ in $D_c^b(V,R/m^n)$, the $H^ij_*A$ are locally constant on each stratum for $i\geq t$. Let $U_n$ (resp. $F_n$) denote the
union of strata $S$ in this stratification of $F$ satisfying $p(S)\leq n$
(resp. $p(S)\geq n$).
For such an $S$, $\overline S-S$ is a union of strata of dimension strictly 
less than $\dim S$. Therefore each $S$ in the stratification of
$U_n-U_{n-1}$ is closed in $U_n$. By transitivity of the intermediate
extension, it will suffice to extend $A$ from $V$ to $V\cup U_t$,
as the complement of the latter in $Y$ is $F_{t+1}$, closed;
$V\cup U_t-V=U_t$ is also closed in $V\cup U_t$, and nonempty.
$U_t$ is a disjoint union of equidimensional strata smooth over $\overline k$
(hence by the note to 2.2.10, the étale topology sees them as smooth),
so that the $H^ij_*A$ are locally constant on $U_t$ for $i\geq t$.
We have reduced to the setting of 2.2.4: $Y=\overline V$, $F=Y-V$
smooth of dimension $m$, $H^ij_*A$ are locally constant on $F$ for
$i\geq t:=p(2m)$, and $j_{!*}A=\tau_{\leq t-1}^Fj_*A$.
As $f$ is smooth of relative dimension $d$, $f^{-1}F$ is also 
equidimensional, now of dimension $m+d$. As pullbacks of locally constant
sheaves, the $H^ij_*f^*A$ are still locally constant on $f^{-1}F$.
If $t'=p(2(d+m))=-d-m$, then shifting by $d$ and applying $\tau^F_{<t'}$ is
the same as applying $\tau^F_{<t}$ and then shifting by $d$. It only remains
to verify that $f^*$ commutes with $\tau^F_{\leq0}$, but this is easy:
letting $i:F\hookrightarrow Y$ be the closed immersion,
$(f^*\tau^F_{\leq 0}A,f^*A,f^*i_*\tau_{>0}i^*A)$ defines
$f^*\tau^F_{\leq0}A$, but $f^*i_*\tau_{>0}i^*A=i_*\tau_{>0}i^*f^*A$
(notating $i$ as usual for the base extension as well), hence also
defines $\tau_{\leq0}^Ff^*A$.

\subsubsection*{Irreducibility of the inverse image of an irreducible local
system on an irreducible scheme by a smooth morphism with geometrically
connected fibers} SGA 1, Exp. IX, 5.6 shows that if $f:S'\ra S$ is
universally submersive (e.g. faithfully flat and quasi-compact) with
geometrically connected fibers, and $S$ is connected, then $S'$ is
connected, and, choosing a geometric point
$s'$ of $S'$ and letting $s$ be the image of $s'$ in $S$, the homomorphism
$\pi_1(S',s')\ra\pi_1(S,s)$ is surjective. This immediately implies that if
$\mathscr L$ is an irreducible local system on $S$, $f^*\mathscr L$ is an
irreducible local system on $S'$. Of course, smooth with geometrically
connected fibers means smooth with geometrically irreducible fibers as these
fibers are themselves smooth (SGA 1, Exp II, 2.1) and therefore regular.
It is a topological fact
(\href{https://stacks.math.columbia.edu/tag/004Z}{Stacks \texttt{004Z}})
that if $Y$ is irreducible and $f:X\ra Y$ is open with irreducible fibers,
then $X$ is irreducible. This verifies that the if $V$ is irreducible
and $f$ is smooth with connected fibers, the inverse image on $f^{-1}V$
of an irreducible local system on $V$ is again irreducible.

\subsection*{4.2.7}
Let $X',X,Y$ be of finite type over a field $k$ and $f:X'\ra X$.
We wish to show that $\boxtimes$ commutes with direct image; i.e. that the
below arrow is an isomorphism.
\begin{equation*}
	f_*K\boxtimes L\ra (f\times\id)_*(K\boxtimes L)
\end{equation*}
We will compute locally about a geometric point
$\xi:=(x,y)\ra X\times Y$ ($x,y$ geometric points of $X,Y$ respectively), 
so that all the objects in $X_x\leftarrow (X\times Y)_\xi\rightarrow Y_y$
are spectra of strictly henselian local rings.
Let $t\ra X_x$ be a geometric point centered on the generic point of an
irreducible component of $X_x$.
\emph{Th. finitude} 2.16 gives that $Y\ra\Spec k$ is universally locally
acyclic, so for any $M$ in $D^+(X\times Y,\ZZ/\ell)$,
\begin{equation*}
	\Gamma((X\times Y)_\xi,M)=\Gamma((X\times Y)_{\xi,t},M),
\end{equation*}
where $(X\times Y)_{\xi,t}$ denotes the geometric fiber in $t$ of
$(X\times Y)_\xi\ra Y_y$. To pass from $X_x$ to $(X_x)_t=t$ we can first
pass to the limit of Zariski neighborhoods of $t$, which is the spectrum of 
an artinian local ring, then kill nilpotents and extend scalars. As lower
star commutes with smooth base change and the étale topology doesn't see
nilpotents, we may therefore assume 
$X=t$, the spectrum of a separably closed extension $k(t)$ of $k$, and 
$Y=Y_y$.
As $X_x\ra\Spec\overline k\leftarrow Y_y$, $\Spec\overline k\times_k Y_y$ is 
the disjoint union of copies of $Y_y$, and $\xi$ picks one of them; i.e.
$(\Spec\overline k\times_k Y_y)_\xi=Y_y$. So we may assume
$Y_y\ra\overline k$, and write
\begin{equation*}
	\Gamma((X\times Y)_{\xi,t},M)=\Gamma(Y_y\times_{\overline k}k',M)
	=\Gamma(Y_y,M),
\end{equation*}
where the second equality is Arcata V 3.3.
We are reduced to $k=\overline k$, $X=\overline k$, $Y=Y_y$, in which case
the formula is
\begin{equation*}
	\Gamma(X',K)\otimes\Gamma(Y,L)\ra\Gamma(X'\times Y,K\boxtimes L),
\end{equation*}
which is obtained from the Künneth formula of \emph{Th. finitude} 1.11 by
smooth base change and passage to the limit along $Y_y\ra Y$.

\subsection*{4.2.8} It would appear that $\boxtimes$ is only right t-exact
in $\ZZ_\ell$-cohomology, due to the possible appearance of $\Tor$.

\subsection*{4.3.1} This is a theorem in $\QQ_\ell$-cohomology and not in
$\ZZ_\ell$-cohomology because the latter has few irreducible objects; in
particular, the category of lisse $\ZZ_\ell$-sheaves is not artinian;
the irreducible lisse $\ZZ_\ell$ sheaves are torsion….

\subsection*{4.3.3} `la monodromie de $\mathcal L$ ne change pas par
réstriction à $U$' $\rightsquigarrow$ SGA 1 Exp. 1 10.3.

\subsection*{4.3.4} The dévissage of the perverse sheaf $\mathcal F$ should
occur over an irreducible affine smooth open so that we can apply the
results of 4.1.10--4.1.12. The sequence 4.1.10.1 has outer terms supported
on $X-U$ and so reduces the problem for $\mathcal F$ to that for
$j_!j^*\mathcal F=j_!\mathcal L[\dim U]$; 4.1.12.3 then reduces
the problem for $j_!\mathcal L[\dim U]$ to that for
$j_{!*}\mathcal L[\dim U]$. This allows us to proceed by induction on the
length of $\mathcal L$. Let $\mathcal L'\subset\mathcal L$ be a simple
lisse subsheaf. It suffices to remark that the kernel of
$j_{*!}\mathcal L'[\dim U]\ra j_{*!}\mathcal L[\dim U]$ is supported on
$X-U$, and the restriction of the cokernel to $U$ has strictly lesser 
length than that of $\mathcal L'$.

\subsection*{4.4 \& Appendix A: t-exactness of nearby \& vanishing cycles}
\label{BBD:AppendixA}
To understand the argument in Appendix A, \cite[\S3.1]{Illusie} is very
helpful.
In Appendix A, the calculation of $R\Gamma$ proceeds along identical lines
to \cite[7.11.3 \& 10.7]{Modular}.

There is a discrepancy between (4.4.2) the the proposition of Appendix A;
namely (4.4.2) says that $\Psi_{\overline\eta}:=R\Psi_{\overline\eta}$ is
right t-exact, while the appendix claims that the same functor has perverse
amplitude $-1$; i.e. that $\Psi_{\overline\eta}[-1]$ is t-exact.
This discrepancy is due to a difference of t-structures; both are relative
to the middle perversity function $p_{1/2}$, but the dimension function
in (4.4.2) is the naïve one, while the dimension function in the appendix
is the rectified dimension function introduced by Artin in
\cite[XIV 2.2]{SGAA}, which is also the one described in Remark (i) to
Appendix A. The point is that if $S$ admits structure morphism to a field
$k$ and $x\in X$ maps to the generic point of $S$,
\begin{equation*}
	\operatorname{tr. deg.}(k(x)/k)=\operatorname{tr. deg.}(k(x)/k(\eta))+1;
\end{equation*}
in general if $x$ has image $s$ in $S$, Artin rectifies the naïve dimension
function by adding $\operatorname{tr. deg.}(k(x)/k(y))$ so that
\begin{equation*}
	\delta(x)=\dim\overline{\{y\}}+\operatorname{tr. deg.}(k(x)/k(y)).
\end{equation*}
This has the effect of shifting the naïve t-structure on
$X_{\overline\eta}$ by 1 to the left, so that if $\Psi_{\overline\eta}$ is
t-exact with respect to the naïve t-structure (4.4.2),
$\Psi_{\overline\eta}[-1]$ is t-exact with respect to the rectified
t-structure. The point is that we would like to think of $S$ as the
henselization of a curve at a regular point; if $S$ were instead the curve
instead of its localization and $X\ra S$ were still of finite type, any
point $x$ of $X$ lying over the generic point of $S$ would have strictly
positive dimension ($\dim\overline{\{x\}}>0$), and indeed if $S$ were a
curve of finite type over a field $k$, $\dim\overline{\{x\}}$ would be
given by precisely $\operatorname{tr. deg.}(k(x)/k)$.

On the matter of invariants of a $\Lambda[G]$-module when $G$ is a finite
group of order invertible in $\Lambda$: the functor `invariants under $G$'
$\Hom_{\Lambda[G]}(\Lambda,-)$
is exact if $\Lambda$ is projective as $\Lambda[G]$-module. The canonical
surjection $\Lambda[G]\twoheadrightarrow\Lambda$ is split by
\begin{equation*}
	1\mapsto\frac1{|G|}\sum_{g\in G}g=:\omega,
\end{equation*}
which recognizes $\Lambda$ as projective $\Lambda[G]$-module.
Therefore when $Q$ is a profinite group of order prime to $\ell$ acting on
a finite $\Lambda$-module ($\Lambda\supset\ZZ/\ell^n$), the functor
`invariants under $Q$' is exact.
This argument can be applied stalkwise, and gives more.
Namely, if $M$ is a $\Lambda[G]$-module, the map
$M\twoheadrightarrow M_G$ factors as
\begin{equation*}
	M\xra{\omega}M^G\hookrightarrow M_G.
\end{equation*}
This shows that the map $M^G\hookrightarrow M_G$ is an isomorphism; the
inverse is given by the map which to a class $[m]\in M_G$ associates
$\omega\, m$.

There is the matter of how to define the nearby cycles:
in \cite[XIII 2.1]{SGA7} the functor $R\Psi$ is defined in a somewhat 
different way from the functor $\Psi$ of \cite[I 2.2]{SGA7}.
Let $(S,\eta,s)$ be a henselian trait and let
$\tilde\eta,\overline\eta$ denote the spectra of
respectively the maximal unramified extension of $k(\eta)$ and the
separable closure of $\eta$.
Let $\tilde S$ denote the normalization of $S$ in $k(\tilde\eta)$; and
$\tilde s$ its closed point; $\tilde s$ is the spectrum of the separable
closure of $k(s)$ and $\tilde S$ is the strict henselization of $S$ at
$\tilde s$.
Let $\overline S$ denote the normalization of $S$ in $k(\overline\eta)$:
$\overline S$ is the spectrum of a valuation ring with value group $\QQ$,
generic point $\overline\eta$, and closed point $\overline s$ with residue 
field a purely inseparable extension of $k(\tilde s)$.
Let $\rho:\overline\eta\ra\eta$ denote the chosen geometric generic point
of $S$. We have a commutative diagram
\begin{equation*}\begin{tikzcd}
	\overline s\arrow[d]\arrow[r,"\overline i"]&
	\overline S\arrow[d]&\overline\eta\arrow[l,"\overline j"']\arrow[d] \\
	\tilde s\arrow[d]\arrow[r,"\tilde i"]&
	\tilde S\arrow[d]&\tilde\eta\arrow[l,"\tilde j"']\arrow[d] \\
	s\arrow[r,"i"]& S&\eta\arrow[l,"j"'].
\end{tikzcd}\end{equation*}
Let $f:X\ra S$ and $G:=\Gal(\overline\eta/\eta)$.
Exposé 1 (and Appendix A) suppose $S$ is strictly henselian to begin with, 
in which case the tilde is superfluous, and define a functor $\psi$.
On the other hand, in Exposé XIII, Deligne defines for (a not necessarily
strictly henselian $S$)
\begin{equation*}
	R\Psi_\eta(K):=\overline i^*R\overline j_*K_{\overline\eta},
	\qquad\qquad K\in D^+(X_\eta,\Lambda).
\end{equation*}
Let $\overline\rho:\overline s\ra s$ denote the morphism obtained by base
change of $\overline S\ra S$ along $i:s\ra S$
(technically the base change of the latter morphism has its source in the
spectrum of a field purely inseparable over $k(\overline s)$, but the map
to $s$ from the spectrum of this larger field factors through our
$\overline\rho$), and define
\begin{equation*}
	\psi(K):=\overline\rho^*i^*Rj_*\rho_*\rho^*K,
	\qquad\qquad K\in D_c^b(X_\eta,\Lambda).
\end{equation*}
When $S$ is strictly henselian, this coincides with the functor 
$i^*Rj_*\rho_*\rho^*K$. But now $\psi$ and $R\Psi_\eta$ coincide on
an arbitrary henselian trait $S$:
% when $S$ is strictly henselian? Henceforth
% assume $S$ strictly henselian and suppress the tilde decorations.
% Let $\rho_i:\eta_i\ra\eta$ be a finite Galois extension with
% $G_i:=\Gal(\eta_i/\eta)$ and let $S_i$ be the normalization of
% $S$ in $k(\eta_i)$ with $i_i$, $j_i$ as above.
% Let $\rho$ denote not only $\overline\eta\ra\eta$ and its base extensions
% but also $\overline S\ra S$ and its base extensions, and likewise for 
% $\rho_i$. Suppose $s=\overline s$ (unimportant). With this notation,
% \begin{equation*}
%	i^*Rj_*\rho_{i*}\rho_i^*K
%	=i^*\rho_{i*}Rj_{i*}\rho_i^*K
%	=i_i^*Rj_{i*}\rho_i^*K
% \end{equation*}
% by proper base change. Passage to the limit along $i$ finds the desired
% \begin{equation*}
%	\psi(K)=i^*Rj_*\rho_*\rho^*K
%	=\overline i^*R\overline j_*\rho^*K=R\Psi(K).
% \end{equation*}
the fact that an integral morphism commutes with all base
extension \cite[VIII 5.6]{SGAA} allows one to write
\begin{equation*}
	\psi(K)=\overline\rho^*i^*Rj_*\rho_*\rho^*K
	=\overline\rho^*i^*\rho_*R\overline j_*K_{\overline\eta}
	=\overline\rho^*\overline\rho_*\overline i^*R\overline j_*K_{\overline\eta}
	=\overline i^*R\overline j_*K_{\overline\eta}
	=R\Psi(K).
\end{equation*}
Note that in the case $X=S$, $R\Psi(K)$ just gives the sheaf 
$K_{\overline\eta}$ on the topos $s\times_s\eta\simeq\eta$, and the
distinguished triangle defining $R\Psi$ can be written as
\begin{equation*}
	\operatorname{sp}^*K_{s}\ra K_{\overline\eta}\ra R\Psi(K)\ra.
\end{equation*}
One obtains the isomorphism
\begin{equation*}
	K\xra\sim R\Gamma(G,\rho_*\rho^*K)\qquad\qquad K\in D^b_c(X_\eta,\Lambda)
\end{equation*}
of Appendix A from the same argument, namely by passage to the limit from
\begin{equation*}
	K\xra\sim R\Gamma(G_i,\rho_{i*}\rho^*_iK)\qquad\qquad K\in D^b_c(X_\eta,\Lambda).
\end{equation*}
This isomorphism can be proved after a base change $\eta_i\ra\eta$
after which it is literally Shapiro's lemma
(compare \cite[XIII \S1]{SGA7}).
(The passage to the limit is justified in light of the isomorphisms
\begin{equation*}
	H^q(G,A)\xra\sim H^q(G/U,A^U)
\end{equation*}
as $U$ runs over open subgroups of $G$ and $A$ is a discrete $G$-module
(c.f. Serre, \emph{Cohomologie Galoisienne} \S2.2).)

\subsubsection*{Appendix A (i)}
In light of the above, the identity $i^*Rj_*K=R\Gamma(G,R\Psi(K))$ is 
definitional \cite[3.1.3]{Illusie}, but perhaps not quite – a detailed 
argument is written in
\hyperref[morel]{the note to Morel's article on gluing perverse sheaves}.
The spectral sequence associated to the
perverse canonical filtration is the spectral sequence associated to
the filtration of complexes given by the $^p\tau_{\leq}$; see
\cite[\S3.1.5]{BBD} for the definition of the filtration and
\href{https://stacks.math.columbia.edu/tag/012N}{\texttt{012N}}
for the formulation of the spectral sequence arising from a filtration of
complexes (see also
\href{https://stacks.math.columbia.edu/tag/015X}{\texttt{015X}}). 
Unfortunately when writing down the spectral sequence in
Appendix A there is a collision of notation, as $a,b$ are
used as indexes in the spectral sequence while $R\Psi(\mathcal F)$ still 
has perverse amplitude $[a,b]$.
The perverse amplitude of $i^*j_*\mathcal F$ is contained in $[-1,0]$ by
\cite[4.1.10 (ii)]{BBD}. The compatibility of $R\Psi$ with change of trait
is \emph{Th. finitude} (3.7).

\subsubsection*{Appendix A (ii)}
$Y$ must denote the closed fiber $X_s$. When $\mathcal G$ is supported on
the closed fiber, $R\Psi_\eta(\mathcal G)=0$ and the claim is clear.
Therefore we assume the support of $\mathcal G$ has nonempty intersection 
with $X_\eta$ and $\mathcal G$ is a simple perverse sheaf.
\cite[4.3.1 \& 4.3.2]{BBD} gives $\mathcal G=j_{!*}j^*\mathcal G$, and
\cite[4.1.12]{BBD} allows us to write the distinguished triangle
\begin{equation*}
	i^*\mathcal G\ra R\Psi(\mathcal G_\eta)\ra R\Phi(\mathcal G)\ra
\end{equation*}
\leqnomode
\begin{equation*}\tag*{as}
	(^pH^{-1}i^*j_*\mathcal F)[1]\ra R\Psi(\mathcal F)\ra R\Phi(\mathcal G)\ra.
\end{equation*}
\reqnomode
This shows that $R\Psi(\mathcal G)[-1]$ is a perverse sheaf, and using
(i) and (*) obtains
\begin{equation*}
	R\Psi(\mathcal G)[-1]
	=\ ^pH^{-1}R\Psi(\mathcal F)/(^pH^{-1}R\Psi(\mathcal F))^G.
\end{equation*}




\addtocontents{toc}{\protect\setcounter{tocdepth}{-1}}
\begin{thebibliography}{SGA5}
	\bibitem[BBD]{BBD} \textit{Faisceaux Pervers}
	par Beilinson, Bernstein, Deligne (\& Gabber!)
	\bibitem[CD]{CD} \textit{Catégories Dérivées (Etat 0)} par Verdier,
	dans SGA $4\frac12$
	\bibitem[D]{Modular} \emph{Les constantes des équations fonctionnelles des fonctions $L$} par Deligne, dans LNM \textbf{349}
	\bibitem[I]{Illusie} \emph{Autour du théorème de monodromie locale} par Illusie, dans Astérisque \textbf{223}
	\bibitem[I2]{Cotangent} \emph{Complexe Cotangent et déformations I} par Illusie, dans LNM \textbf{239}
	\bibitem[SGA5]{SGA5} SGA 5, dirigé par Grothendieck
\end{thebibliography}
\addtocontents{toc}{\protect\setcounter{tocdepth}{1}}
\end{document}
