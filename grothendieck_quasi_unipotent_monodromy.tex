\documentclass[deligne.tex]{subfiles}

\begin{document}
Some notes about Grothendieck's theorem on quasi-unipotent monodromy.
We study the arithmetic proof. It uses a proposition proved by
Grothendieck in the appendix of Serre and Tate's article
\textit{Good Reduction of Abelian Varieties}.

We may assume that $K$ is complete since, following Serre,
\textit{Corps Locaux}, II$\S$3 Cor.~4, completing $K$ leaves the 
decomposition unchanged. Now, we may assume that any matrix in the image of
$\rho$ has coefficients in $\ZZ_\ell$ and is congruent to $1\mod l^2$ as
these are both open conditions, $\rho$ is continuous, and we are free to pass 
to an open subgroup of $I(\overline v)$ by making a finite extension of $K$.

Note also that $\im\rho$ is a pro-$l$ group since, while $\GL(n,\ZZ_l)$ is
not a pro-$l$-group, its first congruence subgroup of matrices congruent to
$1\mod l$ is a pro-$l$-group (c.f., e.g., \S5.1 of \textit{Analytic Pro-$p$
Groups} by Dixon, du Sautoy, Mann \& Segal).
We see therefore that the prime-to-$l$ part of the order of $\GL(n,\ZZ_l)$
is finite. As the image of a pro-$p$ group under a continuous homomorphism 
is pro-$p$, the continuous image of a pro-$p$ group in $\GL(n,\ZZ_l)$ is 
finite. As $\im\rho$ is by construction pro-$l$, the image of a pro-$p$ group 
in $\im\rho$ is $\{1\}$.

Now, if $L$ is a finite extension of $K_l$, we wish to show that the 
polynomial $f(T)=T^l-a$ splits in $L$ for any $a\in L$. If it does not, then 
as $f$ is separable and $L$ contains all $l^\text{th}$ roots 
of unity, $L(\sqrt[l]a)$ is the splitting field of $f$ and is Galois.
The automorphism of $L(\sqrt[l]a)/L$ sending 
$\sqrt[l]a\mapsto \zeta_l\sqrt[l]a$, where $\zeta_l$ is a primitive
$l^{\text{th}}$ root of unity, acts transitively on the roots 
of $f$, hence $f$ is irreducible.
But $K_l$ is the $l$-part of the maximal
tamely ramified extension of $K_{nr}$, hence $l$ cannot divide $[L:K_l]$.
(Recall that $K_t$ is the maximal tamely ramified extension of $K$, and
we have
\begin{ceqn}\begin{equation*}
	\Gal(K_t/K_{nr})\simeq\prod_{q\ne p}\ZZ_q
	\qquad
	\Gal(K_t/K_l)\simeq\prod_{q\ne p,l}\ZZ_q
	\qquad
	\Gal(K_s/K_t)\text{ a pro-$p$ group}
\end{equation*}\end{ceqn}
as $q$ runs over primes, so $\Gal(K_s/K_l)$ is an extension
of a group isomorphic to $\prod_{q\ne p,l}\ZZ_q$ by a pro-$p$ group, and therefore has no finite quotient of order divisible by $l$.)
This allows one to conclude that $l$ does not divide the order of
$\Gal(K_s/K_l)$.
The order of $\im\rho$ is a power of $l$ as it is a pro-$l$ group.

An alternative way to see that $l$ does not divide the order of
$\Gal(K_s/K_l)$ that is more faithful to the original proof proceeds by
showing directly that for a finite extension $L/K_l$, every element of $L$ is
an $l^{\text{th}}$ power. To do this, let $L=K_l[t]/a(t)$ for $a(t)$ 
an irreducible separable polynomial $a(t)=a_nx^n+a_{n-1}x^{n-1}+\ldots+a_0$, 
and suppose $t$ is not an $l^\text{th}$ power in $L$. This implies that the 
polynomial $a_l(t)=a_nx^{ln}+a_{n-1}x^{l(n-1)}+\ldots+a_0$ is irreducible and
separable over $K_l$. 
Let $K'/K$ be a finite Galois extension containing all $l^{\text{th}}$ 
roots of unity and the $a_i$, and contained in $K_l$.
The extension $K'[t]/a_l(t)$ is finite and
separable and is contained in a finite Galois extension $K''$ of $K'$ with
$l$ dividing $[K'':K']$, so $l$
divides the ramification index or the residual degree. If the latter,
making a finite unramified extension of $K'$ produces a contradition on
the irreducibility of $a_l(t)$ over $K_l$. If the former,
replacing $K''$ by $K'[\pi^{1/l}]\subset K_l$ similarly yields a contradiction.
Now to see that every finite Galois extension $L$ of $K_l$ cannot have $l$ 
dividing its degree, note that $\Gal(L/K_l)$ contains a cyclic subgroup $H$
of order $l$, and we claim $L=L^H(a^{1/l})$ for some element $a\in L^H$.
Let $\sigma$ generate $H$, and let $b\in L-L^H$. Then the element
\begin{equation*} c=\sum_{m=1}^l\zeta_l^m\sigma^m(b)\end{equation*}
satisfies $c=\zeta_l\sigma(c)$, where $\zeta_l$ is a primitive $l^\text{th}$
root of unity. So
\begin{equation*}c^l=\prod_{m=1}^l\sigma^m(c)\in L^H,\end{equation*}
and letting $a=c^l$ we find that $L^H(a^{1/l})$ is a nontrivial
subextension of $L^H$, hence must actually coincide with $L$.

\addtocontents{toc}{\protect\setcounter{tocdepth}{-1}}
\begin{thebibliography}{ST}
\bibitem[ST]{ST} Jean-Pierre Serre and John Tate. Good reduction of abelian varieties. \textit{Ann. of Math.} (2), 88:492–517, 1968.	
\end{thebibliography}
\addtocontents{toc}{\protect\setcounter{tocdepth}{1}}

\end{document}
