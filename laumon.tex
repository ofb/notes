\documentclass[deligne.tex]{subfiles}

\begin{document}
As a warm-up, let's recall some Galois theory from SGA 1 in connection with
the beginning of \emph{Sommes trig.} All references in this paragraph are to
SGA 1 Exposé V. The Lang isogeny for $\GG_{a,\FF_q}$ is written
\begin{equation*}
	0\lra\FF_q\lra\GG_a\xra{x^q-x}\GG_a\lra0
\end{equation*}
and is a revêtement étale called the Artin-Schreier revêtement.
The sheaf of sections defines an $\FF_q$-torsor.
The Galois group of the Artin-Schreier revêtement is therefore $\FF_q$, as
a connected torsor under a finite group $G$ has Galois (= automorphism) group
$G$. Just as in the theory of fields, the choice of words `Galois group' in
place of `automorphism group' is reserved for Galois objects in the Galois
category. Remark 5.11 characterizes the Galois objects in a Galois category
$\mathscr C$ equipped with fiber functor $F$ as the connected torsors $X$
under a finite group $G$ (torsor = principal homogeneous space).
The implications go as follows: an object $X$ is galoisian if it is 
connected, not isomorphic to $\emptyset_{\mathscr C}$, the initial object of 
$\mathscr C$ ($\Leftrightarrow F(X)\ne\emptyset$), and $\Aut X$ is transitive 
($\Leftrightarrow$ simply transitive) (N$^\circ$\,4, f) \& 5.4).
$X$ is a torsor under the group opposite $\Aut X$ iff $F(X)$ is a torsor
under the group opposite $\Aut X$; i.e. $\Aut X$ acts simply transitively.
Therefore $X$ is a connected torsor under the group opposite $\Aut X$.
On the other hand, suppose $X$ is a torsor under $G$, i.e. $G$ acts on $X$ 
on the right and on $F(X)$ simply transitively, yielding a natural injection
from $G$ into the group opposite $\Aut X$.
As $G$ acts simply transitively, $\Aut X$ acts transitively.
If moreover $X$ is connected, then N$^\circ$\,4 f) gives that
$\Aut X$ acts transitively iff it acts simply transitively, showing that the
injection above is actually an isomorphism between $G$ and the group opposite
$\Aut X$. This is justification for the fact
`In a Galois category, a connected torsor under a finite group $G$ is a
Galois object with Galois group $G$.'

\subsection*{1.1.1.5} En vue de l'additivité de $t$ (0.9), l'énoncé
$t_{Rf_*\overline\QQ_{\ell,X}}(\infty)=1-q$
résulte de la pureté relative pour
\begin{tikzcd}[column sep=small]
	\A^1\arrow[r,hook,"j"]&\PP^1&\infty\arrow[l,hook']
\end{tikzcd}
(Arcata V 3.4) en ce que
\begin{equation*}
	\begin{cases}
		j_*\QQ_\ell=\QQ_\ell \\
		R^1j_*\QQ_\ell=\QQ_\ell(-1)_\infty \\
		R^qj_*\QQ_\ell=0 \text{ pour } q\geq2.
	\end{cases}
\end{equation*}

\subsection*{1.1.3} The action of Frobenius on the fiber is clearer
in \emph{Sommes trig.} 1.5, 1.6 since Deligne's notation for Frobenius is 
clear and consistent.

The rigidification (1.1.3.1) is a consequence of the fact that there is a
distinguished element of $L^{-1}(1)$, namely 1.

The trivialization (1.1.3.2) depends on the construction of $\mathscr L_\chi$
from a torsor defined by an extension. In the setting of \emph{Sommes trig.}
1.3, the $A$-torsor $T$ is defined as the sheaf of local sections of an 
extension $\pi$. As $H$ represents $m^*T$ in the commutative diagram with
exact rows ($\ulcorner$ = cartesian and $m$ denotes the group law)
\begin{equation*}\begin{tikzcd}
	0\arrow[r]&A\times A\arrow[r]\arrow[d,"m"]&G'\times G'\arrow[r,"\pi\times\pi"]\arrow[d]&G\times G\arrow[r]\arrow[d,equals]&0 \\
	0\arrow[r]&A\arrow[r]\arrow[d,equals]&H\arrow[r]\arrow[d]&G\times G\arrow[r]\arrow[d,"m"]&0 \\
	0\arrow[r]&A\arrow[r]&G'\arrow[r,"\pi"]&G\arrow[r]\arrow[ul,phantom,"\ulcorner",near end]&0,
\end{tikzcd}\end{equation*}
in the language of torsors there is a canonical isomorphism
\begin{equation*}
	m(T\times T)=m^*T.
\end{equation*}
The structure of group on $G$ provides commutative diagrams
\begin{ceqn}\begin{equation*}\begin{tikzcd}
	G&&G \\
	G\times *\arrow[u,"\sim"]\arrow[r,"\id\times1"]\arrow[dr,"\sim"']&G\times G\arrow[ul,"\pr_1"']\arrow[ur,"\pr_2"]\arrow[d,"m"]&*\times G\arrow[u,"\sim"']\arrow[l,"1\times\id"']\arrow[dl,"\sim"] \\
	&G
\end{tikzcd}\qquad
\begin{tikzcd}[column sep=small]
	\pi_1(G,1)&&\pi_1(G,1) \\
	\pi_1(G,1)\arrow[u,"\sim"]\arrow[r]\arrow[dr,"\sim"']&\pi_1(G\times G,(1,1))\arrow[ul,"\pr_1"']\arrow[ur,"\pr_2"]\arrow[d,"m"]&\pi_1(G,1)\arrow[u,"\sim"']\arrow[l]\arrow[dl,"\sim"] \\
	&\pi_1(G,1)
\end{tikzcd}
\end{equation*}\end{ceqn}
The diagram
\begin{equation*}\begin{tikzcd}[column sep=large,row sep=0]
	\pi_1(G\times G,(1,1))\arrow[r,"\pr_1\times\pr_2"]&\pi_1(G,1)\times\pi_1(G,1)\arrow[r,"m"]&\pi_1(G,1) \\ &\scalebox{2}{\rotatebox[origin=c]{-180}{$\circlearrowright$}} &\scalebox{2}{\rotatebox[origin=c]{-180}{$\circlearrowright$}} \\
	&A\times A&A
\end{tikzcd}\end{equation*}
gives the action of $\pi_1(G\times G,(1,1)$ on the torsors $T\times T$ and
$m(T\times T)=m^*T$. Pushing by $\chi^{-1}$ gives a representation of
$\pi_1(G\times G,(1,1))$ on $\overline\QQ_\ell$ which is induced by
\begin{equation*}
	(g_1,g_2)\mapsto\chi^{-1}(g_1)\otimes\chi^{-1}(g_2),
\end{equation*}
demonstrating explicitly an isomorphism
\begin{equation*}
	m^*\mathscr L_\chi\simeq\pr_1^*\mathscr L_\chi\otimes\pr_2^*\mathscr L
\end{equation*}
and hence a trivialization of $\mathscr D_2(\mathscr L_\chi)$ compatible
with the identification of the fibers of $\mathscr D_2(\mathscr L_\chi)$
and $\overline\QQ_\ell$ at $(1,1)$.

(1.1.3.3) = \emph{Sommes trig.} 1.7.7
(see \hyperref[sommes:1.6]{note to \emph{Sommes trig.} 1.6})

(1.1.3.4) = \emph{Sommes trig.} 2.7*.

The remark (1.1.3.7) can be summed up by the morphism of torsors induced by
the commutative diagram below, the projective limit of which defines
$\varprojlim_{I(k)}\mathbf{\mu}_N(k)$:
\begin{equation*}\begin{tikzcd}
	1\arrow[r]&\mu_{NM}(k)\arrow[r]\arrow[d,"{[M]}"]&\GG_{m,k}\arrow[r,"{[NM]}"]\arrow[d,"{[M]}"]&\GG_{m,k}\arrow[r]\arrow[d,equals]&1 \\
	1\arrow[r]&\mu_{N}(k)\arrow[r]&\GG_{m,k}\arrow[r,"{[N]}"]&\GG_{m,k}\arrow[r]&1.
\end{tikzcd}\end{equation*}
As $\FF_q^\times\simeq\ZZ/(q-1)$, $I(\FF_q)=\{n\in\NN_{>0}:n|q-1\}$.

\subsection*{1.2.1}
The pairing $\langle\enskip,\enskip\rangle:E\times_SE'\ra\GG_{a,k}$ should
rather land in $\GG_{a,S}$. See \emph{Sommes trig.} 1.7 c) to make sense of
what this does to $\mathscr L_{\psi}$.

To be maximally pedantic, if $q=p^n$, with the 
notation of (1.1.3.3) we should write $\psi_n$ where $\psi_q$ appears in
the definition of $\hat t$.

\subsection*{1.2.2} Locally $S=\Spec A$,
$E=\Spec A[t_1,\ldots,t_r]$,
$E\times_S E=\Spec A[t_1,\ldots,t_r,t_1',\ldots,t_r']$, and the addition
$E\times_S E\ra E$ is given by $t_i\mapsto t_i+t_i'$ while $[-1]$ is given
by $t_i\mapsto-t_i$. More to the point, writing
$\mathscr E\simeq\textbf{Spec}\operatorname{Sym}(\mathscr E)$ with
$\operatorname{Sym}(\mathscr E)$ the symmetric algebra on a locally free
sheaf $\mathscr E$, then addition of sections gives addition on $E$, and
vice versa, as the sheaf of local sections of $E$ coincides with
$\mathscr E^\vee$.

(1.2.2.1)
Identifying $E$ with $E''$
via $e\mapsto\langle e,\enskip\rangle=-a(e)$, the diagram becomes the base
change by $\pi':E'\ra S$ of addition on $E''$,
\begin{equation*}\begin{tikzcd}
	&E''\times_SE''\arrow[rr,"{(e_1'',e_2'')\mapsto e_1''+e_2''}"]\arrow[dl,"\pr"]\arrow[dr,"\pr"]&&E'' \\
	E''\arrow[dr]&&E''\arrow[dl] \\
	&S
\end{tikzcd}\end{equation*}
and the isomorphism
\begin{equation*}
	\pr_{12}^*\mathscr L(\langle\enskip,\enskip\rangle)\otimes
	\pr_{23}^*\mathscr L(\langle\enskip,\enskip\rangle)
	=\alpha^*\mathscr L(\langle\enskip,\enskip\rangle)
\end{equation*}
becomes (1.1.3.2) on the nose.
\begin{align*}
	\mathscr F'\circ\mathscr F(K)
	&=R\pr_!''(\pr'^*(R\pr_!'(\pr^*(K)\otimes\mathscr L(\langle\enskip,\enskip\rangle)))\otimes\mathscr L(\langle\enskip,\enskip\rangle))[2r] \\
	&\simeq R(\pr''\circ\pr_{23})_!((\pr\circ\pr_{13})^*(K)\otimes\pr_{13}^*\mathscr L(\langle\enskip,\enskip\rangle)\otimes\pr_{23}^*\mathscr L(\langle\enskip,\enskip\rangle))[2r] \\
	&\simeq R(\pr''\circ\pr_{23})_!((\pr\circ\pr_{13})^*(K)\otimes\alpha^* \mathscr L(\langle\enskip,\enskip\rangle))[2r] \\
	&\simeq R(\pr''\circ\pr_{13})_!((\pr\circ\pr_{13})^*(K)\otimes\alpha^* \mathscr L(\langle\enskip,\enskip\rangle))[2r] \\
	&\simeq R\pr''_!(\pr^*K\otimes R\pr_{13!}\alpha^* \mathscr L(\langle\enskip,\enskip\rangle))[2r] \\
	&\simeq R\pr''_!(\pr^*K\otimes \beta^*R\pr''_! \mathscr L(\langle\enskip,\enskip\rangle))[2r].
\end{align*}
In (1.2.2.2), the first equation of the proof is missing a $\pi'^*$; it
should read
\begin{equation*}
	\mathscr F(\pi^*L[r])=\pi'^*L\otimes R\pr_!'\mathscr L(\langle\enskip,\enskip\rangle)[2r].
\end{equation*}
The rest of the proof of (1.2.2.2) is straightforward. It allows us to write
\begin{align*}
	\mathscr F'\circ\mathscr F(K)
	&\simeq R\pr''_!(\pr^*K\otimes \beta^*\mathscr F'(\overline\QQ_{\ell,E'}[r])) \\
	&\simeq R\pr''_!(\pr^*K\otimes \beta^*\sigma''_*\overline\QQ_{\ell,S}(-r)) \\
	&\simeq a_*K(-r),
\end{align*}
as $\beta^*\sigma_*''$ is supported precisely on the locus where $e''=a(e)$,
and after base change along $E\times_{a,\id}E''\ra E\times_SE''$, which
commutes with $\pr_!''$, we find $R\pr_!''\pr^*K=a_*$.
\begin{equation*}\begin{tikzcd}
	&E\times_{a,\id}E''\arrow[dl]\arrow[dr]\arrow[rr]&&E\times_SE''\arrow[dl]\arrow[dr] \\
	E\arrow[dr,"a"']&&E''\arrow[dl,"\id"]\arrow[dr,"\pi''"']&&E\arrow[dl,"\pi"] \\
	&E''&&S
\end{tikzcd}\end{equation*}

\begin{remark}\label{laumon:1.2.2.1rmk}
	Identifying $E\simeq (E^\vee)^\vee$ in the canonical way,
	the involutivity can be written
	\begin{equation*}
		\mathscr F'\circ\mathscr F(K)\simeq [-1]_*K(-r),
	\end{equation*}
	and the inverse image under $[-1]$ of this identity is
	$\mathscr F'_{\psi^{-1}}\circ\mathscr F(K)\simeq K(-r)$, since
	$[-1]^*\mathscr F_{\psi}'=\mathscr F'_{\psi^{-1}}$ in light of the fact 
	that $[-1]^*\psi(\langle e'',e'\rangle)=\psi(\langle-e,e'\rangle)=\psi^{-1}(\langle e,e'\rangle)$.
\end{remark}



(1.2.2.4) The adjunction upon which the equality rests is
\begin{equation*}
	\langle f(s),\varphi \rangle_2=\langle s,f'^*\varphi:=\varphi\circ f\rangle_1,
\end{equation*}
as pullback by $f'$ on a section $\varphi$ of $E_2'$ gives, by definition, 
the element of $E_1'$ which, to a section $s$ of $E_1$, applies $f$ then $\varphi$. We then write
\begin{align*}
	\mathscr F_2(Rf_!K_1)
	&=R\pr_{2!}'(\pr_2^*(Rf_!K_1)\otimes\mathscr L(\langle\enskip,\enskip\rangle))[r_2] \\
	&\simeq R\pr_{2!}'(R(f\times1)_!(\pr_1^*K_1)\otimes\mathscr L(\langle\enskip,\enskip\rangle)_2)[r_2] \\
	&\simeq R(\pr_2'\circ (f\times1))_!(\pr_1^*K_1\otimes(f\times1)^*\mathscr L(\langle\enskip,\enskip\rangle)_2)[r_2] \\
	&\simeq R(\pr_2'\circ (f\times1))_!((\pr_1\circ(1\times f'))^*K_1\otimes(1\times f')^*\mathscr L(\langle\enskip,\enskip\rangle)_1)[r_2] \\
	&\simeq R\pr_{2!}'(1\times f')^*R(f\times1)_!(\pr_1^*K_1\otimes\mathscr L(\langle\enskip,\enskip\rangle)_1)[r_2] \\
	&\simeq f'^*R\pr_{1!}'(\pr_1^*K_1\otimes\mathscr L(\langle\enskip,\enskip\rangle)_1)[r_2] \\
	&\simeq f'^*\mathscr F_1(K_1)[r_2-r_1].
\end{align*}


(1.2.2.5)
If $(\pi')'$ is the morphism of dual 
bundles $S\ra E''$ induced by $\pi'$, then $(\pi')'^*a_*\simeq\sigma^*$.
To make sense of the morphism $(\pi')':S\ra E''$, you have to unpack what it
means to consider $S$ as a vector bundle of rank 0 over $S$. 
The only section of $S$ is the zero section, as at all points of $S$, as a
rank 0 vector bundle returns the vector space $\{0\}$ at every point of $S$.
In other words, we dispose of a tautological isomorphim
$S\xra\sim S\times_k*=S\times\{0\}$.
Thinking about $\pi':E'\ra S$ as sending $(s,e')\mapsto (s,0)$,
the map $(\pi')':S\ra E''$ turns the 0 section of $S$ into a
section of $E''$, i.e. the one which to a section $s'$ of $E'$ returns the 0 
section of $S$. Therefore $(\pi')'$ is the embedding by zero section
$0:S\hookrightarrow E''$; as $-0=0$, the zero section of $E''$
corresponds under the isomorphism $a$ to the zero section of $E$, which
sees $(\pi')'^*a_*\simeq\sigma^*$.

(1.2.2.7) For the isomorphism
\begin{equation*}
	\mathscr F(K_1\boxtimes_SK_2)\simeq\mathscr F(K_1)\boxtimes_S\mathscr F(K_2),
\end{equation*}
perhaps it helps to write the commutative diagram
\begin{equation*}\begin{tikzcd}
	E&E\times_SE\arrow[l]\arrow[r]&E \\
	E\times_SE'\arrow[u]\arrow[d]&(E\times_SE)\times_S(E'\times_SE')\arrow[l,"\pr_1"']\arrow[r,"\pr_2"]\arrow[u]\arrow[d]&E\times_SE\arrow[u]\arrow[d] \\
	E'&E'\times_SE'\arrow[l]\arrow[r]&E'.
\end{tikzcd}\end{equation*}
If one imagines this diagram drawn on the $xy$-plane in 3D with the center
at $(0,0,0)$, places $S$ at $(0,0,1)$, and connects every node of this
diagram to this $S$, every row and column of the $3\times3$ above will then
form a cartesian square with this $S$.
The equality of Fourier transform then follows from the Künneth formula
in light of the isomorphism
\begin{equation*}
	\pr_1^*\mathscr L(\langle\enskip,\enskip\rangle_{E_1})
	\otimes\pr_2^*\mathscr L(\langle\enskip,\enskip\rangle_{E_2})
	\simeq\mathscr L(\langle\enskip,\enskip\rangle_{E_1\oplus E_2}),
\end{equation*}
which is an obvious consequence of the fact that given vector bundles
$E_1,E_2$,
\begin{equation*}
	\langle\enskip,\enskip\rangle_{E_1\oplus E_2}
	=\langle\enskip,\enskip\rangle_{E_1}+\langle\enskip,\enskip\rangle_{E_2}.
\end{equation*}
The character $\psi$ carries this additive identity to a multiplicative 
one, hence the $\otimes$.

To go from here to the stated isomorphism
\begin{equation*}
	\mathscr F(K_1*K_2)\simeq\mathscr F(K_1)\otimes\mathscr F(K_2)[-r],
\end{equation*}
after applying (1.2.2.4) in the given way, one is faced with the problem of
justifying
\begin{equation*}
	s'^*(\mathscr F(K_1)\boxtimes_S\mathscr F(K_2))
	\simeq\mathscr F(K_1)\otimes\mathscr F(K_2).
\end{equation*}
It is useful to reason by adjunction, writing the following adjoint pair of
diagrams.
\begin{equation*}\begin{tikzcd}[column sep=small]
	&E'\arrow[d,"{s'}"]\arrow[ddl,bend right]\arrow[ddr,bend left] \\
	&E'\times_SE'\arrow[dl,"\pr_1"]\arrow[dr,"\pr_2"'] \\
	E'&&E'
\end{tikzcd}\qquad\qquad\begin{tikzcd}[column sep=small]
	&E \\
	&E\times_SE\arrow[u,"s"'] \\
	E\times_S\{0\}\arrow[uur,bend left,"\id"]\arrow[ur,"\sigma_1"']&&\{0\}\times_SE\arrow[uul,bend right,"\id"']\arrow[ul,"\sigma_2"]
\end{tikzcd}\end{equation*}
The point is that given $e,e'$ sections of $E,E'$ respectively,
\begin{equation*}
	\langle e,(\pr_i\circ s')(e')\rangle
	=\langle \sigma_i(e),s'(e') \rangle
	=\langle (s\circ \sigma_i)(e), e'\rangle
	=\langle e,e'\rangle.
\end{equation*}
Ergo,
\begin{equation*}
	s'^*(\mathscr F(K_1)\boxtimes_S\mathscr F(K_2))
	\simeq(\pr_1\circ s')^*\mathscr F(K_1)\otimes_S(\pr_2\circ s')^*\mathscr F(K_2))
	\simeq\mathscr F(K_1)\otimes\mathscr F(K_2).
\end{equation*}

\subsection*{1.2.3} Quelques détails supplémentaires pour les exemples donnés suivent.

(1.2.3.1) With $\sigma'_F:S\ra F'$,
\begin{equation*}
	\mathscr F(i_*\overline\QQ_{\ell,F}[s])
	\simeq i'^*\mathscr F(\overline\QQ_{\ell,F}[s])[r-s]
	\simeq i'^*\sigma_{F*}'(\overline\QQ_{\ell,S}(-s))[r-s];
\end{equation*}
now use proper base change for the cartesian square
\begin{equation*}\begin{tikzcd}
	E'\arrow[d,"{i'}"]&F^\perp\arrow[l,"i^\perp"']\arrow[d,"\pi'"]\arrow[dl,phantom,"\urcorner",very near start] \\
	F'&S\arrow[l,"\sigma_F'"].
\end{tikzcd}\end{equation*}

(1.2.3.2) A few useless words: $e_*\overline\QQ_{\ell,S}$ is the sheaf on
$E$ that is supported precisely on the section $e$ of $E$. Restricted to
the closed subscheme which is the image of $e$, $e_*\overline\QQ_{\ell,S}$
is constant with value $\overline\QQ_{\ell}$. Therefore
$(e_*\overline\QQ_{\ell,S})\boxtimes_SK$ has support contained in
$e\times_S E$, and restricted to this closed subscheme, $s$ is an
isomorphism with inverse $E\xra\sim(-e)\times_SE\xra sE\xra\sim e\times_SE$.
En effet, $\tau_e$ factors as $E\xra\sim e\times_SE\xra sE$.

After applying (1.2.2.7), one uses (1.2.3.1) to compute
\begin{equation*}
	\mathscr F(e_*\overline\QQ_{\ell,S})
	\simeq e^\perp_*\overline\QQ_{\ell,E'}[r]
	\simeq\mathscr L(\langle e,\enskip\rangle)[r].
\end{equation*}

(1.2.3.3)
The isomorphism $\alpha$ gives rise to a nondegenerate bilinear form
$B:E\times_S E\ra\GG_{a,S}$ via $B(e_1,e_2)=\langle e_1,\alpha(e_2)\rangle$,
and $\alpha$ is symmetric if $B$ is; equivalently,
\begin{equation*}
	\langle \alpha^{-1}(e'),\alpha(e)\rangle=\langle e,e'\rangle.
\end{equation*}
This allows the easy verification of the identity
\begin{equation*}\tag{$\dagger$}
	q(e)+\langle e,2e'\rangle=q(e+\alpha^{-1}(e'))-q'(e').
\end{equation*}
The cartesian square
\begin{equation*}\begin{tikzcd}
	E\times E'\arrow[r,"{\id\times[2]}"]\arrow[d]&E\times E'\arrow[d] \\
	E'\arrow[r,"{[2]}"]&E'
\end{tikzcd}\end{equation*}
and proper base change gives
\begin{equation*}
	[2]^*\mathscr F(\mathscr L(q))\simeq
	R\pr_!'(\pr^*\mathscr L(q)\otimes\mathscr L(\langle\enskip,2\enskip\rangle)),
\end{equation*}
where here we write $\mathscr L(\langle\enskip,2\enskip\rangle$ for
$(\id\times[2])^*\mathscr L(\langle\enskip,\enskip\rangle)$.
Let $f$ denote the composition of maps in the diagram
\begin{equation*}\begin{tikzcd}
	E\times E'\arrow[r,"\id\times\alpha^{-1}"]&E\times_SE\arrow[r,"s"]&E.
\end{tikzcd}\end{equation*}
We have the following correspondences between functions and sheaves on
$E\times_SE'$.
\begin{align*}
	&q(e+\alpha^{-1}(e'))\longleftrightarrow f^*\mathscr L(q) \\
	&-q'(e')\longleftrightarrow\mathscr L(-q') \\
	&q(e)\longleftrightarrow\pr^*\mathscr L(q) \\
	&\langle e,2e'\rangle\longleftrightarrow(\id\times[2])^*\mathscr L(\langle\enskip,\enskip\rangle)=:\mathscr L(\langle\enskip,2\enskip\rangle).
\end{align*}
Disposing of this dictionary, the identity $(\dagger)$ gives an isomorphism
of sheaves on $E\times_SE'$
\begin{equation*}
	\pr^*\mathscr L(q)\otimes\mathscr L(\langle\enskip,2\enskip\rangle)
	\simeq\pr'^*\mathscr L(-q')\otimes f^*\mathscr L(q)).
\end{equation*}
Therefore the stated isomorphism rests on showing that
\begin{equation*}
	R\pr_!'(\pr'^*\mathscr L(-q)\otimes f^*\mathscr L(q))
	\simeq\mathscr L(-q')\otimes\pi'^*R\pi_!\mathscr L(q).
\end{equation*}
The following diagram is commutative with cartesian squares.
\begin{equation*}\begin{tikzcd}
	E\times_SE'\arrow[r,"{\id\times\alpha^{-1}}"]\arrow[d,"\pr'"]\arrow[rr,"f",bend left]\arrow[dr,phantom,"\ulcorner",near start]
	&E\times_SE\arrow[r,"s"]\arrow[d,"\pr_2"]\arrow[dr,phantom,"\ulcorner",near start]
	&E\arrow[d,"\pi"] \\
	E'\arrow[r,"\alpha^{-1}"]\arrow[rr,"\pi'",bend right]
	&E\arrow[r,"\pi"]
	&S
\end{tikzcd}\end{equation*}
To see that the right square is cartesian, note that the isomorphism
\begin{align*}
	\beta:E\times_SE&\lra E\times_SE \\
	(e_1,e_2)&\longmapsto(e_1-e_2,e_2)
\end{align*}
induces an isomorphism of cartesian squares
\begin{equation*}\begin{tikzcd}
	E\times_SE\arrow[r,"\pr_1"]\arrow[d,"\beta"]\arrow[dd,"\pr_2"',bend right=50]
	&E\arrow[d,"\id"']\arrow[dd,"\pi",bend left=50] \\
	E\times_SE\arrow[r,"s"]\arrow[d,"\pr_2"]&E\arrow[d,"\pi"'] \\
	E\arrow[r,"\pi"]&S.
\end{tikzcd}\end{equation*}
By the projection formula and proper base change, we conclude
\begin{equation*}
	R\pr_!'(\pr'^*\mathscr L(-q)\otimes f^*\mathscr L(q))
	\simeq\mathscr L(-q')\otimes R\pr_!'f^*\mathscr L(q)
	\simeq\mathscr L(-q')\pi'^*R\pi_!\mathscr L(q).
\end{equation*}
	
(1.2.3.4) Let $p:G\ra S$ denote the structure morphism and its various base
extensions along $\pi,\pi'$.
Let $f=(\pr_G,m)^{-1}$, with transpose $f'=(\pr_G,m')$. The diagram
\begin{equation*}\begin{tikzcd}
	G\times_SE'\arrow[r,"f'"]\arrow[rr,"m'",bend right]&G\times_SE'\arrow[r,"p"]&E'
\end{tikzcd}\end{equation*}
commutes, so that
\begin{equation*}
	m'^*\mathscr F(K)=f'^*p^*\mathscr F(K)
	\simeq f'^*\mathscr F(p^*K)
	\simeq\mathscr F(f_!p^*K).
\end{equation*}
where in the first isomorphism we have used (1.2.2.9) and in the second
(1.2.2.4). The commutative diagram
\begin{equation*}\begin{tikzcd}
	G\times_SE\arrow[rrr,"p",bend left]\arrow[rr,"\id",bend right]\arrow[r,"f"]
	&G\times_SE\arrow[rr,"m",bend right,crossing over]\arrow[r,"{(\pr_G,m)}"]
	&G\times_SE\arrow[r,"p"]&E
\end{tikzcd}\end{equation*}
together with the isomorphism $L\xra\sim f_*f^*L=f_!f^*L$ for all $L$ in 
$D_c^b(G\times_SE,\overline\QQ_\ell)$ and the hypothesis
$m^*K\simeq M\boxtimes_SL$ let us write
\begin{equation*}
	\mathscr F(f_!p^*K)\simeq\mathscr F(m^*K)\simeq\mathscr F(M\boxtimes_SL).
\end{equation*}
Letting $\pr,\pr'$ denote
$E\xleftarrow{\pr}E\times_S E'\xra{\pr'}E'$
and their base extensions by $p$, the commutative diagrams below have
cartesian diamonds marked.
\begin{ceqn}
\begin{equation*}
\begin{tikzcd}[column sep=small]
	&&[-30pt]G\times_SE\times_SE'\arrow[dl,"\pr"]\arrow[dr,"\pr'"']\arrow[dd,phantom,"{\rotatebox{45}{\text{$\urcorner$}}}",near start]&[-30pt] \\
	&G\times_SE\arrow[dl,"p"']\arrow[dr,"\pi"']
	&&G\times_SE'\arrow[dl,"\pi'"]\arrow[dr,"p"] \\
	E&&G&&E'
\end{tikzcd}
\qquad\qquad\begin{tikzcd}[row sep=small]
	&[-30pt]G\times_SE\times_SE'\arrow[dl,"\pr'"]\arrow[dr,"p"']\arrow[dd,"\pr'" description]
	\arrow[dddr,phantom,"\ulcorner",near start]
	&[-30pt] \\
	G\times_SE'\arrow[dd,equals]
	&&E\times_SE'\arrow[dd,"\pr'"] \\
	&G\times_SE'\arrow[dl,"\id"]\arrow[dr,"p"'] 
	\arrow[dd,phantom,"{\rotatebox{45}{\text{$\urcorner$}}}",near start] \\
	G\times_SE'\arrow[dr]&&E'\arrow[dl] \\
	&E'
\end{tikzcd}
\end{equation*}
\begin{equation*}\begin{tikzcd}[column sep=small]
	M\arrow[d,dash]&[10pt]&[-30pt]G\times_SE\times_SE'\arrow[dl,"\pr'"]\arrow[dr,"p"']
	\arrow[dd,phantom,"{\rotatebox{45}{\text{$\urcorner$}}}",near start]
	&[-30pt]&[-20pt]\mathscr L(\langle\enskip,\enskip\rangle)\arrow[dl,dash] \\
	G&G\times_SE'\arrow[l,"\pi'"]\arrow[dr]&&E\times_SE'\arrow[dr]\arrow[dl]
	&L\arrow[d,dash] \\
	&&E'&&E
\end{tikzcd}
\end{equation*}
\end{ceqn}
The projection formula and proper base change find
\begin{align*}
	\mathscr F(M\boxtimes_SL)
	&\simeq R\pr'_!(\pr^*(M\boxtimes_SL)\otimes\mathscr L(\langle\enskip,\enskip\rangle) \\
	&\simeq R\pr'_!(\pr^*\pi^*M\otimes\pr^*p^*L\otimes\mathscr L(\langle\enskip,\enskip\rangle)) \\
	&\simeq R\pr'_!(\pr'^*\pi'^*M\otimes p^*(\pr^*L\otimes\mathscr L(\langle\enskip,\enskip\rangle)) \\
	&\simeq \pi'^*M\otimes R\pr'_!(p^*(\pr^*L\otimes\mathscr L(\langle\enskip,\enskip\rangle))) \\
	&\simeq \pi'^*M\otimes p^*\mathscr F(L)
	=M\boxtimes_S\mathscr F(L).
\end{align*}

\subsection*{1.2.3.5} As
$E_1\times_S E'\simeq E_1\times_{S_1}E_1'\simeq E\times_S E_1'$, the diagram
below has cartesian diamonds.
\begin{ceqn}\begin{equation*}\begin{tikzcd}[column sep=tiny]
	&&S_1 \\
	&&E_1\times_{S_1}E_1'\arrow[dll,"\pr_1"]\arrow[drr,"\pr'_1"']\arrow[d,"f"]
	\arrow[ddl,phantom,"{\rotatebox{15}{\text{$\urcorner$}}}",near start]
	\arrow[ddr,phantom,"{\rotatebox{75}{\text{$\urcorner$}}}",near start]
	\arrow[u,phantom,"{\rotatebox{45}{\text{$\llcorner$}}}",near start] \\
	E_1\arrow[dr,"f_E"]\arrow[uurr]
	&&E\times_SE'\arrow[dl,"\pr"]\arrow[dr,"\pr'"'] 	\arrow[dd,phantom,"{\rotatebox{45}{\text{$\urcorner$}}}",near start]
	&&E_1'\arrow[dl,"f_{E'}"]\arrow[uull] \\
	&E\arrow[dr]&&E'\arrow[dl] \\ &&S
\end{tikzcd}\end{equation*}\end{ceqn}
In light of the fact that $\mathscr L(\langle\enskip,\enskip\rangle)$ on
$E_1\times_{S_1}E_1'$ coincides with the inverse image under $f$ of
$\mathscr L(\langle\enskip,\enskip\rangle)$ on $E\times_SE'$,
proper base change and the projection formula for $f$ give
\begin{align*}
	\mathscr F(Rf_{E!}K_1)
	&\simeq R\pr'_!(Rf_!\pr_1^*K_1\otimes\mathscr L(\langle\enskip,\enskip\rangle))
	\simeq R(\pr'\circ f)_!(\pr_1^*K_1\otimes f^*\mathscr L(\langle\enskip,\enskip\rangle) \\
	&\simeq R(f_{E'}\circ\pr_1')_!(\pr_1^*K_1\otimes \mathscr L(\langle\enskip,\enskip\rangle)
	\simeq Rf_{E'!}\mathscr F_1(K_1).
\end{align*}

\subsection*{1.3.1–1.3.2}
I would like to advocate for something of a shortcut through these sections.
\subsubsection*{t-exactness of $\mathscr F$}
Laumon deduces the t-exactness of $\mathscr F$ from the fact that the
`forget supports' map is an isomorphism. However, the following direct
argument (lifted from the appendix to the reprinted 
Astérisque \textbf{100}) is immediate.
As in \hyperref[laumon:1.2.2.1rmk]{the remark to the note to (1.2.2.1)},
$\mathscr F_{\psi^{-1}}\circ\mathscr F_{\psi}(K)=K(-r)$.
As $\mathscr F_{\psi}$ is the composition of exact functors $\pr^*[n]$,
$\mathscr L(\langle\enskip,\enskip\rangle)$ and the left t-exact functor
$\pr_!$ (BBD 4.1.2), it is left t-exact. But it is also left adjoint to
its inverse, and this inverse is also left t-exact since up to a Tate twist
it coincides with $\mathscr F_{\psi^{-1}}$. Therefore $\mathscr F_{\psi}$ is
also right t-exact (BBD 1.3.17 (iii)).

\subsubsection*{`Forget supports'}
The theorem (1.3.1.1) in Laumon's paper states that for all $K$ in
$D_c^b(E,\overline\QQ_\ell)$, the `forget supports' map
\begin{equation*}
	R\pr_!'(\pr^*K\otimes\mathscr L(\langle\enskip,\enskip\rangle))
	\ra R\pr_*'(\pr^*K\otimes\mathscr L(\langle\enskip,\enskip\rangle))
\end{equation*}
is an isomorphism. Laumon refers the reader to his paper with Katz, where
they give an involved geometric proof that ends up yielding more.
But Verdier gave the first proof of this isomorphism, and his proof is very
short and completely formal. It can be found in Katz's 1988 Séminaire
Bourbaki talk `Travaux de Laumon.'

As for the proof of (1.3.2.1), surely the stated isomorphism should read
\begin{equation*}
	R\underline\Hom(\mathscr F_{\psi}(K),\pi'^!L)
	\simeq R\pr_*'(\pr^!(R\underline\Hom(K,\pi^!L))\otimes\mathscr L_{\psi^{-1}}(\langle\enskip,\enskip\rangle)).
\end{equation*}

\subsection*{1.4.1}
\begin{equation*}\begin{tikzcd}
	A\times A'\arrow[r,"\pr'"]\arrow[d,"\alpha\times\alpha'"]
	&A'\arrow[d,"\alpha'"] \\
	D\times D'\arrow[r,"\overline\pr'"]&D'
\end{tikzcd}\end{equation*}
\begin{align*}
	\alpha_!'\mathscr F(K)
	&=R\overline\pr'_*(\alpha\times\alpha')_!(\pr^*K\otimes\mathscr L(xx'))[1]
\\	&=R\overline\pr'_*((\alpha\times\alpha')_!\pr^*K\otimes \overline{\mathscr L}(xx'))[1]
\\	&=R\overline\pr'_*(\overline\pr^*\alpha_!K\otimes \overline{\mathscr L}(xx'))[1].
\end{align*}

\subsection*{1.4.2}\label{laumon:1.4.2}
On a curve $X$ with $j:U\hookrightarrow X$
a dense smooth open with complement $F$ and $A$ a lisse sheaf on $U$,
$A[1]$ is perverse on $U$ and Verdier's formula \cite[2.2.4]{BBD} gives
\begin{equation*}
	j_{!*}A[1]=\tau^F_{<0}Rj_*A[1]=j_*A[1].
\end{equation*}
The simple perverse sheaves on a curve come as (a) $i_*$ of an irreducible
$\overline\QQ_\ell$-sheaf on a closed point or (b) from $j_{!*}[1]$ of an
irreducible sheaf on a dense open.
($T_1$) takes care of (a), but we must verify that a $K$ of type (b)
is either ($T_2$) or ($T_3$). Given a $K$ of type ($T_2$),
\begin{equation*}
	K|\overline A=(\pr_{\overline A})_*(\mathscr L(x.\overline s')\otimes\pr_{\overline s'}^*F')[1]
\end{equation*}
where $\overline s'$ denotes the geometric fiber of $s'$, a discrete set
on which $F'$ is a constant sheaf, so that $K|\overline A[-1]$ has all of
its constituents (as a lisse sheaf \cite[1.1.6]{weilii}) isomorphic to
$\mathscr L(x.a')$ for some $a'\in\overline k$.
\begin{lemma*}
	Suppose $K=j_{!*}F[1]=j_*F[1]$ for $F$ irreducible lisse on dense 
	$U\subset A$, and $K=\mathscr F'(K')$, where $K'=j'_*F'[1]$, $F'$
	irreducible lisse on dense $U'\subset A'$.
	Then $K$ is $(T_3)$.
\end{lemma*}
\begin{proof}
Let $\vep$ denote $\Spec\overline k\ra\Spec k$ and its various extensions.
The constituents of $\vep^*K'$ coincide with $j_*'[1]$ of the
constituents of the lisse sheaf $\vep^*F'$
(exact sequences of lisse sheaves give rise to distinguished triangles
concentrated in degree 0, apply triangulated functor $j_{!*}'[1]$ followed
by $^pH^0$);
in particular they are all of type (b).
As $\mathscr F$ induces an equivalence of perverse sheaves on
$A$ and $A'$, $\mathscr F$ is \emph{a fortiori} exact
(likewise for $\mathscr F'$), so that the constituents of $\vep^*K'$
coincide with those of $\mathscr F\circ\mathscr F'(\vep^*K')$; in
particular, they are still of type (b), and this implies that none of
the constituents of $\mathscr F'(\vep^*K')$ are of type ($T_2$).
As the formation of Fourier transform commutes with any base change
(1.2.2.9), this implies the same for the constituents of
$\vep^*\mathscr F'(K')=\vep^*K$; i.e. that none are isomorphic to
$\mathscr L(x.a')[1]$ for some $a'\in\overline k$.
In light of \cite[4.3.2]{BBD} or the lemma in
\hyperref[laumon:4.3.2]{the note to 4.3.2 below},
which says that if $F$ is a lisse sheaf on a normal connected curve, the
unit of adjunction $F\ra j_*j^*F$ is an isomorphism, this implies that $K$
is of type ($T_3$): if $\vep^*F$ had a constituent isomorphic to
$\mathscr L(x.a')|\overline U$, by the above $\vep^*K$ would have a
constituent isomorphic to
$\overline j_*(\mathscr L(x.a')|\overline U)[1]\simeq\mathscr L(x.a')[1]$,
where $\overline j=\vep^*j_*:\overline U\hookrightarrow\overline A$.
\end{proof}
\begin{corollary*}
	$\mathscr F$ exchanges $(T_3)$ and $(T_3')$.
\end{corollary*}
\begin{proof}
	Given a simple perverse sheaf $K'$ of type $(T_3')$, $\mathscr F'(K')$
	is simple and is not $(T_1)$ by (1.4.2.1 (i)), therefore must be of type
	(b); i.e. $\mathscr F'(K')$ satisfies the hypotheses of the lemma,
	so is ($T_3$).
\end{proof}
\begin{corollary*}[Dichotomy]
	An irreducible lisse $\overline\QQ_\ell$-sheaf $\mathcal F$ on a 
	dense open $U\hookrightarrow A$
	\begin{enumerate}[label=(\greek*)]
	\item has every constituent of $\vep^*\mathcal F$ isomorphic to $\mathscr L(x.a')|\overline U$ for various $a'\in\overline k$, or
	\item has no constituent of $\vep^*\mathcal F$ isomorphic to $\mathscr L(x.a')|\overline U$, for any $a'\in\overline k$.
	\end{enumerate}
\end{corollary*}
\begin{remark}
	In analogy with the Fourier transform on function spaces on $\RR$,
	\begin{align*}
		\text{constant functions}&\leftrightarrow (T_2) \\
		L^2(\RR)&\leftrightarrow (T_3) \\
		\text{point masses}&\leftrightarrow (T_1).
	\end{align*}
\end{remark}
\begin{remark}
	It is tempting to observe that if we were in the abelian category of
	constructible sheaves (perverse of pervervity $p=0$) shifted by 1, of
	course $j_!A\hookrightarrow j_*A$, although in the category of perverse
	sheaves for the middle perversity, $j_*A[1]$ is simple.
	In this category,
	\begin{equation*}
		\ker(j_!A[1]\ra j_*A[1])=i_*H^0i^*Rj_*A=i_*R^1i^!j_!A
		\qquad\text{placed in degree 0,}
	\end{equation*}
	where $i$ denotes the immersion of the complement \cite[4.1.2]{BBD}.
	The point is that although $i^!j_!A=0$, $R^1i^!j_!A$ vanishes iff
	$j_*A$ extends to a lisse sheaf on $X$. Assuming it doesn't, $j_!A[1]$
	is not simple, as it admits a nontrivial subobject
	$i_*\,^pH^0Ri^!j_!A[1]=:i_*\,^pi^!j_!A[1]$; this is nothing other than
	$i_*R^1i^!j_!A$ placed in degree 0, and coincides with the largest
	sub-object of $j_!A[1]$ in the essential image (via $^pi_*$) of the
	category of perverse sheaves on $S$ (for the middle perversity – this is
	simply the category of constructible sheaves on the finite set $S$)
	\cite[1.4.25]{BBD}.
\end{remark}

\subsection*{2.1.1}\label{laumon:2.1.1}
A first curiosity: are the conventions (0.3) \emph{en rigeur}?
Evidently $T$ isn't essentially of finite type over $k$, but perhaps it is
implicit that it is the spectrum of a ring ind-étale over a $k$-algebra of
finite type.
Does the inclusion $k\{\pi\}\subset R\subset k[[\pi]]$ require that
$k\subset R$ by assumption, or as in the complete case, does a
coefficient field exist automatically for $R$?
The answer is that we must assume that $k\subset R$, as the following
stupid example shows.

Let $k_1=\FF_q(t)$, $A_1:=k_1[\pi]_{(\pi)}$, the local ring at 0 of
$\A^1_{k_1}$, and put
\begin{align*}
	A_n&:=A_1[x_n]/(x_n^{p^n}-(\pi+t)) \\
	A&:=\varinjlim_n A_n.
\end{align*}
For each $n$, $A_n$ is a d.v.r. with uniformizer $\pi$.
For $m<n$ the map goes $x_m\mapsto x_n^{p^{n-m}}$.
Let $A^h$ denote the henselization of $A$; both $A$ and $A^h$
have the perfect closure $k_1^{p^{-\infty}}$ of $k_1$ as residue field, but
neither contain $k_1^{p^{-\infty}}$.

\begin{remark}
As a partial converse, if $A$ is an \emph{excellent} henselian d.v.r.
with perfect residue field, then its completion $\hat A$ contains a
\emph{canonical} coefficient field, and Artin's approximation theorem gives 
that $A$ contains a coefficient field.
\end{remark}

As $R$ is equicharacteristic, the inertia admits this simple description;
c.f. \hyperref[weilii:1.7.11]{note to Weil II, 1.7.11}.

One must always repair to \emph{Corps Locaux} Ch. IV for the ultra-mystical 
`upper numbering' filtration on $I$. In Proposition 3 of \S1, it is claimed
that $s(f)-f$ has all its coefficients divisible by $s(y)-y$.
If we let $\mathfrak p_{K'}$ denote the maximal ideal of $A_{K'}$, the
definition of $i_{G/H}$ means that $s(y)-y$ is of order $i_{G/H}(s)$ in
$A_{K'}$; i.e. $(s(y)-y)=\mathfrak p_{K'}^{i_{G/H}(s)}$.
Lemma 1 then shows that all the coefficients of $s(f)-f$ have order
$\geq i_{G/H}(s)$, hence are divisible by $s(y)-y$.

In light of \emph{Corps Locaux} IV \S3 Prop. 14 \& Rmk. 1, if $L/K$ is an
infinite Galois extension with Galois group G, one defines
$G^v:=\varprojlim G(L'/K)^v$ as $L'$ runs over the set of finite Galois
sub-extensions of $L$.
This description shows that $G^v$ is a compact subgroup of $G$, hence 
closed in $G$, hence also in the compact open subgroup $I=G_0=G^0$,
(provided of course $v\geq0$). It also shows that $G^v$ is normal, as it is 
a projective limit of the normal groups $G(L'/K)^v$
(\emph{Corps Locaux} IV Prop. 1). Left continuity
\begin{equation*}
	G^v=\bigcap_{w<v}G^w
\end{equation*}
amounts to the statement that if $s\in G$ is not in $G^v$, then
$s\not\in G^w$ for some $w<v$.
An element $s\in G$ belongs to $G^v$ if for every
finite Galois subextension $L\supset L'\supset K$ with $\Gal(L'/K)=H$,
$i_{G/H}(s)\geq\psi(v)$, so it will suffice to show that if there is
some $L'$ as above with $i_{G/H}(s)<\psi(v)$, then there is some $w<v$ such
that $i_{G/H}(s)<\psi(w)$.
As $\psi$ is continuous and increasing, this is trivial.

Laumon considers the induced filtration on $I=G_0=G^0$. The filtration is 
separated,
\begin{equation*}
	\bigcap_{\lambda\geq0}I^{(\lambda)}=\{1\},
\end{equation*}
as the same is true for $G/H$ for every normal open subgroup $H$ of $G$
(\emph{Corps Locaux} IV Prop. 1).

It's clear that $I^{(\lambda\;+)}\subset I^{(\lambda)}$ as $I^{(\lambda)}$
is closed. But why $I^{(0\;+)}=P$? On the level of a finite Galois 
extension $L'$ of $K$, \emph{Corps Locaux} IV \S2 explains that over a
perfect field of characteristic $p$, $G_1$ is a $p$-group and the quotient
$G_0/G_1$ is sent isomorphically by the inertia character to a subgroup of
the group of roots of unity of the residue field of $\overline L'$; this is
the tame inertia, and $G_1$ is the wild inertia.
It is necessary to switch to the upper numbering 
filtration in order for the filtration to play well with quotients, and
as the index $(G_0:G_1)$ increases (corresponding to more tame inertia in
the extension $L'$), $\varphi(1)$ approaches 0 from the right.
This means that (provided the maximal tamely ramified extension of $K$ is
not finite over $K_r$, the maximal unramified extension of $K$)
for every $\vep>0$ and $s\in I^{(0+\vep)}$ there
exists an extension $L'$ of $K$ with $1/(G_0:G_1)<\vep$, so that
$s\not\in G_1$, $G:=\Gal(L'/K)$.
So $P$ coincides with the completion of $\cup I^{(0+\vep)}$, which 
coincides with the closure of $\cup I^{(0+\vep)}$ in $I$.

\subsection*{2.1.2}\label{laumon:2.1.2}
(2.1.2.2) The kernel of the representation of $P$ is closed and of finite
index, hence open, hence by \emph{Corps Locaux} IV Prop 14. \& Prop. 1,
$I^{(\lambda)}$ acts trivially for $\lambda\gg0$.
The corollary (2.1.2.3) is Maschke's theorem.
In the definition of \emph{pente}, $\RR_+:=\RR_{\geq0}$, and
$\lambda$ may equal $0$; $\lambda=0$ iff $W$ is trivial.
If $H$ denotes the kernel of the representation $P\ra\GL(V)$, then
$P/H$ is a finite group and $(P/H)^v=P^vH/H$
(\emph{Corps Locaux} IV Prop. 14). The slope $\lambda$ coincides with the
largest real number $v$ such that $(P/H)^v=\{1\}$; in the vocabulary of
\emph{Corps Locaux} IV \S3 Rmk. 1, this is the largest break in the 
filtration on $P/H$.
The canonical slope decomposition (2.1.2.4) of $V$ is a decomposition as
$I$- or $G$-module since if $W$ is a simple $P$-submodule of $V$ of slope
$\lambda$ and $g\in G$, then $gW$ is still simple as $P$-module and still
of slope $\lambda$, as the groups $I^{(v)}$ are normal subgroups of $G$,
for all $v\geq0$. Therefore $V_\lambda$ is preserved by $G$.

(2.1.2.8) To see that $L_\psi(1/\pi)$ has slope 1, it will suffice to show
that if $\Gamma:=\Gal(\eta'/\eta)$, then $\Gamma=\Gamma_1$ and 
$\Gamma_2=\{1\}$. It would follow that $\varphi(1)=1$ for this extension so
that $\Gamma^1=\Gamma_1=\Gamma$ and $\Gamma^{1+\vep}=\{1\}$ for all
$\vep>0$. With the criterion of \emph{Corps Locaux} IV \S2 Prop. 5, it
would suffice to show that for all $s\in\Gamma$,
\begin{align*}
	s(\pi')/\pi'&\equiv1\mod (\pi') \\
	s(\pi')/\pi'&\not\equiv1\mod (\pi')^2.
\end{align*}
As $s(\pi')/\pi'=1/(1+\alpha\pi')$ for $\alpha\in\FF_p$,
\begin{equation*}
	s(\pi')/\pi'\equiv 1-\alpha\pi'\mod(\pi')^2.
\end{equation*}


\subsection*{2.1.4} Schur's lemma gives that every simple tame $I$-module has rank 1.
The action of $\Gal(\overline k/k)$ on $\hat\ZZ(1)(\overline k)$ by inner
automorphisms coincides with the action of Galois on roots of unity;
c.f. \href{https://stacks.math.columbia.edu/tag/0BU5}{Stacks tag \texttt{0BU5}}.

\subsection*{2.1.5} The equivalence of the three conditions can be seen as
follows: (i) trivially implies (ii) and (iii) as $V^I$ (resp. $V_I$) is the
largest subobject (resp. quotient) on which $I$ acts trivially.
The decomposition $V=\oplus V_\lambda$ permits us to assume $V=V_\lambda$.
If $\lambda>0$, then by definition $V_\lambda^I=0$ and moreover $P$ acts
nontrivially. Restricting the action of $P$ to a simple $P$-submodule $W$ 
of $V_\lambda$, one finds $W_P=0$, hence $(V_{\lambda})_P=0$ and
\emph{a fortiori} $(V_{\lambda})_I=0$. Moreover, $V_\lambda$ has no
nontrivial subquotient on which $I$ acts trivially, as such a subquotient
would be a direct sum of simple $P$-modules of slope $\lambda$, so
(i), (ii), and (iii) are automatically verified and we consider $V=V_0$,
a tame $G$-module. The existence of a geometrically constant subquotient of
$V$ as $G$-module implies the same as $I$-module, and therefore we need
only show that (ii)$\Leftrightarrow$(iii) and the combination implies that
there exists no $I$-module subquotient of the tame $I$-module $V$.
As $T:=I/P\simeq\hat\ZZ(1)(\overline k)$ is procyclic with topological
generator, say, $t$, and the representation $V$ of $T$ is continuous,
\begin{equation*}
	0\ra V^T\ra V\xra{t-1}V\ra V_T\ra0
\end{equation*}
is exact, and shows that $V^T=0\Leftrightarrow V_T=0$.
Given $T$-submodules
\begin{equation*}
	V=V_0\supset V_1\supset V_2\supset0,
\end{equation*}
$V_1^T=0\Rightarrow (V_1)_T=0$ for each $i$, so if $V_1/V_2$ is
$T$-invariant, the quotient $V_1\ra V_1/V_2$ factors through $(V_1)_T=0$,
and $V_1/V_2=0$. So, $V$ has no $T$-invariant subquotient.

Duality $V\mapsto V^\vee$ sends $\mathscr G_{(0,\infty[}$ into itself
as $(V^\vee)^I=(V_I)^\vee$.

\subsection*{2.2.1}\label{laumon:2.2.1}
Of course, the various functions are extended additively with sign, so that
e.g. for a perverse sheaf $K$ on $X$, $s_x(K)\leq0$ for $x\in|X|$.

(2.2.1.1) Let
\begin{tikzcd}[column sep=18pt]x\arrow[r,hook,"i"]&X&X-x\arrow[l,hook',"j"'].\end{tikzcd}
There is a distinguished triangle
($i^!=Ri^!$ etc.)
\begin{equation*}
	i^!K\ra i^*K\ra i^*j_*j^*K\ra
	\hspace{100pt}
\end{equation*}
As $K$ is perverse, $\mathscr H^{-1}(i^!K)=0$ and $\mathscr H^i(j^*K)=0$
for $j\ne-1$. Representing $j^*K$ by a complex $I$ of injectives in degrees
$\geq-1$
\begin{equation*}
	0\ra\cdots\ra0\ra I_{-1}\xra{d^{-1}}I_0\ra I_1\ra\cdots
\end{equation*}
we have $\mathscr H^{-1}(j^*K)=\ker d^{-1}$. As $j_*$ is left exact,
\begin{ceqn}\begin{equation*}
	\mathscr H^{-1}(j_*j^*K)=\mathscr H^{-1}(j_*I)=\ker j_*(d^{-1})
	=j_*(\ker d^{-1})=j_*\mathscr H^{-1}(j^*K)=j_*j^*\mathscr H^{-1}(K),
\end{equation*}\end{ceqn}
so that
\begin{equation*}
	0\ra i^*\mathscr H^{-1}(K)\ra i^*j_*(j^*\mathscr H^{-1}(K))
\end{equation*}
is exact, proving
$\mathscr H^{-1}(K)_{\overline x}\subset j_*(j^*\mathscr H^{-1}(K))_{\overline x}$
and the inequality $r(K)\leq r_x(K)$, as
$r(\mathscr H^0(K))=0$, and showing that
$\mathscr H^{-1}(K)$ is lisse at $x$ iff
$I_x$ acts trivially on $\mathscr H^{-1}(K)_{\overline x}$ iff
$r_x(\mathscr H^{-1}(K))=r(\mathscr H^{-1}(K))$; of course
$\mathscr H^0(K)_{\overline x}=0$ iff $r_x(\mathscr H^0(K))=0$.
As $r_x(K)=r_x(\mathscr H^0(K))-r_x(\mathscr H^{-1}(K))$ and
$r(K)=-r(\mathscr H^{-1}(K))$, $r_x(K)\geq r(k)$ with equality iff
(i) holds. Of course, $s_x$ measures wild ramification and so
(ii)$\Leftrightarrow$(iii) trivially.
See \hyperref[laumon:reprise]{Reprise}.

\subsection*{2.2.2}\label{laumon:2.2.2} The tame quotient
$\pi_1(\xi,\overline\xi)\ra\pi_1(\xi,\overline\xi)^{\mathrm{mod}}$
has not actually been defined; `tame quotient' in (2.1.1) meant
$I\twoheadrightarrow I/P$. This tame quotient corresponds to
$G\twoheadrightarrow G/P$.

(2.2.2.1) It suffices to show $\overline i_*^{\mathrm{mod}}$ is an
isomorphism in light of the short exact sequence of $\pi_1$ (SGA 1 6.11) 
which expresses the fundamental group as extension of $\Gal(\overline k/k)$ 
by the geometric fundamental group.

(2.2.2.2) This mysterious theorem is found in Katz,
\emph{Local-to-global extensions of representations of fundamental groups,}
where he also proves a cohomological formula for the $\ell$-adic Swan
representation.
In light of \hyperref[laumon:reprise]{Reprise} below, some cursory analysis
of the meaning of this theorem can be made.
The discussion doesn't change if one replaces $k$ by $\overline k$.
First of all, $\Gal(\overline\xi/k(u))$ surjects onto
$\pi_1(\GG_{m,k},\overline\xi)$, and we can describe the kernel in terms of
the monodromy at all geometric closed points of $\GG_{m,k}$.
\begin{equation*}
	G:=\Gal(\overline\xi/k(u))
	\twoheadrightarrow\pi_1(\GG_{m,k},\overline\xi)
	\twoheadrightarrow\pi_1(\GG_{m,k},\overline\xi)^{\mathrm{mod}.\infty}
	\twoheadrightarrow\pi_1(\GG_{m,k},\overline\xi)^{\mathrm{mod}}
\end{equation*}
With the notation (2.2.1),
the kernel of the first map is topologically generated by
$\{I_x\}_{x\in\GG_{m,k}}$.
The kernel of the second map is topologically generated by $P_{\infty}$.
The kernel of the third map is topologically generated by $P_0$.
(Both $P_\infty,P_0\subset G$.)
The injection $\pi_1(\xi,\overline\xi)\hookrightarrow G$ corresponds to the
Galois extension $k(\xi)/k(u)$.

That $\pi_1(\xi,\overline\xi)$ injects into
$\pi_1(\GG_{m,k},\overline\xi)^{\mathrm{mod}.\infty}$
corresponds simply to the statement that membership in the latter group
puts no condition on the monodromy at 0. An object in the latter Galois
category corresponds to a finite cover of $\PP^1-\{0\}$ which is étale away
from $\infty$ and tamely ramified at infinity. The injection
$\pi_1(\xi,\overline\xi)\hookrightarrow\pi_1(\GG_{m,k},\overline\xi)^{\mathrm{mod}.\infty}$
corresponds to the statement that the constraints on monodromy at all 
closed points of $\PP^1$ other than 0 imposed by membership in 
$\pi_1(\GG_{m,k},\overline\xi)^{\mathrm{mod}.\infty}$ put in effect
`no constraints' on the monodromy at 0.

The isomorphism
$\pi_1(\xi\otimes_k\overline k,\overline\xi)^{\mathrm{mod}}\simeq\pi_1(\GG_{m,\overline k},\overline\xi)^{\mathrm{mod}}$
is the statement that the Galois category of étale covers of the generic
point of $(\A^1_{\overline k})_{(0)}$ tamely ramified at the closed point 
is equivalent to the category of étale covers of $\GG_{m,\overline k}$
which are tamely ramified at 0 and $\infty$.
The proof given notes that both are isomorphic to $\hat\ZZ(1)(\overline k)$,
but this fact for $\GG_{m,\overline k}$ is not proved in (1.1.3.7), which 
introduces the Kummer coverings of $\GG_{m,\overline k}$.
What's needed is Grothendieck's
comparison theorem for curves (SGA 1 XIII 2.12), which immediately shows
that the tame fundamental group of $\PP^1_{\overline k}-\{\infty\}$ is
trivial (this is the automorphism group of a fiber functor on the Galois
category consisting of finite covers of $\PP^1_{\overline k}$ étale over
$\A^1_{\overline k}$ and tame at $\infty$)
and that $\pi_1(\GG_{m,\overline k},\overline\xi)^{\mathrm{mod}}$
is freely generated by a generator of the tame inertia at 0, hence is
indeed isomorphic to $\hat\ZZ(1)(\overline k)$.
In light of this, we know that the Kummer coverings of (1.1.3.7) do exhaust
the set of finite maps to $\PP^1_k$ étale over $\GG_{m,\overline k}$ and
tamely ramified at 0 and $\infty$.

\subsection*{Intermezzo I: Katz-Gabber extensions}
The paper is by Katz,
\emph{Local-to-global extensions of representations of fundamental groups}.

(1.2.3) En effet, when you pull $E$ back to $X\otimes_K K^{\mathrm{sep}}$,
the only monodromy is geometric. The map
$\pi_1(X\times_K K^{\mathrm{sep}},\overline x)\ra\Aut(E(\overline x))$
is continuous with open kernel $U$ corresponding to the Galois $G$-torsor
$Z$, where of course $G=\pi_1(X\times_KK^{\mathrm{sep}},\overline x)/U$.
Pulling back to $Z$ kills the monodromy on $E$ (i.e. $E$ `splits').
Now the point is that in the exact sequence
($\overline X:=X\times_K K^{\mathrm{sep}}$)
\begin{equation*}\tag{$\dagger$}
	e\ra\pi_1(\overline X,\overline x)\ra\pi_1(X,\overline x)\ra\Gal(K^{\mathrm{sep}}/K)\ra e,
\end{equation*}
the surjection is a quotient map, \emph{a fortiori} open. Let $V$ be 
defined by the exact sequence
\begin{equation*}
	e\ra V\ra\pi_1(X,\overline x)\ra\Aut(E(\overline x))\ra e
\end{equation*}
and let $V'$ be the image of $V$ in $\Gal(K^{\mathrm{sep}}/K)$; $V'$
is open and we let the finite Galois extension $K'$ correspond to any such
containing $V'$ and over which $Z$ is defined..

(1.3) A unique $p$-Sylow is the same as a normal
$p$-Sylow. A normal $p$-Sylow in a group $G$ is characteristic as it is the
unique subgroup with its order (any $p$-subgroup of $G$ is contained in 
it).

(1.3.2) In proof of $1)\Rightarrow 4)$, the statement that `the unique
open normal subgroup of $\pi_1(\mathbf G_{m,L},\overline x)$ of index 
$N\geq1$ prime to $p$ is the one corresponding to the $N$th power covering 
$[N]$ of $\GG_{m,L}$ by itself' is easily seen: denoting such a subgroup by
$U$, $U$ corresponds to a finite étale connected torsor $T$ with group
$\pi_1(\mathbf G_{m,L},\overline x)/U$, and the homomorphism from this
group into the group opposite $\Aut T=\Aut T(\overline x)$ is an
isomorphism \cite[5.11]{SGA1}. So $T$ is a cover of $\mathbf G_{m,L}$
tamely ramified at 0 and $\infty$ since $p\nmid {|\Aut T(\overline x)|}$,
and as in the \hyperref[laumon:2.2.2]{note to (2.2.2)},
$\pi_1(\GG_{m,L},\overline\xi)^{\mathrm{mod}}\simeq\hat\ZZ(1)(L)$, so the
statement is obvious.

Given a special $E\ra\GG_{m,K}$, the preimage of the unique $p$-Sylow of
the geometric monodromy of $E$ in $\pi_1(\GG_{m,K^{\mathrm{sep}}},\overline x)$
is an open subgroup with index $N_1$ prime to $p$; by the above claim,
$[N_1]^*E$ has geometric monodromy a $p$-group and is still at worst tamely
ramified at 0 so that for some $N_2$, $[N_2]^*[N_1]^*E$ extends to an étale
cover of $\A^1_K$; now let $N=N_1N_2$.

The proof of $4) \Rightarrow 7)$ is self-evident after you observe that
\begin{equation*}\begin{tikzcd}
	&\GG_{m,K}\arrow[dl,"{[N]}"']\arrow[dr,"\operatorname{Trans}_b"] \\
	\GG_{m,K}\arrow[dr,"\operatorname{Trans}_a"']&&\GG_{m,K}\arrow[dl,"{[N]}"] \\
	&\GG_{m,K}
\end{tikzcd}\end{equation*}
commutes, since the left composition corresponds to
$T\mapsto aT\mapsto aT^N$ as $[N]$ is $K$-linear, while the right 
composition corresponds to $T\mapsto T^N\mapsto (bT)^N=aT^N$.

(1.4) When Katz writes `an action of $G$ on $E$ covering its action on
$\GG_{m,K'}$,' for example, `covering' means, in the language of
\cite[XIII 1.1]{SGA7}, `compatible with.' Moreover the action of $G$ on $E$ 
should be continuous.

In the proof of the main theorem, once reduced to $(N,K')=(1,K)$ over the
field $K'$, there is `nothing to prove' when $p=1$ since in this case the
words `monodromy group a $p$-group' means `trivial monodromy group,' and
so putting
\begin{ceqn}\begin{align*}
&\mathcal A := \text{category of finite étale coverings of $\A^1_{K'}$ 
with trivial monodromy, \emph{and}} \\
&\mathcal B := \text{category of finite étale coverings of
$\Spec K'((T^{-1}))$ with trivial monodromy,}
\end{align*}\end{ceqn}
both $\mathcal A$ and $\mathcal B$ are isomorphic to the category of
finite étale coverings of $\Spec K'$: in the case of $\mathcal A$, this 
follows from the exact sequence ($\dagger$) in light of the fact that
$\pi_1(\PP^1_{K^{\mathrm{sep}}})=1$;
in the case of $\mathcal B$, this follows from the same exact sequence and
the observation that $K^{\mathrm{sep}}((T^{-1}))$ is the function field of
the strict henselization $\PP^1_{(\infty)}$ of $\PP^1_{K'}$ at $\infty$; in
both cases, the condition `trivial mondromy' means that the revêtement
étale extends over $\infty$
(resp. the closed point of the strict henselization), and
$\pi_1(\PP^1_{K^{\mathrm{sep}}})=1=\pi_1(\PP^1_{(\infty)})$ as any
connected revêtement étale of the spectrum of a strictly henselian ring is 
trivial.

(1.4.2) If $R$ is a ring, the idempotents of $R[[x]]$ are in bijection with
those of $R$, for, given an idempotent $f\in R[[x]]$, write 
$f(x)=r_0+f_1(x)$ with $f_1(x)$ a power series with constant term 0;
$r_0$ must be idempotent and if $f_1(x)\ne0$ it has a nonzero term of
lowest degree, say $r_nx^n$. Then
\begin{equation*}
	r_0+f_1(x)=f(x)=f(x)^2=r+2r_0f_1(x)+f_1(x)^2
\end{equation*}
so that $2r_0r_n=r_n$ and therefore also $2r_0r_n=2r_0^2r_n=r_0r_n$ so
$r_0r_n=0$, contradicting $r_n\ne0$.

Using this fact, we find that the idempotents of $R$ are in bijection with
those of $R[T]$; writing $R((T^{-1}))=R[T][[T^{-1}]]$ finds that the 
idempotents of $R((T^{-1}))$ are in bijection with those of $R[T]$ and so
also with those of $R$.

(1.4.4) It's helpful to recall \emph{Arcata} II (2.1).

(1.4.5) To see $\mathscr P$ is surjective on $T^{-1}R[[T^{-1}]]$, pick a
power series in $T^{-1}$ with no constant term $b=\sum_{i>0}b_iT^{-i}$.
To hit $b$ with an element $c$ of $T^{-1}R[[T^{-1}]]$, we proceed in the
usual way: given $b$ at step $i>0$ with $b_j=0$ for $j<i$ and 
$c_1,\ldots,c_j$ already fixed, put $c_i:=b_i$ and replace $b$ by
$b-\mathscr P(c_iT^{-i})$. Then $c:=\sum_{i>0}c_iT^{-i}$ has
$\mathscr P(c)=b$.

(1.4.6) Recall Serre, \emph{Cohomologie Galoisienne} 2.3 Prop. 8 for
\begin{equation*}
	H^*(\varprojlim G_i,\varinjlim A_i)=\varinjlim H^*(G_i,A_i).
\end{equation*}

(1.4.8) As tensor product commutes with colimits, the assertion about
$K^{\mathrm{sep}}\otimes_K K((T^{-1}))$ is clear from the isomorphism
$K'\otimes_K K((T^{-1}))\xra\sim K'((T^{-1}))$ which in turn follows from 
the basic fact that a finitely generated module $M$ over a noetherian local
ring $A$ with maximal ideal $\mathfrak m$ has
$\hat M\simeq M\otimes_A\hat A$ for the $\mathfrak m$-adic topology; in our
case $A=K[T^{-1}]$, $M=K'[T^{-1}]$, and the isomorphism of fields above is
the statement over the generic fiber; i.e.
\begin{align*}
	&K'\otimes_K K[[T^{-1}]]=K'[T^{-1}]\otimes_{K[T^{-1}]}K[[T^{-1}]] \xra\sim K'[[T^{-1}]]\quad\text{\emph{and}} \\
	&K'\otimes_K K((T^{-1}))=K'[T^{-1}]\otimes_{K[T^{-1}]}K((T^{-1}))\xra\sim K'((T^{-1})).
\end{align*}
We see that the field $K'((T^{-1}))$ is the fraction field of the
henselian d.v.r. $K'[[T^{-1}]]$ and the colimit of henselian local rings
along local ring homomorphisms is henselian local, it suffices to observe
that the colimit is also along étale, \emph{a fortiori} unramified ring
maps that the colimit is a henselian a d.v.r.

That the fraction field of a henselian d.v.r. has the same Galois
theory as that of its completion is a result of Berkovich
– \hyperref[pf:berkovich_lemma]{c.f. proof of the lemma in}
\hyperref[laumon:2.4.1]{note to (2.4.1)}.
This fact can also be obtained from the structure we have found for the
Galois group: the unramified and tame parts are identical, and the free
pro-p quotients can be shown to be mapped isomorphically: it amounts to
showing that the map on étale $H^1(-,\ZZ/p)$ from the (spectra of the)
maximal pro-p Galois extension of $K^{\mathrm{sep}}((T^{-1}))$ to that of $K^{\mathrm{sep}}\otimes_K K((T^{-1}))$ is an isomorphism.

(1.4.10) The point is that the map of $\pi_1$s of (1.4.7) factors as
\begin{ceqn}\begin{equation*}
	\pi_1(\Spec(K^{\mathrm{sep}}((T^{-1})),\overline x_1)\ra
	\pi_1(\GG_{m,K^{\mathrm{sep}}},\overline y_1)(\text{\emph{tame at 0}})
	\twoheadrightarrow
	\pi_1(\GG_{m,K^{\mathrm{sep}}},\overline y_1)(\text{\emph{special}})
\end{equation*}\end{ceqn}
by functoriality of the inverse image. This composite is an isomorphism
and its inverse, preceded by the (tame-at-0)-to-(special) quotient, 
provides the retraction. Why is the retraction unique?
Identifying $\pi_1(\Spec(K^{\mathrm{sep}}((T^{-1})),\overline x_1)$ with
its image under the canonical injection, retractions are in bijection with 
complements to $\pi_1(\Spec(K^{\mathrm{sep}}((T^{-1})),\overline x_1)$:
closed normal subgroups $N$ of
$\pi_1(\GG_{m,K^{\mathrm{sep}}},\overline y_1)(\text{\emph{tame at 0}})$
such that
\begin{align*}
	&N\cap\pi_1(\Spec(K^{\mathrm{sep}}((T^{-1})),\overline x_1)=\{1\}
	\quad\text{\emph{and}} \\
	&N\cdot\pi_1(\Spec(K^{\mathrm{sep}}((T^{-1})),\overline x_1)= \pi_1(\GG_{m,K^{\mathrm{sep}}},\overline y_1)(\text{\emph{tame at 0}}).
\end{align*}
Recall that
$\pi_1(\GG_{m,K^{\mathrm{sep}}},\overline y_1)(\text{\emph{special}})$
is the maximal pro-`group with unique $p$-Sylow subgroup' quotient of
$\pi_1(\GG_{m,K^{\mathrm{sep}}},\overline y_1)(\text{\emph{tame at 0}})$.
In the language of \emph{Profinite Groups} by Ribes \& Zalesskii, the class
$\mathcal C$ of finite groups with unique $p$-Sylow subgroup is
a \emph{variety} of finite groups, hence \emph{a fortiori} a 
\emph{formation} of finite groups. Putting
\begin{ceqn}\begin{align*}
	R_{\mathcal C}:=\bigcap\{N:\ &N\text{ open normal subgroup of }
	\pi_1(\GG_{m,K^{\mathrm{sep}}},\overline y_1)(\text{\emph{tame at 0}}),
	\\
	&\pi_1(\GG_{m,K^{\mathrm{sep}}},\overline y_1)(\text{\emph{tame at 0}})/N\in\mathcal C\},
\end{align*}\end{ceqn}
(3.4) of \emph{op. cit.} says that $R_{\mathcal C}$ is a characteristic subgroup, the sequence
\begin{ceqn}\begin{equation*}
	1\ra R_{\mathcal C}\ra
	\pi_1(\GG_{m,K^{\mathrm{sep}}},\overline y_1)(\text{\emph{tame at 0}})\ra
	\pi_1(\GG_{m,K^{\mathrm{sep}}},\overline y_1)(\text{\emph{special}})\ra1
\end{equation*}\end{ceqn}
is exact, and given any closed normal subgroup $K$ of
$\pi_1(\GG_{m,K^{\mathrm{sep}}},\overline y_1)(\text{\emph{tame at 0}})$
such that
$\pi_1(\GG_{m,K^{\mathrm{sep}}},\overline y_1)(\text{\emph{tame at 0}})/K$
is pro-$\mathcal C$, then $K$ is a subgroup of $R_{\mathcal C}$.
This last point implies that the only $N$ as above is $R_{\mathcal C}$,
hence the retraction is unique.

Since specialness is geometric, the following diagram of profinite groups
and continuous homomorphisms is commutative with exact rows and columns 
(compare 1.3.3).
\begin{ceqn}\begin{equation*}\begin{tikzcd}[column sep=small]
	&1\arrow[d]& 1\arrow[d] \\
	&R_{\mathcal C}\arrow[r,equals]\arrow[d]
	&R_{\mathcal C}\arrow[d] \\
	1\arrow[r]&\pi_1(\GG_{m,K^{\mathrm{sep}}},\overline y_1)(\text{\emph{tame at 0}})\arrow[r]\arrow[d]
	&\pi_1(\GG_{m,K},\overline y_1)(\text{\emph{tame at 0}})\arrow[r]\arrow[d]&\Gal(K^{\mathrm{sep}}/K)\arrow[d,equals]\arrow[r]&1 \\
	1\arrow[r]&\pi_1(\GG_{m,K^{\mathrm{sep}}},\overline y_1)(\text{\emph{special}})\arrow[r]\arrow[d]
	&\pi_1(\GG_{m,K},\overline y_1)(\text{\emph{special}})\arrow[r]\arrow[d]
	&\Gal(K^{\mathrm{sep}}/K)\arrow[r]&1 \\
	&1&1
\end{tikzcd}\end{equation*}\end{ceqn}
In more words, clearly the kernel of the tame-to-special quotient for $K$
contains that for $K^{\mathrm{sep}}$ (the functor of restriction to 
geometric monodromy corresponds to the functor `reciprocal image along
$\Spec K^{\mathrm{sep}}\ra\Spec K$'); to see it is no larger, note that
the special coverings of $G_{m,K_i}$ are special coverings of $G_{m,K}$ for
all Galois extensions $K_i$ of $K$; therefore an element of the kernel must
fix all the special coverings of $G_{m,K_i}$ for each $i$, and
$K^{\mathrm{sep}}=\cup_i K_i$.

This diagram implies the uniqueness of the retraction for $K$ restricting
to the given one for $K^{\mathrm{sep}}$.
In more words, as before, retractions of
\begin{equation*}
	\pi_1(\GG_{m,K},\overline y_1)(\text{\emph{tame at 0}})
	\twoheadrightarrow
	\pi_1(\GG_{m,K},\overline y_1)(\text{\emph{special}})
\end{equation*}
are in bijection with closed normal subgroups $N$ satisfying the same
conditions as before with $K^{\mathrm{sep}}$ replaced by $K$ and
$\overline x_1$ replaced by $\overline x$. The condition that the 
retraction must restrict to the given one for $K^{\mathrm{sep}}$ implies
that $N$ must contain $R_{\mathcal C}$. But as seen in the diagram above,
$N=R_{\mathcal C}$. Therefore, as before, there is only one possible choice
for $N$.

(1.6.4) Finitely presented since constructible – \emph{Th. finitude}.

(1.6.6) See Katz, \emph{Gauss Sums, Kloosterman Sums, and Monodromy Groups} 
(2.1.1) \& (2.3.3).

(1.6.8) There are 3 claims to verify: (1) that the isomorphism class of a
projective $\ZZ_\ell[G]$-module can be deduced by character; (2) that
$\chi(\A^1_K,-)=\chi_c(\A^1_K,-)$, and (3) that
$\chi_c(\GG_{m,K},\mathcal F^{\mathrm{can}})=-\operatorname{swan}_\infty(\mathcal F)$.

(1)
Referring to Serre, \emph{Linear Representations of Finite Groups},
(14.4) gives that for a mixed-characteristic local field $K$ with valuation
ring $A$ with residue field $A/\mathfrak m=:k$ and $G$ a finite group,
every projective $A[G]$-module is a direct sum of projective indecomposable
$A[G]$-modules unique up to isomorphism so that two projective
$A[G]$-modules are isomorphic iff their images in the Grothendieck group of
the category of projective $A[G]$-modules coincide. (16.1) gives that two
projective $A[G]$-modules $P,P'$ are isomorphic if the $K[G]$-modules
$P\otimes K,P'\otimes K$ are.
Let $R_K(G)$ denote the Grothendieck group of finite-dimensional
$K[G]$-modules. (14.5) gives that $\langle E,F\rangle=\dim\Hom^G(E,F)$ on
$K[G]$-modules $E,F$ induces a bilinear form
\begin{equation*}
	R_K(G)\times R_K(G)\ra\ZZ
\end{equation*}
and (14.6) shows that it is compatible with extension of field
$K\subset K'$, which induces an injection $R_K(G)\ra R_{K'}(G)$.
Over a large enough field (e.g. the algebraic closure of $K$), the above
bilinear form is nondegenerate and induces an isomorphism of $R_K(G)$ onto
its dual. The identity
\begin{equation*}
	\Hom_G(E,F)\simeq\check E\otimes_G F
\end{equation*}
and the fact that the Swan representation is self-dual
\cite[Exp. X (3.8) \& (4.4)]{SGA5} show that indeed it suffices to show
that
\begin{equation*}
	\dim_{\overline\QQ_\ell}(H^1(\A^1_K,j_!\mathcal F^{\mathrm{can}}))
	=\operatorname{swan}_\infty(\mathcal F),
\end{equation*}
where the latter is the Swan conductor $b(M)$ of Serre (19.3), in light of
the isomorphism
\begin{equation*}
	H^1(\A^1_K,j_!((\operatorname{Reg}_{G;\ZZ_\ell})^{\mathrm{can}}))\otimes_{\ZZ_\ell[G]}M\xra\sim H^1(\A^1_K,j_!\mathcal F^{\mathrm{can}}).
\end{equation*}

(2)
The equality $\chi(X,\mathcal F)=\chi_c(X,\mathcal F)$ for a proper smooth
curve $X$ over an algebraically closed field $k$ was proven by Grothendieck
\cite[Exp. X (7.12)]{SGA5}, and later by Laumon for any $X$ separated and
finite type over $k$ algebraically closed in his 1981 article
\emph{Comparaison de caractéristiques d'Euler-Poincaré en cohomologie l-adique}.

(3) The formula
$\chi_c(\GG_{m,K},\mathcal F^{\mathrm{can}})=-\operatorname{swan}_\infty(\mathcal F)$
can be deduced from the Grothendieck-Ogg-\v Savarevi\v c formula as appears
in Laumon (2.2.1.2). Laumon's Swan conductor coincides with Serre's
Swan conductor $b(M)$ after passing from upper to lower numbering
\begin{multline*}
	s(M)=\sum_{\Lambda\in\Lambda(M)}\lambda.r(V_\lambda)
	=\int_0^\infty\codim M^{G^\lambda}\;d\lambda \\
	=\int_0^\infty\frac{\codim M^{G_r}}{[G_0:G_r]}\;dr
	=\sum_{i=1}^\infty\frac{\codim M^{G_i}}{[G_0:G_i]}=b(M).
\end{multline*}
Incidentally, if $a(M)$ denotes the Artin conductor, putting
\begin{equation*}
	t(M):=\int_{-1}^0\codim M^{G^\lambda}\;d\lambda=\int_{-1}^0\frac{\codim M^{G_r}}{[G_0:G_r]}\;dr=\codim M^{G_0}=\codim M^{I_\infty}
\end{equation*}
allows us to write $a(M)=t(M)+s(M)$, expressing the Artin conductor as the
sum of the tame conductor $t(M)$ and the Swan (or wild) conductor $s(M)$.


\subsection*{2.3.1}\label{laumon:2.3.1}
(2.3.1.1) The negative sign before $\chi_c$ is due to the shift $[1]$ in
the formula for Fourier transform. The contribution in the formula of
Grothendieck-Ogg-\v Safarevi\v c from $a_\infty(K)$ is
$r(K)-s(F_{\overline\eta_\infty}\otimes\mathscr L(x.s')_{\overline\eta_\infty})$
and of course $r_\infty(K')=0$ as our $K$ on $\PP^1$ is obtained as 
extension by zero of $K$ on $A$. The Swan conductor $s_\infty$ doesn't 
care about the stalk of $K$ at $\infty$ but only about the 
restriction of $K$ to the generic point $\eta_{\infty}$ of the henselian
trait $(\PP^1_k)_{(\infty)}$. As $\mathscr L$ is locally constant on $A$,
$a_s(K\otimes\mathscr L(x.s'))=a_s(K)$ for $s\in S$.

To understand $\mathscr L(x.a')_{\overline\eta_\infty}$ when $a'\ne0$,
let's again write the Artin-Schreier covering
\begin{equation*}
	0\ra\FF_p\ra\GG_{a,k}\xra{t^p-t}\GG_{a,k}\ra0.
\end{equation*}
The induced map $k[t_1]\ra k[t_2]$ on coordinate rings is
$t_1\mapsto t_2^p-t_2$, as can be readily seen by remembering what $F^*$
and $+$ do on the coordinate ring of $\GG_{a,k}$. The Artin-Schreier sheaf
$\mathcal A$ is the sheaf of local sections of this covering.
It is an $\FF_p$-torsor on $\GG_{a,k}=\Spec k[t_1]$. As the above map on
coordinates makes $t_2$ integral over $k[t_1]$ with equation of integral
dependence $t_2^p-t_2=t_1$, $\mathcal A$ is represented over
$\GG_{a,k}=\Spec k[t_1]$ by the revêtement étale
$\Spec k[t_1,t_2]/(t_2^p-t_2-t_1)$.
Pushing $\mathcal A$ by the character $\psi^{-1}$ gives $\mathscr L_{\psi}$
but it is conceptually easier to keep working with $\mathcal A$.
The pullback of $\mathcal A$ to $A$ via $A\times a'\ra A\times A'\ra\GG_a$
corresponds to the revêtement étale of $A=\Spec k[x]$ given by
$\Spec k[x,t]/(t^p-t-xa')$, this is the one-parameter family of
Artin-Schreier coverings
\begin{equation*}
	t^p-t=xa',\qquad a'\in\overline k.
\end{equation*}
Let $\FF=\overline\FF_p$.
Given $x\in A(\FF),x'\in A'(\FF)$, we can consider 
$x,x'$ as elements of $\FF_q$ for some $q=p^n$,
$\mathscr L(xx')=\psi^{-1}(N_{\FF_q/\FF_p}(xx'))$.
When $k$ is merely a perfect field of characteristic $p$, $\mathscr L(x.a')$
corresponds to the push of the above $\FF_p$-torsor $\mathcal A$ by
$\psi^{-1}:\FF_p\ra\overline\QQ_\ell^\times$. The sheaf
$\mathcal A$ corresponds to the representation of $\Gal(\eta'/\eta)$
discussed in \hyperref[laumon:2.1.2]{the note to (2.1.2.8)};
likewise $\mathscr L(x.a')_{\overline\eta_\infty}$ corresponds to
$L_{\psi}(1/xa')$; here $xa'$ corresponds to $\pi$ and indeed
$1/xa'$ is a uniformizer for the strictly henselian trait
$(\PP^1_{\overline k})_{(\infty)}$.
So $\mathscr L(x.a')_{\overline\eta_\infty}$ has slope 1.

If there is an $a'_1\in\overline k$ such that
\begin{equation*}
	((F_{\overline\eta_\infty})_1\otimes\mathscr L(x.a'_1)_{\overline\eta_\infty})^{I_\infty^{(1)}}\ne0,
\end{equation*}
then for all $a_2'\ne a_1'\in\overline k$,
\begin{equation*}
	((F_{\overline\eta_\infty})_1\otimes\mathscr L(x.a'_2)_{\overline\eta_\infty})^{I_\infty^{(1)}}
	=((F_{\overline\eta_\infty})_1\otimes\mathscr L(x.a'_1)_{\overline\eta_\infty}\otimes\mathscr L(x.(a'_2-a'_1))_{\overline\eta_\infty})^{I_\infty^{(1)}}=0,
\end{equation*}
as $\mathscr L(x.(a'_2-a'_1))_{\overline\eta_\infty}$ has slope 1.

Formula (i): in light of the above discussion, (2.1.2.7) gives that for
almost all $a'\in\overline k$,
\begin{equation*}
s(F_{\overline\eta_\infty}\otimes\mathscr L(x.a')_{\overline\eta_\infty})
=r((F_{\overline\eta_\infty})_{[0,1[})+r((F_{\overline\eta_\infty})_1)+ s((F_{\overline\eta_\infty})_{]1,\infty[}).
\end{equation*}\
By (2.2.1.1), $r(K')=r_{s'}(K')$ for all but the finitely many $s'\in S'$.
Combining these two facts with the Grothendieck-Ogg-\v Safarevi\v c formula for $r_{s'}(K')$ gives
\begin{align*}
	r(K')&=r(F_{\overline\eta_\infty})+\sum_{s\in S}\deg(s).a_s(K)-(r((F_{\overline\eta_\infty})_{[0,1[})+r((F_{\overline\eta_\infty})_1)+ s((F_{\overline\eta_\infty})_{]1,\infty[})) \\
	&=\sum_{s\in S}\deg(s).a_s(K)+r((F_{\overline\eta_\infty})_{]1,\infty[})-s((F_{\overline\eta_\infty})_{]1,\infty[}).
\end{align*}

The difference between formul\ae\ (i) and (ii) is in
$s((F_{\overline\eta_\infty})_1\otimes\mathscr L(x.s')_{\overline\eta_\infty})$;
in the generic formula (i) we could discard the finitely many geometric
points of $A'$ where this differs from $r((F_{\overline\eta_\infty})_1)$.
Therefore,
\begin{equation*}
	r_{s'}(K')-r(K')=-s((F_{\overline\eta_\infty})_1\otimes\mathscr L(x.s')_{\overline\eta_\infty})+r((F_{\overline\eta_\infty})_1).
\end{equation*}
The difference between formul\ae\ (i) and (iii) is that in (i) we applied
(2.1.2.7) while in (iii) we cannot, so that
\begin{align*}
	r(K')_{0'}-r(K')
	&=-s(F_{\overline\eta_\infty})
	+(r((F_{\overline\eta_\infty})_{[0,1[})+r((F_{\overline\eta_\infty})_1)+ s((F_{\overline\eta_\infty})_{]1,\infty[})) \\
	&=-s((F_{\overline\eta_\infty})_{[0,1[})+ r((F_{\overline\eta_\infty})_{[0,1[}).
\end{align*}

\subsection*{Intermezzo II: SGA 7 Exposé XIII}\label{sga7:XIII}
\S1 In the definition of a compatible action of $G$ on a sheaf of sets
$\mathcal F$ on $\overline Y$, to an étale $a:U\ra\overline Y$ and $g\in G$
we must associate an isomorphism
\begin{equation*}\begin{tikzcd}
	\overline Y\times_{u(g),a}U\simeq &[-35pt]U\arrow[r,"u(g)"]\arrow[d,"a"]\arrow[dr,phantom,"\ulcorner",near start]&U\arrow[d,"a"] \\
	&\overline Y\arrow[r,"u(g)"]& \overline Y
\end{tikzcd}\rightsquigarrow
\begin{tikzcd}
	\mathcal F(U\xra{u(g)}U\xra{a}\overline Y)\xra{\sigma(g)(U)}\mathscr F(U\xra a\overline Y)
\end{tikzcd}\end{equation*}
% 	\mathcal F(U\ra\overline Y\xra{u(g)}\overline Y)\xra{\sigma(g)(U)}\mathscr F(U\ra\overline Y)
so as to induce morphisms of sheaves
$\sigma(g):u(g)^*\mathcal F\ra\mathcal F$ in a compatible way so that
$\sigma(gh)=\sigma(g)\sigma(h)$.
In effect, there is an obvious choice: $\mathcal F(u(g))$, and this choice
explains why, given $\mathcal G$ on $Y$, the action of
$\Gal(\overline k,k)$ on $\overline{\mathcal G}$ by transport of structure
is compatible with the action of the same group on $\overline Y$
(action `by transport of structure' means the above action).
\emph{A priori} the compatible action of $G$ on $\mathcal F$ need not even
factor through $u$.

(1.2.7) Let $f:S'\ra S$ be a surjective morphism of henselian traits.
Then $\eta'\mapsto\eta,s'\mapsto s$, and we can choose
$\overline s',\overline s,\overline\eta',\overline\eta$ so that
$\overline s'\ra s$ factors through $\overline s$ and likewise
$\overline\eta'\ra\overline\eta\ra\eta$ so that the diagram below commutes.
\begin{equation*}\begin{tikzcd}
	Y\times_s\overline s'\arrow[d,"\overline s'"]\arrow[r,"f"]
	&Y\times_s\overline s\arrow[d,"\overline s"] \\
	Y\times_s s'\arrow[r,"f"]&Y
\end{tikzcd}\end{equation*}
This implies that for $\mathcal F$ a sheaf on $Y$,
$f^*\mathcal F_{\overline s}:=f^*\overline s^*\mathcal F=\overline s'^*f^*\mathcal F$,
so that $\Gal(\overline s'/s')$ acts on
$f^*\mathcal F_{\overline s}$ \emph{via} the homomorphism
\begin{equation*}\Gal(\overline s'/s')\ra\Gal(\overline s/s)
\qquad\text{induced by the restriction of $f$ to $s'$;}\end{equation*}
likewise, $\Gal(\overline\eta'/\eta')$ acts on $f^*\mathcal F_{\overline\eta}$ \emph{via} the homomorphism
\begin{equation*}\Gal(\overline\eta'/\eta')\ra\Gal(\overline\eta/\eta)
\qquad\text{induced by the restriction of $f$ to $\eta'$.}\end{equation*}

(1.3) Typo: $\overline X:=X\times_S\overline S$.
Recall that $\overline S$ is the normalization of $S$ in
$\overline\eta$, which is the spectrum of a strictly henselian d.v.r.; 
its separably closed residue field may be
an inseparable extension of $k(s)$, but we take $\overline s$ to denote the
spectrum of this separably closed field, which can be considered as the
closed point of $\overline S$ or, by light abuse of notation (0.2.4), as
defining a geometric point of $S$.

It is perhaps worth comparing $i^*j_*j^*$ with
$\overline i^*\overline j_*\overline j^*$ in the definition of $\Psi_\eta$.
Let $\mathcal F$ be a sheaf on $Y\ra S$;
then $\mathcal F_{\overline\eta}$ carries action of
$\Gal(\overline\eta/\eta)$. In effect, the stalk of
$i^*j_*j^*\mathcal F$ is $\mathcal F_{\overline\eta}^I$, 
while $\overline i^*\overline j_*\overline j^*\mathcal F=\mathcal F_{\overline\eta}$ endowed with continuous action of
$\Gal(\overline\eta/\eta)$ compatible with the action of the latter on
$Y_{\overline s}$. In galoisian terms of (1.2.2), this is the 
observation that $\Gal(\overline\eta/\overline\eta)=\{e\}$.

In (1.3.6.2), the $\psi$ is omitted on the left hand side because it
induces an equivalence (see note to (2.1.8) below).
The $X_s\times_s\eta$-component of both sides is readily computed from
the definitions, keeping in mind that $\psi_{\overline\eta}$ is a sheaf
on $X_{\overline s}$; namely,
\begin{equation*}
	\Psi_{\eta}(f_*\mathcal F)
	=\overline i^*\overline j_*(f_*(F)_{\overline\eta})
	=\overline i^*\overline j_*\Gamma(X_{\overline\eta},\mathcal F),
\end{equation*}
which, as $\Gal(\overline\eta/\eta)$-module, is simply
$\Gamma(X_{\overline\eta},\mathcal F)$ (sitting on $\overline s$).
With $f:X_{\overline s}\ra\overline s$,
\begin{equation*}
	f_*\Psi_{\eta}(\mathcal F)
	=\Gamma(X_{\overline s},\Psi_{\overline\eta}(\mathcal F)).
\end{equation*}


(1.3.8) The commutativity of $f_!$ with change of base does not require
that $f$ be quasi-finite, only separated and locally of finite type
\cite[Exp. XVII 6.1.4]{SGAA}.
The morphism $f_!\overline j_*\ra\overline j_*f_!$ follows from
the definition of $f_!$ \cite[Exp. XVII 6.1.2]{SGAA} and the observation
that the sections of $\overline j_*f_!$ over $X'$ are the sections of
$\overline j_*f_*$ over $X_{\overline\eta}$ with support proper over 
$X'_{\overline\eta}$ while the sections of $f_!\overline j_*$ are the same,
only they must now have support proper over the larger space $\overline X'$. Therefore the map is induced simply by the inclusion
$\overline\eta\hookrightarrow\overline S$.
When $f$ is finite it is proper and all these maps are isomorphisms.

(1.3.9) Commute with direct image $\rightsquigarrow$
\cite[Exp. XVIII 3.1.12.3]{SGAA}.

$\overline i^*f^!\ra f^!\overline i^*\rightsquigarrow$
\cite[Exp. XVIII 3.1.14.2]{SGAA}.

$f$ étale $\rightsquigarrow$ \cite[Exp. XVIII 3.1.8]{SGAA}.

(1.3.10) The definition of this arrow is clarified with a diagram in
\cite[3.7]{thfin}, where the derived version of this map is shown to be an
isomorphism, provided you assume $\mathcal F$ is of torsion prime to the
residual characteristic of $S$.

(1.4) The business of securing the injectivity of $\varphi'$ is the usual 
trick (SGA V Exp. XV p.\,479). Namely, the cone of $\spe^*K_s$
(where here $\spe=\spe\circ j$ in the sense of (1.2.2))
is the cone of the identity map on this complex; i.e. the complex
$C(\spe^*K_s)^i:=(\spe^*K_s)^i\oplus(\spe^*K_s)^{i-1}$ with differential
$d^i(x,y)=(d^ix,x-d^{i-1}y)$. The identity map on this complex is homotopic
to 0 via the homotopy $h(x,y)=(y,0)$, so it is acyclic and
$K'_\eta=K_\eta\oplus C(\spe^*K_s)$ is homotopic to $K_\eta$.
Replacing $\varphi$ by $\varphi':=(\varphi,\id,0)$
(as $\spe^*K_s$ coincides with $K_{\overline s}$ with action of
$\Gal(\overline\eta/\eta)$ factoring through $\Gal(\overline s/s)$ so that
$\varphi'$ remains equivariant), $\varphi'$ is now a termwise split 
injection with termwise splitting given by projection to the second factor.
In (1.4.2.2) $\spe$ again coincides with $\spe\circ j$ (1.2.2).
En termes imagés, $\Phi(K)$ keeps track of the sections of $K_\eta$ which 
do not come from residual extension of sections of $K_s$; bref, the
discrepancy between $K_\eta$ and $K_s$.

(2.1.2) `l'image réciproque $F_{\overline\eta}$ est acyclique'
$\rightsquigarrow$ \cite[VII 5.7]{SGAA}.

(2.1.3) $S_{\mathrm{nr}}$ is a strictly henselian trait 
coinciding with the normalization of $S$ in $\eta_{\mathrm{nr}}$.

Claim: $\overline X_{(\overline x)}\simeq
X_{(\overline x)}\times_{S_{\mathrm{nr}}}\overline S$. As
$\overline X=X\times_SS_{\mathrm{nr}}\times_{S_{\mathrm{nr}}}\overline S$
and $X_{(\overline x)}$ coincides with the strict henselization of
$X\times_SS_{\mathrm{nr}}$ at $\overline x$,
$X_{(\overline x)}\times_{S_{\mathrm{nr}}}\overline S$ is pro-étale over
$\overline X$.
On the other hand, let $\eta'$ be a finite separable extension of
$\eta_{\mathrm{nr}}$ and $S'$ the normalization of $S_{\mathrm{nr}}$ in
$\eta'$. Then $X_{(\overline x)}\times_{S_{\mathrm{nr}}}S'$ is finite over
$X_{(\overline x)}$, so splits as a disjoint union of henselian local ring
spectra indexed by the points in the closed fiber
$\overline x\times_{S_{\mathrm{nr}}}S'$.
As the closed fiber of the map $S'\ra S_{\mathrm{nr}}$ is radicial,
the map $\overline x\times_{S_{\mathrm{nr}}}S'\ra\overline x$ is injective
and hence we see that $\overline x\times_{S_{\mathrm{nr}}}S'$ is one point
(\href{https://stacks.math.columbia.edu/tag/01S2}{Stacks tag \texttt{01S2}})
so that $X_{(\overline x)}\times_{S_{\mathrm{nr}}}S'$, and hence by passage 
to the limit $X_{(\overline x)}\times_{S_{\mathrm{nr}}}\overline S$ too,
are strictly henselian local
(\href{https://stacks.math.columbia.edu/tag/04GI}{Stacks tag \texttt{04GI}}).
It is now immediate that the generic fiber of the strict henselization of
$\overline X$ at $\overline x$ coincides with
\begin{equation*}
	(X_{(\overline x)}\times_{S_{\mathrm{nr}}}\overline S)_{\overline\eta}=
	X_{(\overline x)}\times_{\eta_{\mathrm{nr}}}\overline\eta.
\end{equation*}

(2.1.5) Considering $\mathcal F$ as concentrated in degree 0, then so too
is $\spe^*i^*\mathcal F$. We write the
long exact sequence of cohomology associated to the stalk at $\overline x$ 
(technically the stalk at the point $(\overline x,\overline\eta)$ of the
topos $X_s\times_s\eta$) of the distinguished triangle (2.1.2.4)
\begin{equation*}
	0\ra\mathcal F_{\overline x}\xra\sim
	(\overline j_*\overline j^*\mathcal F)_{\overline x}\ra
	R^0\Phi(\mathcal F)\ra0\ra
	R^1\Psi(\mathcal F_\eta)_{(\overline x,\overline\eta)}
	\ra R^1\Phi(\mathcal F)\ra\cdots
\end{equation*}
with the second arrow an isomorphism as $\mathcal F$ is lisse
(c.f. \hyperref[laumon:reprise]{Reprise}).
(2.1.3) gives that 
\begin{equation*}
	R^i\Psi(\mathcal F_\eta)_{(\overline x,\overline\eta)}\simeq
	H^i(X_{(\overline x)}\times_{\eta_{\mathrm{nr}}}\overline\eta,\mathcal F)
\end{equation*}
and the note to (2.1.3) above shows that
$X_{(\overline x)}\times_{\eta_{\mathrm{nr}}}\overline\eta$ is a variety of
vanishing cycles of $f$ at the point $\overline x$ in the sense of
Arcata V 1.3 (more properly it should be called a variety of nearby 
cycles); as $f$ is smooth hence (universally) locally acyclic
(Arcata V 2.1) and the restriction of $\mathcal F$ to $X_{(\overline x)}$
is constant, this same definition 1.3 gives that
$R\Psi(\mathcal F)_{(\overline x,\overline\eta)}$ is
connective (i.e. acyclic in degrees $>0$) so that
$R\Phi(\mathcal F)_{(\overline x,\overline\eta)}=0$.

In the case of a complex $K$ in $D^+(X,\Lambda)$ with $f$ smooth and
the $\mathscr H^i(K)$ lisse, we can reduce to the previous paragraph with
the help of the spectral sequence
\begin{equation*}
	E_{II}^{p,q}=R^p\Psi_\eta(\mathscr H^q(K_s))
	\Rightarrow\mathscr H^{p+q}\Psi_\eta(K_s)
\end{equation*}
which finds that the map $\spe^*i^*K\ra R\Psi_\eta(K_s)$ is a
quasi-isomorphism.

See also \cite[2.12]{thfin}.

(2.1.6) $Rf_*$ is described as a functor
$D^+(Y\times_sS,\Lambda)\ra D^+(Y'\times_sS,\Lambda)$.

c) $f$ quasi-fini $\Rightarrow f_!$ exact $\rightsquigarrow$ \cite[XVII 6.1.4]{SGAA}.

d) $f$ quasi-fini $\Rightarrow Rf^!$ is the right derived functor of
$f^!\rightsquigarrow$ \cite[XVIII 3.1.8 (i)]{SGAA}.

(2.1.7) Climbing to the highest heights of pedantry, to derive (1.3.6.1),
start by representing our $K$ in $D^+(X,\Lambda)$ by a complex of
injectives $I_1$; as $f_*$ preserves injectives, 
$R\Psi\;Rf_*K=\Psi\;f_*I_1$. By (1.3.6.1) we find a morphism of complexes
$\Psi\;f_*I_1\ra f_*\;\Psi I_1$. Taking a quasi-isomorphism
$\Psi I_1\ra I_2$ into a complex of injectives $I_2$ composes to give
\begin{equation*}
	R\Psi\;Rf_*K=\Psi\;f_*I_1\ra f_*\,\Psi I_1\ra f_* I_2=Rf_*\;R\Psi K.
\end{equation*}
The same method gives (2.1.7.2) and the first arrow in the below for
(2.1.7.3).
\begin{equation*}
	Rf_!\;R\Psi=R\tilde f_*\;j_!\;R\Psi\ra R\tilde f_*\;R\Psi\;j_!\ra
	R\Psi\;R\tilde f_*\;j_!=R\Psi\;Rf_!.
\end{equation*}
For $f$ quasi-finite, $f^!$ is right adjoint to the exact functor $f_!$
and so preserves injectives.

Deligne shows \cite[3.7]{thfin} that the formation of nearby cycles $R\Psi$
commutes with change of trait; i.e. the base-change morphism (2.1.7.5) is
an isomorphism (for sheaves of torsion prime to the residual characteristic 
of $S$).
(Confusingly, he calls $R^i\Psi_\eta$ vanishing cycles. This is surely 
because, according to Illusie, \emph{Grothendieck and vanishing cycles},
 when Grothendieck introduced the functors $R\Psi$, $R\Phi$, he
called both these `functors of vanishing cycles.')

(2.1.8)
Typo: $K$ belongs in $D^+(X,\Lambda)$.
When $X'=S$, $\Psi$ induces an equivalence of sheaves on $S$ with sheaves
on the topos $s\times_s S$ (1.2.2 b)) and is therefore omitted.
The $X_s\times_s\eta$-component of the left-hand side of (2.1.8.1) is
\begin{equation*}
	\R\Psi_\eta\;Rf_*K=\overline i^*\overline j_*\overline j^*Rf_*K
	=\overline i^*\overline j_*(Rf_*K)_{\overline\eta}
	=\overline i^*\overline j_*R\Gamma(X_{\overline\eta},K).
\end{equation*}
As in the note to (1.3) above, as $\Gal(\overline\eta/\eta)$-module,
$\overline i^*\overline j_*R\Gamma(X_{\overline\eta},K)
=R\Gamma(X_{\overline\eta},K)$, but the left-hand side can be considered
as sitting on $\overline s$ while the right-hand side sits on
$\overline\eta$. This is pedantry.
In any event, the $X_s\times_s\eta$-component of the right-hand side of
(2.1.8.1) coincides with the right-hand side of (2.1.8.3) in light of
(2.1.6.2), which in itself is pedantic and simply recognizes that the 
object $R\Psi_{\overline\eta}(K)$ sits on $X_{\overline s}$, and we take
the stalk at the point $(\overline x,\overline\eta)$ of $X_s\times_s\eta$
as in (2.1.3). In other words, with $f:X_{\overline s}\ra\overline s$,
\begin{equation*}
	Rf_*\;R\Psi_{\eta}(K)
	=Rf_*\overline i^*\overline j_*(K_{\overline\eta})
	=R\Gamma(X_{\overline s},\overline i^*\overline j_*(K_{\overline\eta})
	=R\Gamma(X_{\overline s},R\Psi_{\overline\eta}(K)).
\end{equation*}
The long exact sequence (2.1.8.9) exists when $f$ is proper because in that
case (2.1.8.3) is an isomorphism.

\subsection*{2.3.2}
The translation of \hyperref[laumon:2.2.1]{the note to (2.2.1.1)} into
$G_{s'}$-modules is that
\begin{equation*}
	\mathscr H^{-1}(K'_{\overline s'})\subset
	F'^{I_{s'}}_{\overline\eta_{s'}},
\end{equation*}
confirming the exactness on the left of the exact sequence of
$G_{s'}$-modules.

(2.3.2.1)
(i) The point is that the Artin-Schreier $\FF_p$-torsor $\mathcal A$
(c.f. \hyperref[laumon:2.3.1]{note to 2.3.1}) is trivialized by
the base change $\GG_{a,k}\xra{t^p-t}\GG_{a,k}$ by the Lang isogeny for
$\GG_{a,k}$, which is a revêtement étale.
Pulling back to $A\times_k A'\ra\GG_{a,k}$ trivializes $\mathscr L(x.x')$
over this revêtement étale and reduces to the stated theorem
(see also \hyperref[sga7:XIII]{note to (2.1.5) in Intermezzo II}).

(In galoisian terms, if $\overline x$ is a geometric point of $\GG_{a,k}$,
$\mathscr L_\psi$ is defined by
\begin{equation*}
	\pi_1(\GG_{a,k},\overline x)\ra\FF_p\xra{\psi^{-1}}\overline\QQ_\ell^\times
\end{equation*}
The Artin-Schreier covering $\GG_{a,k}\xra{t^p-t}\GG_{a,k}$ corresponds to
the open subgroup of $\pi_1(\GG_{a,k},\overline x)$ which coincides with
the kernel of this representation.)

(ii) Use (1.4.1.1) and \cite[Exp. XIII 2.1.7.1]{SGA7} for $R\Psi$;
proper base change for $\spe^*i^*$ and (TR3) gives the desired isomorphism
\begin{equation*}\begin{tikzcd}
	\spe^*i^* R\overline\pr'_*\arrow[r]\arrow[d,"\sim"]&
	R\Psi_{\eta_{s'}} R\overline\pr'_*\arrow[r]\arrow[d,"\sim"]&
	R\Phi R\overline\pr'_*\arrow[r]\arrow[d]&\ \\
	R\overline\pr'_*\spe^*i^*\arrow[r]&
	R\overline\pr'_*R\Psi_{\eta_{s'}}\arrow[r]&
	R\overline\pr'_*R\Phi\arrow[r]&\ 
\end{tikzcd}\end{equation*}
Let $M:=\overline\pr^*(\alpha_!K)\otimes\overline{\mathscr L}(x.x')[1]$.
As $R\overline\pr'_*\spe^*=\spe^*R\overline\pr'_*$, this diagram 
computes to
\begin{equation*}\begin{tikzcd}
	\spe^*K'_{\overline s'}\arrow[r]\arrow[d,"\sim"]&
	R\Psi_{\eta_{s'}}(K')\arrow[r]\arrow[d,"\sim"]&
	R\Phi(K')\arrow[r]\arrow[d]&\ \\
	\spe^*R\Gamma(D_{\overline s'},i^*(M))\arrow[r]&
	R\Gamma(D_{\overline s'},R\Psi_{\overline\eta_{s'}}(M))\arrow[r]&
	R\Phi(M)_{(\overline\infty,\overline s')}\arrow[r]&\ 
\end{tikzcd}\end{equation*}
where here $R\Gamma(D_{\overline s'},i^*(M))=R\overline\pr'_*(i^*M)_{\overline s'}$
for $\overline\pr':D\times_k s'\ra s'$.
En effet, (i) gives that $R\Phi(M)$ is supported on
$\infty\times_k\eta_{s'}$ (the topos; literally as a sheaf it sits on
$\infty\times_k\overline s'\hookrightarrow D_{\overline s'}$);
if $k$ denotes the inclusion of this point into $D\times_k\eta_{s'}$,
$R\Phi(M)=k_*L$ so that
$R\overline\pr'_*R\Phi(M)=R\Gamma(D_{\overline s'},k_*L)=L_{(\overline\infty,\overline s')}$.
Compare with \cite[Exp. XIII 2.1.8]{SGA7} and note to (2.1.8) in \hyperref[sga7:XIII]{Intermezzo II}.

(2.3.2.3) = (2.3.2.1) (iii) + the exact sequence at the top of the
page + (2.3.1.1).

(2.3.3.1) Typos: in the first displayed equation of (ii) the stalk is at
$(\overline s,\overline\infty')$ not $(\overline s,\overline\infty)$.
In the first displayed equation of the proof obviously it is 
$D\times_k\overline\infty'$ not $D\otimes_k\overline\infty'$.

In the proof of (iii), the point is that as the support of
$R\Psi_{\overline\eta_{\infty'}}=R\Phi_{\overline\eta_{\infty'}}$
is contained in $(S\cup\infty)\times_k\eta_{\infty'}$.
Let $M_s$ denote the restriction of
$R\Psi_{\overline\eta_{\infty'}}(\overline\pr^*(\alpha_!K)\otimes\overline{\mathscr L}(x.x')[1])$
to $s\times_k\overline\infty'$
(it is the fiber at $\overline\eta_{\infty'}$ of a sheaf on the topos
$s\times_k\eta_{\infty'}$, hence literally a sheaf on
$s\times_k\overline\infty'$), and put
$s:s\times_kD'_{(\infty')}\hookrightarrow D\times_kD'_{(\infty')}\xra{\pr'}D'_{(\infty')}$
($s\in S\cup\infty$). Then
\begin{equation*}
	R^{-1}\Gamma(D\times_k\overline\infty',R\Psi_{\overline\eta_{\infty'}}(\overline\pr^*(\alpha_!K)\otimes\overline{\mathscr L}(x.x')[1]))
	=\bigoplus_{s\in S\cup\infty}s_{\overline\infty'*}(M_s)
\end{equation*}
with the notation of \cite[2.1.6]{SGA7}. Fix some $s\in S\cup\infty$.
$s_{\infty'}:s\times_k\infty'\ra\infty'$ is a
revêtement étale that realizes
$G_{s\times_k\infty'}\hookrightarrow G_{\infty'}$ as an open subgroup,
and $s\times_k\overline\infty'$ is topologically a disjoint union of copies
of $\overline\infty'$ indexed by the (algebraic) geometric points centered
on $s$ (there are $\deg(s)$ many). As $G_{\infty'}$-module,
\begin{equation*}
	s_{\overline\infty'*}(M_s)
	=G_{\infty'}\otimes_{G_{s\times_k\infty'}}M_s
	=:\operatorname{Ind}^{G_{\infty'}}_{G_{s\times_k\infty'}}(M_s).
\end{equation*}

The proof of (i) is given by (1.3.1.2), the proof of which is relegated to
the paper of Katz-Laumon \emph{Transformation de Fourier et Majoration de
Sommes Exponentiels} and is discussed next.

\subsection*{Intermezzo III: Katz-Laumon (2.4)}
All notation and references in this section are to
\emph{Transformation de Fourier et Majoration de Sommes Exponentiels}
unless otherwise noted.
Let us first discuss how their proof of (2.4.1) gives the theorem (2.4.4)
about universal strong local acyclicity.
With the notation of the proof of (2.4.1), and letting $M$ throughout 
denote constant sheaf with value the $A$-module $M$,
the translation trick allows us to replace $\PP^1\times\A^1$ by
$\PP^1\times\GG_m$.
The claim is trivial away from $\infty\times\GG_m$.
It suffices to show that $\widetilde\pr_1$ is universally locally acyclic
rel. $M\otimes^L j_!\mu^*\mathscr L_\psi$. By (2.4.2), this will be true
iff the same is true rel. $M\otimes^L\tilde f_*A_{\tilde X}$.
By the projection formula for $\tilde f$ \& Leray, this is the same as
$\widetilde\pr_1\circ\tilde f$ being universally locally acyclic
rel. $M$. As proven, there is a neighborhood of $\infty\times\GG_m$ over
which $\widetilde\pr_1\circ\tilde f$ factors as a surjective radicial
morphism followed by a smooth morphism; hence by universal local acyclicity
of a smooth morphism, $\widetilde\pr_1\circ\tilde f$ is universally locally
acyclic rel. any locally constant sheaf, in particular rel. $M$.

In the proof itself, the formula for $[a]^*\mu^*\mathscr L_{\psi}$
depends on \cite[1.7.1]{Trig}. The extension of the Artin-Schreier covering
$\tilde X$ is defined as the finite covering of $\PP^1\times\GG_m$ defined
by $X_0T_1^q-X_0T_1T_0^{q-1}=X_1T_0^qy$. It suffices to verify the various
properties of $\tilde X$ when $S=\FF_q$.
Note that $\tilde X$ is integral as $t^q-t-xy$ doesn't have a root over
$k(x,y)$, and $\FF_q$ acts transitively on the roots by
$t\mapsto t+\gamma,\gamma\in\FF_q$.
Pick any point $\Spec\FF_q(y)$ of $\GG_{m,\FF_q}$, corresponding to a
choice of 
$y\in\overline\FF_q^\times$ or $y$ as a transcendental generator of
$\FF_q(y)/\FF_q$. 
On one hand, the point $\infty\in\PP^1_{\FF_q(y)}$ has one point of
$\tilde X\times_{\GG_m}\Spec\FF_q(y)$ lying over it, since when $X_0=0$,
$T_0$ is nilpotent.
On the other, $T_0=0$ implies $X_0=0$ as then $T_1$ is a unit.
Therefore $T_0$ generates the maximal ideal of the local ring of the point
of $\tilde X\times_{\GG_m}\Spec\FF_q(y)$ lying over
$\infty\times_{\FF_q}\FF_q(y)$, which 
shows that every point of $\tilde f^{-1}(\infty\times_{\FF_q}\FF_q(y))$ is
totally ramified over $\infty\times_{\FF_q}\FF_q(y)$, and also shows that
regardless of whether or not we extend scalars to $\FF:=\overline\FF_q$,
the following is true: The points of $\tilde f^{-1}(\infty\times\GG_m)$
are in bijection with the points of $\infty\times\GG_m$, hence also with
the points of $\GG_m$; given a point $a\in\GG_m$,
the maximal ideal $\mathfrak m$ of the corresponding point of $\tilde X$ is 
generated by $1+\operatorname{codim} a$ elements: if $a$ is closed,
$\mathfrak m$ is generated by $T_0$ and the minimal polynomial of $y$, and
if $a$ is the generic point, $\mathfrak m$ is generated by $T_0$ only.
This implies that $\tilde f^{-1}(\infty\times_{\FF_q}\GG_{m,\FF_q})$ is
geometrically regular over $\FF_q$, hence, in light of the lemma in
\hyperref[BBD:2.2.10]{the note to \cite[2.2.1.0]{BBD}},
smooth over $\FF_q$.

Checking that $\tilde X$ is étale over $\A^1\times\GG_m$ amounts to showing
that for any choice of $x\in\FF,y\in\FF^\times$, the closed subscheme 
$Q\hookrightarrow\PP^1_{\FF_q(x,y)}$ cut out by
$T_1^q-T_1T_0^{q-1}=T_0^qxy$ is étale over $\FF_q(x,y)$.
The subscheme $Q$ has empty intersection with $T_0=0$, and away from
$T_0=0$, $Q$ is defined by the separable polynomial $t^q-t=xy$, and as
above, $\FF_q$ acts transitively on the roots, so that $Q$ either is the
disjoint union of $q$ copies of $\Spec k(x,y)$ or is the spectrum of a 
separable field extension of $k(x,y)$. In any event: étale,
and $\tilde f^{-1}(\A^1\times\GG_m)$ is smooth over $\FF_q$.

The proof of (2.4.2) hits a snag when it is only assumed, as stated, that
there exists a nontrivial additive character $\psi:\FF_q\ra A^\times$.
The representation $f_*A_X$ coincides with the regular $A[\FF_q]$-module
$A[\FF_q]$. If we write multiplicatively, this is the ring $A[x]/(x^q-1)$
acting on itself as a free module of rank 1. The point is that of course
the separable polynomial $x^q-1$ need not split into linear factors over 
$A$ if it has one root distinct from 1 in $A$. What is needed is a 
primitive root of this polynomial; i.e. a root which is not a root of any 
polynomial $x^{p^m}-1$, $m<n$ (suppose $q=p^n$). (Actually, $x^q-1$ splits 
as a product of $n$ polynomials each of degree $p$, as
$(x^{p^m}-1)|(x^{p^{m+1}}-1)$.)
The existence of such a root in $A^\times$ implies that
$A[\FF_q]$ decomposes completely into $q$ $A$-modules of rank 1 indexed by
the $q$ distinct characters $\FF_q\ra A^\times$; conversely, if such a root
fails to exist, there will be irreducible factors of $x^q-1$ of degree 
$d>1$ corresponding to irreducible factors of $A[\FF_q]$ free of rank $d$
as $A$-module.

With this proviso, the decomposition of $f_*A_X$ is achieved, and all
that is left to note for the decomposition of $\tilde f_*A_{\tilde X}$
is that the stalk over a point of the base of the latter are free
$A$-modules of rank which coincides with the number of points in the 
geometric fiber \cite[VIII 5.5]{SGAA}, and we have shown that every 
geometric point of $\infty\times\GG_m$ has precisely one point of
$\tilde X$ lying over it, so that the stalk of $\tilde f_*A_{\tilde X}$ is
free of rank 1 at every geometric point of $\infty\times\GG_m$.
As the direct image of a constant sheaf on the generic point of a normal
integral scheme is constant
(c.f. remark at the end of \hyperref[laumon:reprise]{Reprise}),
\begin{equation*}
	\tilde f_*A_{\tilde X}=\tilde f_*j'_*A_X=j_*f_*A_X
	=A_{\PP^1\times\GG_m}\oplus\big(\bigoplus_{\psi'\ne1}j_*\mu^*\mathscr L_{\psi'}\big)),
\end{equation*}
and by rank considerations we must have 
$j_*\mu^*\mathscr L_{\psi'}=j_!\mu^*\mathscr L_{\psi'}$ for all
nontrivial $\psi$.

To see this last fact directly, observe that given a geometric point 
$\overline x$ centered on a point $x\in\tilde X$ with image
$y\in\infty\times\GG_m$, and letting $\overline y$ denote the image of
$\overline x$, let $\tilde X_{(\overline x)}$, resp.
$Y_{(\overline y)}$ denote the strict henselizations of $\tilde X$, resp.
$Y=\PP^1\times\GG_m$ at $\overline x$ and $\overline y$, respectively.
They are regular local rings.
The map on generic points $\eta_{\overline x}\ra\eta_{\overline y}$ is
still given by $t^q-t=xy$, as this still gives a finite extension of degree 
$q$ of $Y_{(\overline y)}$ so that if $\xi$ denotes the generic point of
$\GG_m$, the uniformizer of $Y_{(\overline y)}\times_{\GG_m}\xi$ is the
$q^{\text{th}}$ power of that of $X_{(\overline x)}\times_{\GG_m}\xi$.
It follows that provided $\psi'\ne1$,
$\Gal(\overline\eta_y/\eta_{\overline y})$ acts nontrivially on the 
reciprocal image of $\mathscr L_{\psi'}$ to $Y_{(\overline y)}$,
as $\Gal(\eta_{\overline x}/\eta_{\overline y})$ does.
As the stalk of $j_*\mathscr L_{\psi'}$ at $\overline y$ coincides with
the invariants of the $\Gal(\overline\eta_y/\eta_{\overline y})$-module
$A$, $j_*(\mathscr L_{\psi'})_{\overline y}=0$.

In the last paragraph of the proof, we can do all this assuming $S=\FF_q$. 
Fix coordinates $(\tau,y)$ on
$\tilde f^{-1}(U)$ via
\begin{equation*}
	\FF_q[\tau,y,y^{-1}]\ra\FF_q[\tau,\xi,y,y^{-1}]/(\xi(1-\tau^{q-1})-\tau^qy).
\end{equation*}
The fiber of this map over the closed subscheme of $\A^1\times\GG_m$ 
defined by $\tau^{q-1}$ is empty. Therefore $(\tau,y)$ factors through
the complement $U'\times\GG_m$ of this closed subscheme, which corresponds
to inverting $(1-\tau^{q-1})$, after which the above map on functions 
becomes an isomorphism. As $\xi=\tau^qy/(1-\tau^{q-1})$ in $(\tau,y)$
coordinates, $\tilde\pr_1\circ\tilde f$ takes the stated form.
The following diagram commutes.
\begin{ceqn}\begin{equation*}\begin{tikzcd}
	&&[50pt]\tilde f^{-1}(U)\arrow[d,"{(\tau,y)}"',"{\rotatebox{-90}{\text{$\sim$}}}"]\arrow[dll,"\tilde\pr_1\circ\tilde f"',bend right=10] \\
	\A^1&U'\times\GG_m\arrow[l,"\upsilon"']&U'\times\GG_m\arrow[l,"\kappa"'] \\
	\FF_q[\xi]\arrow[r,"\xi\mapsto\tau'y'"]
	&\FF_q[\tau',y',y'^{-1},(1-\tau'^{q-1})^{-1}]\arrow[r,"{(\tau',y')\mapsto(\tau^q,y/(1-\tau^{q-1}))}"]
	&\FF_q[\tau,y,y^{-1},(1-\tau^{q-1})^{-1}]
\end{tikzcd}\end{equation*}\end{ceqn}
The map $\upsilon:U'\times\GG_m\ra\A^1$ is smooth since the coordinate ring 
of the former can be written as
$S:=\FF_q[\xi,y',a,b]/(ay'-1,b(1-\xi a)-1)$, and
\begin{equation*}
	\det\begin{pmatrix}y'&-b\xi \\ 0 & 1-\xi a \end{pmatrix}
	=y'(1-\xi a)
\end{equation*}
is invertible in $S$ with inverse $ab$, so that $\FF_q[\xi]\ra S$ is
standard smooth and hence smooth
(Stacks tags
\href{https://stacks.math.columbia.edu/tag/00T6}{\texttt{00T6}} \&
\href{https://stacks.math.columbia.edu/tag/00T7}{\texttt{00T7}}).

To verify that $\kappa$ is radicial and surjective, it suffices to show
$\kappa$ is bijective and induces purely inseparable residual extensions.
This can be done by verifying that for each $x\in U'\times\GG_m$,
the scheme-theoretic fiber $\kappa^{-1}(x)$ is isomorphic to the spectrum
of an artinian local ring with purely inseparable residual extension.

Picking a point $x$ amounts to choosing a field
$\FF_q(\alpha,\beta)$ such that $\alpha^{q-1}\ne1$;
as this is trivially satisfied when $\alpha$ is transcendental and
Frobenius is an isomorphism $\FF_q(\alpha)$ when $\alpha$ is algebraic,
$\alpha^{q-1}\ne1$ is equivalent to the condition that $\alpha\not\in\FF_q$.
The fiber $\kappa^{-1}(x)$ in this case is isomorphic to the spectrum of 
the finite $\FF_q(\alpha,\beta)$-algebra
\begin{align*}
	A:=\;&\FF_q(\alpha,\beta)[\tau,y,y^{-1},(1-\tau^{q-1})^{-1}]/(\tau^q-\alpha,y-\beta(1-\tau^{q-1})) \\
	\simeq\; &\FF_q(\alpha,\beta)[\tau,(1-\tau^{q-1})^{-1}]/(\tau^q-\alpha);
\end{align*}
$\alpha$ (resp. $\beta$) is algebraic over $\FF_q$ iff $x$ does not
map to the generic point of $U'$ (resp. $\GG_m$).
If $\alpha$ is transcendental then
\begin{equation*}
	A\simeq\FF_q(\alpha,\beta)[\tau]/(\tau^q-\alpha)
\end{equation*}
is a field which is purely inseparable over $\FF_q(\alpha,\beta)$.
If $\alpha$ is algebraic then Frobenius induces an isomorphism
of $\FF_q(\alpha)$ and there exists a
$\sqrt[\uproot{4}q]\alpha\in\FF_q(\alpha)$ so that
$\tau^q-\alpha=(\tau-\sqrt[\uproot{4}q]\alpha)^q$;
$A$ is therefore isomorphic to a localization of a local Artinian ring with
trivial residual extension, and we need check only that $\kappa^{-1}(x)$
is nonempty; i.e. that $A\ne0$.
This is true iff $1-\tau^{q-1}$ is not nilpotent, in which case it's 
already a unit in $A$. As $\alpha\not\in\FF_q$ by assumption and
$\alpha\ne \sqrt[\uproot{4}q]\alpha$ so that
$(\sqrt[\uproot{4}q]\alpha)^{q-1}\ne1$ and $1-\tau^{q-1}$ indeed
doesn't go to zero under $\tau\mapsto\sqrt[\uproot{4}q]\alpha$.

\subsection*{2.4.1}\label{laumon:2.4.1} Back to Laumon.
The point is that $\pi/\pi'$, $\pi'/\pi$, and $1/\pi\pi'$
determine three distinct rational maps $T\times_kT'\ra\GG_{a,k}$
(regular over $T\times\eta',\eta\times T',\eta\times\eta'$, respectively), 
and $\mathscr L_\psi(\pi/\pi')$, $\mathscr L_\psi(\pi'/\pi)$, and
$\mathscr L_\psi(1/\pi\pi')$ are the reciprocal images of $\mathscr L_\psi$
defined over these open loci of $T\times T$ via these various maps.

By way of preface, it would seem from the discussion below that $T$
implicitly is assumed to coincide with $\Spec k\{x\}$.
Certainly it would seem that Néron desingularization should not be 
necessary for the result (2.4.2.1) cited as `immediate,' or a somewhat
esoteric result about the Galois group of generic points of henselian 
d.v.r.s to connect (2.4.2.2) to the discussion in (2.2.2).
Perhaps there is an implicit assumption I've missed, or far simple ways to
obtain these results with no assumption on $T$ other than
$k\{\pi\}\subset R\subset k[[x]]$.
As remarked in \hyperref[laumon:2.1.1]{the note to (2.1.1)}, one must
assume $k\subset R$; in contrast with an equicharacteristic complete 
d.v.r., an equicharacteristic henselian d.v.r. need not contain a
coefficient field.

(2.4.2.1) As (i), (ii), and (iii) are interchangeable, we dissect (i)
as an example.
The map $\pi':\eta'\ra\eta_{\infty'}$ is specified by
$\pi'\mapsto1/x'$. As $1/x'$ is a uniformizer for the local ring of $D'$
at $\infty'$ (and also its henselization $D'_{(\infty')}$), from the data 
of $\pi'\mapsto1/x'$ we get a morphism $T'\ra\Spec\mathcal O_{D',\infty'}$
and by the universal property of henselization
(Raynaud, \emph{Anneaux Locaux Henséliens}, VIII Déf. 1) a unique
factorization of this morphism to a commutative triangle
\begin{equation*}\begin{tikzcd}
	\Spec\mathcal O_{D',\infty'}&D'_{(\infty')}\arrow[l] \\
	&T'\arrow[ul]\arrow[u,"\pi'"']
\end{tikzcd}\end{equation*}
which recovers $\pi'$ on the generic fiber.
Note that $R\Psi_{\eta_{\infty'}}=R\Phi_{\eta_{\infty'}}$ as in (2.3.3.1).
It is asserted that the base-change morphism
\begin{equation*}
	(\pi\times\pi')^*R\Psi_{\eta_{\infty'}}\ra
	R\Psi_{\eta'}(\pi\times\pi')^*
\end{equation*}
is an isomorphism. As $\pi\times\pi'=(1\times\pi')\circ(\pi\times1)$, we
first deal with $(1\times\pi')^*$, which is the morphism induced by change
of trait $\pi':T'\ra D'_{(\infty')}$ and is an isomorphism by
\cite[3.7]{thfin}.
On the other hand, letting $V_0:=k[\pi]_{(\pi)}$ and $T_0:=\Spec V_0$, the
morphism $\pi:T\ra A$ factors as $T\xra{\pi_1}T_0\ra A$ where the second
morphism is ind-étale and therefore commutes with base change by passage to 
the limit. It doesn't seem `immediate' to me that
\begin{equation*}
	(\pi_1\times1)^*R\Psi_{\eta'}\ra
	R\Psi_{\eta'}(\pi_1\times1)^*
\end{equation*}
is an isomorphism. The cleanest way I know to show this is to
invoke Néron desingularisation in the form \cite[I 0.5.1]{SGA7}.
In the paragraphs following the statement of this lemma (0.5.1),
it's explained how the conditions of the lemma are satisfied in the case of
$\pi_1:T\ra T_0$. The residual extension is trivial in our case; to see
that the extension $\eta/\eta_0$ is separable, it suffices to show that
$k(\eta)/k(\pi)$ is separable. For this one need only show that
$k(\pi^{1/p})\otimes_{k(\pi)}k(\eta)\simeq k(\eta)[t]/(t^p-\pi)$ is
separable. Assuming $p>1$, this is true iff $\pi$ does not have a
$p^{\text{th}}$ root in $k(\eta)$, which is true as $\pi$ is a uniformizer.
Néron says that $\pi_1$ is a limit of smooth morphisms and therefore
commutes with $R\Psi$.

On a related note, I know of a funny way to show the more general
\begin{lemma*}
	Let $f:T\ra T_0$ be a surjective morphism of henselian traits and suppose
	\begin{enumerate}[label=(\greek*)]
	\item the special fiber $T\times_{T_0}s_0$ is reduced,
	\item $k(s_0)\subset k(s)$ is purely inseparable, and
	\item the maximal purely inseparable extension of the completion $\widehat{k(\eta_0)}$ is dense in $\widehat{k(\eta_1)}$.
	\end{enumerate}
	Then for every scheme $X_0\ra T_0$,
	$f$ induces an equivalence of topoi
	$\widetilde X_{\acute{\mathrm{e}}\mathrm t}\ra \widetilde{(X_{0})}_{\acute{\mathrm{e}}\mathrm t}$, where $X:=X_0\times_{T_0}T$.
\end{lemma*}
This applies in the situation where $T_0=\Spec R_0$ is equicharacteristic
and $R_0$ contains a coefficient field mapping isomorphically onto its 
residue field, which is assumed perfect, as assumed in (2.4.1),
because of the inclusion $k\{\pi\}\subset R_0\subset k[[\pi]]$.
\begin{proof}\label{pf:berkovich_lemma}
By (2.4.1–2.4.3) in  Berkovich,
\emph{Étale cohomology for non-Archimedean analytic spaces},
Publ. Math. IHES 78,
reciprocal image $L\mapsto L\otimes_{k(\eta_0)}k(\eta)$ induces an
equivalence of categories of finite separable extensions of $k(\eta_0)$ and
$k(\eta)$, and hence induces an isomorphism
$\Gal(\overline\eta/\eta)\xra\sim\Gal(\overline\eta/\eta_0)$.
The same is true for the closed fiber (replacing $\eta$ by $s$) if we
restrict to finite Galois extensions by e.g.
\href{https://stacks.math.columbia.edu/tag/030M}{Stacks tag \texttt{030M}}.
Recall \cite[IV 9.5.4]{SGAA}, which says in our case that the category of 
sheaves on $X\ra T$ is equivalent to the category of triples
\begin{equation*}
	(\mathcal F_s,\mathcal F_\eta,\varphi:\mathcal F_s\ra i^*j_*\mathcal F_\eta)
	\qquad\text{where}
\begin{tikzcd}\eta\arrow[r,hook,"j"]&T&s,\arrow[l,hook',"i"']\end{tikzcd}
\end{equation*}
($\mathcal F_s$ a sheaf on $X\times_T s$,
$\mathcal F_\eta$ a sheaf on $X\times_T \eta$)
via the functor $\mathcal F\mapsto(\mathcal F_s,\mathcal F_\eta,\varphi)$
where
\begin{equation*}
	\varphi:\mathcal F_s=i^*\mathcal F\ra i^*j_*j^*\mathcal F
	=\operatorname{sp}_*\mathcal F_\eta,
\end{equation*}
with quasi-inverse the functor which associates to 
$(\mathcal F_s,\mathcal F_\eta,\varphi)$ the object $W$ defined by the
cartesian square
\begin{equation*}\begin{tikzcd}
	W\arrow[d]\arrow[r]\arrow[dr,phantom,"\ulcorner" near start]&j_*\mathcal F_\eta\arrow[d] \\
	i_*\mathcal F_s\arrow[r,"i_*\varphi"]&i_*i^*j_*\mathcal F_\eta.
\end{tikzcd}\end{equation*}
Decorating everything with a subscript 0, the same is true for
$X_0\ra T_0$.
Last, recall \cite[XIII 1.3.3 (ii)]{SGA7}, which says that $Y\ra\Spec k$
with separable closure $\overline k$ and a sheaf $\mathcal F$ on $Y$,
the functor $\mathcal F\mapsto\overline{\mathcal F}$ induces an equivalence
between the category of sheaves of sets on $Y$ with that of sheaves of sets
on $\overline Y$ equipped with a continuous action of $\Gal(\overline k/k)$
compatible with the action of $\Gal(\overline k/k)$ on $\overline Y$.
\end{proof}

(2.4.2.2) By the Berkovich result in the proof of the previous lemma,
we may assume $T=\Spec k\{x\}$; now invoke not (2.2.2.1) but rather the
discussion following (2.2.2.2). A choice of retraction for $i_*$ (in the
notation of (2.2.2)) corresponds to an extension of $V$ to a lisse
$\overline\QQ_\ell$-sheaf $\mathcal F$ on $\GG_{m,k}$.
Let $j:\GG_{m,k}\hookrightarrow A$; for (i) put $K:=j_!\mathcal F$ and for
(ii) and (iii) put $\alpha_!j_!\mathcal F[1]$. In all three cases,
$K$ is a perverse sheaf \cite[4.1.3]{BBD} and $\pi^*K=V_![1]$.

\subsection*{2.5}
In order to complete the proof (4.4) of Deligne's main theorem in Weil II,
it is only necessary to have (2.5.3.1) (i) (Kummer does not make an 
appearance). The proof of (2.5.3.1) (i) is an elegant synthesis of
(2.3.2)–(2.4.2), and the only necessary auxiliary computation is that of
$\mathscr F(\overline\QQ_{\ell,A-\{0\}})|\eta_{\infty'}$. There is an exact sequence of
perverse sheaves on $A$ \cite[4.1.10]{BBD}
\begin{equation*}
	0\ra\overline\QQ_{\ell,\{0\}}
	\ra\overline\QQ_{\ell,A-\{0\}}[1]
	\ra\overline\QQ_{\ell,A}[1]\ra0.
\end{equation*}
Applying $\mathscr F$ gives the exact sequence of perverse sheaves on $A'$
\begin{equation*}
	0\ra\overline\QQ_{\ell,A'}[1]
	\ra\mathscr F(\overline\QQ_{\ell,A-\{0\}})[1]
	\ra\overline\QQ_{\ell,\{0'\}}(-1)\ra0
\end{equation*}
which finds
$\mathscr F(\overline\QQ_{\ell,A-\{0\}})|\eta_{\infty'}= \overline\QQ_{\ell,\eta_{\infty'}}$ and completes the proof of (i).


\subsection*{4.2.2}
(4.2.2.2) Write $\mathscr F:=F$ to harmonize with Deligne's notation.
The condition that `the action of $I_s$ be unipotent of echelon 2.'
corresponds, in the vocabulary of \cite[1.7.2]{weilii}, to the property that
the filtration of local monodromy on $\mathscr F_{\overline\eta}$ has at
most 3 nonzero graded pieces; i.e. if $I$ does not act trivially on
$\mathscr F_{\overline\eta}$, then the nilpotent operator $N$ arising from
the logarithm of the unipotent part of the local monodromy has $N\ne0$ but
$N^2=0$. In this case, as described in \cite[1.6.1]{weilii}, the filtration
has $M_1=\mathscr F_{\overline\eta}$, $M_0=F_{\overline\eta}^I$, and
$M_{-1}=\im N$, and $\Gr_1^M=F_{\overline\eta}/F_{\overline\eta}^I$, which
\cite[1.8.4]{weilii} says is $\iota$-pure of weight $w+1$.

(4.2.2.3) Recall the decomposition (1.6.14.3) of Weil II and the fact that
there is a typo in it (c.f. \hyperref[weilii:1.6.14]{note to 1.6.14}).
Recall that, as in the proof of \cite[1.8.4]{weilii}, if $s\in S$, the fiber
$(j_*\mathscr F)_{\overline s}$ coincides with
$\ker N=\mathscr F^I_{\overline\eta}$, where $N$ is the logarithm of the
local monodromy (we can assume that all of $I$ acts unipotently).
Recall that $N$ is compatible with the filtration of local monodromy,
that $\Gr_i^M(\ker N)\xra\sim P_i$ \cite[1.6.6]{weilii}, and that
$P_i=0$ for $i>0$ \cite[1.6.4]{weilii}. The eigenvalues of $F_s$ on
$j_*(\mathscr F)_{\overline s}$ are in bijection with the eigenvalues of 
$F$ on the $P_i(\mathscr F_{\overline\eta})$, where $F$ is the
conjugation class of liftings of Frobenius in the Weil group
(or one such \cite[1.7.4]{weilii}). Now Laumon's proof goes through after
you correct all the typos: $P_{-i}(\mathscr F_{\overline\eta})$ is
$\iota$-pure of weight $w+i$ \cite[1.8.4]{weilii},
\begin{equation*}
	P_{-i}(\check{\mathscr F}_{\overline\eta})
	\simeq P_{-i}(\mathscr F_{\overline\eta})^\vee(i),
\end{equation*}
and $\alpha$ is an eigenvalue of Frobenius on some
$P_{-i}(\mathscr F_{\overline\eta})$ (for this you need to know that the filtration of local monodromy is stable under the action of $W(\overline\eta,\eta)$, which is true by \cite[1.7.5, 1.8.5]{weilii}).

\subsection*{4.3.1}
(4.3.1.1) It is implicitly claimed that
$H^0(U,\mathscr F)=0$, which is a consequence of the fact that $\mathscr F$
is assumed irredudible and not geometrically constant.
If $\overline u$ is a geometric point of $\overline U=U\times_{\FF_q}\FF$,
there is an exact sequence
\begin{equation*}
	e\ra\pi_1(\overline U,\overline u)\ra\pi_1(U,\overline u)\ra\Gal(\FF/\FF_q)\ra e
\end{equation*}
(SGA 1 6.1), and $\mathscr F$ is not geometrically constant if
$\pi_1(\overline U,\overline u)$ does not act trivially on $\mathscr F_{u}$;
i.e. the reciprocal image of $\mathscr F$ on $\overline U$ is not constant.
As $\mathscr F_{\overline u}^{\pi_1(\overline U,\overline u)}$ is stable
under $\pi_1(U,\overline u)$, it must be 0, as $\mathscr F$ is assumed
irreducible, so $\Gamma(U,\mathscr F)=0$.

\subsection*{4.3.2}\label{laumon:4.3.2}
Following Deligne, we decorate with a subscript 0
those objects over $\FF_q$ and remove this subscript to indicate the
extension of scalars to $\FF$.

(4.3.2.1) Reduction to $X_0=D_0$. We notate $j_0:U_0\hookrightarrow X_0$ 
and $\mathscr F_0$ lisse on $U_0$.
The choice of a nonconstant meromorphic function on $X_0$ gives rise to a
finite morphism $f_0:X_0\ra D_0$.
The only condition on the function is that it induce a morphism with 
nonempty étale locus. As $X_0$ is smooth, this is easy: just pick a point
$x\in X_0$ and a generator of the local ring $\mathscr O_{X,x}$
(SGA 1 I 9.11). Leray gives
\begin{equation*}
	R\Gamma(D,f_{0*}j_*\mathscr F_0)\simeq R\Gamma(X,j_*\mathscr F_0).
\end{equation*}
It suffices to find some open $j'_0:U_0'\hookrightarrow D_0$ a lisse
$\overline\QQ_\ell$-sheaf $\mathscr F'_0$ on $U_0'$, $\iota$-pure of weight 
$w$, and an identification
$f_{0*}j_{0*}\mathscr F_0\simeq j_{0*}'\mathscr F_0'$ on $D_0$.
\begin{lemma*}
	Let $\mathscr F$ be a lisse sheaf on a normal connected curve $S$ and
	$j:U\hookrightarrow S$ an open immersion.
	The unit of adjunction $\mathscr F\ra j_*j^*\mathscr F$ is an 
	isomorphism.
\end{lemma*}
Admitting the lemma, pick some nonempty $U'_0$ such that
$X_0\times_{D_0}U_0'\ra U_0'$ is étale, and let
$u_0:V_0:=U_0\cap(X_0\times_{D_0}U_0')\hookrightarrow U_0$.
As the diagram
\begin{equation*}\begin{tikzcd}
	V_0\arrow[r,"u_0"]\arrow[d,"f_0|V_0"]
	&U_0\arrow[r,"j_0"]&X_0\arrow[d,"f_0"] \\
	U_0'\arrow[rr,"j_0'"]&&D_0
\end{tikzcd}\end{equation*}
commutes,
\begin{equation*}
	j'_{0*}(f_0|V_0)_*(u_0^*\mathscr F_0)
	=f_{0*}j_{0*}u_{0*}u_0^*\mathscr F_0
	\simeq f_{0*}j_{0*}\mathscr F_0,
\end{equation*}
and the sheaf $(f_0|V_0)_*(u_0^*\mathscr F_0)$ is lisse. More precisely,
if $\overline\eta$ denotes a geometric generic point of $V_0$
(and its image in $U_0'$), and $u_0^*\mathscr F_0$ is defined by the 
monodromy representation
$\pi_1(V_0,\overline\eta)\ra\Aut\mathscr F_{\overline\eta}$, the étale
$f_0|V_0$ induces a morphism
\begin{equation*}
	\pi_1(V_0,\overline\eta)\ra\pi_1(U_0',\overline\eta)
\end{equation*}
which is an injection of the former group onto the open subgroup of the
latter corresponding to the revêtement étale $f_0|V_0$.
The sheaf $(f_0|V_0)_*(u_0^*\mathscr F_0)$ is defined by the induced
representation
\begin{equation*}
	\mathscr F_{\overline\eta}\otimes_{\pi_1(V_0,\overline\eta)}\pi_1(U_0',\overline\eta).
\end{equation*}
Therefore we take $\mathscr F_0':=(f_0|V_0)_*(u_0^*\mathscr F_0)$.
\begin{proof}[Proof of lemma]
	The lemma can be recovered as a corollary of
	\hyperref[laumon:reprise]{the \emph{reprise} below},
	and, perhaps more cheekily, from \cite[4.3.2]{BBD} in light of
	\hyperref[laumon:1.4.2]{the note to 1.4.2} above.
	A simple direct proof goes as follows.
	Checking the statement fiberwise at a geometric point $\overline s$
	centered on a point $s\in S-U$ reduces us to the setting of
	a henselian trait $(S,\eta,s,\overline\eta,\overline s)$
	\cite[0.6]{weilii} and
	$\mathscr F$ a lisse sheaf on $S$. We have the usual exact sequence
	\begin{equation*}
		e\ra I\ra\Gal(\overline\eta/\eta)\ra\Gal(\overline s/s)\ra e.
	\end{equation*}
	where by SGA 1 V 8.2 \& Arcata IV 2.2 the map
	$\Gal(\overline\eta/\eta)\twoheadrightarrow\Gal(\overline s,s)$
	factors as
	\begin{equation*}
		\Gal(\overline\eta/\eta)\twoheadrightarrow\pi_1(S,\overline\eta)
		\xra\sim\pi_1(S,\overline s)
		\xra\sim\pi_1(s,\overline s)\simeq\Gal(\overline s,s).
	\end{equation*}
	The stalk $(\eta_*\eta^*\mathscr F)_{\overline s}$ can be identified
	with $\mathscr F_{\overline\eta}^I$, where here
	$\mathscr F_{\overline\eta}$ is the
	$\Gal(\overline\eta,\eta)$-representation defined by
	$\Gal(\overline\eta,\eta)\ra\pi_1(S,\overline\eta)$.
	The factorization above shows that $I$ dies in this quotient and 
	therefore acts trivially.
\end{proof}
Returning to the proof of (4.3.2.1), Poincaré duality on $D$ as stated rests
on
\begin{theorem*}[Deligne, SGA 4$\frac12$ \emph{Dualité} 1.3]
	Let $j:U\hookrightarrow S$ be a dense open of a regular scheme $S$ purely
	of dimension 1 and $\mathscr F$ a locally constant constructible sheaf
	of $\ZZ/n$-modules on $U$. One has $Dj_*\mathscr F=j_*D\mathscr F$,
	i.e. $\underline\Hom(j_*\mathscr F,\ZZ/n)=j_*\underline\Hom(F,\ZZ/n_U)$
	and $\underline\Ext^i(j_*\mathscr F,\ZZ/n)=0$ for $i>0$.
\end{theorem*}
Let $\overline A:=A\otimes_{\FF_q}\overline k$ and $\overline D$ likewise.
To calculate $H^i_c(\overline A,j_*\mathscr F)$, the perfect pairing
\begin{equation*}
	H^i_c(\overline A,j_*\mathscr F)\otimes H^{2-i}(\overline A,j_*\check{\mathscr F})\ra\overline\QQ_\ell(-1)
\end{equation*}
and Artin's theorem give $H^0_c(\overline A,j_*\mathscr F)\simeq0$ and
$H^2_c(\overline A,j_*\mathscr F)\simeq M(-1)$,
as $H^0(\overline A,j_*\check{\mathscr F})=\check M$.
The exact sequence
\begin{equation*}
	0=H_c^0(\overline A,j_*\mathscr F)\ra H^0(\overline D,j_*\mathscr F)
	\ra\mathscr F_{\overline\infty}\xra\partial H_c^1(\overline A,j_*\mathscr F)
	\ra H^1(\overline D,j_*\mathscr F)\ra0
\end{equation*}
reduces the calculation of $H^1_c(\overline A,j_*\mathscr F)$ to that of
$H^1(\overline D,j_*\mathscr F)$, as $\partial=0$.
Relative purity (Arcata V 3.4) gives that 
$j_*\mathscr F$ is constant on $\overline A$, and the standard computation
$H^1(\overline X,\ZZ/\ell^n(1))\simeq\underline{\operatorname{Pic}}(\overline X/\overline k)_{\ell^n}$ for a smooth connected curve
$\overline X$ over $\overline k$ shows that $H^1(\overline D,\ZZ/\ell^n)=0$
and hence the same is true for $H^1(\overline D,j_*\mathscr F)$
(Arcata III 3.1, IV 6.2).








\subsection*{Reprise: le sorite des faisceaux localement constants}\label{laumon:reprise}
	The action of local inertia on a normal curve $S$ determines the net
	locus (and therefore the étale locus by SGA 1 I 9.11) of the 
	normalization of $S$ in a finite separable extension of its function 
	field in the following way.
	Let $\overline\eta$ be a geometric point centered on the generic point
	$\eta$ of $S$, $L$ a finite Galois extension of the function field of
	$S$, and $S'$ the normalization of $S$ in $L$.
	Let $\overline s$ be a geometric point centered on a closed point of $S$,
	$S_{(s)}$ and $S_{(\overline s)}$ the spectra of the henselization and
	strict henselization, respectively of the local ring of $S$ at $s$.
	Both are henselian traits.
	Let $S'_{(s)}:=S'\times_SS_{(s)}$; $S'_{(s)}\ra S_{(s)}$ is generically
	étale.
	As $S_{(s)}$ is a henselian local ring, $S'_{(s)}$ splits as a disjoint
	union of henselian local rings $S'_{(s_i')}$ indexed by
	$\xi_i\in\Spec L\otimes_{k(\eta)}k(\eta_s)$ (in bijection with the
	scheme-theoretic fiber of $S'\ra S$ over $s$)
	Let $s'_i$ denote the corresponding closed point of $S'_{(s_i')}$
	(and by abuse of notation the corresponding point in the scheme-theoretic
	fiber of $S'\ra S$ over $s$).
	$S'_{(s_i')}$ is pro-étale over $S'$ and henselian, with the same residue
	field as $\mathscr O_{S',s_i'}$, therefore coincides with the
	henselization of $\mathscr O_{S',s_i'}$, hence is also normal.
	$S'_{(s_i')}$ is finite and generically étale over $S_{(s)}$,
	\emph{a fortiori} integral. Hence $S'_{(s_i')}$ coincides with the
	normalization of $S_{(s)}$ in $k(\xi_i)$.
	Considering $s_i'$ as a point in the scheme-theoretic fiber over $s$,
	$S'\ra S$ is étale at $s_i'$ iff $S_{(s_i')}\ra S_{(s)}$ by faithfully
	flat descent
	(\href{https://stacks.math.columbia.edu/tag/02VN}{Stacks tag \texttt{02VN}}).
	Fixing some $i$ and letting $s':=s_i',\xi:=\xi_i$ we reduce to studying
	the normalization of the henselian trait $S_{(s)}$ in a finite separable
	extension $k(\xi)$ of its function field.
	Let $\overline\eta_s$ be a geometric point centered on the generic
	point $\eta_s$ of $S_{(s)}$.
	The local inertia at $s$ is defined by the exact sequence
	\begin{equation*}
		e\ra I_{s}\ra\Gal(\overline\eta/\eta_s)\ra\Gal(\overline s/s)\ra e.
	\end{equation*}
	Compare the following lemma to \cite[Exp.\,V \S2]{SGA1}.
	\begin{lemma*}
		$S_{(s')}\ra S_{(s)}$ is étale iff $I_s$ acts trivially on $S_{(s')}$.
	\end{lemma*}
	\begin{proof}
	`$\Rightarrow$' Arcata IV 2.2.
	`$\Leftarrow$' $k(\xi)$ is a finite separable extension of $k(\eta_s)$;
	let $K$ denote the Galois closure of $k(\xi)$ in $k(\overline\eta)$,
	$S_{(s)}=:\Spec A,S_{(s')}=:\Spec A'$, and $B$ the normalization of $A$ 
	in $K$. $A\subset (A')^{I_s}=A'\subset B^{I_s}$ and $A\subset B^{I_s}$ is
	étale (\href{https://stacks.math.columbia.edu/tag/09EH}{Stacks tag \texttt{09EH}}),
	so $S_{(s')}\ra S_{(s)}$ is étale.
	\end{proof}


	A question that should be easy is: describe the category of lisse sheaves
	on a normal connected curve in terms of the Galois group of the function
	field. More precisely, given a lisse sheaf on an open subscheme $S$
	of a smooth complete curve $\overline S$, describe the locus of
	$\overline S$ over which the sheaf can be extended to a lisse sheaf.
	In the case of finite coefficients, the answer goes like this.
	Let $\eta$ be a geometric point centered on the generic point $\eta$
	of the curve $S$. A l.c.c. sheaf $\mathscr F$ on $S$ is the same as a
	revêtement étale $X\ra S$; i.e. an open subgroup of
	$\pi_1(S,\overline\eta)$.
	$S$ in turns corresponds to a closed subgroup $Q$ of
	$\Gal(\overline\eta,\eta)$ via the exact sequence
	\begin{equation*}
		e\ra Q\ra\Gal(\overline\eta,\eta)\ra\pi_1(S,\overline\eta)\ra e.
	\end{equation*}
	To each point $s\in S$ we attach an inertia subgroup $I_s$ which is a
	subgroup of $\Gal(\overline\eta/\eta_s)$, where $\eta_s$ is the generic
	point of $S_{(s)}$. As $S_{(s)}$ is a projective limit of revêtements
	étales of $S$, $I_s$ embeds into
	$\Gal(\overline\eta/\eta)$ via
	\begin{equation*}
		I_s\subset\Gal(\overline\eta/\eta_s)\subset\Gal(\overline\eta/\eta).
	\end{equation*}
	We must have that each $I_s$ be contained in $Q$, since as we saw above,
	the map $\Gal(\overline\eta/\eta_s)\twoheadrightarrow\Gal(\overline s/s)$
	factors through the projection
	$\Gal(\overline\eta/\eta_s)\twoheadrightarrow\pi_1(S,\overline\eta)$.
	On the other hand, there may be other points in $\overline S-S$ with
	nontrivial monodromy; i.e. the corresponding inertia acts nontrivially.
	This means that the direct image along $S\hookrightarrow\overline S$ of
	the sheaf on $S$ represented by $X$ is not locally constant on a
	neighborhood of such a point.
	Geometrically, we can take the normalization $X'$ of $\overline S$ in the
	function field of $X$; $X'\times_{\overline S}S\simeq X$ (SGA 1 I 10.2).
	$X'$ is étale at a point if it is net there (SGA 1 I 9.11), so
	$\mathscr F$ extends to a lisse sheaf over $U\hookrightarrow\overline S$
	iff $X'\times_{\overline S}U\ra U$ is net.
	By the lemma above, this is true iff the inertia at each point $u\in U$
	acts trivially on the $\Gal(\overline\eta/\eta_u)$-representation
	corresponding to the sheaf on $\overline S_{(u)}$ represented by
	$X'\times_{\overline S}\overline S_{(u)}$.
	Properly said, $I_u$ acts on the fiber of this sheaf at $\overline\eta$.
	If $I_u$ acts trivially, then $X'$ is net in the fiber over $u$ and $X$ 
	extends (via $X'$) to a revêtement étale of
	$S\cup\{u\}$. (By the above lemma, is isomorphic to the direct image
	of $\mathscr F$ under $S\hookrightarrow S\cup\{u\}$.)
	Thinking about $I_u$ as a subgroup of $\Gal(\overline\eta,\eta)$,
	the condition that $I_u$ act trivially on
	$X'\times_{\overline S}\overline S_{(u)}$ is equivalent to that $I_u$
	act trivially on the
	$\Gal(\overline\eta/\eta)$-representation $\mathscr F_{\overline\eta}$.
	Therefore, the locus of $\overline S$ over which $\mathscr F$ can be
	extended to a lisse sheaf coincides with the union of the
	$s\in\overline S$ such that $I_s$ acts trivially on
	$\mathscr F_{\overline\eta}$. This justifies the
\begin{proposition*}
	Let $S$ be a normal connected curve with generic point $\eta$ 
	and $\mathscr C$ the category with objects pairs $(\mathscr F,U)$ with 
	$U$ a nonempty open of $S$ and $\mathscr F$ a l.c.c. sheaf on $U$,
	modulo the equivalence relation $(\mathscr F,U)\sim(\mathscr G,V)$ if
	$\mathscr F|U\cap V\simeq\mathscr G|U\cap V$, and morphisms
	\begin{equation*}
		\Hom_{\mathscr C}((\mathscr F,U)\ra(\mathscr G,V))
		=\varinjlim_{W\subset U\cap V}\Hom(\mathscr F|W,\mathscr G|W),
	\end{equation*}
	limit taken over nonempty opens $W$ contained in $U\cap V$.
	$\mathscr C$ is equivalent to the category of finite separable extensions
	of the function field $k(\eta)$ of $S$ and $k(\eta)$-algebra morphisms.
	By Grothendieck's Galois theory, this category is in turn equivalent to
	the category of finite sets with continuous $\Gal(\overline\eta/\eta)$
	action.
	Given such an extension $L$ of $k(\eta)$, the normalization $X$ of $S$ in
	$L$ is étale over a nonempty open $U\subset S$ and represents a l.c.c.
	sheaf on $U$. The maximal $U\subset S$ such that $X\times_SU\ra U$ 
	is étale coincides with the open subscheme of $S$ with closed points
	\begin{equation*}
		|U|=\{s\in S:I_s\text{ acts trivially on }L\}
	\end{equation*}
	where here $I_s$ acts via
	$I_s\subset\Gal(\overline\eta/\eta_s)\subset\Gal(\overline\eta/\eta)$,
	$\eta_s$ the generic point of $S_{(s)}$.
	If $\mathscr F$ is the sheaf of local sections for $X$ over this $U$,
	the pair $(\mathscr F,U)$ is distinguished in its class by
	the property that $U$ is maximal for the filtered partial order given
	by inclusion. Given $j:V\hookrightarrow U$ and
	$(\mathscr G,V)\sim(\mathscr F,U)$, $\mathscr F\xra\sim j_*\mathscr G$.
\end{proposition*}
\begin{corollary*}
	Given a lisse ($\ZZ_\ell,\QQ_\ell,R,E_\lambda,\overline\QQ_{\ell}$)-sheaf
	$\mathscr F$ on a nonempty open subscheme $U$ of a normal connected curve 
	$S$ with geometric generic point $\eta$, the maximal locus $U'\subset S$ 
	over which $\mathscr F$ extends to a lisse sheaf $\mathscr F'$ is defined
	by its set of closed points
	\begin{equation*}
		|U'|=\{s\in S:I_s\text{ acts trivially on }\mathscr F_{\overline\eta}\}.
	\end{equation*}
	If $j:U\hookrightarrow U'$ denotes the open immersion,
	$\mathscr F'\xra\sim j_*\mathscr F$.
\end{corollary*}
\begin{proof}
	Disregrading the module structure, the sheaf $\mathscr F$ is represented
	by a projective system of revêtements étales, to each of which the
	proposition compatibly applies.
\end{proof}
\begin{corollary*}
	Let $\mathscr F$ be a lisse sheaf on a normal connected curve $S$ and
	$j:U\hookrightarrow S$ an open immersion.
	The unit of adjunction $\mathscr F\ra j_*j^*\mathscr F$ is an 
	isomorphism.
\end{corollary*}
\begin{corollary*}
	Let $A\in\{\ZZ_\ell,\QQ_\ell,R,E_\lambda,\overline\QQ_{\ell}\}$.
	The category $\mathscr C$ as above with l.c.c. sheaves replaced by lisse
	$A$-sheaves on a normal connected curve $S$ is equivalent to the full 
	subcategory of $A$-modules of finite type with continuous action of
	$\Gal(\overline\eta/\eta)$ generated by those $A$-modules $V$ with
	the property that $I_s$ acts trivially on $V$ for all but finitely many
	$s\in|S|$.
\end{corollary*}
\begin{proof}
	By SGA 5 Exp. VI
	(c.f. \hyperref[sec:st1.2]{note to \emph{Sommes trig.} 1.2}),
	such a sheaf is equivalent the data of its monodromy
	representation, which is equivalent to a projective system of 
	representations. Each representation in the projective system corresponds
	to a revêtement étale of some nonempty open $U\subset S$ together with
	the appropriate module structure on its fiber. By the proposition, each
	revêtement étale is unramified over an open set which can be calculated
	from the action of the local inertia. In order that there exist a
	nonempty open $U\subset S$ over which all the revêtements étales in the
	projective system are unramified (so that the corresponding projective
	system defines a lisse sheaf on $U$) it is necessary and sufficient that
	the local inertia at all but finitely many points act trivially on all
	the revêtements étales in the projective system. This is true for 
	some $I_s$ iff $I_s$ acts trivially on the fiber of the lisse sheaf
	considered as $A$-module.
\end{proof}
\begin{corollary*}
	The kernel of the map
	\begin{equation*}
		\Gal(\overline\eta/\eta)\twoheadrightarrow\pi_1(U,\overline\eta)
	\end{equation*}
	is topologically generated by the subgroups $\{I_s:s\in|U|\}$.
\end{corollary*}
\begin{proof}
	On one hand, the category of l.c.c. sheaves on $U$ is a Galois category
	with group $\pi_1(U,\overline\eta)$. On the other, it is equivalent to
	the category of finite $\Gal(\overline\eta/\eta)$-sets on which
	$I_s$ acts trivially for all $s\in U$.
\end{proof}
\begin{remark}
	The same argument can be used to extend a lisse sheaf on a normal scheme
	over the generic point of a divisor. If the scheme is moreover smooth,
	the sheaf can be extended over the points of codimension $>1$ by the
	purity theorem of Zariski-Nagata (SGAA XVI 3.3).
\end{remark}
\begin{remark}
It is a simple exercise to see that if $g:\eta\ra X$ is the inclusion
of the generic point to a normal integral scheme $X$, and $M_\eta$
is a constant sheaf on $\eta$ with constant value $M$, then
$g_*M$ is likewise constant. Let $M_X$ denote the constant 
sheaf on $X$ with constant value $M$. We need to show
\begin{equation*}
	M_X\rightarrow g_*g^*M_X=g_*M_\eta\tag{$\dagger$}.
\end{equation*}
is an isomorphism. Fix a geometric point $x$ of $X$ and let $X_{(x)}$ denote the (spectrum of the) strict henselization of $X$ at $x$; $X_{(x)}$ is a normal domain as $X$ is normal. The map on stalks is
\begin{equation*}
	M\rightarrow\Gamma(\eta\times_X X_{(x)},M_\eta).
\end{equation*}
$\eta_x:=\eta\times_X X_{(x)}$ is the spectrum of the field of fractions of 
the strict henselization of $X$ at $x$, which is separable algebraic over
$k(\eta)$.
Letting $I_x:=\operatorname{Gal}(\overline\eta/\eta_x)$,
\begin{equation*}
	\Gamma(\eta\times_X X_{(x)},M_\eta)=M^{I_x},
\end{equation*}
where here we have identified $M$ with the stalk of $M_\eta$ at a geometric point centered on $\eta$. So the condition that the map ($\dagger$) be an isomorphism at the geometric point $x$ is the same as $I_x$ acting trivially on $M$, considered as $\operatorname{Gal}(\overline\eta/\eta)$-module. In particular, this condition is satisfied by the constant sheaf, which corresponds to $M$ with trivial action of Galois.

(We implicitly use here the description of étale morphism with normal integral target in EGA $\text{IV}_4$ 18.10.7.)
\end{remark}

\subsection*{4.4}
The short exact sequence in the second paragraph is the exact sequence of
the first paragraph of (2.3.2) in light of the fact that $j'_*F'=j'_{!*}F'$
is concentrated in degree $-1$. The computation
\begin{equation*}
	\mathscr H^{-1}(\mathscr F(j_*F[1])_{\overline 0'})
	=H^1_c(A\otimes_{\FF_q}\overline k,j_*F)
\end{equation*}
follows from proper base change along the closed immersion
$\{0'\}\hookrightarrow A'$ applied to either (1.4.1.1), yielding
\begin{equation*}
	\mathscr H^{-1}(\mathscr F(j_*F[1])_{\overline 0'})
	=H^1(\alpha_!j_*F\otimes\overline{\mathscr L}(x.0))
	=H^1_c(A\times_{\FF_q}\overline0,j_*F)
\end{equation*}
or simply from (1.3.1.1) directly, yielding
\begin{align*}
	\mathscr H^{-1}(\mathscr F(j_*F[1])_{\overline 0'})
	&=\mathscr H^{-1}(R\pr_!'(\pr^*j_*F[1]\otimes\mathscr L(x.x'))_{\overline 0'}[1]) \\
	&=H^1_c(A\times_{\FF_q}\overline0,j_*F\otimes\mathscr L(x.0))
	=H^1_c(A\times_{\FF_q}\overline k,j_*F).
\end{align*}
In any event, this $G_{0'}$-module coincides with the stalk of $j'_*F'$ at
$\overline 0'$, which is what is expressed by the equation
\begin{equation*}
	(F'_{\overline\eta_{0'}})^{I_{0'}}
	=H^1_c(A\times_{\FF_q}\overline k,j_*F).
\end{equation*}
For the rightmost term of the short exact sequence, $I_{0'}$ acts trivially
on $F_{\overline\infty}$ as $F$ is by assumption unramified at infinity.
let $\pi:\Spec k\{\pi\}:=T\ra D$ and $\pi':\Spec k\{\pi'\}:=T'\ra A_{(0')}$
be defined by $\pi\mapsto 1/x$ and $\pi'\mapsto x'$. We have
\begin{align*}
	R^{-1}\Phi_{\overline\eta_{0'}}(j_*'F')
	&=R^{-1}\Phi_{\overline\eta_{0'}}(\overline\pr^*(\alpha_!j_*F)
	\otimes\overline{\mathscr L}(x.x')[1])_{(\overline\infty,\overline0')}
	\tag{2.3.2.1) (iii} \\
	&=(\pi\times\pi')^*R^{-1}\Phi_{\overline\eta_{0'}}(\overline\pr^*(\alpha_!j_*F)\otimes\overline{\mathscr L}(x.x')[1]) \\
	&=R^1\Phi_{\overline \eta_{0'}}(\pr^*(F_{\overline\infty!})\otimes{\mathscr L}(\pi'/\pi))_{(\overline t,\overline t')}
	\tag{2.4.2.1) (ii} \\
	&=\mathscr F^{(\infty,0')}(F_{\overline\infty}) \tag{2.4.2.3} \\
	&=F_{\overline\infty}\otimes\mathscr F^{(\infty,0')}(\overline\QQ_\ell)
	=F_{\overline\infty}(-1). \tag{2.5.3.1) (i}
\end{align*}
The second-to-last equality is true because $F$ is assumed unramified at
infinity, so that the action of
$G_\infty:=\Gal(\overline\eta_{\infty}/\eta_{\infty})$
on $F_{\overline\infty}$ factors through
$G_\infty\twoheadrightarrow G_\infty/I_\infty=\Gal(\FF/\FF_q)$, which is
procyclic generated by Frobenius, and by Schur's lemma
$F_{\overline\infty}$ splits as a direct sum of 1-dimensional torsion
sheaves (in the sense of \cite[1.2.7]{weilii}) so that we may assume
$F_{\overline\infty}\simeq\overline\QQ_\ell^{(b)}$ for 
$b\in\overline\QQ_\ell^\times$ an $\ell$-adic unit, which is the reciprocal
image of a sheaf on $\Spec\FF_p$. As twisting by such a sheaf induces an
exact autoequivalence of the category (or derived category) of
$\overline\QQ_\ell$-sheaves, we find that
$\overline\QQ_\ell^{(b)}\otimes\mathscr F^{(\infty,0')}(V)=\mathscr F^{(\infty,0')}(\overline\QQ_\ell^{(b)}\otimes V)$
for any $G_\infty$-module $V$.

A few words on the application of the Deligne lemmata: it is used here that
$F$ is not ramified at infinity, so that if $\alpha$ denotes simultaneously 
$A\hookrightarrow D$ and $U\hookrightarrow U\cup\{\infty\}$ (depending on
the context), $\alpha_*F$ is a lisse sheaf on $U\cup\{\infty\}$.
This is not necessary to conclude by (4.2.2.1) that for each eigenvalue of
Frobenius $\beta$ on $F_{\overline\infty}$,
\begin{equation*}
	w_{q}(\beta)\leq w,
\end{equation*}
but it is necessary to connect the corresponding statement for $F^\vee$ to
that for $F$ in the following way.
Namely, we find that for each eigenvalue of Frobenius $\gamma$ on
$(F^\vee)_{\overline\infty}$,
\begin{equation*}
	w_{q}(\gamma)\leq -w,
\end{equation*}
and as $\alpha_*F$ and $\alpha_*F^\vee$ are both lisse,
$(\alpha_*F)^\vee=\alpha_*(F^\vee)$ and
\begin{ceqn}\begin{equation*}
	(F^\vee)_{\overline\infty}:=(\alpha_*F^\vee)_{\overline\infty}
	=(\alpha_*\underline\Hom(F,\overline\QQ_\ell))_{\overline\infty}
	=\underline\Hom(\alpha_*F,\overline\QQ_\ell)_{\overline\infty}
	=\Hom(F_{\overline\infty},\overline\QQ_\ell)
	=(F_{\overline\infty})^\vee,
\end{equation*}\end{ceqn}
which allows us to conclude from $w_{N(\infty)}(\gamma)\leq -w$ that
\begin{equation*}
	w_{q}(\beta)\geq w.
\end{equation*}
This shows in our application that $F'$ is $\iota$-pure of $\iota$-weight
$w$ and (4.2.2.1) gives that for each eigenvalue $\nu$ of $F_{0'}$ on
$(j_*'F')_{\overline0'}=H^1_c(A\times_{\FF_q}\overline k,j_*F)$,
\begin{equation*}
	w_{q}(\nu)\leq w+1.
\end{equation*}
In the effort to show that $F'$ is $\iota$-pure, since $F$ is unramified
at infinity, the $P_\infty$-module
$(F\oplus\check F(-w))_{\overline\eta_\infty}$ is trivial; i.e. purely of
slope 0, so that $\mathscr F_{\psi}(j_*(F\oplus\check F(-w))[1])$
is of the form $j'_*G'_1[1]$ and
$\mathscr F_{\psi^{-1}}(j_*(F\oplus\check F(-w))[1])$ is of the form
$j'_*G'_2[1]$ and their direct sum is of the form $j'_*G'[1]$ for
the lisse $\QQ_\ell$-sheaf $G':=G'_1\oplus G'_2$ on $A'-\{0\}$.

For the \emph{coup de grâce}, it would be better to refer to (1.1.1),
especially (1.1.1.3), and (1.2.1.2) than to (3.1). The point is that we
can follow \emph{Rapport sur la formula des traces} (1.6) and (3.1) (with
$t^f$ replaced by $t$ so that degree is measured over $\FF_q$, not $\FF_p$,
and shifting by $[-1]$ to cancel the shift of $[1]$ on $G$ and
in the definition of $\mathscr F$)
to find that
\begin{align*}
	&\det(1-t\; F_{x'}^*,G') \\
	=&\det(1-t\; F_{x'}^*,H_c^0(A\times_{\FF_q}\overline x',j_*(F\oplus\check F(-w))\otimes\mathscr L_\psi(x.x')[1])) \\
	&\times\det(1-t\; F_{x'}^*,H_c^0(A\times_{\FF_q}\overline x',j_*(F\oplus\check F(-w))\otimes\mathscr L_{\psi^{-1}}(x.x')[1]))
\end{align*}
Now the point is that as $G'$ is concentrated in degree zero,
letting $\mathscr L=\mathscr L_\psi$ or $\mathscr L_{\psi^{-1}}$,
\begin{ceqn}\begin{equation*}
	R\Gamma(A\times_{\FF_q}\overline x',j_*(F\oplus\check F(-w))\otimes\mathscr L(x.x'))
	=H^1_c(A\times_{\FF_q}\overline x',j_*(F\oplus\check F(-w))\otimes\mathscr L(x.x'))[-1].
\end{equation*}\end{ceqn}
Grothendieck's trace formula gives therefore that
$\iota\det(1-t\; F_{x'}^*,G')$ can be computed as stated in terms of the
polynomials $Q_{x,x'}$ in light of \cite[1.7.6 \& 1.7.7]{Trig}
(compare (1.1.1.2) and (1.1.3.3)), which allow us
to write for each $x\in|A\times_k x'|$
\begin{equation*}
	F_x^*|(j_*(F\oplus\check F(-w))\otimes\mathscr L_\psi(x.x'))_{\overline x}
	=\psi(\Tr_{k(x)/\FF_p}(x.x'))\;F_x^*|(j_*(F\oplus\check F(-w)))_{\overline x}.
\end{equation*}
If $P(t)\in\RR[t]$ and $\alpha=e^{i\theta}$, then
$P(\alpha t)P(\alpha^{-1}t)\in\RR[t]$ as well, since
$P(\alpha t)P(\overline\alpha t)$ is fixed by complex conjugation.
% If $P(t)=\sum_{i=0}^np_it^i$, the coefficient of the term of degree
% $0\leq m\leq 2n$ in $P(\alpha t)P(\alpha^{-1}t)$ is
% \begin{equation*}
% 	\sum_{i=\max\{m-n,0\}}^{\min\{m,n\}}
%	p_ip_{m-i}\alpha^i\overline\alpha^{m-i}.
% \end{equation*}
% If $i=m-i$, $\alpha^i\overline\alpha^{m-i}\in\RR$.
% Otherwise $i\ne m-i$ and
% $p_ip_{m-i}\alpha^i\overline\alpha^{m-i}+p_{m-i}p_i\alpha^{m-i}\overline\alpha^i\in\RR$.
\vfill\flushright Q.E.D.





\addtocontents{toc}{\protect\setcounter{tocdepth}{-1}}
\begin{thebibliography}{Sommes trig.}
	\bibitem[Weil II]{weilii} Deligne, Weil II.
	\bibitem[Laumon]{Laumon} Laumon, \emph{Transformation de Fourier, constants d'équations fonctionnelles et conjecture de Weil}.
	\bibitem[SGA 1]{SGA1} SGA 1.
	\bibitem[SGA 7]{SGA7} SGA 7.
	\bibitem[Sommes trig.]{Trig} \textit{Application de la formule des traces aux sommes trigonométriques}
	dans SGA $4\frac12$.
	\bibitem[Th. finitude]{thfin} \textit{Théorèmes de finitude en cohomologie $\ell$-adique}
	dans SGA $4\frac12$.
\end{thebibliography}
\addtocontents{toc}{\protect\setcounter{tocdepth}{1}}
\end{document}
