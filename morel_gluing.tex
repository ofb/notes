\documentclass[deligne.tex]{subfiles}

\begin{document}
\subsection*{0. Introduction}
We use the notation
$\begin{tikzcd}\GG_{m,k}\arrow[r,hook,"j"]&\A^1_k&\{0\}\arrow[l,hook',"i"']\end{tikzcd}$.
The nearby cycles functor $\Psi_f$ is relative to the morphism 
$f:X\ra\A^1_k$; let $Y:=f^{-1}(0)$ and $U:=X-Y$.
We use the setup of (2.2.2) in Laumon's Fourier transform
article: write $\A^1_k=\Spec(k[u])$ with generic point $\eta=\Spec(k(u))$
and Zariski trait $(\A^1_k)_{0}:=\Spec(k[u]_{(u)})$ at 0;
let $k\{u\}$ denote the henselization of the local ring $k[u]_{(u)}$, 
let $(\A^1_k)_{(0)}:=\Spec(k\{u\})$ with generic point
$\xi=\Spec(k\{u\}[u^{-1}])$, so
that the inclusion $k[u,u^{-1}]\hookrightarrow k\{u\}[u^{-1}]$
induces a morphism of schemes $\iota:\xi\ra\GG_{m,k}$.
We also use $i,j$ to denote the inclusion of the closed and generic points
to $(\A^1_k)_{(0)}$.

Fix a geometric point $\rho:\overline\xi\ra\xi$, which via $\xi\ra\eta$
we also consider to be centered on $\eta$.
% if $\pi_1(\GG_{m,k},\overline\xi)^{\text{mod}.\infty}$ is the quotient of
% the étale fundamental group classifying the finite revêtements étales of
% $\GG_{m,k}$ which are tamely ramified at $\infty$, we have a commutative
% diagram
% \begin{equation*}\begin{tikzcd}
%	&\pi_1(\GG_{m,k},\overline\xi)\arrow[d,twoheadrightarrow] \\
%	\pi_1(\xi,\overline\xi)\arrow[r,"{\iota_*^{\text{mod}.\infty}}"]\arrow[ur,"\iota_*"]
%	&\pi_1(\GG_{m,k},\overline\xi)^{\text{mod}.\infty}.
% \end{tikzcd}\end{equation*}
% \begin{theorem*}
%	The natural morphism $i_*^{\text{mod}.\infty}$ is injective and admits a
%	continuous retraction $r$.
%\end{theorem*}
Given an object $K$ of $D_c^b(X_\xi)$, 
$\Gal(\overline\xi,\xi)$ acts on the nearby cycles
$R\Psi_\xi(K)[-1]$ of \cite[Exp. XIII]{sga7} relative to the morphism $f$ 
(abusively letting $\iota$ also denote the inclusion of $\xi$ into
$\A^1_k$).
(See \hyperref[BBD:AppendixA]{the note to BBD Appendix A} for a detailed
comparison of two ways of writing $R\Psi_\xi$.)

Let $S$ denote $(\A^1_k)_{(0)}$ with closed point $0$.
Let $\tilde\xi$ denote the maximal tamely ramified extension of $\xi$ and
let $\overline\A^1_k,\GG_{m,k},\overline S$ denote the normalisations of
$\A^1_k,\GG_{m,k}$, and $S$, respectively, in $\overline\xi$.
Likewise for tildes.
Note that $\tilde 0$ is the spectrum of the separable closure of $k$
and % $\tilde S$ is the strict henselisation of $S$ at $\tilde 0$ while
$\tilde S$ is the spectrum of a valuation ring with value group
$\ZZ_{(p)}$ (the localization of $\ZZ$ at the prime $(p)$) while
$\overline S$ is the spectrum of a valuation ring with value group $\QQ$
and with residue field a purely inseparable extension of $k(\tilde s)$.
The commutative diagram on the right is obtained by localising the one
on the left near $0$.
\begin{equation*}\begin{tikzcd}
	\overline 0\arrow[d]\arrow[r,"\overline i"]&
	\overline\A^1_k\arrow[d]&\overline\GG_{m,k}\arrow[l,"\overline j"']\arrow[d] \\
	\tilde 0\arrow[d]\arrow[r,"\tilde i"]&
	\tilde\A^1_k\arrow[d,"\tilde\pi"]&\tilde\GG_{m,k}\arrow[l,"\tilde j"']\arrow[d,"\tilde\pi"] \\
	0\arrow[r,"i"]&\A^1_k&\GG_{m,k}\arrow[l,"j"'].
\end{tikzcd}\qquad\qquad
\begin{tikzcd}
	\overline 0\arrow[d]\arrow[r,"\overline i"]&
	\overline S
	\arrow[d]&\overline\xi\arrow[l,"\overline j"']\arrow[d] \\
	\tilde 0\arrow[d]\arrow[r,"\tilde i"]&
	\tilde S\arrow[d,"\tilde\pi"]&\tilde\xi\arrow[l,"\tilde j"']\arrow[d] \\
	0\arrow[r,"i"]& S&\xi\arrow[l,"j"'].
\end{tikzcd}\end{equation*}
Let $\pi$ denote the composition
$\overline\A^1_k\ra\tilde\A^1_k\xra{\tilde\pi}\A^1_k$ and its various base
extensions to morphisms landing in $\GG_{m,k}$ or $S$.
To get $\pi_1(\GG_{m,k},1)$ to act on nearby cycles, make the 
\begin{definition*}
	For $K$ in $D_c^b(U)$, set 
	$\Psi_f:=\overline i^*\overline j_*\pi^*K[-1]$.
	% $\Psi_f:=i^*j_*\pi_*\pi^*K$.
\end{definition*}
(We notate also by $i,j$ etc. the corresponding base extensions via
the morphism $f$.)

As $i$ factors as $\{0\}\ra(\A^1_k)_{(0)}\ra\A^1_k$,
$\overline i$ factors as $\{\overline 0\}\ra\overline S\ra\overline\A^1_k$
so that $\Psi_f[1]$ coincides with the usual $R\Psi_\xi$ for the henselian 
trait $S$. As $i$ also factors as $\{0\}\ra(\A^1_k)_0\ra\A^1_k$,
$\Psi_f$ also coincides with
$\overline i^*\overline j_*K_{\overline\xi}[-1]$
which carries an action of $\Gal(\overline\xi/\eta)$.
(`Coincide,' means in this case `coincide' in $D_c^b(Y_{\overline k})$, but
these various ways of writing $\Psi_f$ endow it with actions by different
groups; in this case, one is a subgroup of the other.)
When $\mathscr L$ is a lisse sheaf on $\GG_{m,k}$, this action of
factors through the quotient
$\Gal(\overline\xi/\eta)\twoheadrightarrow\pi_1(\GG_{m,k},1)$.
For an arbitrary $\QQ_\ell$-sheaf, the lisse locus may be a proper subscheme 
of $\GG_{m,k}$ in which case the action of $\Gal(\overline\xi/\eta)$ may not
factor through the stated quotient, but rather will factor through one with
a smaller kernel.
Regardless, Remark 0.2 about a local system $\mathscr L$ is now obvious.


% Let $\pi_1(\GG_{m,k},1)$ act on $\Psi_f(K)$ via
% \begin{equation*}
%	\pi_1(\GG_{m,k},1)\xra\sim\pi_1(\GG_{m,k},\overline\xi)
%	\twoheadrightarrow\pi_1(\GG_{m,k},\overline\xi)\xra r
%	\pi_1(\xi,\overline\xi),
% \end{equation*}
% the last group acting naturally on the nearby cycles by the definition of
% the topos $X\times_s\xi$, where $s$ denotes the closed point of
% $(\A^1_k)_{(0)}$.

On the functorial exact triangle $\Psi_f\xra{T-1}\Psi_f\ra i^*j_*\ra$:
this is discussed with more detail in the proof of t-exactness of (shifted) 
nearby and vanishing cycles in \cite[Appendix A]{BBD}.
The point is, with $X\ra S$ of finite type over a strictly henselian trait
$(S,\eta,s)$ which is essentially of finite type over a field $k$ of 
characteristic $\ne\ell$, starting with $K\in D_c^b(X_\xi)$ and
letting $G:=\Gal(\overline\xi/\xi)$, $K\xra\sim R\Gamma(G,\rho_*\rho^*K)$
and therefore
\begin{equation*}
	i^*j_*K=i^*j_*R\Gamma(G,\rho_*\rho^*K).
\end{equation*}
Writing $G$ as an extension of $\ZZ_\ell(1)$ by a pro-group $Q$
of order prime to $\ell$, since invariants by $Q$ is an exact functor,
$R\Gamma(G,-)$ is represented on $D_c^b(X_\xi\times BG)$ by
\begin{equation*}
	K\mapsto\operatorname{Cone}(K^Q\xra{T-1}K^Q)[-1]
\end{equation*}
(an explication of why can be found in \cite[10.7]{Modular}; in short, it
is because a $\ZZ_\ell(1)$-module admits a 2-step acyclic resolution by
coinduced modules),
so, keeping in mind $\Psi_f:=R\Psi_\xi[-1]$, in order to see that the triangle
\begin{equation*}
	\Psi_f(K^Q)\xra{T-1}\Psi_f(K^Q)\ra i^*j_*(K^Q)\ra
\end{equation*}
is distinguished, it suffices to see that $i^*j_*(K^Q)=(i^*j_*K)^Q$, which
is clear, and then simply use functoriality.

When $S$ is no longer strictly henselian, the situation is situation is the
same, except of course $\Gal(\overline k/k)$ could be nontrivial, so
$R\Gamma(\pi_1(\GG_{m,\overline k},1),\Psi_f(K))=i^*j_*(K)$, `packaged' as
a sheaf on $Y_{\overline k}$ with action of $\Gal(\overline k/k)$
(recall \cite[XIII 1.1.3]{SGA7}: the functor
$\mathcal F\mapsto\{\overline{\mathcal F}$ endowed with action of
$\Gal(\overline k/k)\}$ is an equivalence of categories).
This is why in the notes Morel writes `the last term should be base changed
from $Y$ to $Y_{\overline k}$,' but this is a distinction without a
difference.

Morel takes $T$ to be a topological generator of the prime-to-$p$
quotient of $\pi_1(\GG_{m,\overline k},1)$, while in the above $T$ is a
topological generator of the maximal pro-$\ell$ quotient $\ZZ_\ell(1)$ of
$\Gal(\overline\xi/\xi)$.
This is insignificant: the invariants under $Q$ functor is exact on a
pro-$\ell$ module provided $Q$ has order prime to $\ell$ so we could take
$T$ to be a topological generator of the prime-to-$p$ quotient and take
invariants under the maximal pro-$p$ subgroup (which is a normal Sylow)
or take $T$ to be a topological generator of the maximal pro-$\ell$ quotient
and take invariants under everything else (which is the approach taken 
above).
Let's first analyse the maximal prime-to-$p$ quotient of both groups; we
will see that they coincide, which will imply that the maximal pro-$\ell$
quotients also coincide.
The maximal prime-to-$p$ quotient of $\pi_1(\GG_{m,\overline k},1)$
coincides with $\pi_1(\GG_{m,\overline k},1)^{\text{mod}}$,
and, after base changing to $\overline k$, the maximal prime-to-$p$ quotient
of $\Gal(\overline\xi,\xi\otimes_k\overline k)$ coincides with
$\Gal(\overline\xi,\xi\otimes_k\overline k)^{\text{mod}}$, the tame quotient
of $\Gal(\overline\xi,\xi\otimes_k\overline k)$. Both these groups are
isomorphic to $\hat\ZZ(1)(\overline k)$; c.f. \cite[2.2.2.1]{Laumon} and
\hyperref[laumon:2.2.2.2]{the note to 2.2.2.2}.
To see that the prime-to-$p$ quotient of $\pi_1(\GG_{m,\overline k},1)$
coincides with $\pi_1(\GG_{m,\overline k},1)^{\text{mod}}$, simply recall
that $\Gal(\overline\xi,\eta\otimes_k\overline k)\twoheadrightarrow\pi_1(\GG_{m,\overline k},1)$
and the former group admits a dévissage
\begin{equation*}
	1\ra P_0\ra\Gal(\overline\xi,\eta\otimes_k\overline k)\ra\hat\ZZ(1)(\overline k)\ra1
\end{equation*}
where the wild inertia $P_0$ is a $p$-group so that the maximal prime-to-$p$ 
quotient of $\Gal(\overline\xi,\eta\otimes_k\overline k)$ is
$\hat\ZZ(1)(\overline k)$; moreover the same is true of the maximal
prime-to-$p$ quotient of the subgroup
$\Gal(\overline\xi,\xi\otimes_k\overline k)\subset\Gal(\overline\xi,\eta\otimes_k\overline k)$, and this subgroup surjects onto
$\Gal(\overline\xi,\xi\otimes_k\overline k)^{\text{mod}}\simeq\hat\ZZ(1)(\overline k)$. We have a commutative diagram \cite[2.2.2]{Laumon}
\begin{equation*}\begin{tikzcd}
	\Gal(\overline\xi,\eta\otimes_k\overline k)\arrow[r,twoheadrightarrow]
	&\pi_1(\GG_{m,\overline k},1)\arrow[r,twoheadrightarrow]
	&\pi_1(\GG_{m,\overline k},1)^{\text{mod}}
	\\
	&\Gal(\overline\xi,\xi\otimes_k\overline k)
	\arrow[u,hook]
	\arrow[r,twoheadrightarrow]
	&\Gal(\overline\xi,\xi\otimes_k\overline k)^{\text{mod}}
	\arrow[u, phantom,"\text{\rotatebox{-90}{$\simeq$}}"]\arrow[r,phantom,"\simeq"]
	&\hat\ZZ(1)(\overline k)
\end{tikzcd}\end{equation*}
and from the description of $\Gal(\overline\xi,\eta\otimes_k\overline k)$
it follows that
$\pi_1(\GG_{m,\overline k},1)^{\text{mod}}\simeq\hat\ZZ(1)(\overline k)$ is
the maximal prime-to-$p$ quotient of $\pi_1(\GG_{m,\overline k},1)$.
Moreover, a topological generator of either the maximal prime-to-$p$ or
pro-$\ell$ quotients of $\Gal(\overline\xi,\xi\otimes_k\overline k)$
is carried onto the same in $\pi_1(\GG_{m,\overline k},1)$.

Regardless of whether one takes $T$ to be a topological generator of the
maximal pro-$\ell$ quotient or the maximal prime-to-$p$ quotient (we will
choose the latter to conform to Morel's notes), there is still the issue 
that after extending scalars $k\hookrightarrow\overline k$ the triangle
\begin{equation*}
	\Psi_f(K^Q)\xra{T-1}\Psi_f(K^Q)\ra i^*j_*(K^Q)\ra
\end{equation*}
is distinguished, where we have taken $Q$-invariants, whereas Morel doesn't
mention taking $Q$-invariants when she writes this triangle.
This is most likely because she is using tamely-ramified nearby cycles
$\tilde\Psi_f$ to begin with.
\begin{definition*}
	For $K\in D_c^b(U)$, let
	$\tilde\Psi_f(K):=\tilde i^*\tilde j_*\tilde\pi^*K[-1]$ denote tame 
	nearby cycles.
\end{definition*}
(Compare \cite[Exp.\,I 2.7]{SGA7}.)
Note that $\tilde\Psi_f(K)=R\Gamma(Q,\Psi_f(K))=\Psi_f(K)^Q=\Psi_f(K^Q)$,
where $Q$ is the wild inertia of $\Gal(\overline\xi,\xi)$.
Therefore, when Morel writes $\Psi_f$, she is probably implicitly writing
$\tilde\Psi_f$, and we will do the same in what follows.


\subsection*{1. Unipotent nearby cycles}
We retain the notation from the previous section.
We write $\FF_\ell$ for Morel's $F$. Now $\FF_\ell$ can be
$\QQ_\ell$, $E_\lambda$, $\overline\QQ_\ell$, etc.

(1.1) We construct the unipotent nearby cycles.
To find that the endomorphism ring is finite over $\FF_\ell$, simply
use the fact that $R\underline\Hom$ preserves constructibility and write
\begin{equation*}
	\End_{D_c^b}(\Psi_f K)=H^0R\Hom(\Psi_f K,\Psi_f K).
\end{equation*}
Let $P:=\Psi_fK$ and $E=\End P$. We have $\FF_\ell[T]\ra E$ with kernel
$a(T)$, a nonconstant polynomial as $\dim_{\FF_\ell}E<\infty$.
Write $a=bc$ with $(T-1)\nmid b$ and $c=(T-1)^m$ for some $m\in\NN$.
As $(b,c)=1$, there exist $x,y\in\FF_\ell[T]$ s.t. $1=xb+yc$ ($1=\id_P$).
As $a=0$ in $E$, we have $(xb)(yc)=xya=0$,
\begin{align*}
	&xb=xb(xb+yc)=(xb)^2+xya=(xb)^2,
\end{align*}
and similarly $(yc)^2=yc$, all in $E$. Let $e_1=xb$ and $e_2=yc$.
We have found a pair of orthogonal idempotents $e_1$ and $e_2$ with
$e_1+e_2=1:=\id_P$, and by
\hyperref[BBD:2.2.18]{the note to BBD 2.2.18}, the triangulated category
$D_c^b(Y)$ is Karoubi complete: every idempotent splits. More precisely,
given an object $P$ of this category and an idempotent $e:P\ra P$, there is
an object $P_e$ and a retraction $P_e\xra i P\xra p P_e$ with $pi=\id_{P_e}$
and $ip=e$. In particular, this means that $i$ is a monomorphism, $p$ an
epimorphism, and $P_e$ is simultaneously a limit and colimit of the diagram
$P\rightrightarrows P$, where the parallel morphisms are $\id_P$ and $ip=e$,
via the morphisms $i$ and $p$, respectively. Moreover, $P_e$ is an absolute
(co)limit, i.e. preserved by every functor. 

Therefore from the $e_i$ we get $P_i,i_i,p_i$ ($i=1,2$), and to verify that
$P=P_1\oplus P_2$ it only remains
(\hyperref[https://stacks.math.columbia.edu/tag/0103]{\texttt{0103}})
to check that $p_2\circ i_1=0=p_1\circ i_2$. As $i_2$ is a monomorphism and
$p_1$ an epimorphism, in order to show $p_2\circ i_1=0$ it suffices to write
\begin{equation*}
	i_2\circ p_j\circ i_1\circ p_1=e_2\circ e_1=0.
\end{equation*}
Therefore $P=P_1\oplus P_2$.
Given any polynomial $q(T)$, we consider $q(T)$ as endomorphism of $P$.
As $i_2\circ p_2\circ q(T)\circ i_1 \circ p_1=e_2 q(T) e_1=q(T)e_2e_1=0$,
in fact $p_2\circ q(T)\circ i_1=0$ and similarly $p_1\circ q(T)\circ i_2=0$.
This means that $q(T)$ descends to endomorphisms of $P_1$ and $P_2$, and
the decomposition $P=P_1\oplus P_2$ is $\FF_\ell[T]$-equivariant.
Moreover, to calculate $q(T)$ on $P_1$, it suffices to know
$p_1\circ q(T)\circ i_1$.

We see that as $(T-1)^m=c$ and $a=bc=0$ as
endomorphism of $P$, $(T-1)^me_1=cxb=xa=0$.
Therefore $p_1\circ (T-1)^m\circ i_1=0$ ($p_1$ is an epimorphism) and
$(T-1)^m=0$ on $P_1$.
On the other hand,
$p_2\circ (T-1)(y(T-1)^{m-1})\circ i_2=p_2\circ e_2\circ i_2=\id_{P_2}$
shows that $(T-1)$ is invertible as endomorphism of $P_2$.
This shows that $T-1$ is nilpotent on $P_1$ and an automorphism of $P_2$.

We set $\Psi_f^{\mathrm{un}}K:=P_1$ and $\Psi_f^{\mathrm{nu}}K:=P_2$,
call the former the unipotent nearby cycles, and the latter the
non-unipotent nearby cycles. We have proved a $\FF_\ell[T]$-linear 
decomposition 
\begin{equation*}
	\Psi_fK=\Psi_f^{\mathrm{un}}K\oplus\Psi_f^{\mathrm{nu}}K.
\end{equation*}
Nothing in the above discussion changes if we replace
$\Psi_f$ by $\tilde \Psi_f$, so we have similarly
\begin{equation*}
	\tilde\Psi_f(K)=\tilde\Psi_f^{\text{un}}(K)\oplus\tilde\Psi_f^{\text{nu}}(K)
\end{equation*}
and a distinguished triangle
\begin{equation*}
	\tilde\Psi_f^{\text{un}}\xra{T-1}\tilde\Psi_f^{\text{un}}\ra i^*j_*\ra.
\end{equation*}
(That $\tilde\Psi_f^{\mathrm{un}}$ preserves constructibility is deduced
trivially from the fact that $\Psi_f^{\mathrm{un}}$ does, and as
taking $Q$-invariants is an exact functor on $\ell$-adic sheaves,
$H^i\tilde\Psi_f^{\mathrm{un}}=H^i(\Psi_f^{\mathrm{un}})^Q= (H^i\Psi_f^{\mathrm{un}})^Q$; now recall that every subsheaf of a
constructible sheaf of modules over a noetherian ring on a noetherian scheme
is constructible.)

\subsection*{2. Some local systems on $\GG_{m,k}$}
The local systems $\mathcal L_a$ appear to
be stand-ins for Sasha's $\Gr I^{a,b}$, which, by the way, come with the 
pairing $\langle\;,\,\rangle$. Let's look more closely at the analogy.
The action of $\pi_1(\GG_{m,k},1)$ on $I^{a,b}$ is also via the 
projection onto the maximal pro-$\ell$ quotient $\ZZ_\ell(1)$, and
$t\in\ZZ_\ell(1)$ acts on $I^{a,b}$ by $\tilde t=\exp(\log\tilde t)$.
The point is that $\log\tilde t\in A^{1*}$ in Sasha's notation; i.e.
$A^\circ\simeq\FF_\ell[[\log\tilde t]]$, and therefore $\log\tilde t$
acts on $\Gr I^{-a,0}$ by Morel's $N$.
Therefore indeed Morel's $\mathcal L_a\simeq\Gr I^{-a,0}$.
Morel claims an isomorphism $\mathcal L_a^\vee\simeq\Gr I^{0,a}$, but
if we want to upgrade the naïve isomorphism of sheaves on $\Spec k$ to one
on $\GG_{m,k}$ we run into the issue that the logarithm of geometric 
monodromy acts on $\mathcal L_a^\vee$ by $-\log\tilde t$, not $\log\tilde t$
and so $t$ acts by $\exp(-\log\tilde t)$. Therefore a
$\pi_1(\GG_{m,k},1)$-equivariant isomorphism is not provided by the identity
morphism but rather by the isomorphism
\begin{equation*}\begin{tikzcd}[column sep=tiny]
	I^{0,a}\arrow[d,"\text{\scalebox{1.5}{\rotatebox{-90}{$\sim$}}}"]\arrow[r,equals]
	&F\arrow[r,phantom,"\oplus"]\arrow[d,"\id"]
	&F(1)\arrow[r,phantom,"\oplus"]\arrow[d,"-\id"]
	&\cdots\arrow[r,phantom,"\oplus"]
	&F(a)\arrow[d,"(-1)^a\id"] \\
	\mathcal L_a^\vee\arrow[r,equals]
	&F\arrow[r,phantom,"\oplus"]
	&F(1)\arrow[r,phantom,"\oplus"]
	&\cdots\arrow[r,phantom,"\oplus"]
	&F(a).
\end{tikzcd}\end{equation*}
In Sasha's notation, $(\Gr I^{-a,0})^\vee\simeq\Gr I^{0,a}$, and the maps
$\alpha,\beta$ are just the $\Gr$ of maps $I^{-a,0}\hookrightarrow I^{-b,0}$
and $I^{-b,0}\twoheadrightarrow I^{-b,a-b}$. To obtain the statement about
$D(\alpha_{a,b})$, we see before passing to $\Gr$ that
$D(\alpha_{a,b})(-1)[-2]:I^{0,b}\ra I^{0,a}$ is just reduction modulo $I^a$.
Passing to $\Gr$, this is the same as
\begin{equation*}
	\beta_{a,b}(b):\mathcal L_b(b)=(\Gr I^{-b,0})(b)\twoheadrightarrow
	(\Gr I^{-a,0})(a)=\mathcal L_a(a).
\end{equation*}

\subsection*{3. Beilinson's construction of $\Psi^{\mathrm{un}}_f$}
Everything is clear except perhaps at first glance the expression
$\ker(N^a,\Psi_f^{\mathrm{un}} K)$, as $N$ is not an endomorphism of
$\Psi_f^{\mathrm{un}}$ but rather sends it to $\Psi_f^{\mathrm{un}}(-1)$.
Write $N$ as a linear map
$\Psi_f^{\mathrm{un}}(1)\ra\Psi_f^{\mathrm{un}}$; what this means is that
$N$ assigns, linearly in $\FF_\ell$, an endomorphism of
$\Psi_f^{\mathrm{un}}$ to an element of $\FF_\ell(1)$.
Therefore there is an unambiguous meaning to
$\ker(N^a,\Psi_f^{\mathrm{un}} K)$: if $t_1,\ldots,t_a$ are $a$ nonzero
elements of $\FF_\ell(1)$, $\ker(Nt_a\ldots Nt_1,\Psi_f^{\mathrm{un}} K)$ is
independent of the choice of $t_i$.

(3.3) Returning to the various ways of defining $\Psi_f$, we settled upon
$\tilde\Psi_f^{\mathrm{un}}$. This functor with $f=\id$, as a direct factor
of tame cycles, no longer has the the property of sending every shifted 
local system $\mathcal L[1]$ on $\GG_{m,k}$ to $L_{\{0\}}$
(where $L$ the representation of $\pi_1(\GG_{m,k},1)$ corresponding to
$\mathcal L$). But it still does for these $\mathcal L_a$, as 
$\pi_1(\GG_{m,\overline k},1)$ acts on $\mathcal L_a$ through the tame
quotient, and moreover $T$ acts unipotently by construction.

(3.4) is indeed obvious, but it appears to have a typo.
Start with $\Psi_f^{\mathrm{un}}(K)$, express the latter as
$\ker(N^{a+1},\Psi_f^{\mathrm{un}}(K))$ for $a\gg0$ and then via the 
isomorphism $\gamma$ as $\ker(N,\Psi_f^{\mathrm{un}}(K\otimes\mathcal L_a))$;
i.e. for $a\gg0$ the map
\begin{align*}
	\Psi_f^{\mathrm{un}}(K)&\ra\ker(N,\Psi_f^{\mathrm{un}}(K\otimes\mathcal L_a)) \\
	x&\mapsto(x,-Nx,\ldots,(-N)^ax)
\end{align*}
is an embedding. The target is identified with
$^pH^{-1}i^*j_*(K\otimes\mathcal L_a))$, which now has $N$ acting only by
$1\otimes N=1\otimes(\beta_{a,a+1}\circ\alpha_{a,a+1})$; i.e. by
\begin{equation*}
	(x,-Nx,\ldots,(-N)^ax)\mapsto(-Nx,\ldots,(-N)^ax,0);
\end{equation*}
via the above embedding, this means that
$1\otimes N=1\otimes(\beta_{a,a+1}\circ\alpha_{a,a+1})$ acts by
$-N$, not $N$, on $\Psi_f^{\mathrm{un}}(K)$.

(3.5) Sasha's construction of
$\Psi_f^{\mathrm{un}}$ has the merit of behaving simply with respect to
duality and admitting a description in terms of basic functors, but, as
remarked above, whether it lands in $Y$ or $Y_{\overline k}$ is 
insignificant since it lands in $Y_{\overline k}$ with action of
$\Gal(\overline k/k)$ coming from the split-exact arithmetic-geometric
sequence of $\pi_1$, and therefore can be regarded as a sheaf on $Y_k$ to
begin with.

\subsection*{4. Duality} As before,
$\FF_\ell=\QQ_\ell,E_\lambda,\overline\QQ_\ell$, etc.
\begin{lemma*}
	Let $f:X\ra S$ be a separated morphism of finite type between schemes of
	finite type over $k$ and $\mathcal L$ a lisse $\FF_\ell$-sheaf on $S$.
	Then $f^!(\mathcal L)\xleftarrow\sim f^*\mathcal L\otimes f^!\FF_\ell$.
\end{lemma*}
\begin{proof}
The counit of adjunction $f_!f^!\ra\id$ defines an arrow in the $H^0$ of
\begin{ceqn}\begin{equation*}
	R\Hom(\mathcal L\otimes f_!f^!\FF_\ell,\mathcal L)
	=R\Hom(f_!(f^*\mathcal L\otimes f^!\FF_\ell),\mathcal L)
	=R\Hom(f^*\mathcal L\otimes f^!\FF_\ell,f^!(\mathcal L))
\end{equation*}\end{ceqn}
which is evidently locally an isomorphism.
\end{proof}
The lemma allows one to write, for $K\in D_c^b(X,\FF_\ell)$,
\begin{multline*}
	D(K\otimes f^*\mathcal L_a)
	=R\Hom(K,D(f^*\mathcal L_a^\vee))
	=R\Hom(K,f^!D(\mathcal L_a^\vee)) \\
	=R\Hom(K,f^*\mathcal L_a^\vee\otimes f^!\FF_\ell))
	=D(K)\otimes f^*\mathcal L_a^\vee
	\simeq D(K)\otimes f^*\mathcal L_a(a).
\end{multline*}

(4.1) As $D$ is involutive and $D(X(a))=D(X)(-a)$, the diagram
\begin{small}\begin{ceqn}\begin{equation*}\mathclap{\begin{tikzcd}[ampersand replacement=\&,column sep=50pt]
	^pH^0i^*j_*(K\otimes\mathcal L_a) \arrow[r,"\alpha_{a,a+b+1}"]
	\& ^pH^0i^*j_*(K\otimes\mathcal L_{a+b+1}) \arrow[r,"\beta_{b,a+b+1}"]
	\&[25pt] ^pH^0i^*j_*(K\otimes\mathcal L_b) \\
	\coker_a(K)\arrow[r,"\alpha_{a,a+b+1}"]\arrow[d,equals]
	\arrow[u,"\rsim"]
	\&\coker_{a+b+1}(K)\arrow[r,"\beta_{b,a+b+1}"]\arrow[d,equals]
	\arrow[u,"\rsim"]
	\&\coker_b(K)(-a-1)\arrow[d,equals]
	\arrow[u,"\rsim"] \\
	D(\ker_a(DK)(a))\arrow[r,"{D(\beta_{a,a+b+1}(a))}"]
	\&D(\ker_{a+b+1}(DK)(a))(-b-1)\arrow[r,"D(\alpha_{b,a+b+1}(a))(-b-1)"]
	\&D(\ker_b(DK)(a))(-b-1) \\
	D(\Psi_f^{\mathrm{un}}(DK))(-a)\arrow[r,"D(N^{b+1})(-a)"]
	\arrow[u,"\rsim"]
	\&D(\Psi_f^{\mathrm{un}}(DK))(-a-b-1)\arrow[r,"D(\id)(-a-b-1)"]
	\arrow[u,"\rsim"]
	\&D(\Psi_f^{\mathrm{un}}(DK))(-a-b-1)
	\arrow[u,"\rsim"]
\end{tikzcd}}\end{equation*}\end{ceqn}\end{small}
commutes, where the first line is isomorphic to the second by
\cite[4.1.12.4]{BBD}, and we have used that the map $\alpha_{a,b}$ induces
the map $\mathcal L_b^\vee=\mathcal L_b(b)\xra{\beta_{a,b}(b)}\mathcal L_a(a)=\mathcal L_a^\vee$
and vice versa. We need that $N^{b+1}=0$ on $\Psi_f^{\mathrm{un}}(DK)$, and
I don't see why this is secured if $N^{b+1}=0$ on $\Psi_f^{\mathrm{un}}(K)$,
so let's modify the hypothesis to say `$N^{a+1}=0$ on
$\Psi_f^{\mathrm{un}}(K)$ and $N^{b+1}=0$ on $\Psi_f^{\mathrm{un}}(DK)$.'

\subsection*{5. The maximal extension functor}
The $\beta$s in the commutative diagram defining $\gamma_{a,a-1}$
are $\beta_{a-1,a}$, not $\beta_{a,a+1}$.

(5.1) It is of course the map
$\beta_{a-1,a}:\mathcal L_a\ra\mathcal L_{a-1}(-1)$
that is surjective. The snake lemma gives an isomorphism
\begin{equation*}
	\coker\gamma_{a,a-1}\simeq i_* ^pH^0i^*j_*(K\otimes\mathcal L_{a-1})(-1).
\end{equation*}
To see that the last vertical arrow in the last commutative diagram of the 
proof is an isomorphism, it is easier to identify it with
\begin{multline*}
	i_*\Psi_f^{\mathrm{un}}K(-1)
	=\ker(j_!(K\otimes f^*\mathcal L_{a-1}(-1))\ra j_*(K\otimes f^*\mathcal L_{a-1}(-1))) \\
	\xra{\alpha_{a-1,a}(-1)}
	\ker(j_!(K\otimes f^*\mathcal L_a(-1))\ra j_*(K\otimes f^*\mathcal L_a(-1)))
	=i_*\Psi_f^{\mathrm{un}}K(-1),
\end{multline*}
which is the identity by the results of \S3, noting that the square below
commutes
\begin{equation*}\begin{tikzcd}
	\mathcal L_a\arrow[r,"\alpha_{a,a+1}"]\arrow[d,"\beta_{a-1,a}"]
	&\mathcal L_{a+1}\arrow[d,"\beta_{a,a+1}"] \\
	\mathcal L_{a-1}(-1)\arrow[r,"\alpha_{a-1,a}(-1)"]
	&\mathcal L_a(-1).
\end{tikzcd}\end{equation*}
The remaining claims are now clear; given
$0\ra K_1\ra K_2\ra K_3\ra0$ exact, one deduces exactness of
$0\ra \Xi_fK_1\ra \Xi_fK_2\ra \Xi_fK_3$ by finding an integer $a$ such that
$\ker(j_!(K_i\otimes f^*\mathcal L_a)\xra{\gamma_{a,a-1}}j_*(K_i\otimes f^*\mathcal L_{a-1})(-1))=\Xi_f K_i$ for $i=1,2,3$ and then applying the
snake lemma to the following commutative diagram with exact rows
\begin{ceqn}\begin{equation*}\begin{tikzcd}
	0\arrow[r]
	&j_!(K_1\otimes f^*\mathcal L_a)\arrow[r]\arrow[d,"\gamma_{a,a-1}"]
	&j_!(K_2\otimes f^*\mathcal L_a)\arrow[r]\arrow[d,"\gamma_{a,a-1}"]
	&j_!(K_3\otimes f^*\mathcal L_a)\arrow[r]\arrow[d,"\gamma_{a,a-1}"]
	&0 \\
	0\arrow[r]
	&j_*(K_1\otimes f^*\mathcal L_a)\arrow[r]
	&j_*(K_2\otimes f^*\mathcal L_a)\arrow[r]
	&j_*(K_3\otimes f^*\mathcal L_a)\arrow[r]
	&0.
\end{tikzcd}\end{equation*}\end{ceqn}

(5.4) Typo: it should read
\begin{ceqn}\begin{equation*}
	\coker(j_!(K\otimes f^*\mathcal L_b)(1)\ra j_*(K\otimes f^*\mathcal L_{b-1}))
	=\coker(j_!(K\otimes f^*\mathcal L_{b-1})\ra j_*(K\otimes f^*\mathcal L_{b-1}))
\end{equation*}\end{ceqn}
The point is, writing
\begin{equation*}
	\gamma_{a,a+1}=j_!(K\otimes\mathcal L_a)\ra j_*(K\otimes\mathcal L_a)\xra{\alpha_{a,a+1}}j_*(K\otimes\mathcal L_{a+1})
\end{equation*}
shows that $\ker\gamma_{a,a+1}=\ker(j_!(K\otimes\mathcal L_a)\ra j_*(K\otimes\mathcal L_a))=\Psi_f^{\mathrm{un}}(K)$ and the snake lemma
begins with
\begin{equation*}
	0\ra\Psi_f^{\mathrm{un}}\xra{\alpha_{a,a+b+1}}\Psi_f^{\mathrm{un}}\ra
	\ker(\gamma_{b,b-1})(-a-1)\ra\cdots
\end{equation*}
where $\alpha_{a,a+b+1}$ induces an isomorphism of $\Psi_f^{\mathrm{un}}$.
The dual statement about cokernels, after writing
\begin{ceqn}\begin{equation*}
	\gamma_{b,b-1}(a-1)=j_!(K\otimes\mathcal L_b)(a-1)\xra{\beta_{b,b-1}(a-1)}j_!(K\otimes\mathcal L_{b-1})(a-2)\ra j_*(K\otimes\mathcal L_{b-1})(a-2),
\end{equation*}\end{ceqn}
finds that
\begin{align*}
	\coker\gamma_{b,b-1}(a-1)
	&\simeq\coker(j_!(K\otimes\mathcal L_{b-1})(-a-2)\ra j_*(K\otimes\mathcal L_{b-1})(a-2)) \\
	&\simeq\Psi_f^{\mathrm{un}}K(-a-2)(-b+1)
\end{align*}
so that, denoting
$\coker(j_!(K\otimes\mathcal L_{a+b+1})\ra j_*(K\otimes\mathcal L_{a+b+1}))$
by $\coker(j_!\ra j_*)$, the snake lemma ends in
\begin{ceqn}\begin{equation*}\begin{tikzcd}
	\coker(\gamma_{a,a+1})\arrow[r]\arrow[d,equals]
	&\underset{a+b+1}{\coker}(j_!\ra j_*)
	\arrow[r,"\beta_{b,a+b+1}"]\arrow[d,equals]
	&[15pt]\coker(\gamma_{b,b-1}(-a-1))\arrow[d,"\rsim"]\ra0 \\
	\coker(\gamma_{a,a+1})\arrow[r]\arrow[d,equals]
	&\underset{a+b+1}{\coker}(j_!\ra j_*)
	\arrow[r,"\beta_{b-1,a+b+1}"]\arrow[d,"\rsim"]
	&[15pt]\underset{b-1}{\coker}(j_!\ra j_*)(-a-1)\arrow[d,"\rsim"]\ra0 \\
	\coker(\gamma_{a,a+1})\arrow[r]
	&\Psi_f^{\mathrm{un}}K(-a-b-1)
	\arrow[r,"\sim"]
	&\Psi_f^{\mathrm{un}}K(-a-2)(-b+1)\ra0,
\end{tikzcd}\end{equation*}\end{ceqn}
using (4.1). Of course
\begin{equation*}
	\beta_{b-1,a+b+1}:\underset{a+b+1}{\coker}(j_!\ra j_*)\xra\sim
	\underset{b-1}{\coker}(j_!\ra j_*)(-a-1)
\end{equation*}
is an isomorphism from the dual statement about $\alpha$.

(5.6) Just apply $\beta_{a,a+1}$ to the exact sequence of (5.5) for $a\gg0$.

(5.7) If we denote the map $i_*\Psi_f^{\mathrm{un}}\ra\Xi_f$ by $\upsilon$,
then the composition coincides with
\begin{gather*}
	i_*\Psi_f^{\mathrm{un}}D\xra\upsilon\Xi_fD=D\Xi_f\xra{D\upsilon}Di_*\Psi_f^{\mathrm{un}}=i_*\Psi_f^{\mathrm{un}}(-1)D,\qquad\text{so that}\\
	D(D\upsilon\circ\upsilon)=D\upsilon\circ DD\upsilon=D\upsilon\circ\upsilon.
\end{gather*}
The morphism $\Xi_f\ra i_*\Psi_f^{\mathrm{un}}(-1)$ is
induced by
$\beta_{a-1,a}:j_!(K\otimes\mathcal L_a)\ra j_!(K\otimes\mathcal L_{a-1})(-1)$
for $a\gg0$.
The dual morphism fits into a commutative diagram ($a\gg0$)
\begin{equation*}\begin{tikzcd}
	\coker(j_!(K\otimes\mathcal L_{a-1})(a)\ra j_*(K\otimes\mathcal L_{a-1})(a))\arrow[r,"\alpha_{a-1,a}(a)"]
	&[10pt]\coker(\gamma_{a-1,a}(a)) \\
	\ker(j_!(K\otimes\mathcal L_{a-1})\ra j_*(K\otimes\mathcal L_{a-1}))
	\arrow[u,"\rsim"]\arrow[r,"\alpha_{a-1,a}"]
	&\ker(\gamma_{a,a-1})\arrow[u,"\rsim"] \\
	i_*\Psi_f^{\mathrm{un}}(K)\arrow[u,"\rsim"]\arrow[r,hook]
	&\Xi_f(K).\arrow[u,"\rsim"]
\end{tikzcd}\end{equation*}
To see that the upper square commutes, it helps to recall that both
vertical arrows are coboundaries in the snake lemma applied to two similar
diagrams, and a morphism between these two diagrams can be built on
$\alpha_{a-1,a}$ (applied to the upper right corners of the diagrams) and
its twist (applied to the lower left corners).
Therefore
$D\upsilon\circ\upsilon=N:i_*\Psi_f^{\mathrm{un}}\ra\Psi_f^{\mathrm{un}}(-1)$ 
and $DN=N$.

We can also study the composition
$\Xi_f\ra i_*\Psi_f^{\mathrm{un}}(-1)\ra\Xi_f(-1)$.
The first map is induced by $\beta_{a,a-1}$ and the second by
$\alpha_{a-1,a}(-1)$, as already remarked.
We have (5.2) that
\begin{equation*}
	\ker(\gamma_{a,a-1})\xra{\beta_{a-1,a}}\ker(\gamma_{a-1,a-2}(-1))
\end{equation*}
induces $N$ on $\Xi_f$ for $a\gg0$. Since $\gamma_{a,a-1}$ factors as
$\beta_{a-1,a}$ followed by
$j_!(K\otimes\mathcal L_{a-1})(-1)\ra j_*(K\otimes\mathcal L_{a-1})(-1)$,
$\beta_{a-1,a}$ applied to $\ker(\gamma_{a,a-1})$ must actually factor as
\begin{ceqn}\begin{equation*}
	\ker(\gamma_{a,a-1})\xra{\beta_{a-1,a}}
	\ker(j_!(K\otimes\mathcal L_{a-1})\ra j_*(K\otimes\mathcal L_{a-1}))(-1)
	\hookrightarrow
	\ker(\gamma_{a-1,a-2}(-1)).
\end{equation*}\end{ceqn}
The middle is $i_*\Psi_f^{\mathrm{un}}(K)(-1)$, and if we postcompose the
above morphism by $\alpha_{a-1,a}(-1)$ we don't change it, as
$\alpha_{a-1,a}$ induces an isomorphism $\Xi_fK\xra\sim\Xi_fK$ for $a\gg0$.
We have shown that $N:\Xi_f\ra \Xi_f$ factors as
\begin{equation*}
	\Xi_f\twoheadrightarrow i_*\Psi_f^{\mathrm{un}}(-1)
	\hookrightarrow \Xi_f(-1)
\end{equation*}
and the former map induces via the inclusion
$i_*\Psi_f^{\mathrm{un}}\hookrightarrow\Xi_f$ 
the morphism $N$ on $i_*\Psi_f^{\mathrm{un}}$; i.e.
\begin{equation*}
	i_*\Psi_f^{\mathrm{un}}\hookrightarrow\Xi_f\twoheadrightarrow
	i_*\Psi_f^{\mathrm{un}}(-1)\qquad\text{also composes to $N$.}
\end{equation*}
Last, it is clear that the square
\begin{equation*}\begin{tikzcd}[row sep=small, column sep=small]
	j_!\arrow[rr,hook]\arrow[dr]\arrow[dd]
	&&[-25pt]\ker(\gamma_{a-1,a-2})\arrow[dd,"\rsim"]\arrow[dl,equals] \\
	&\Xi_f\arrow[dr,equals]\arrow[dl] \\
	j_*&&\coker(\gamma_{a-1,a}(a))\arrow[ll,twoheadrightarrow]
\end{tikzcd}\end{equation*}
commutes, where the leftmost down arrow is the canonical morphism, which
shows that this morphism is also sent to itself by $D$.

\subsection*{6. The unipotent vanishing cycles functor}\label{sec:MorelPhi}
\begin{remark}
The first paragraph discusses the construction of vanishing cycles.
However, (6.2) shows that the functor $\Phi_f^{\mathrm{un}}$ constructed
is a direct factor of the the tame vanishing cycles which appears in the
distinguished triangle
\begin{equation*}
	\Psi_f^{\mathrm{un}}j^*\xra{\mathrm{can}}\Phi^{\mathrm{un}}_f\ra i^*\ra
\end{equation*}
where as before $\Psi_f^{\mathrm{un}}=\tilde\Psi_f^{\mathrm{un}}$.
\end{remark}

In light of \cite[4.1.10.1]{BBD}, $\Psi_f^{\mathrm{un}}$ sends $M(X)$ to
$M(Y)$ because it has support in $Y$.
(5.6) $\rightsquigarrow\vep(-1)\circ N=0$.
(5.2) $\rightsquigarrow N\circ\delta=0$.
It is easy to see that $\can(-1)\circ\var$ induces the map
$\Xi_f\twoheadrightarrow\Psi_f^{\mathrm{un}}(-1)\hookrightarrow\Xi_f$.
We studied this composition in the note to the previous section and showed
that it coincides with $N$.

(6.2) There is an error in the second part of the proof that doesn't change
the conclusion. The following diagram is commutative with exact rows and
columns.
\begin{equation*}\begin{tikzcd}\tag{$\dagger$}\label{morel:6diag}
	&&0\arrow[r]\arrow[d]
	&i_*\ ^pH^{-1}i^*K\arrow[d] \\
	&&j_!j^*K\arrow[r,equals]\arrow[d,"d^{-1}"]
	&j_!j^*K\arrow[d,"\adj"] \\
	0\arrow[r]&i_*\Psi_f^{\mathrm{un}}j^*K\arrow[r]\arrow[d,"\kappa"]
	&\ker d^0\arrow[d]\arrow[r]
	&K\arrow[d]\arrow[r]&0 \\
	0\arrow[r]&\ker v\arrow[r]\arrow[d]
	&i_*\Phi_f^{\mathrm{un}}K\arrow[r,dashed,"v"]\arrow[d]
	&i_*\ ^pH^0i^*K\arrow[r]\arrow[d]&0 \\
	&0&0&0
\end{tikzcd}\end{equation*}
The snake lemma gives the exact sequence
\begin{equation*}
	0\ra\ker\kappa\ra j_!j^*K\ra j_!j^*K/i_*\ ^pH^{-1}i^*K\ra\coker\kappa\ra0
\end{equation*}
so that $\kappa$ is surjective and $\ker\kappa=i_*\ ^pH^{-1}i^*K$.
This doesn't change the conclusion that
$i_*\coker\can\xra\sim i_*\ ^pH^0i^*K$.
Moreover these conclusions for $\ker(i_*\can)$, $i_*\coker\can$ imply the 
results without the $i_*$ because $i_*$ is t-exact and fully faithful.

\subsection*{7. The functor $\Omega_f$}
$\Omega_f$ can be defined as $\ker(\vep+\adj)$ or as $\ker(\vep-\adj)$,
and the isomorphism of complexes $C^\dotp K\xra\sim C'^\dotp K$ (6.1)
carries the latter isomorphically onto the former.
In particular, this gives an isomorphism $\Omega_f\simeq\ker d^0$, and, in
light of the middle row of the diagram \eqref{morel:6diag} above, the
second short exact sequence. The first short exact sequence is (6.1) on the 
nose.

The argument that gives the quasi-isomorphism
$(\Psi_f^{\mathrm{un}}j^*\xra\can\Phi_f^{\mathrm{un}})\ra i^*$
actually establishes the existence of a distinguished triangle
\begin{equation*}
	\Psi_f^{\mathrm{un}}j^*\xra\can\Phi_f^{\mathrm{un}}\ra i^*\ra
\end{equation*}
which is the one we expect from the `usual' definition of
$\Phi_f^{\mathrm{un}}$. But while the first two functors take perverse 
sheaves to perverse sheaves, $i^*$ has perverse amplitude in $[-1,0]$.
It is strange to phrase this as a `quasi-isomorphism,' since
$\Psi_f^{\mathrm{un}}j^*\xra\can\Phi_f^{\mathrm{un}}$ is not in
$D^b_c(Y)$, while $i^*K$ is usually not a perverse sheaf.
I prefer just to say that we have the distinguished triangle above.

\subsection*{8. Gluing}
$D^\dotp (c)$ is a complex because the composition of the two differentials
gives $N-N=0$.

(8.1) Both directions of the proof are wrong. The first part doesn't work
because $\Phi_f^{\mathrm{un}}D^\dotp(c)$ does not have the stated form.
Note that the form that is claimed has $b\circ a=0$ but this should instead
give $N$. The second part doesn't work because the square on the right in
the morphism of exact sequences doesn't commute: tracing it down-right
gives the canonical $\Omega_f\ra i_*\Psi_f^{\mathrm{un}}j^*K(-1)$ while
tracing it right-down gives the null morphism.

Fortunately, Sasha developed a clever device in his appendix on diads that
gets the job done.
Conforming to the language of the next section, we consider a category 
$M(X)_1^\sharp$ of diads of perverse sheaves on $X$.
An object of this category is a diagram
\begin{equation*}
	C_-\xhookrightarrow{\alpha_-=(\alpha_-^1,\alpha_-^2)}
	A\oplus B\xtwoheadrightarrow{\alpha_+=(\alpha_+^1,\alpha_+^2)}C_+
\end{equation*}
in which $\alpha_-^1$ is a monomorphism and $\alpha_+^1$ an epimorphism.
Morphisms are defined pointwise and such a morphism of diads is a
(mono/epi/iso)morphism if it is pointwise; one has
$(M(X)_1^\sharp)^\circ=(M(Y)^\circ)_1^\sharp$ and exact functors
$M(X)\ra M(Y)$ induce exact functors $M(X)_1^\sharp\ra M(Y)_1^\sharp$.
The category $M(X)_1^\sharp$ is endowed with an involutive autoequivalence
$r$ called the reflection functor, which we compute explicitly
\hyperref[sec:reflection]{in the next section}.
The category of perverse sheaves embeds into
$M(X)_1^\sharp$ via the functor $C:M(X)\hookrightarrow M(X)_1^\sharp$ which
sends a perverse sheaf $K$ to the diad
\begin{equation*}
	j_!j^*K\xra{\delta\oplus\adj}\Xi j^*K\oplus K\xra{\vep-\adj}j_*j^*K.
\end{equation*}
Likewise the category of gluing data embeds into $M(X)_1^\sharp$ via the 
functor $D$ which sends a gluing datum $c=(K_U,K_V,u,v)$ to the diad
\begin{equation*}
	i_*\Psi_f^{\mathrm{un}}K_U\xra{\can\oplus u}\Xi_fK_U\oplus i_*K_Y\xra{\can-v}i_*\Psi_f^{\mathrm{un}}K_U(-1).
\end{equation*}
Identifying the category of perverse sheaves and the category of gluing 
data with their respective essential images in $M(X)_1^\sharp$ via $C$
(resp. $D$), $r$ exchanges these two subcategories of $M(X)_1^\sharp$, and
in doing so coincides with the functors $F$ and $G$. To verify this, we use
\hyperref[sec:reflection]{our explicit computation of $r$ in the next section}
to check that
\begin{align*}
	&r(C(K))=i_*\Psi_f^{\mathrm{un}}j^*K\xra{\can\oplus\can}\Xi_fj^*K\oplus i_*\Psi_f^{\mathrm{un}}K\xra{\can-\var}i_*\Psi_f^{\mathrm{un}}j^*K(-1), \\
	&r(D(c))=j_!K_U\xra{\delta\oplus\adj}\Xi_fK_U\oplus H^0(D(c))
	\xra{\vep-\adj}j_*K_U.
\end{align*}
As $r$ is an involution, it induces an equivalence between
the category of perverse sheaves and the category of gluing data, embedded
as subcategories of $M(X)_1^\sharp$, completing the proof.

We now provide details of the proofs of the
statements of Sasha's appendix on monads and diads. Reich provides details
for the key statement about the equivalence of diads of type $A_1^\#$ and
$A_2^\#$ but his proof is wrong (he claims certain maps obtained from a diad
of type $A_2^\#$ should yield an exact sequence when they don't even form a
complex).

\subsection*{Sasha's Appendix A1 \& A2}
Let $\mathcal A$ denote an exact category.
\begin{proof}[Proof of Sasha's Lemma A1]
Let $\mathcal A^\flat=:\mathcal A_0^\flat$ denote the category of 
monads and $\mathcal A^\flat_i$ Sasha's $A_i^\sim$.
Let $T_i:\mathcal A^\flat_i\ra\mathcal A^\flat_{i+1}$ be the functors
described, where $i\in\ZZ/3\ZZ$.
	
Define the functor $T_0^{-1}:\mathcal A^\flat_1\ra\mathcal A^\flat_0$ by
\begin{equation*}
	(P_{-1}\xhookrightarrow{\gamma_{-1}}P_0\xhookrightarrow{\gamma_0}P_1)
	\mapsto(P_{-1}\xhookrightarrow{\gamma_0\circ\gamma_{-1}}P_1\twoheadrightarrow P_1/P_0).
\end{equation*}
$T_0^{-1}$ is clearly an inverse to $T_0$.

Turning to $\mathcal A^\flat_2$, first note that the following square is a
pushout and pullback square, and therefore $\vep_-$ is an admissible
epimorphism and $\vep_+$ an admissible monomorphism.
\begin{equation*}\begin{tikzcd}
	L_-\arrow[r,twoheadrightarrow,"\vep_-"]\arrow[d,hook,"\delta_-"]
	&B\arrow[d,hook,"-\vep_+"] \\
	A\arrow[r,twoheadrightarrow,"\delta_+"]&L_+
\end{tikzcd}\end{equation*}
This is true because to give a morphism to $A$ and $B$ which agrees on
$L_+$ via $\delta_+$ and $-\vep_+$ is the same as giving a kernel for 
$A\oplus B\xra{\delta_++\vep_+}L_+$, and dually. We therefore have a
commutative diagram in which the right square is bicartesian, and claim that
the dashed arrow is an isomorphism.
\begin{equation*}\begin{tikzcd}
	\ker\vep_-\arrow[r,hook]\arrow[d,dashed]
	&L_-\arrow[r,twoheadrightarrow,"\vep_-"] \arrow[d,hook,"\delta_-"]
	&B\arrow[d,hook,"-\vep_+"] \\
	\ker \delta_+\arrow[r,hook]&A\arrow[r,twoheadrightarrow,"\delta_+"]&L_+
\end{tikzcd}\end{equation*}
To give a morphism to $A$ which $\delta_+$ sends to zero is (since $\vep_+$
is a monomorphism) the same as giving a morphism to $A\oplus B$ which is in
the kernel of $\delta_++\vep_+$ and with null projection to $B$.
This is the same as giving a morphism to $L-$ which $\vep_-$ sends to zero,
which is the same as giving a morphism to $\ker\vep_-$. Therefore every 
monad in $\mathcal A^\flat_2$ has $\vep_-$ an admissible epimorphism and
$\vep_+$ an admissible monomorphism, and
\begin{equation*}
	(L_-\xhookrightarrow{(\delta_-,\vep_-)}
	A\oplus B\xtwoheadrightarrow{(\delta_+,\vep_+)}L_+)
	\simeq
	(L_-\xhookrightarrow{(\delta_-,\vep_-)}A\oplus
	L_-/\ker\vep_-\xtwoheadrightarrow{(\delta_+,\vep_+)}A/\ker\vep_-),
\end{equation*}
where $\vep_+$ is induced by $\delta_-$. The composition of functors
$T_0\circ T_2$ sends
\begin{equation*}
	(L_-\xhookrightarrow{(\delta_-,\vep_-)}
	A\oplus B\xtwoheadrightarrow{(\delta_+,\vep_+)}L_+)
	\mapsto
	(\ker\vep_-\hookrightarrow L_-\hookrightarrow A),
\end{equation*}
and by the above isomorphism is clearly an equivalence with inverse
\begin{equation*}
	T_1:(\ker\vep_-\hookrightarrow L_-\hookrightarrow A)\mapsto
	(L_-\hookrightarrow A\oplus
	L_-/\ker\vep_-\twoheadrightarrow A/\ker\vep_-).
\end{equation*}
Therefore $T_1$ is also an equivalence with inverse $T_0\circ T_2$
and $T_2$ is an equivalence with inverse $T_1\circ T_0$:
\begin{equation*}\begin{tikzcd}
	\mathcal P:=P_-\hookrightarrow P\xtwoheadrightarrow{\alpha_+} P_+
	\arrow[d,rightsquigarrow,"T_0"] \\
	P_-\hookrightarrow\ker\alpha_+\hookrightarrow P
	\arrow[d,rightsquigarrow,"T_1"] \\
	\ker\alpha_+\hookrightarrow P\oplus H(\mathcal P)\twoheadrightarrow P/P_-
	\arrow[d,rightsquigarrow,"T_2"] \\
	\mathcal P.
\end{tikzcd}\end{equation*}
\end{proof}
In {\S}A2 the arrows $A\oplus B^i\ra D_+$ are admissible epimorphisms, not
monomorphisms.
\begin{proof}[Proof of Sasha's Lemma A2]
Let $\mathcal A^\sharp=:\mathcal A^\sharp_0$ denote the category of diads
and $\mathcal A^\sharp_i$ the related categories.
Let $T_0^\sharp:\mathcal A^\sharp_0\ra\mathcal A^\sharp_1$ and
$T_1^\sharp:\mathcal A^\sharp_1\ra\mathcal A^\sharp_2$ be as described; $T_0^\sharp$ is obviously an equivalence.
Note that $\mathcal A_1^\sharp$ is a full subcategory of
$\mathcal A_0^\flat$ and $\mathcal A_2^\sharp$ is a full subcategory of
$\mathcal A_2^\flat$. Therefore $T_1^\sharp$ is just the restriction of
$T_1\circ T_0$ to $\mathcal A_1^\sharp$. We have to check that $T_1^\sharp$ 
actually lands in $\mathcal A_2^\sharp\subset\mathcal A_2^\flat$. Once that
is established, if we can show that $T_2$ (the inverse of $T_1\circ T_0$),
restricted to $\mathcal A_2^\sharp$, lands in $\mathcal A_1^\sharp$,
Lemma A1 then implies that $T_1^\sharp$ is an equivalence with inverse
$T_2|_{\mathcal A_2^\sharp}$.
We have
\begin{equation*}\begin{tikzcd}[%
    column sep=-10pt
    ,/tikz/column 1/.append style={anchor=base east}
    ,/tikz/column 3/.append style={anchor=base west}
    ]
	\mathcal Q:=C_-\xhookrightarrow{\alpha_-=(\alpha_-^1,\alpha_-^2)}
	A&\oplus \arrow[d,rightsquigarrow,"T_1\circ T_0=T_1^\sharp"] 
	&B\xtwoheadrightarrow{\alpha_+=(\alpha_+^1,\alpha_+^2)}C_+
	\\
	\ker(\alpha_+)\hookrightarrow A\oplus B&\oplus H&(\mathcal Q)
	\twoheadrightarrow (A\oplus B)/C_-
\end{tikzcd}\end{equation*}
and to show that $T_1^\sharp$ lands in $\mathcal A_2^\sharp$ we must secure
that $\ker(\alpha_+)\hookrightarrow A\oplus H(\mathcal Q)$ and
$A\oplus H(\mathcal Q)\twoheadrightarrow(A\oplus B)/C-$.
The commutative diagram below has exact rows.
\begin{equation*}\begin{tikzcd}
	C-\arrow[r,hook]\arrow[d,hook]
	&\ker(\alpha_+)\arrow[r,twoheadrightarrow]\arrow[d]
	&H(\mathcal Q)\arrow[d,equals] \\
	A\arrow[r,leftrightarrow]\arrow[d,twoheadrightarrow]
	&A\oplus H(\mathcal Q)\arrow[r, leftrightarrow]\arrow[d]
	&H(\mathcal Q)\arrow[d,equals] \\
	C_+&(A\oplus B)/C_-\arrow[l,twoheadrightarrow]
	&\ker(\alpha_+)/C_-\arrow[l,hook']
\end{tikzcd}\end{equation*}
Both claims follow from the five lemma for exact categories.

It remains only to show that $T_2|_{\mathcal A_2^\sharp}$ lands in
$\mathcal A_1^\sharp\subset\mathcal A_0^\flat$; then it will be an inverse
to $T_1^\sharp$. We have
\begin{equation*}\begin{tikzcd}[%
    column sep=-10pt
    ,/tikz/column 1/.append style={anchor=base east}
    ,/tikz/column 3/.append style={anchor=base west}
    ]
	D_-\xhookrightarrow{(\gamma_-,\delta_-^1,\delta_-^2)}
	A\oplus &B^1\arrow[d,rightsquigarrow,"T_2"]&\oplus B^2
	\xtwoheadrightarrow{(\gamma_+,\delta_+^1,\delta_+^2)}D_+ \\
	\ker(\delta_-^2)\hookrightarrow A&\oplus &B^1
	\twoheadrightarrow\coker((\gamma_-,\delta_-^1):D_1\ra A\oplus B^1)
\end{tikzcd}\end{equation*}
and we need to verify that $\ker\delta_-^2\ra A$ is admissible
monomorphism and $A\ra\coker((\gamma_-,\delta_-^1):D_1\ra A\oplus B^1)$
is admissible epimorphism. We know from A1 that $\delta_-^1$ and
$\delta_-^2$ are admissible epimorphisms while $\delta_+^1$ and $\delta_+^2$
are admissible monomorphisms. If $\ker\delta_-^2\ra A$ admitted a cokernel, 
then it would follow from the obscure axiom applied to the commutative 
diagram with exact rows
\begin{equation*}\begin{tikzcd}
	\ker\delta_-^2\arrow[d]\arrow[r,hook]
	&D_-\arrow[d,hook]\arrow[r,twoheadrightarrow]
	&B^2\arrow[d,equals] \\
	A\arrow[r,hook]&A\oplus B^2\arrow[r,twoheadrightarrow]&B^2
\end{tikzcd}\end{equation*}
that $\ker\delta_-^2\ra A$ would be an admissible monomorphism.
En effet,
\begin{equation*}
	\coker(\ker\delta_-^2\ra A)
	=\coker(D_-\xhookrightarrow{(\gamma_-,\delta_-^2)}A\oplus B^2),
\end{equation*}
as both can be described as the colimit of the following diagram.
\begin{equation*}\begin{tikzcd}
	D_-\arrow[r,twoheadrightarrow,"-\delta_-^2"]\arrow[d,"\gamma_-"]&B^2 \\ A
\end{tikzcd}\end{equation*}
This identification additionally shows that the diagram
\begin{equation*}\begin{tikzcd}
	\ker\delta_-^2\arrow[r,hook]\arrow[d,hook]
	&A\arrow[d,hook]\arrow[r,twoheadrightarrow]
	&\coker(\ker\delta_-^2\ra A)\arrow[d,equals] \\
	D_-\arrow[r,hook]&A\oplus B^1\arrow[r,twoheadrightarrow]
	&\coker((\gamma_-,\delta_-^1):D_-\ra A\oplus B^1)
\end{tikzcd}\end{equation*}
commutes so that
$A\twoheadrightarrow\coker((\gamma_-,\delta_-^1):D_-\ra A\oplus B^1)$ is
an admissible epimorphism.
(Moreover the square to the left in this diagram is bicartesian \cite[2.12]{buhler}.)
We've proved that the functors
\begin{equation*}\begin{tikzcd}
	\mathcal A_1^\sharp\arrow[r,"T_1^\sharp",shift left]
	&\mathcal A_2^\sharp\arrow[l,"T_2|_{\mathcal A_2^\sharp}",shift left]
\end{tikzcd}\end{equation*}
are mutually inverse equivalences.
\end{proof}
\subsubsection*{The reflection functor}\label{sec:reflection}
We wish to compute $T_2|_{\mathcal A_2^\sharp}\circ r\circ T_1^\sharp$, 
which is an involution of the category $\mathcal A_1^\sharp$ 
(since $r$ is an involution of $\mathcal A_2^\sharp$ and
$T_2|_{\mathcal A_2^\sharp}=(T_1^\sharp)^{-1}$) that we will also
denote by $r$; the diagram below illustrates the composition.
\begin{equation*}\begin{tikzcd}
	\mathcal Q:=C_-\xhookrightarrow{\alpha_-=(\alpha_-^1,\alpha_-^2)}
	A\oplus B\xtwoheadrightarrow{\alpha_+=(\alpha_+^1,\alpha_+^2)}C_+
	\arrow[d,rightsquigarrow,"T_1\circ T_0=T_1^\sharp"] \\
	\ker(\alpha_+)\xhookrightarrow{(\gamma_-,\delta_-^1,\delta_-^2)}
	A\oplus B\oplus H(\mathcal Q)
	\xtwoheadrightarrow{(\gamma_+,\delta_+^1,\delta_+^2)}
	(A\oplus B)/C_-
	\arrow[d,leftrightsquigarrow,"r"] \\
	\ker(\alpha_+)\xhookrightarrow{(\gamma_-,\delta_-^2,\delta_-^1)}
	A\oplus H(\mathcal Q)\oplus B
	\xtwoheadrightarrow{(\gamma_+,\delta_+^2,\delta_+^1)} (A\oplus B)/C_-
	\arrow[d,rightsquigarrow,"T_2"] \\
	\ker(\delta_-^1)\hookrightarrow A\oplus H(\mathcal Q)\twoheadrightarrow
	\coker(\ker\alpha_+\hookrightarrow A\oplus H(\mathcal Q))
\end{tikzcd}\end{equation*}
It is easy to see that $\ker(\delta_-^1:\ker\alpha_+\ra B)=\ker(\alpha_+^1)$
because to give a morphism to $\ker(\alpha_+)$ which goes to zero
under $\ker\alpha_+\hookrightarrow A\oplus B\twoheadrightarrow B$ is the
same as giving a morphism to $A$ which goes to zero under $\alpha_+^1$.
As for $\coker(\ker\alpha_+\hookrightarrow A\oplus H(\mathcal Q))$, to give
a map from $A\oplus H(\mathcal Q)$ that kills $\ker(\alpha_+)$ is to give a
map $a_1$ from $A$ and $a_2$ from $H(\mathcal Q)$ such that $a_1-a_2$ kills
$\ker(\alpha_+)$; as
$C_-\hookrightarrow\ker(\alpha_+)\twoheadrightarrow H(\mathcal Q)$ is
short exact, to give a map from $A$ which coincides on $\ker(\alpha_+)$ with
a map from $H(\mathcal Q)$ is to give a map from $A$ which annihilates
$C_-$. Therefore
$\coker(\ker\alpha_+\hookrightarrow A\oplus H(\mathcal Q))=\coker(C-\hookrightarrow A)$, so
\begin{equation*}
	r(\mathcal Q)=\ker(\alpha_+^1)\xhookrightarrow{} A\oplus H(\mathcal Q)
	\twoheadrightarrow \coker(\alpha_-^1)=A/C_-.
\end{equation*}
The map $\ker(\alpha_+^1)\hookrightarrow A$ is the natural inclusion
and $A\twoheadrightarrow A/C_-$ the natural projection;
the map $\ker(\alpha_+^1)\ra H(\mathcal Q)$ coincides with $a$ in the
below commutative diagram (with rows that are not exact)
\begin{equation*}\begin{tikzcd}
	\ker(\alpha_+^1)\arrow[rr,bend left,"a"]\arrow[r,hook]\arrow[d,hook]
	&\ker(\alpha_+)\arrow[r,twoheadrightarrow]\arrow[d,hook]
	&H(\mathcal Q)\arrow[d,hook,"\iota"]\arrow[dd,bend left=60pt,"b"] \\
	A\arrow[r,hook]&A\oplus B\arrow[r,twoheadrightarrow]
	&(A\oplus B)/C_-\arrow[d,"\pi"] \\
	&&A/C_-.
\end{tikzcd}\end{equation*}
Finally, the map $H(\mathcal Q)\ra A/C_-$ coincides with $-b$ in the diagram
above, the negative sign appearing because we have
$-\gamma_0:P_0/P_{-1}\ra P_1/P_{-1}$ in the definition of the functor $T_1$
so that $\delta_+^2:H(\mathcal Q)\ra(A\oplus B)/C_-$
coincides with $-\iota$ in the composition
\begin{equation*}
	H(\mathcal Q)\xra{\delta_+^2}(A\oplus B)/C_-\ra A/C_-.
\end{equation*}




\addtocontents{toc}{\protect\setcounter{tocdepth}{-1}}
\begin{thebibliography}{BBD}
	\bibitem[B]{glue} \emph{How to glue perverse sheaves} par Beilinson
	\bibitem[BBD]{BBD} \textit{Faisceaux Pervers}
	par Beilinson, Bernstein, Deligne (\& Gabber!)
	\bibitem[M]{M} \emph{Beilinson's construction of nearby cycles and gluing} par Sophie Morel
\end{thebibliography}
\addtocontents{toc}{\protect\setcounter{tocdepth}{1}}
\end{document}
