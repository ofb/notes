\documentclass[deligne.tex]{subfiles}

\begin{document}
$p,q,\ell,\FF_q,\FF$: $p$ is a prime number, $q=p^f$ is a power of $p$ and 
$\FF$ an algebraic closure of the field $\FF_q$; $\ell$ is a prime number
$\ne p$.

$X_0,X$: $X_0$ is a scheme on $\FF_q$, $X=X_0\times_{\FF_q}\FF$.
If $\mathscr F_0$ is an (étale) sheaf on $X_0$, $\mathscr F$ denotes its 
inverse image on $X$.

\subsection{Mise en garde}
The exposition beginning with~\ref{sec:houz} is based on SGA 5, Exposé XIV 
by C. Houzel. This approach is explicit at the expense being somewhat
anti-conceptual and overloaded with cumbersome notation, which makes it
hard to remember. On the other hand, Deligne's approach to Frobenius is
conceptually beautiful and easy to remember, but the first time I read it
I couldn't understand it. Now that I do, I write these notes.
Deligne uses $\varphi$ to denote what he calls the `Frobenius substitution';
it is the well-known topological generator of $\Gal(\FF/\FF_q)$.
He uses $F$ to denote the `Frobenius endomorphism', notated $\fr_{X_0}$
below, which is the endomorphism of a scheme $X_0/\FF_q$ obtained by
$x\mapsto x^q$ on the structure sheaf $\mathscr O_{X_0}$.
Finally, he uses $F^*$ to denote the action of Frobenius on sheaves on a
scheme over $\FF_q$ (and base extensions of such), and their cohomology.

\subsection{Frobenius following Deligne}
\subsubsection{Representing the Frobenius correspondence}
Let $Q$ be any scheme and $\mathscr G$ an (étale) sheaf on $Q$.
Let $Q_{\mathrm{pet}}$ denote the category (Et/$Q$) of algebraic spaces
étale over $Q$, equipped with the étale topology.
We `recall' the following
\begin{proposition*} $\mathscr G$ is represented over $Q_{\mathrm{pet}}$ by
l'espace étalé $[\mathscr G]$ of $\mathscr G$: an algebraic space, locally 
separated and étale over $Q$.
If moreover $\mathscr G$ is
\begin{enumerate}[label=(\greek*)]
	\item locally constant constructible
	\item locally constant,
	\item constructible,
\end{enumerate}
then $[\mathscr G]$ may be taken to be
\begin{enumerate}[label=(\greek*)]
	\item a scheme finite étale over $Q$ (SGAA Exp. IX 2.2).
	\item a scheme étale over $Q$ (SGAA Exp. IX 2.2).
	\item étale and finitely presented as an algebraic space over $Q$ (SGAA Exp. IX 2.7).
\end{enumerate}
\end{proposition*}
In other words, $\mathscr G$ is the sheaf of local sections of the space
$[\mathscr G]$: for each $V\in\mathrm{Et/}Q$,
\begin{equation*}
	\mathscr G(V)=\Hom_Q(V,[\mathscr G]).
\end{equation*}
M. Artin constructs the espace étalé $[\mathscr G]$ associated to an 
arbitrary étale sheaf $\mathscr G$ on a scheme $Q$ in
\href{http://math.uchicago.edu/~barrett/resources/Artin_theoremes_de_representabilite_pour_les_espaces_algebriques.pdf}{\emph{Théorèmes de représentabilité pour les espaces algébriques}} VII \S1.
A sketch: put
\begin{equation*}
	(\mathcal U\ra\mathscr G):=\coprod_{U,\xi\in\mathscr G(U)}(U\xra\xi\mathscr G),
\end{equation*}
the sum executed over affine schemes $U$ étale over $Q$ and the
$\xi\in\mathscr G(U)$. The canonical morphism
$\mathcal U\times_\mathscr G\mathcal U\ra\mathcal U\times_Q\mathcal U$
induced by $\xi\in\mathscr G(U)$ and $\eta\in\mathscr G(V)$
($U,V$ affine étale schemes over $Q$) is an open immersion and defines an 
étale equivalence relation on $\mathcal U$. Let $[\mathscr G]$ be the 
quotient of $\mathcal U$ by this equivalence relation; it is an algebraic
space over $Q$, in general not separated, but only locally separated;
in other words,
$\mathcal U\times_{\mathscr G}\mathcal U\ra\mathcal U\times_Q\mathcal U$ 
is not necessarily a closed immersion.

Now let $X_0$ be a scheme over $\FF_q$ and $\mathscr F_0$ a sheaf on $X_0$.
The formation of the espace étalé $[\mathscr F_0]$ of $\mathscr F_0$ yields
the commutative square
\begin{equation*}\begin{tikzcd}[]
	{[\mathscr F_0]}\arrow[r,"F"]\arrow[d]&{[\mathscr F_0]}\arrow[d] \\
	X_0\arrow[r,"F"]&X_0.
\end{tikzcd}\end{equation*}
and hence a morphism $[\mathscr F_0]\ra F^*[\mathscr F_0]$.
Replacing $S$ with $X_0$, and $X_0$ with $[\mathscr F_0]$, 
\ref{sec:relFrobDef} below finds that this morphism is
relative Frobenius $\Fr_{[\mathscr F_0]/X_0}$; as $[\mathscr F_0]$ is étale
over $X_0$, \ref{sec:relFrobProp} tells us that this morphism is an
isomorphism $[\mathscr F_0]\xra\sim F^*[\mathscr F_0]$ with inverse
Deligne's Frobenius correspondence (1.2.1)
\begin{equation*}
	F^*:F^*\mathscr F_0\xra\sim\mathscr F_0.
\end{equation*}
This morphism is the same as the one constructed with different notation in
\ref{sec:FrobCor}.
The point is that by putting additional hypotheses on $\mathscr F_0$, one
may assume that $[\mathscr F_0]$ is in fact a scheme étale over $X_0$.

\subsubsection{The Frobenius correspondence and the Frobenius endomorphism}
Let $\mathscr F_0$ be an abelian sheaf on $X_0$. We wish to elucidate
Deligne's approach in 1.8 of \emph{Rapport} to show that
\begin{equation*}
	F^{*-1}=\varphi\qquad\text{(on $H_c^i(X,\mathscr F)$)}.
\end{equation*}
Letting $Y_0=\Spec\FF_q$, as written
$H_c^i(X,\mathscr F)=[R^if_!\mathscr F_0](\FF)$, and, noting $F=\id$ on 
$Y_0$ we have the (stupid) diagram
\begin{equation*}
\begin{tikzcd}
&{[R^if_!\mathscr F_0]}\arrow[dr,"F"]\arrow[drr, dashed,"F",bend left=20] \\
&&{[R^if_!\mathscr F_0]}\arrow[r,"\id"']\arrow[d] &{[R^if_!\mathscr F_0]}\arrow[d] \\
&&Y_0\arrow[r,"\id"] &Y_0
\end{tikzcd}
\end{equation*}
which just serves to connect this discussion to that of the previous section 
and show that the morphism defining the inverse of the
Frobenius correspondence $F^*$ on $[R^if_!\mathscr F_0]$ is indeed
$F:[R^if_!\mathscr F_0]\ra [R^if_!\mathscr F_0]$; as
$F$ acts on geometric points by $\varphi$, we see that
\begin{equation*}
	F^{*-1}:[R^if_!\mathscr F_0](\FF)\ra [R^if_!\mathscr F_0](\FF)
\end{equation*}
coincides with $\varphi$.

\subsection{Frobenius following Houzel}\label{sec:houz}
\begin{definition*}
We denote by $\fr_{X_0}$ the morphism of schemes $X_0\ra X_0$ which is the 
identity on the underlying topological space $|X_0|$ and acts on the
structure sheaf $\mathscr O_{X_0}$ by $g\mapsto g^q$.

This morphism is called \emph{absolute Frobenius}.
\end{definition*}

\subsubsection{}\label{sec:relFrobDef}
If the structure morphism $X_0\ra\FF_q$ factors through some scheme $S$,
then we denote by
$X_0^{(q/S)}:=X_0\times_{S,\fr_S}S$ the fiber product of $g:X_0\ra S$ by the morphism 
$\fr_S:S\ra S$ with projection $\pi_{X_0/S}:X_0^{(q/S)}\ra X_0$.
The absolute Frobenius $\fr_{X_0}$ then factors through the morphism $\pi_{X_0/S}$.
We can form the diagram
\begin{equation*}
\begin{tikzcd}
&X_0\arrow[dr,"\Fr_{X_0/S}"]\arrow[drr, dashed,"\fr_{X_0}",bend left=20]\arrow[ddr,"g"',bend right=20] \\
&&X_0^{(q/S)}\arrow[r,"\pi_{X_0/S}"']\arrow[d,"g^{(q)}"] &X_0\arrow[d,"g"] \\
&&S\arrow[r,"\fr_S"] &S
\end{tikzcd}
\end{equation*}
\begin{definition*}
    The morphism $\Fr_{X_0/S}$ is called \emph{relative Frobenius}.
\end{definition*}
\subsubsection{Frobenius acts on geometric points}
Consider the set of geometric points $X(\FF)=X_0(\FF)$.
Frobenius acts on this set by $\varphi\in\Gal(\FF/\FF_q)$,
$\varphi(x)=x^q$. In particular, as $\FF_q$ is perfect, $X^F=X_0(\FF_q)$, where
$X^F$ denotes the geometric points fixed by Frobenius. In slightly more words, consider
a geometric point $\overline x\rightarrow X_0$ centered on $x$. The fiber
$X\times_{X_0}x$ is isomorphic to the spectrum of $A=\FF\otimes_{\FF_q}k(x)$.
As $k(x)/\FF_q$ is separable, $A\sim\prod_{[k(x):\FF_q]}\FF$, and $[k(x):\FF_q]$ is also
equal to the number of $\FF_q$-embeddings $k(x)\rightarrow\FF$. Such an embedding is
fixed by $\varphi$ iff $k(x)=\FF_q$, and by $\varphi^f$ iff $k(x)\subset \FF_{q^f}$.
So for every point $x\in|X_0|$ with $[k(x):\FF_q]=f$, there are $f$
geometric points centered on $x$; $F$ acts transitively by $\varphi$ on this set, and
$F^f$ fixes each of these geometric points.

\subsubsection{Behavior of relative Frobenius}\label{sec:relFrobProp}
The relative Frobenius $\Fr_{X_0/S}$ is integral, surjective, and radicial, hence
a universal homeomorphism. This is clear when $S=\FF_q$;
i.e. for $\fr_{X_0}=\pi_{X_0/S}\circ\Fr_{X_0/S}$; it follows that
$\Fr_{X_0/S}$ is radicial \cite[I 3.5.6 (ii)]{EGA}.
Moreover, $\pi_{X_0/S}$ is separated and radicial, therefore
$\Fr_{X_0/S}$ is integral \cite[II 6.1.5 (v)]{EGA} and surjective.

Suppose moreover that $g:X_0\ra S$ is étale. The same is true of
$g^{(q)}:X_0^{(q)}\ra S$, and therefore $\Fr_{X_0/S}:X_0\ra X_0^{(q)}$ is étale.
As $\Fr_{X_0/S}$ is also radicial and surjective, it is an isomorphism.

\subsubsection{Frobenius correspondence}\label{sec:FrobCor}
Let $X_0$ be a scheme over $\FF_q$ and $\mathscr F_0$ a sheaf of sets on
$\et{(X_0)}$ (the small étale site of $X_0$ whose underlying category is the 
category of schemes étale over $X_0$).
We have for all $U\ra X$ étale $(\fr_{X_0})_\ast\mathscr F_0(U)=\mathscr F_0(U^{(q/X)})$.
The isomorphism $\mathscr F_0(\Fr_{U/X_0}):(\fr_{X_0})_\ast\mathscr F_0(U)\ra \mathscr F_0(U)$
is natural in $U$ and induces an isomorphism of sheaves
\begin{equation*}
    \mathscr F_0(\Fr_{/X_0}):(\fr_{X_0})_\ast\mathscr F_0\xrightarrow{\sim}\mathscr F_0;
\end{equation*}
by adjunction applied to $\mathscr F_0(\Fr_{/X_0})^{-1}$ we obtain a morphism
\begin{equation*}
    F^\ast : \fr_{X_0}^\ast \mathscr F_0\rightarrow\mathscr F_0.
\end{equation*}
As $\fr_{X_0}$ is integral, surjective, and radicial, $\fr_{X_0}^\ast:\et{(X_0)}\ra\et{(X_0)}$
is an equivalence of sites, and the functors
\begin{equation*}
    (\fr_{X_0})_\ast,\fr_{X_0}^\ast:\widetilde{\et{(X_0)}}\longrightarrow \widetilde{\et{(X_0)}}
\end{equation*}
are autoequivalences and quasi-inverses, where $\widetilde{\et{(X_0)}}$ denotes the
étale topos on $X_0$ \cite[Exp. VIII, 1.1]{SGAA}.
Therefore $F^\ast$ is also an isomorphism.
\begin{definition*}
    The isomorphism $F^\ast : \fr_{X_0}^\ast\mathscr F_0\longrightarrow\mathscr F_0$ is called
    the \emph{Frobenius correspondence}.
\end{definition*}

\subsubsection{Frobenius acts on cohomology}
Consider $\mathscr F_0$ a sheaf of $\Lambda$-modules, for some commutative ring $\Lambda$.
The canonical morphism $\alpha:\mathscr F_0\ra{\fr_{X_0}}_\ast\fr_{X_0}^\ast\mathscr F_0$
gives rise to
\begin{equation*}
    \Gamma(X_0,\mathscr F_0)\xra{\alpha}\Gamma(X_0,\fr_{X_0}^\ast\mathscr F_0)
    \xra{F^\ast}\Gamma(X_0,\mathscr F_0);
\end{equation*}
we also denote the composition of these maps by $F^\ast$.
When $\mathscr F_0=\sh{\Lambda}$, this composition is easily seen to coincide with
$\id_{\Gamma(X_0,\sh{\Lambda})}$,
and as every section $s\in\Gamma(X_0,\mathscr F_0)$ corresponds to a morphism
$s:\sh{\Lambda}\ra\mathscr F_0$ and $F^\ast$ is evidently functorial in $\mathscr F_0$,
we find $F^\ast\circ\Gamma(X_0,s)=\Gamma(X_0,s)\circ F^\ast=\Gamma(X_0,s)$, ergo
$F^\ast s=s$, so $F^\ast$ induces the identity on ${\Gamma(X_0,\mathscr F_0)}$.
Recalling the definition of the Frobenius correspondence via adjunction, this action of Frobenius on $\Gamma(X_0,\mathscr F_0)$ coincides with the composition
\begin{equation*}
    \Gamma(X_0,\mathscr F_0)\xrightarrow{\mathscr F_0(\Fr_{/X_0})^{-1}}
    \Gamma(X_0,(\fr_{X_0})_\ast\mathscr F_0)=\Gamma(X_0,\mathscr F_0).\tag{$\ast$}
\end{equation*}
Considering $\mathscr F_0$ now as an object of $D^+(X_0,\Lambda)$, we have 
${\fr_{X_0}}_\ast$ preserves injective objects; hence the composition
\begin{equation*}
    F^\ast:R\Gamma(X_0,\mathscr F_0)\xra{\mathscr F_0(\Fr_{/X_0})^{-1}} R\Gamma(X_0,{\fr_{X_0}}_\ast\mathscr F_0)\lra R\Gamma(X_0,\mathscr F_0)
\end{equation*}
can be computed by applying $(\ast)$ term-by-term to an injective resolution of
$\mathscr F_0$, whence we see that $F^\ast$ acts by identity on $R\Gamma(\mathscr F_0)$,
and hence
\begin{equation*}
    F^\ast:H^i(X_0,\mathscr F_0)\lra H^i(X_0,\mathscr F_0)
\end{equation*}
is the identity for all $i$.

Suppose that $X_0$ is separated and of finite type over $\FF_q$.
Then we can replace $\Gamma$ in the above discussion by $\Gamma_c$ to find that
Frobenius acts on compactly supported cohomology
\begin{equation*}
    F^\ast : H_c^i(X_0,\mathscr F_0)\lra H_c^i(X_0,\mathscr F_0).
\end{equation*}

\setcounter{section}{2}\setcounter{subsection}{0}
\subsection{The trace formula for $\Spec{\FF_{q^n}}\ra\Spec{\FF_q}$}
Let $X_0=\Spec{\FF_{q^n}}$, $q=p^f$,
and $\mathscr F_0$ be a constructible $\Qell$
sheaf on $X_0$, $\mathscr F$ its inverse image on $X$.
In this case, the cohomological description of the
$L$-function $L(X_0,\mathscr F_0)$ reads very simply
\begin{equation*}
    \tag{$\ast$}\det(1-F^\ast_x t^{d(x)},\mathscr F)^{-1}
    =\det(1-F^\ast t^f,H^0_c(X,\mathscr F))^{-1},
\end{equation*}
where $d(x)=[k(x):\FF_p]$. We first need to make precise how Frobenius
is acting on the left and right sides.

On the left side, we fix a geometric point $\overline x\rightarrow x=X$ and
construct the action of Frobenius on the fiber $\mathscr F_{\overline x}$
by picking the smallest power of $F_{(q)}$ which actually fixes the
geometric point $\overline x$, namely $F_{(q^n)}=F_{(q)}^n$.
The notation $F^\ast_x$ denotes the endomorphism of
$\mathscr F_{\overline x}$ induced by $F^\ast_{(q^n)}$.
Up to isomorphism, $(F_x,\mathscr F_{\overline x})$ do not depend on the
choice of geometric point $\overline x\rightarrow x$, and the trace,
determinant of this action are denoted by $\Tr(F_x^\ast,\mathscr F)$, etc.

Now, on the right side, we have the Frobenius correspondance on cohomology.
We will make use of the identity
\begin{equation*} {F^\ast}^{-1}=\varphi,\end{equation*}
where $\varphi$ is the Frobenius considered as the topological generator
of $\Gal(\FF,\FF_q)$ (c.f. remark below).
The data of a $\Qell$-sheaf on $X$ is equivalent to the data of a
finite-dimensional $\Qell$-vector space $V=\mathscr F_{\overline x}$ on
which $\Gal(\overline k(x)/k(x))=\pi_x$ acts continuously.
There is a canonical isomorphism
\begin{equation*}
    \pi_x=\Gal(\overline k(x),k(x))\simeq\hat\ZZ
\end{equation*}
furnished by the Frobenius element
\begin{equation*}
    \varphi_x\in\Gal(\overline k(x),k(x))=\Gal(\FF/\FF_{q^n}),\qquad
    \varphi_x(\lambda)=\lambda^{q^n},
\end{equation*}
so that the action of $\pi_x$ on $V=\mathscr F_{\overline x}$ is known once
one knows the automorphism $(\varphi_x)_V$ (under the one condition that
$(\varphi_x)^\nu\ra\id_V$ as $\nu\ra0$ multiplicatively). If
\begin{equation*} \pr_x:X=\Spec(\FF_{q^n})\ra\Spec\FF_q=e\end{equation*}
is the canonical morphism, $\pi_x$ is identified via ${\pr_x}_\ast$
with a subgroup of the analogous Galois group $\pi_e$ for $e=\Spec(\FF_q)$,
itself topologically generated by $\varphi$, and via this identification
we have the identity \begin{equation*} \varphi_x=\varphi^n.\end{equation*}
The sheaf ${\pr_x}_\ast(\mathscr F)$ is defined by the induced module
\begin{equation*}
    {\pr_x}_\ast(\mathscr F)_{\overline e}\simeq\mathscr F_{\overline x}
    \otimes_{\pi_x}\pi_e,
\end{equation*}
from which one deduces that, letting $f=\varphi_x^{-1}$,
$\varphi^{-1}$ acts on ${\pr_x}_\ast(\mathscr F)_{\overline e}$ by
\begin{equation*}
f^{(n)}:(x_1,\ldots,x_n)\mapsto(f(x_n),x_1,\ldots,x_{n-1}),
\end{equation*}
where here we have written ${\pr_x}_\ast(\mathscr F)_{\overline e}$ with 
respect to a basis as a free $\pi_x$-module of rank $n$.
Now the formula ($\ast$) is a matter of verifying the formula
\begin{equation*}
    \det(1-f t^n)=\det(1-f^{(n)} t)
\end{equation*}
for $f$ acting on a free module of rank $n$. This is Deligne's
corollary 3.4.

\remark Perhaps one way to think about the identification
$F^\ast=\varphi^{-1}$ is by setting up the usual diagram
\begin{equation*}
\begin{tikzcd}
&X\arrow[dr,"\Fr_{X/X_0}"]\arrow[drr, dashed,"\fr_X",bend left=20]\arrow[ddr,"g"',bend right=20] \\
&&X^{(q/X_0)}\arrow[r,"\pi_{X/X_0}"']\arrow[d,"g^{(q)}"] &X\arrow[d,"g"] \\
&&X_0\arrow[r,"\fr_{X_0}"] &X_0
\end{tikzcd}
\end{equation*}
with $g$ the base extension of the map $\Spec\FF\rightarrow\Spec\FF_q$ that
arises from fixing an algebraic closure of $\FF_q$, and
then observing that $\Fr_{X/X_0}=\varphi\times_{\FF_q}\id_{X_0}$.
Recalling that $F^\ast$ on $\mathscr F_0/X_0$ is induced by 
$\Fr_{U/X_0}^{-1}$ for $U\ra X_0$ étale, and by functoriality
$F^\ast$ on $\mathscr F/X$ is induced by pulling back the same, hence by
$\Fr_{U/X_0}^{-1}$ for $U\ra X$ étale, in particular we have that
$F^\ast$ on $H^0_c(X,\mathscr F)$ is induced by
$\mathscr F(\Fr_{X/X_0})^{-1}=\mathscr F(\varphi^{-1}\times_{\FF_q}\id_{X_0})$.

\setcounter{section}{3}\setcounter{subsection}{0}
\subsection{Le sorite de la notation}
It is very important to note that in Deligne's notation,
$\Tr(F^\ast_x,\mathscr F)$ and $\Tr(F^\ast,\mathscr F_x)$ are traces of
possibly different operators on the fiber $\mathscr F_x$.
Namely, if $\mathscr F$ is a $\QQ_\ell$-sheaf on $X_0$ a scheme separated and
of finite type over $\FF_q$, then $\mathscr F_{\overline x}$ is a
$\QQ_\ell$ vector space, for a choice of geometric point $\overline x$ centered
on a closed point $x$ of $X_0$. Then $F^\ast_x$ denotes the 
Frobenius $F^\ast_{q^n}=F^{\ast n}_q$
raised to the power of the residue field extension $n=[\deg k(x) : \FF_q]$.
This power of Frobenius is the least that fixes each geometric point centered 
on $x$, and the notation $\Tr(F^\ast_x,\mathscr F)$ means
$\Tr(F^\ast_x,\mathscr F_{\overline x})$.

On the other hand, if, say, $x\in X^{F^n}$ is a geometric point centered on a
point of $X_0$ defined over $\FF_{q^n}$, $\Tr(F^\ast, \mathscr F_x)$ denotes
(absolute) $q$-power Frobenius acting on the fiber. So, unless $x\in X^F$,
$\Tr(F^\ast,\mathscr F_x)$ and $\Tr(F^\ast_x,\mathscr F)$ are traces of
different operators on the same vector space, the latter a power of the other.

\subsection{Le sorite des faisceaux localement constants}
\subsubsection*{The case of locally constant sheaves of sets}
Let $X$ be a scheme and $\mathscr L$ a locally constant sheaf of 
sets with finite fibers on $X$.
(With additional assumptions on $X$, the case of a
locally constant sheaf of sets with infinite fibers is reduced to the finite
case in the course of the discussion of Weil II \nameref{1.7.8}.)
We know that $\mathscr L=h_U$ for some $U\ra X$ revêtement étale.
We know that every revêtement étale of $X$ is étale-locally on $X$ trivial;
namely for some $V\ra X$ étale, $U_V\sim\coprod V$. We wish to show that we may
take $V\ra X$ to be a revêtement étale
(with no further work, we could then take it to be a \emph{galoisian} 
revêtement, i.e. a connected torsor for the automorphism group of the fiber, 
as principal Galois objects in a Galois category form a cofinal system).

First note that if $f:X\ra Y$ is any morphism of schemes and $V\ra Y$ is étale,
then $f^\ast h_V\sim h_{V\times_Y X}$.
To see this, observe
\begin{equation*}
    \Hom(f^\ast h_V,\mathscr G)
    =\Hom(h_V, f_\ast\mathscr G)
    =f_\ast\mathscr G(V)
    =\mathscr G(V\times_Y X)
    =\Hom(h_{V\times_Y X},\mathscr G).
\end{equation*}
Evidently, this argument holds true for any morphism of sites.

So, it will suffice to show that any revêtement étale can be trivialized
after base extension by a revêtement étale. To see this, assume $X$ connected 
and let $U\ra X$ be a revêtement étale of constant degree $d$ and proceed by 
recurrence on $d$, the case $d=1$ being trivial.
(Of course, in the special case that $U\ra X$ is galoisian with Galois group $G$,
$U\times_X U\sim U\times G$ is a trivial $G$-torsor, and we are done.)

As $U\ra X$ is étale, hence net, and finite, the diagonal morphism
$U\ra U\times_X U$ is simultaneously an open and closed immersion, hence
an isomorphism onto a connected component of
$U\times_X U$, allowing us to write $U\times_X U=U\coprod Z$ with
$Z\ra U$ of constant degree $d-1$. By hypothesis, there exists a revêtement
étale $V\ra U$ such that $Z\times_U V\sim\coprod_{d-1}V$.
Our desired revêtement is then simply the composition $V\ra U\ra X$:
\begin{equation*}
    V\times_X U=V\times_U U\times_X U=V\times_U(U{\textstyle\coprod} Z)
    =V{\textstyle\coprod} (V\times_U Z)={\textstyle\coprod}_d V.
\end{equation*}

\subsubsection*{The case of locally constant constructible sheaves}
Let $\Lambda$ be a commutative, noetherian torsion ring. We adapt the above
discussion to locally constant constructible (l.c.c.) sheaves of
$\Lambda$-modules. Let $\mathscr F$ be a l.c.c. sheaf of $\Lambda$-modules
on a connected scheme $X$. Then the fibers of $\mathscr F$ are finite sets 
and the above discussion yields a revêtement étale $f:V\ra X$ with $V$ a
(connected) galoisian cover with Galois group $H$
(i.e. $V$ is a $H$-torsor) such that $f^\ast\mathscr F$ is a constant sheaf.
Its constant value $H^0(V,f^\ast\mathscr F)$ is a $\Lambda[H]$-module.

The sheaf $f_\ast\Lambda$ on $X$, together with the natural action of $H$,
is a rank 1 l.c.c. sheaf of $\Lambda[H]$-modules.
Relative to the natural action of $H$ on $f_\ast f^\ast\mathscr F$, the
trace morphism $f_\ast f^\ast\mathscr F\ra\mathscr F$ factors by an
isomorphism
\begin{equation*} (f_\ast f^\ast\mathscr F)_H\ra\mathscr F, \end{equation*}
and we have
$f_\ast f^\ast\mathscr F=f_\ast M=f_\ast\Lambda\otimes_\Lambda M$
with the diagonal action of $H$.

The above discussion shows that a l.c.c. sheaf of $\Lambda$-modules on a
connected scheme $X$ is determined by its restriction to the small étale
site of $X$. Sheaves on the small étale site $\mathscr U$ are in turn
determined by Grothendieck's Galois theory: fixing a geometric point
$\overline x$ of $X$ and putting $G:=\pi_1(X,\overline x)$, the functor
\begin{align*}
	\operatorname{Sh}(\mathscr U)&\ra \text{finite }G\text{-sets} \\
	\mathscr F&\mapsto\mathscr F_{\overline x}
\end{align*}
admits the inverse
\begin{align*}
	\text{finite }G\text{-sets}&\ra\operatorname{Sh}(\mathscr U) \\
	\mathscr F_{\overline x}&\mapsto
	[V\in\mathscr U\mapsto\Hom_G(V_{\overline x},\mathscr F_{\overline x})].
\end{align*}
To verify this, as the torsors are cofinal in a covering of any
$V\in\mathscr U$, we may cover $V$ by a torsor $W$ with Galois group $H$
and combine the equalizer description of $\mathscr F(V)$
\begin{equation*}
	\mathscr F(V)\ra\mathscr F(W)
	\rightrightarrows\mathscr F(W\times_V W)
\end{equation*}
with the description of such in the case of a Galois torsor;
c.f.~\cite[I \S5]{SGA4.5}.

The discussion in the previous section can be rephrased using the
monodromy representation of a l.c.c. sheaf. Namely, let $\mathscr F$
be a l.c.c. sheaf of $\Lambda$-modules on a connected scheme $X$ pointed
by a geometric point $\overline x$ as above; $\mathscr F$ corresponds
to a representation $\pi_1(X,\overline x)\ra\GL(\mathscr F_{\overline x})$.
As the latter is a finite group, the kernel of this representation is of
finite index, and as the Galois coverings are cofinal, we can find a Galois
cover of $X$ corresponding to a open subgroup contained in the kernel.
The sheaf $\mathscr F$ becomes constant when restricted to this cover.

\subsubsection*{The case of lisse sheaves} Let $E\subset\overline\QQ_\ell$
be an finite extension of $\QQ_\ell$ with valuation ring $R$, integral
closure of $\ZZ_\ell$ in $E$, $\mathfrak m$ the maximal ideal of $R$.
Every $\overline\QQ_\ell$-sheaf $\mathscr G$ is obtained as
$\mathscr F\otimes_E\overline\QQ_\ell$ for some $E$ and some torsion-free 
(i.e. flat) constructible $E$-sheaf $\mathscr F$. This means that
$\mathscr F=``\operatorname{lim\;proj}"\mathscr F_n$, the latter a flat
$R$-sheaf. A lisse $R$-sheaf has all the $\mathscr F_n$ locally constant
sheaves of $R/\mathfrak m^n$-modules, and for each $n$, the above discussion
shows that the functor `fiber at $\overline x$' gives an equivalence of
categories between the category of lisse $R/\mathfrak m^n$-sheaves and the
category of $R/\mathfrak m^n$-modules of finite type together with a
continuous action of $\pi_1(X,\overline x)$. Since the $\mathscr F_n$ have
$\mathscr F_n\otimes_{R/\mathfrak m^{n+1}}R/\mathfrak m^n\xrightarrow{\sim}\mathscr F_{n-1}$, by passing to the limit we get an equivalence between
the category of lisse $R$-sheaves and the category of finite $R$-modules
with continuous action of $\pi_1(X,\overline x)$.

\addtocontents{toc}{\protect\setcounter{tocdepth}{-1}}
\begin{thebibliography}{SGA 4$\tfrac12$}
\bibitem[EGA]{EGA} \textit{Eléments de géométrie algébrique} par A. Grothendieck.
\bibitem[Gr]{Gr} \textit{Formule de Lefschetz et Rationalité des Fonctions
$L$} par A. Grothendieck.
\bibitem[SGAA]{SGAA} SGA 4 ed. Grothendieck, Artin, Verdier.
\bibitem[SGA5]{SGA5} SGA 5, Exposé XV par C. Houzel.
\bibitem[SGA 4$\tfrac{1}{2}$]{SGA4.5}
SGA 4$\frac{1}{2}$, \textit{Rapport sur la formule des traces} par P. Deligne.
\end{thebibliography}
\addtocontents{toc}{\protect\setcounter{tocdepth}{1}}

\end{document}