\documentclass[deligne.tex]{subfiles}

\begin{document}
\subsection*{1.2}\label{sec:st1.2}
Suppose $X$ is locally noetherian and $X'$ is a connected
component of $X$ pointed by a geometric point $a\ra X'$.
Via Grothendieck's Galois theory, over $X'$, the $A$-torsor $T$ can be
identified (after a choice $e$ of identity for the stalk of $T$ at $a$) with
the set $A$ with (continuous) left action of $\pi_1(X',a)$ and right action
of $A$.

Given a homomorphism $\tau:A\ra B$, we wish to produce a map of sets
$\tau:A\ra B$ so that $\tau(ta)=\tau(t)\tau(a)$ and give the set $B$ with 
right action of $B$ a left action of $\pi_1(X',a)$ with respect to which
$\tau$ is equivariant. The given homomorphism of groups sets up a map of sets
compatible with the actions of $A$ and $B$, and the formula
\begin{equation*}
	g\tau(t):=\tau(gt)
\end{equation*}
defines a left $\pi_1$ action on $\tau(A)$. For $s\in B$ not necessarily in
$\tau(A)$,
\begin{equation*}
	gs:=g\tau(e)s=\tau(ge)s
\end{equation*}
extends the left action of $\pi_1$ to all of $B$.
The $B$-torsor $\tau(T)$ is then identified with the set $B$ with right
action of $B$ and the given left action of $\pi_1$, with respect to which
$\tau:T\ra\tau(T)$ is equivariant by construction.
This discussion depends on the choice of identity for the torsor $T$
(the identity for $\tau(T)$ is obtained from the choice for $T$ and the
homomorphism $\tau$) and therefore only defines the torsor $\tau(T)$ up to
isomorphism.

Let $R$ (resp. $E_\lambda$) be a finite $\ZZ_\ell$-algebra
(resp. finite $\QQ_\ell$-algebra).
Recall (SGA 5 Exp. VI ($\alpha$) 1.2.4, ($\beta$) 1.4.1, ($\gamma$) 1.4.2, ($\delta$) 1.4.3) that the category of
\begin{enumerate}[label=(\greek*)]
	\item lisse $\ZZ_\ell$-sheaves
	\item lisse $R$-sheaves
	\item lisse $\QQ_\ell$-sheaves
	\item lisse $E_\lambda$-sheaves
\end{enumerate}
on a connected locally noetherian scheme $X$, pointed by a geometric point
$a\ra X$ is equivalent to the category of
\begin{enumerate}[label=(\greek*)]
	\item $\ZZ_\ell$-modules of finite type on which $\pi_1(X):=\pi_1(X,a)$ acts continuously for the $\ell$-adic topology
	\item $R$-modules of finite type equipped with a continuous and $R$-linear action of $\pi_1(X)$ on the underlying $\ZZ_\ell$-module of finite type
	\item $\QQ_\ell$-vector spaces of finite dimension equipped with a continuous action of $\pi_1(X)$
	\item $E_\lambda$-vector spaces of finite dimension equipped with a continuous and $E_\lambda$-linear action of $\pi_1(X)$ on the underlying $\QQ_\ell$-vector space.
\end{enumerate}
Given an $A$-torsor $T$ on $X$ and a finite-dimensional $E_\lambda$-vector
space $V$, the data of an 
$E_\lambda$-sheaf $\mathscr F$, lisse of rank $\dim V$, together with a
morphism of sheaves $\rho:T\ra\underline{\operatorname{Isom}}(V,\mathscr F)$
satisfying $\rho(ta)=\rho(t)\rho(a)$ is the same (after a choice of identity
$e$ for the stalk of $T$ at $a$) as the data of a continuous
and $E_\lambda$-linear action of $\pi_1(X)$ on the underlying
$\QQ_\ell$-vector space of $V$ and a $\pi_1$-equivariant group homomorphism
$\rho:A\ra\GL(V)$, where the action of $\pi_1$ on $\GL(V)$ is induced by the
action of $\pi_1$ on the second factor of $\Hom(V,V)$.
Given $\rho:A\ra\GL(V)$ a linear representation of $A$, there is a unique
continuous and $E_\lambda$-linear action of $\pi_1$ on the $\QQ_\ell$-vector
space underneath $V$ that makes $\rho$ $\pi_1$-equivariant:
\begin{align*}\tag{$\dagger$}
	\pi_1(X)\times V&\ra V \\
	(g,v)&\mapsto \rho(ge)v
\quad\rightsquigarrow\quad\rho(ga)=\rho(gea)=\rho(ge)\rho(a)=g\rho(a).
\end{align*}

\subsection*{1.3}
Torsors can be discussed in the language of schemes or sheaves; the
distinction comes down to whether the torsor is representable as a sheaf,
and this distinction motivates the introduction of algebraic spaces.
In SGA 1 Exp. V \S2 and SGAD Exp. III \S0 \& Exp. IV \S5
the notion of principal homogeneous space is developed;
these are the representing objects for certain (sheaf) torsors.
There, an $A$-torsor is called a principal homogeneous space under $A$.
More precisely, given a category $\mathscr C$, let
$\widehat{\mathscr C}$ denote its category of set-valued presheaves
$\textbf{Hom}(\mathscr C^\circ,(\textbf{Set}))$.
If $A$ is a $\widehat{\mathscr C}$-group acting on $X$ an object of
$\widehat{\mathscr C}$, $X$ is formally principal homogeneous under $A$
(i.e. an $A$-pseudo torsor) if the equivalent conditions below are satisfied:
\begin{enumerate}[label=(\roman*)]
	\item for each object $S$ of $\mathscr C$, the set $X(S)$ is empty or principal homogeneous under $A(S)$;
	\item the morphism of functors $A\times X\ra X\times X$ defined setwise by $(a,x)\mapsto(ax,x)$ is an isomorphism.
\end{enumerate}
It amounts to the same to say that the canonical morphism of functors
\begin{equation*} X\times A\ra X\times X \end{equation*} is an isomorphism.
If $\mathscr C$ is equipped with a topology, then one says that the
$S$-object $X$ with $S$-group of operators $A$ is fibered principal
homogeneous under $A$ (i.e. is an $A$-torsor) if it is locally trivial; i.e.
there exists a covering family $\{S_i\ra S\}$ such that for each $i$, the
$S_i$-functor $X\times_S S_i$ with $S_i$-functor-group of operators
$A\times_S S_i$ is trivial.

The category $\widehat{\mathscr C}$ has a final object
$\underline{\mathbf{e}}$ which sends an
object of $\mathscr C$ to $\{\emptyset\}$, the set with one element.
This functor is representable iff $\mathscr C$ admits a final object.
The `sections' functor $\Gamma$ is defined on $\widehat{\mathscr C}$ as
$\Hom(-,\underline{\mathbf{e}})$ and on $\mathscr C$ via $X\mapsto h_X$;
if $\mathscr C$ admits a final object, this latter functor is isomorphic to
$\Hom(e,-)$.
SGAD Exp. IV 5.1.2, 5.1.3 observes that $X$ is formally principal 
homogeneous under $A$, there is an isomorphism
\begin{equation*}
	\Gamma(X)\xra{\sim}\operatorname{Isom}_{A\text{-obj.}}(A,X)
\end{equation*}
of principal homogeneous sets under $\Gamma(A)$; therefore an isomorphism
of $A$-objects
\begin{equation*}
	X\xra{\sim}\underline{\operatorname{Isom}}_{A\text{-obj.}}(A,X).
\end{equation*}
The proof is simply that to each section $x$ of $X$ one associates the
morphism $A\ra X$ defined setwise by $a\mapsto xa$.
This implies that an object with group of operators is trivial iff it is
formally principal homogeneous and possesses a section.

The algebraic group schemes in the given extension can also be
considered as sheaves for the fpqc topology on $S$. En effet, the
surjectivity of $\pi$ implies that $\pi$ is faithfully flat, and therefore
an fpqc covering.
\begin{proposition*}[SGAD Exp. IV 5.1.7.1]
Let $\mathscr C$ denote a category possessing a final object, stable
by fiber products, and equipped with a subcanonical topology $\mathcal T$
(such as \textbf{Sch} equipped with fppf, fpqc, étale).
Suppose $\pi:G'\ra G$ is a morphism of $\mathscr C$-groups which is covering
for the topology $\mathcal T$, and $A=\ker\pi$.
Then $G$ represents the quotient sheaf $G'/A$, and $\pi$ is an $A_G$-torsor;
i.e. $G'$ is an $A$-torsor on $G$.
\end{proposition*}
Therefore it makes sense to say `$G'$ is an $A$-torsor on $G$' or that
`the sheaf $T$ of local sections of $\pi$, $T=\Hom_G(-,G')$, is an $A$-
torsor on $G$'; the former represents the latter.
The $A$-torsor $G'$ on $G$ is indeed locally trivial for fpqc: pulling back
along the faithfully flat covering $\pi$, we find
\begin{align*}
	G'\times_G A&\ra G'\times_G G' \\
	(g',a)&\mapsto(g',g'a)
\end{align*}
is indeed an isomorphism, as can be checked setwise.

In order to understand how to add torsors, it is instructive to first recall
how to add extensions of abelian groups. Suppose
\begin{align*}
	&0\ra A\ra G'\ra G\ra 0 \\
	&0\ra A\ra G''\ra G\ra 0
\end{align*}
are exact sequences of abelian groups. $G'$ and $G''$ are $A$-torsors in
\textbf{Set}. The Baer sum $G'+G''$ is constructed from the direct sum of
extensions by pushout along addition for $A$ and pullback along the diagonal 
for $G$.
\begin{equation*}\begin{tikzcd}
	0\arrow[r]&A\oplus A\arrow[dr,phantom,"\lrcorner",near end]\arrow[r]\arrow[d,"+"] &G'\oplus G\arrow[r]\arrow[d] &G\oplus G\arrow[r]\arrow[d]&0 \\
	0\arrow[r]&A\arrow[r]&+_*(G'\oplus G'')\arrow[r]&G\oplus G\arrow[r] \arrow[dl,phantom,"\llcorner",near end]&0 \\
	0\arrow[r]&A\arrow[r]\arrow[u]&\Delta^*+_*(G'\oplus G'')\arrow[r]\arrow[u]&G\arrow[r]\arrow[u,"\Delta"]&0
\end{tikzcd}\end{equation*}
Here, the square marked with $\llcorner$ is cartesian while the one marked 
with $\lrcorner$ is cocartesian.
The Baer sum of $G'$ and $G''$ is $\Delta^*+_*(G'\oplus G'')$.
Now consider the case of 1.3 where we are given an extension of commutative
algebraic group schemes over $S$, and $G'$ is an $A$-torsor over $G$.
Pulling back $G'$ along addition for $G$ yields the fiber product
$+_G^*G'=G'\times_G G\times_S G$.
If $X$ is a scheme over $S$, to give a morphism over $S$ to $+_G^*G'$
amounts to giving two objects $g_1,g_2\in G(X)$ and an object in
$g\in G'(X)$ such that $\pi g=g_1+g_2$.
This data is equivalent to the data of two objects $g_1',g_2'$ in $G'(X)$
mapping to $g_1,g_2$, respectively, modulo the relation which considers two
such pairs equivalent if their sum is the same.
This generalizes pushing out by addition on $A$ to the case of torsors.
The pullback of torsors is given on representing objects by fiber product,
so that given an $S$-morphism $f:X\ra G$, $f^*T=\Hom_X(-,X\times_{f,\pi} G')$.
This torsor is trivial if there is a section $X\ra X\times_{f,\pi} G'$ over $X$;
i.e. a commutative diagram
\begin{equation*}\begin{tikzcd}[column sep=tiny]
	X\arrow[rr]\arrow[dr,"f"']&&G'\arrow[dl,"\pi"] \\
	&G.
\end{tikzcd}\end{equation*}
In other words, the torsor $f^*T$ is defined up to isomorphism by the image
of $f$ in $\Hom_S(X,G)/\pi\Hom_S(X,G')$.

If $g:X\ra G$ is another $S$-morphism, then $g^*T$ is represented by
$X\times_{g,\pi} G'$.
The fiber products $X\times_{f,\pi} G'$ and $X\times_{g,\pi} G'$
are $A$-torsors over $X$. In analogy with Baer sum, a sum of torsors is
given by pushing out by the addition on $A$ followed by pulling back by the
diagonal $\Delta_X:X\ra X\times_S X$. Since $G'$ is defined by
an extension and both torsors come from pulling back $G'$ (i.e. $T$),
this is the same as pulling back $G'$ along the composition of morphisms
\begin{equation*}\begin{tikzcd}
	X\arrow[r,"{(f,g)}"]&G\times_S G\arrow[r,"+"]&G.
\end{tikzcd}\end{equation*}
More explicitly, the morphism $X\ra G\times_S G$ factors as
\begin{equation*}\begin{tikzcd}
	X\arrow[r,"\Delta"]&X\times_S X\arrow[rr,"{(f\circ\pr_1,g\circ\pr_2)}"]&&G\times_S G
\end{tikzcd}\end{equation*}
and therefore pulling back along this morphism corresponds to pulling back
$(f^*T,g^*T)$ along the diagonal $\Delta_X$.
Of course, the fiber product $G'\times_{G'}(G'\times_S G')\simeq G'\times_S G'$
corresponds to the $A\times A$-torsor $(T,T)$.
As we have already seen, pulling back $G'$ along $G\times_S G\xra{+} G$
is the quotient of $(T,T)$ by the relation the action of $(a_1,a_2)$ and
$(a_1',a_2')$ if $a_1+a_2=a_1'+a_2'$. Therefore, $(f+g)^*T$ is the quotient
of $(f^*T,g^*T)$ by this same relation, which verifies the formula 
\begin{equation*} (f+g)^*T=f^*T+g^*T. \end{equation*}

\subsubsection*{The Lang torsor}
For the development of the Lang isogeny and its properties, see
Borel, \emph{Algebraic Groups} \S16. There, he proves that $\mathscr L$ is
surjective when $G^0$ is connected.
If the group $G^0$ is smooth (i.e. geometrically reduced), then $\mathscr L$
is seen to be étale (as a morphism of schemes) by computing its differential
and Milne, \emph{Algebraic Groups}, 1.63.
That the functor $G_0^F=G_0(\FF_q)$ is a consequence of the description of
$F$ on the set $G_0(\FF)$; see \emph{Rapport} 1.1.

\subsection*{1.6}\label{sommes:1.6} If $G_0$ is a commutative algebraic group defines over
$\FF_q$, then the norm map
\begin{equation*}
N:G_0(\FF_{q^n})\ra G_0(\FF_q)\subset G_0(\FF_{q^n})\end{equation*}
is defined, as $F$ is a group homomorphism.
In the case $G_0=\GG_a$, then $\GG_a(\FF_{q^n})=\FF_{q^n}$
and $N$ coincides with the field \emph{trace}
$N(x)=x+x^q+\cdots+x^{q^{n-1}}$, as the group operation is
addition. If $G_0=\GG_m$, on the other hand, then
$\GG_m(\FF_{q^n})=\FF_{q^n}^\times$ and $N$ coincides with the field 
\emph{norm} $N(x)=x\cdot x^q\cdots x^{q^{n-1}}$, as the group operation is
multiplication.

\subsection*{1.7} The formul\ae\
$\mathscr F(\chi,f_0)=f_0^*\mathscr F(\chi,\id_{G_0})$, etc. result from the
fact that the pullback of lisse sheaves corresponds to the homomorphism of
$\pi_1$ induced by $f_0$; i.e. if $x\ra X_0$ is a geometric point and 
$g=f_0(x)$, then both sides of these identities result from the composition 
of maps
\begin{equation*}
	\pi_1(X_0,x)\ra\pi_1(G_0,g)\xra{L_0} G_0(\FF_q)\xra{\chi^{-1}} E_\lambda^*,
\end{equation*}
where the $L_0$ indicates the homomorphism of groups defining the
Lang torsor $L_0(G_0)$.

The formula that motivates this whole business
\begin{equation*} F_x^*=\chi f_0(x)\end{equation*}
follows from the hint that the fiber in $x$ of the morphism 1.2.2 commutes
with $F_x^*$. This is true for the following reason: the geometric point 
$x\in X^F$ and $f_0$ give homomorphisms
\begin{equation*}	
	\Gal(\FF,\FF_q)\ra\pi_1(X_0,x)\ra\pi_1(G_0,x)
\end{equation*}
and $F_x^{*-1}$ coincides with the image of the Frobenius substitution
$\varphi\in\Gal(\FF,\FF_q)$. Commutation of $F_x^*$ with $f_0^*$ is clear 
and it remains to show that $\rho_x$ of 1.2.2 is $\pi_1(X,x)$-equivariant.
The formula ($\dagger$) of \hyperref[sec:st1.2]{the note to 1.2} shows this
explicitly.

\subsection*{1.8} (ii) The inverse image by $f$ of the the $E_\lambda$-sheaf
obtained from $\chi^{-1}(L_0(G_0))$ by extension of scalars from $\FF_q$ to
$k$ is probably better written $f^*\mathscr F(\chi)_1$ than $\mathscr F(\chi f)$.

\subsection*{1.10} The subtlety in this argument is building a bridge 
between $\CC$ and $\FF_q$. Of course, the isomorphisms on singular
cohomology with $\QQ$ coefficients imply isomorphisms on singular cohomology
with $\QQ_\ell$ coefficients for all $\ell$.
The quadric hypersurface $X_0'$ and hyperplane $Y_0'$ in
$\PP_0^{2N}$ over $\FF_q$ are each defined by the vanishing of
a homogeneous polynomial of degree two and one, respectively, in the ring
$\FF_q[X_0,X_1,\ldots,X_{2N}]$; let $f_0$ denote the polynomial defining
$X_0'$. As $X_0'$ is a nonsingular variety over a perfect field, it is
smooth over $\FF_q$. Let $m(x)$ be the minimal polynomial for a primitive
element of $\FF_q$ over $\FF_p$; lifting the coefficients of $m$ to $\ZZ$,
$m$ remains irreducible, and defines a finite extension of domains
$\ZZ\ra\ZZ[x]/(m(x))=:A$ so that $A/(p)\simeq\FF_q$.
The coefficients of $f_0$ now admit lifts to $A$ and define a projective
quadric $X_S'$ in $\PP^{2N}_S$ which is, in particular, flat over
$S:=\Spec A$, and such that the fiber over $(p)$ is $X'_0$.
Recalling EGA IV 12.2.4, the set of points $s\in S$ such that $(X'_S)_s$ is
smooth over $k(s)$ is open. In particular, the fiber over the generic point
$\xi\in S$ is smooth over $k(\xi)$, which is a finite extension of $\QQ$.
The strict henselization $\tilde A$ of $A$ at $(p)$ is a regular local ring 
with spectrum $\tilde S$; $X'_{\tilde S}\ra\tilde S$ is proper and smooth,
and $k(\tilde\xi)$ is an algebraic extension of $k(\xi)$.
Let $\Spec\CC=t\ra\tilde S$ be a geometric point centered on $\tilde\xi$,
and put $X:=(X'_{\tilde S})_t$.
The specialization morphism
\begin{equation*}
	H^*(X',\QQ_\ell)\xra\sim H^*(X,\QQ_\ell)
\end{equation*}
is an isomorphism (Arcata V 3.1). As $X$ is a projective nonsingular quadric
in $\PP^{2N}_\CC$, the comparison theorem between ordinary cohomology and 
étale cohomology (Arcata V 3.5.1) allow us to apply the transcendental
argument to conclude.

\subsection*{1.13}
To compute $R^if_!\QQ_\ell$, we may assume $Y$ is the spectrum of a 
separably closed field and $X=\A^1$. By Poincaré duality and Artin's 
theorem, $H^0_c(\A^1,\QQ_\ell)=0$. As for $H^1_c$, the short exact sequence
of sheaves on $\PP^1$
\begin{equation*}
	0\ra(\QQ_\ell)_{\A^1}\ra\QQ_\ell\ra(\QQ_\ell)_\infty\ra0
\end{equation*}
gives rise to a long exact sequence of cohomology
\begin{equation*}
	0\ra\QQ_\ell\ra\QQ_\ell\ra H^1_c(\A^1,\QQ_\ell)\ra H^1(\PP^1,\QQ_\ell)=0\qquad(\text{as }\operatorname{Pic}^0(\PP^1)=0),
\end{equation*}
verifying $H^1_c(\A^1,\QQ_\ell)=0$.
Evidently $H^0(\A^1,\QQ_\ell)=\QQ_\ell$; by Poincaré we conclude
\begin{equation*}
	\begin{cases}
	R^if_!\QQ_\ell=0\qquad i\ne2 \\
	R^2f_!\QQ_\ell=\QQ_\ell(-1).
\end{cases}\end{equation*}

\subsection*{2.3*} Leray spectral sequence for cohomology with proper 
support is a particular case of spectral sequence of composed functors,
but the proof that for all composition of morphisms $f=gh:X\xra h Y\xra g Z$
we have `well-behaved' transitivity isomorphisms between $Rf_!$ and
$Rg_!Rh_!$ is not straight forward; it is SGAA Exp. XVII 5.1.8, which also
proves that the functors $Rf_!$ are `way out' and triangulated.
The proof of transitivity is formal, relying on \S3 of the same exposé,
which reduces the problem to the analogous ones for proper morphisms and
open immersions, provided one `compatibility' isomorphism, which is 5.1.6.
With transitivity in hand, the spectral sequence
\begin{equation*}
	E_2^{pq}=R^pg_!R^qh_!(K)\Rightarrow R^{p+1}f_!(K),
\end{equation*}
valid for $K$ in $D(X,\mathscr A_X)$ ($\mathscr A$ a sheaf of rings), is
a consequence of a general spectral sequence written down by Verdier:
for all $L$ in $D(Y,\mathscr A_Y)$
\begin{equation*}
	E_2^{pq}=R^pg_!(\mathscr H^q(L))\Rightarrow R^{p+q}g_!(L);
\end{equation*}
see \emph{Des catégories dérivées des catégories abéliennes} 4.4.6.


\subsection*{2.4*} Künneth formula in cohomology with proper support is
SGA 4 Exp. XVII 5.4.

\subsection*{2.5*} The spectral sequence (2.5.2)* is the spectral sequence
associated to a filtration of chain complex. Namely, the complements of the 
closed subsets in the filtration on $X$ gives rise to a filtration on $X$
by open subsets $U_p\subset U_{p+1}$.
To calculate cohomology with proper support, fix a compactification
$j:X\hookrightarrow\overline X$.
In what follows, we write $\mathscr F$ for the sheaf and its various
inverse images. Let $j_p:U_p\hookrightarrow\overline X$ and
$i_{p+1}:U_{p+1}-U_p=X_p-X_{p+1}\hookrightarrow\overline X$ be the immersions. The filtration $j_{p!}\mathscr F\subset j_{p+1!}\mathscr F$ on
$j_!\mathscr F$ has successive quotients isomorphic to $i_{p+1!}\mathscr F$.
In light of the spectral sequence (2.3.1)*, the spectral sequence associated
to a filtered injective resolution of $j_!\mathscr F$
(\href{https://stacks.math.columbia.edu/tag/05TT}{Stacks \texttt{05TT}})
gives rise 
(\href{https://stacks.math.columbia.edu/tag/015W}{Stacks \texttt{015W}})
to the spectral sequence
\begin{equation*}
	E_1^{p,q}=H^{p+q}(\overline X,i_{p+1!}\mathscr F)
	=H_c^{p+q}(X_p-X_{p+1},\mathscr F)
	\Rightarrow H^{p+q}(\overline X,j_!\mathscr F)=H_c^{p+q}(X,\mathscr F).
\end{equation*}

\begin{remark}
Let $C$ be a curve of finite type over an algebraically closed field $k$, 
$P$ the set of rational points of the curve, $\mathscr F$ the constant sheaf
with value $A$ on $C$, and $x\in C(k)$.
The fiber of $P_*P^*\mathscr F$ is not $\mathscr F_x$;
rather, it can be identified with functions from $P$ to $A$ continuous for
the discrete topology on $P$, modulo the relation that $f\sim g$ if
$f(x)=g(x)$ and $f$ and $g$ disagree on only finitely many points.
\end{remark}

\subsection*{2.6*}
SGAA Exp. V \S3 introduces the \v Cech complex and the Cartan-Leray spectral
sequence associated to a covering, which is simply the spectral sequence of
the composition of functors, where the first is inclusion of sheaves into 
presheaves and the second is $\check H^0$ (it is shown that the $\check H^i$ 
associated to the \v Cech complex are indeed right derived functors of
$\check H^0$ on presheaves). The condition on the morphisms in the covering
is simply that fibered products are representable, which is true in the
category of schemes.
However, this is not done for cohomology with support.
In the introduction to SGAA Exp. Vbis, the Leray spectral sequence is 
discussed for an open covering and for a locally finite closed covering,
and this guides SGAA Exp. XVII 6.2.8--6.2.10, which discusses in full detail
the construction of the so-called `extraordinary' spectral sequence
(2.6.2)*. One can also obtain the
more `ordinary' spectral sequence of (2.6.1)* from the argument there by
replacing the trace morphism $u_!u^*\ra\id$ in the case of $u:Y\ra X$ 
separated, étale, surjective, and finite type with the unit of adjunction
$\id\ra u_*u^*$ when $u$ arises from a finite covering by closed subschemes.
Then instead of getting a left resolution of the sheaf, one gets a right
resolution, as in the usual \v Cech resolution.

Given $(X_i)_{i\in I}$ a finite covering by closed subschemes, let
$u:Y=\coprod_i X_i:=Y\ra X$,
\begin{equation*}
	Y_n=\underbrace{Y\times_X\cdots\times_X Y}_{n+1\text{ times}},
\end{equation*}
and $u_n:Y_n\ra X$. The sheaves $\mathscr F_n=u_{n*}u_n^*\mathscr F$ form,
via the units of adjunction $\mathscr F\ra u_{n*}u_n^*\mathscr F$, a
simplicial sheaf coaugmented by $\mathscr F$.
As discussed in SGAA Exp. Vbis, if $\mathscr C^*(\mathscr F)$ denotes an
injective resolution of $\mathscr F$, the double complex
\begin{equation*}
	\mathscr F\rightarrow (u_{p*}\mathscr C^q(u_p^*(\mathscr F)))_{p,q}
\end{equation*}
defines a resolution of $\mathscr F$ by injective sheaves.

\begin{equation*}\begin{tikzcd}
	Y\arrow[rr,"u"]\arrow[dr,"f"']&&X\arrow[dl,"g"] \\
	&\Spec k
\end{tikzcd}\end{equation*}
If $f_n$ is the projection of $Y_n$ to $\Spec k$, then
$Rf_{n!}=Rg_!u_{n!}=Rg_!u_{n*}$. Applying the functor $g_!$ to the double
complex and filtering by semi-simplicial degree gives rise to the
\v Cech spectral sequence
\begin{equation*}
	E_1^{p,q}=R^qf_{p!}(u_p^*\mathscr F)\Rightarrow R^{p+q}g_!\mathscr F;
\end{equation*}
this is (2.6.1)* for cohomology with support.

The problem when taking the $X_i$ to be open is that $u_{n*}\ne u_{n!}$ and
there is no longer a coaugmentation map $\id\ra u_{n!}u_n^*$.
The trace morphism, however, provides an augmentation map
$u_{n!}u_n^*\ra\id$ and one proceeds by resolving to the left to produce the
extraordinary \v Cech resolution and spectral sequence
\begin{equation*}
	E_1^{-p,q}=R^qf_{p!}(u_p^*\mathscr F)\Rightarrow R^{-p+q}g_!\mathscr F;
\end{equation*}
this is (2.6.2)*.
For more on the extraordinary \v Cech resolution, see
SGAA Exp. XVII 6.2.8--6.2.10.

\addtocontents{toc}{\protect\setcounter{tocdepth}{-1}}
\begin{thebibliography}{Sommes trig.}
	\bibitem[Sommes trig.]{Trig} \textit{Application de la formule des traces aux sommes trigonométriques}
	dans SGA $4\frac12$.
\end{thebibliography}
\addtocontents{toc}{\protect\setcounter{tocdepth}{1}}
\end{document}
